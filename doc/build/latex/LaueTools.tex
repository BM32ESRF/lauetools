%% Generated by Sphinx.
\def\sphinxdocclass{report}
\documentclass[letterpaper,10pt,english]{sphinxmanual}
\ifdefined\pdfpxdimen
   \let\sphinxpxdimen\pdfpxdimen\else\newdimen\sphinxpxdimen
\fi \sphinxpxdimen=.75bp\relax

\PassOptionsToPackage{warn}{textcomp}
\usepackage[utf8]{inputenc}
\ifdefined\DeclareUnicodeCharacter
 \ifdefined\DeclareUnicodeCharacterAsOptional
  \DeclareUnicodeCharacter{"00A0}{\nobreakspace}
  \DeclareUnicodeCharacter{"2500}{\sphinxunichar{2500}}
  \DeclareUnicodeCharacter{"2502}{\sphinxunichar{2502}}
  \DeclareUnicodeCharacter{"2514}{\sphinxunichar{2514}}
  \DeclareUnicodeCharacter{"251C}{\sphinxunichar{251C}}
  \DeclareUnicodeCharacter{"2572}{\textbackslash}
 \else
  \DeclareUnicodeCharacter{00A0}{\nobreakspace}
  \DeclareUnicodeCharacter{2500}{\sphinxunichar{2500}}
  \DeclareUnicodeCharacter{2502}{\sphinxunichar{2502}}
  \DeclareUnicodeCharacter{2514}{\sphinxunichar{2514}}
  \DeclareUnicodeCharacter{251C}{\sphinxunichar{251C}}
  \DeclareUnicodeCharacter{2572}{\textbackslash}
 \fi
\fi
\usepackage{cmap}
\usepackage[T1]{fontenc}
\usepackage{amsmath,amssymb,amstext}
\usepackage{babel}
\usepackage{times}
\usepackage[Bjarne]{fncychap}
\usepackage{sphinx}

\usepackage{geometry}

% Include hyperref last.
\usepackage{hyperref}
% Fix anchor placement for figures with captions.
\usepackage{hypcap}% it must be loaded after hyperref.
% Set up styles of URL: it should be placed after hyperref.
\urlstyle{same}

\addto\captionsenglish{\renewcommand{\figurename}{Fig.}}
\addto\captionsenglish{\renewcommand{\tablename}{Table}}
\addto\captionsenglish{\renewcommand{\literalblockname}{Listing}}

\addto\captionsenglish{\renewcommand{\literalblockcontinuedname}{continued from previous page}}
\addto\captionsenglish{\renewcommand{\literalblockcontinuesname}{continues on next page}}

\addto\extrasenglish{\def\pageautorefname{page}}

\setcounter{tocdepth}{1}



\title{LaueTools Documentation}
\date{Mar 13, 2020}
\release{}
\author{JS Micha, O. Robach. S. Tardif}
\newcommand{\sphinxlogo}{\vbox{}}
\renewcommand{\releasename}{}
\makeindex

\begin{document}

\maketitle
\sphinxtableofcontents
\phantomsection\label{\detokenize{index::doc}}



\chapter{Introduction}
\label{\detokenize{intro::doc}}\label{\detokenize{intro:welcome-to-lauetools-s-documentation}}\label{\detokenize{intro:introduction}}
LaueTools is a \sphinxstylestrong{python} written package aiming at analysing from 1 to few 10000s microdiffraction Laue patterns coming from synchrotron CRG-IF BM32 beamline at ESRF.

With pip installer, it is now relatively easy to install LaueTools and put hands immediately on data. See {\hyperref[\detokenize{installation:installation}]{\sphinxcrossref{\DUrole{std,std-ref}{Installation}}}} page.


\section{Tools}
\label{\detokenize{intro:tools}}
LaueTools has got :
\begin{itemize}
\item {} 
Modules to be imported in your own scripts like any \sphinxstylestrong{scientific library}.

\item {} 
LaueTools has got several {\hyperref[\detokenize{GUIs:guis}]{\sphinxcrossref{\DUrole{std,std-ref}{Graphical User Interfaces}}}} to interact graphically with data so that to accelerate the design of future scripts and to set the most finely of parameters for batch processing.

\item {} 
\sphinxstylestrong{Notebooks} to guide the typical way you can handle the data (visualisation, selection, analysis, post processing)

\end{itemize}


\section{Developers}
\label{\detokenize{intro:developers}}
Developers are welcome to improve the code readability and to add new functionalities. Please contact us.


\section{Browse Modules and Functions}
\label{\detokenize{intro:browse-modules-and-functions}}\begin{itemize}
\item {} 
\DUrole{xref,std,std-ref}{genindex}

\item {} 
\DUrole{xref,std,std-ref}{modindex}

\end{itemize}


\chapter{Installation}
\label{\detokenize{installation::doc}}\label{\detokenize{installation:installation}}\label{\detokenize{installation:id1}}
Some dependencies are rather usual (numpy, scipy, matplotlib) while others are more uncommon but very useful (fabio, networkx).

GUIs are based on wxpython graphical libraries which can be tricky to install (Sorry. We are working on it).


\section{How to Get LaueTools code}
\label{\detokenize{installation:how-to-get-lauetools-code}}\begin{itemize}
\item {} 
Download the very last version of the code at \sphinxstylestrong{gitlab.esrf.fr} (but you are also welcome to fork this project):
\begin{quote}

\sphinxurl{https://gitlab.esrf.fr/micha/lauetools}
\end{quote}

\item {} 
or Download last (or older releases) on \sphinxstylestrong{pypi} by means of pip
\begin{quote}

\sphinxurl{https://pypi.org/project/LaueTools/}

if pip is installed:

\fvset{hllines={, ,}}%
\begin{sphinxVerbatim}[commandchars=\\\{\}]
\PYG{n}{pip} \PYG{n}{install} \PYG{n}{lauetools}
\end{sphinxVerbatim}
\end{quote}

\end{itemize}


\section{Build LaueTools Documentation}
\label{\detokenize{installation:build-lauetools-documentation}}
Documentation can be generated, by installing sphinx and a cool html theme:

\fvset{hllines={, ,}}%
\begin{sphinxVerbatim}[commandchars=\\\{\}]
\PYG{n}{pip} \PYG{n}{install} \PYG{n}{sphinx}

\PYG{n}{pip} \PYG{n}{install} \PYG{n}{sphinx}\PYG{o}{\PYGZhy{}}\PYG{n}{rtd}\PYG{o}{\PYGZhy{}}\PYG{n}{theme}
\end{sphinxVerbatim}

You may need rinohtype:

\fvset{hllines={, ,}}%
\begin{sphinxVerbatim}[commandchars=\\\{\}]
\PYG{n}{pip} \PYG{n}{install} \PYG{n}{RinohType}
\end{sphinxVerbatim}

Then from /LaueTools/Documentation folder which contains \sphinxtitleref{Makefile} and 2 folders \sphinxtitleref{build} and \sphinxtitleref{source}, build the documentation

\fvset{hllines={, ,}}%
\begin{sphinxVerbatim}[commandchars=\\\{\}]
make html
\end{sphinxVerbatim}

Files in html format will be browsed in /build/html folder with any web navigator. You can start with index.html.


\chapter{Getting started}
\label{\detokenize{getStarted::doc}}\label{\detokenize{getStarted:getting-started}}

\section{Launch Graphical User Interfaces of LaueTools}
\label{\detokenize{getStarted:launch-graphical-user-interfaces-of-lauetools}}\begin{itemize}
\item {} 
start Lauetools GUIs from command line :

\end{itemize}

To deal with relative import, the package name ‘LaueTools’ must be specified to the python interpreter as following
\begin{quote}

Examples:
\begin{itemize}
\item {} 
python -m LaueTools.LaueToolsGUI

\item {} 
python -m LaueTools.LaueSimulatorGUI

\item {} 
python -m LaueTools.PeaksearchGUI

\end{itemize}
\end{quote}

The first main GUI, LaueToolsGUI can open also the two last GUIs (LaueSimulatorGUI, PeaksearchGUI)

There are additional basic GUIs for batch processing located in FileSeries folder:
\begin{itemize}
\item {} 
python -m LaueTools.FileSeries.Peak\_Search

\item {} 
python -m LaueTools.FileSeries.Index\_Refine

\item {} 
python -m LaueTools.FileSeries.Build\_summary

\item {} 
python -m LaueTools.FileSeries.Plot\_Maps2

\end{itemize}
\begin{itemize}
\item {} 
within interactive python (say, ipython -i), GUI can be started thanks to a start() function:
\begin{itemize}
\item {} 
In {[}1{]} : import LaueTools.LaueToolsGUI as LTGUI

\item {} 
In {[}2{]} : LTGUI.start()

\end{itemize}

\end{itemize}

\begin{sphinxadmonition}{note}{Note:}
in the LaueTools folder :
\begin{itemize}
\item {} 
neither \textgreater{} python LaueToolsGUI

\item {} 
nor in \textgreater{}ipython -i :  \textgreater{} run LaueToolsGUI  will work…

\end{itemize}
\end{sphinxadmonition}


\subsection{Use LaueTools module as a library}
\label{\detokenize{getStarted:use-lauetools-module-as-a-library}}
With pip installation, LaueTools package will be included to python packages. Therefore any module will be callable as the following:
\begin{quote}

-In {[}1{]} : import LaueTools.readmccd as rmccd

-In {[}2{]} : rmccd.readCCDimage(‘myimage.tif’)
\end{quote}

In jupyter-notebook, it is also simple in the same manner:
\begin{quote}

\noindent\sphinxincludegraphics{{notebook0}.jpg}
\end{quote}


\chapter{Conventions}
\label{\detokenize{conventions::doc}}\label{\detokenize{conventions:conventions}}

\section{Mathematics and Conventions}
\label{\detokenize{conventions:mathematics-and-conventions}}

\chapter{Graphical User Interfaces}
\label{\detokenize{GUIs::doc}}\label{\detokenize{GUIs:guis}}\label{\detokenize{GUIs:graphical-user-interfaces}}
LaueTools provides two types of GUI:
\begin{itemize}
\item {} 
GUI with graphical interaction when handling data

\item {} 
GUI used as input file parameters for batch processing

\end{itemize}


\section{Interactive Graphical User Interfaces}
\label{\detokenize{GUIs:interactive-graphical-user-interfaces}}
The main steps of analysis are Laue peaks search, Laue Pattern indexation and unit Cell Refinement. Detector geometry calibration (DetectorCalibrationBoard) and Laue Pattern of Polycrystals (LaueSimulatorGUI) are also available.


\subsection{Peak Search (PeaksearchGUI)}
\label{\detokenize{GUIs:peak-search-peaksearchgui}}

\subsection{Indexation (LaueToolsGUI)}
\label{\detokenize{GUIs:indexation-lauetoolsgui}}

\subsection{Crystal unit cell refinement (LaueToolsGUI)}
\label{\detokenize{GUIs:crystal-unit-cell-refinement-lauetoolsgui}}

\subsection{Detector Geometry Calibration (DetectorCalibrationBoard)}
\label{\detokenize{GUIs:detector-geometry-calibration-detectorcalibrationboard}}

\subsection{Laue pattern simulation of assembly of crystals (LaueSimulatorGUI)}
\label{\detokenize{GUIs:laue-pattern-simulation-of-assembly-of-crystals-lauesimulatorgui}}

\section{Batch Processing Graphical User Interfaces}
\label{\detokenize{GUIs:batch-processing-graphical-user-interfaces}}

\chapter{Tutorials}
\label{\detokenize{Tutorials::doc}}\label{\detokenize{Tutorials:tutorials}}\label{\detokenize{Tutorials:id1}}

\section{Basics of Laue Pattern peak search and Unit cell Refinement}
\label{\detokenize{Basic_PeakSearch_IndexRefine::doc}}\label{\detokenize{Basic_PeakSearch_IndexRefine:basics-of-laue-pattern-peak-search-and-unit-cell-refinement}}

\subsection{This Notebook is a part of Tutorials on LaueTools Suite.}
\label{\detokenize{Basic_PeakSearch_IndexRefine:this-notebook-is-a-part-of-tutorials-on-lauetools-suite}}
Author: J.-S. Micha

Last Revision: August 2019

tested with python3

\sphinxstylestrong{Objectives}
\begin{itemize}
\item {} 
Load and display Laue pattern images

\item {} 
Perform a Peak Search

\item {} 
Perform the indexation of a Laue spots list

\item {} 
Perform the crystal orientation and unit cell refinement

\end{itemize}

Setting absolute path to LaueTools Modules if Lauetools has not been
installed with pip. It is assumed that this notebook is located in a
subfolder (normally Notebooks)

\fvset{hllines={, ,}}%
\begin{sphinxVerbatim}[commandchars=\\\{\}]
\PYG{n}{LaueToolsCode\PYGZus{}Folder} \PYG{o}{=} \PYG{l+s+s1}{\PYGZsq{}}\PYG{l+s+s1}{..}\PYG{l+s+s1}{\PYGZsq{}}
\PYG{k+kn}{import} \PYG{n+nn}{sys}\PYG{o}{,}\PYG{n+nn}{os}
\PYG{n}{abspathLaueTools} \PYG{o}{=}\PYG{n}{os}\PYG{o}{.}\PYG{n}{path}\PYG{o}{.}\PYG{n}{abspath}\PYG{p}{(}\PYG{n}{LaueToolsCode\PYGZus{}Folder}\PYG{p}{)}
\PYG{n+nb}{print}\PYG{p}{(}\PYG{l+s+s1}{\PYGZsq{}}\PYG{l+s+s1}{abspathLaueTools}\PYG{l+s+s1}{\PYGZsq{}}\PYG{p}{,}\PYG{n}{abspathLaueTools}\PYG{p}{)}
\PYG{n}{sys}\PYG{o}{.}\PYG{n}{path}\PYG{o}{.}\PYG{n}{append}\PYG{p}{(}\PYG{n}{LaueToolsCode\PYGZus{}Folder}\PYG{p}{)}
\end{sphinxVerbatim}

\fvset{hllines={, ,}}%
\begin{sphinxVerbatim}[commandchars=\\\{\}]
\PYG{n}{abspathLaueTools} \PYG{o}{/}\PYG{n}{home}\PYG{o}{/}\PYG{n}{micha}\PYG{o}{/}\PYG{n}{LaueToolsPy3}\PYG{o}{/}\PYG{n}{LaueTools}
\end{sphinxVerbatim}

\fvset{hllines={, ,}}%
\begin{sphinxVerbatim}[commandchars=\\\{\}]
\PYG{k+kn}{import} \PYG{n+nn}{LaueTools}
\PYG{n}{LaueTools}\PYG{o}{.}\PYG{n+nv+vm}{\PYGZus{}\PYGZus{}file\PYGZus{}\PYGZus{}}
\end{sphinxVerbatim}

\fvset{hllines={, ,}}%
\begin{sphinxVerbatim}[commandchars=\\\{\}]
\PYG{l+s+s1}{\PYGZsq{}}\PYG{l+s+s1}{/home/micha/LaueToolsPy3/LaueTools/\PYGZus{}\PYGZus{}init\PYGZus{}\PYGZus{}.py}\PYG{l+s+s1}{\PYGZsq{}}
\end{sphinxVerbatim}

\fvset{hllines={, ,}}%
\begin{sphinxVerbatim}[commandchars=\\\{\}]
\PYG{c+c1}{\PYGZsh{}\PYGZpc{}matplotlib inline}
\PYG{o}{\PYGZpc{}}\PYG{n}{matplotlib} \PYG{n}{notebook}

\PYG{k+kn}{import} \PYG{n+nn}{time}\PYG{o}{,}\PYG{n+nn}{copy}\PYG{o}{,}\PYG{n+nn}{os}

\PYG{c+c1}{\PYGZsh{} Third party modules}
\PYG{k+kn}{import} \PYG{n+nn}{matplotlib}     \PYG{c+c1}{\PYGZsh{} graphs and plots}
\PYG{k+kn}{import} \PYG{n+nn}{matplotlib}\PYG{n+nn}{.}\PYG{n+nn}{pyplot} \PYG{k}{as} \PYG{n+nn}{plt}
\PYG{k+kn}{import} \PYG{n+nn}{numpy} \PYG{k}{as} \PYG{n+nn}{np}    \PYG{c+c1}{\PYGZsh{} numerical arrays}

\PYG{c+c1}{\PYGZsh{} LaueTools modules}

\PYG{k+kn}{import} \PYG{n+nn}{LaueTools}\PYG{n+nn}{.}\PYG{n+nn}{IOLaueTools} \PYG{k}{as} \PYG{n+nn}{IOLT}   \PYG{c+c1}{\PYGZsh{} read and write ASCII file  (IO)}
\PYG{k+kn}{import} \PYG{n+nn}{LaueTools}\PYG{n+nn}{.}\PYG{n+nn}{readmccd} \PYG{k}{as} \PYG{n+nn}{RMCCD} \PYG{c+c1}{\PYGZsh{} read CCD and detector binary file, PeakSearch methods}
\end{sphinxVerbatim}

\fvset{hllines={, ,}}%
\begin{sphinxVerbatim}[commandchars=\\\{\}]
\PYG{n}{LaueToolsProjectFolder} \PYG{o}{/}\PYG{n}{home}\PYG{o}{/}\PYG{n}{micha}\PYG{o}{/}\PYG{n}{LaueToolsPy3}\PYG{o}{/}\PYG{n}{LaueTools}
\end{sphinxVerbatim}
\begin{sphinxalltt}
/home/micha/anaconda3/lib/python3.6/site-packages/h5py/\_\_init\_\_.py:36: FutureWarning: Conversion of the second argument of issubdtype from \sphinxtitleref{float} to \sphinxtitleref{np.floating} is deprecated. In future, it will be treated as \sphinxtitleref{np.float64 == np.dtype(float).type}.
  from .\_conv import register\_converters as \_register\_converters
\end{sphinxalltt}

\fvset{hllines={, ,}}%
\begin{sphinxVerbatim}[commandchars=\\\{\}]
module Image / PIL is not installed
Cython compiled module \PYGZsq{}gaussian2D\PYGZsq{} for fast computation is not installed!
module Image / PIL is not installed
\end{sphinxVerbatim}

Considering single image analysis (that belong to the LaueTools
distribution)

\fvset{hllines={, ,}}%
\begin{sphinxVerbatim}[commandchars=\\\{\}]
\PYG{n}{t0} \PYG{o}{=} \PYG{n}{time}\PYG{o}{.}\PYG{n}{time}\PYG{p}{(}\PYG{p}{)}
\PYG{n}{LaueToolsExamplesFolder} \PYG{o}{=} \PYG{n}{os}\PYG{o}{.}\PYG{n}{path}\PYG{o}{.}\PYG{n}{join}\PYG{p}{(}\PYG{n}{LaueToolsCode\PYGZus{}Folder}\PYG{p}{,}\PYG{l+s+s1}{\PYGZsq{}}\PYG{l+s+s1}{Examples}\PYG{l+s+s1}{\PYGZsq{}}\PYG{p}{)}

\PYG{n}{imageindex} \PYG{o}{=} \PYG{k+kc}{None}
\PYG{n}{imagefolder} \PYG{o}{=}\PYG{n}{os}\PYG{o}{.}\PYG{n}{path}\PYG{o}{.}\PYG{n}{join}\PYG{p}{(}\PYG{n}{LaueToolsCode\PYGZus{}Folder}\PYG{p}{,}\PYG{l+s+s1}{\PYGZsq{}}\PYG{l+s+s1}{LaueImages}\PYG{l+s+s1}{\PYGZsq{}}\PYG{p}{)}
\PYG{n}{imagefilename} \PYG{o}{=} \PYG{l+s+s1}{\PYGZsq{}}\PYG{l+s+s1}{Ge\PYGZus{}blanc\PYGZus{}0000.mccd}\PYG{l+s+s1}{\PYGZsq{}}

\PYG{c+c1}{\PYGZsh{}imagefolder =os.path.join(LaueToolsCode\PYGZus{}Folder,\PYGZsq{}LaueImages\PYGZsq{})}
\PYG{c+c1}{\PYGZsh{}imagefilename = \PYGZsq{}CdTe\PYGZus{}I999\PYGZus{}03Jul06\PYGZus{}0200.mccd\PYGZsq{}}
\end{sphinxVerbatim}

Considering analysis of one image in dataset

\sphinxstylestrong{For information:} select image file of interest, in case of set of
images with index. Then, splitting imagefilename allows to loop over
images: prefix+index.extension

\fvset{hllines={, ,}}%
\begin{sphinxVerbatim}[commandchars=\\\{\}]
\PYG{o}{\PYGZpc{}}\PYG{o}{\PYGZpc{}}\PYG{n}{script} \PYG{n}{false}
\PYG{c+c1}{\PYGZsh{} just to show (cell not executed)}

\PYG{n}{imagefolder} \PYG{o}{=}\PYG{l+s+s1}{\PYGZsq{}}\PYG{l+s+s1}{/home/micha/LaueProjects/VO2/ToScript/Data\PYGZus{}VO2}\PYG{l+s+s1}{\PYGZsq{}}

\PYG{n}{prefixfilename}\PYG{o}{=} \PYG{l+s+s1}{\PYGZsq{}}\PYG{l+s+s1}{CT30\PYGZus{}}\PYG{l+s+s1}{\PYGZsq{}}
\PYG{n}{imageindex}\PYG{o}{=}\PYG{l+m+mi}{20}

\PYG{n}{imagefilename} \PYG{o}{=} \PYG{n}{prefixfilename}\PYG{o}{+}\PYG{l+s+s1}{\PYGZsq{}}\PYG{l+s+si}{\PYGZpc{}04d}\PYG{l+s+s1}{.mccd}\PYG{l+s+s1}{\PYGZsq{}}\PYG{o}{\PYGZpc{}}\PYG{n}{imageindex}
\PYG{n+nb}{print}\PYG{p}{(}\PYG{l+s+s2}{\PYGZdq{}}\PYG{l+s+s2}{imagefilename :}\PYG{l+s+s2}{\PYGZdq{}}\PYG{p}{,}\PYG{n}{imagefilename}\PYG{p}{)}
\PYG{c+c1}{\PYGZsh{} you should see: imagefilename : CT30\PYGZus{}0020.mccd}
\end{sphinxVerbatim}

\sphinxstylestrong{Read image file, get data and display it}

Function \sphinxcode{\sphinxupquote{readCCDimage()}} returns \sphinxcode{\sphinxupquote{dataimage}} as a 2D numpy array
with the proper dimensions and orientation given by \sphinxcode{\sphinxupquote{framedim}} and the
geometrical transformations labelled by \sphinxcode{\sphinxupquote{fliprot}}

\fvset{hllines={, ,}}%
\begin{sphinxVerbatim}[commandchars=\\\{\}]
\PYG{n+nb}{print}\PYG{p}{(}\PYG{l+s+s1}{\PYGZsq{}}\PYG{l+s+s1}{Displaying }\PYG{l+s+si}{\PYGZpc{}s}\PYG{l+s+se}{\PYGZbs{}n}\PYG{l+s+s1}{\PYGZsq{}}\PYG{o}{\PYGZpc{}}\PYG{n}{imagefilename}\PYG{p}{)}
\PYG{n}{dataimage}\PYG{p}{,} \PYG{n}{framedim}\PYG{p}{,} \PYG{n}{fliprot} \PYG{o}{=} \PYG{n}{RMCCD}\PYG{o}{.}\PYG{n}{readCCDimage}\PYG{p}{(}\PYG{n}{imagefilename}\PYG{p}{,}\PYG{n}{dirname}\PYG{o}{=}\PYG{n}{imagefolder}\PYG{p}{,}\PYG{n}{CCDLabel}\PYG{o}{=}\PYG{l+s+s1}{\PYGZsq{}}\PYG{l+s+s1}{MARCCD165}\PYG{l+s+s1}{\PYGZsq{}}\PYG{p}{)}
\PYG{n}{fullpathimagefile}\PYG{o}{=} \PYG{n}{os}\PYG{o}{.}\PYG{n}{path}\PYG{o}{.}\PYG{n}{join}\PYG{p}{(}\PYG{n}{imagefolder}\PYG{p}{,}\PYG{n}{imagefilename}\PYG{p}{)}

\PYG{n}{fig}\PYG{p}{,} \PYG{n}{ax} \PYG{o}{=} \PYG{n}{plt}\PYG{o}{.}\PYG{n}{subplots}\PYG{p}{(}\PYG{n}{figsize}\PYG{o}{=}\PYG{p}{(}\PYG{l+m+mi}{4}\PYG{p}{,}\PYG{l+m+mi}{4}\PYG{p}{)}\PYG{p}{)}

\PYG{n}{ax}\PYG{o}{.}\PYG{n}{imshow}\PYG{p}{(}\PYG{n}{dataimage}\PYG{p}{,}\PYG{n}{vmin}\PYG{o}{=}\PYG{l+m+mi}{0}\PYG{p}{,}\PYG{n}{vmax}\PYG{o}{=}\PYG{l+m+mi}{2000}\PYG{p}{)}
\PYG{n}{ax}\PYG{o}{.}\PYG{n}{set\PYGZus{}title}\PYG{p}{(}\PYG{l+s+s1}{\PYGZsq{}}\PYG{l+s+si}{\PYGZpc{}s}\PYG{l+s+s1}{\PYGZsq{}}\PYG{o}{\PYGZpc{}}\PYG{n}{imagefilename}\PYG{p}{)}
\end{sphinxVerbatim}

\fvset{hllines={, ,}}%
\begin{sphinxVerbatim}[commandchars=\\\{\}]
\PYG{n}{Displaying} \PYG{n}{Ge\PYGZus{}blanc\PYGZus{}0000}\PYG{o}{.}\PYG{n}{mccd}

\PYG{n}{nb} \PYG{n}{elements} \PYG{l+m+mi}{4194304}
\PYG{n}{framedim} \PYG{p}{(}\PYG{l+m+mi}{2048}\PYG{p}{,} \PYG{l+m+mi}{2048}\PYG{p}{)}
\PYG{n}{framedim} \PYG{n}{nb} \PYG{n}{of} \PYG{n}{elements} \PYG{l+m+mi}{4194304}
\end{sphinxVerbatim}

\fvset{hllines={, ,}}%
\begin{sphinxVerbatim}[commandchars=\\\{\}]
\PYG{o}{\PYGZlt{}}\PYG{n}{IPython}\PYG{o}{.}\PYG{n}{core}\PYG{o}{.}\PYG{n}{display}\PYG{o}{.}\PYG{n}{Javascript} \PYG{n+nb}{object}\PYG{o}{\PYGZgt{}}
\end{sphinxVerbatim}



\fvset{hllines={, ,}}%
\begin{sphinxVerbatim}[commandchars=\\\{\}]
\PYG{n}{Text}\PYG{p}{(}\PYG{l+m+mf}{0.5}\PYG{p}{,}\PYG{l+m+mi}{1}\PYG{p}{,}\PYG{l+s+s1}{\PYGZsq{}}\PYG{l+s+s1}{Ge\PYGZus{}blanc\PYGZus{}0000.mccd}\PYG{l+s+s1}{\PYGZsq{}}\PYG{p}{)}
\end{sphinxVerbatim}

\sphinxstylestrong{*peaksearch*} is performed in two main steps: - 1) blobs or local
maxima finder - 2) for blob, refinement starting from blob average
center.

For the first step, \sphinxcode{\sphinxupquote{readCCDimage()}} is called to obtain raw data if
no different data array is provided with the argument
\sphinxcode{\sphinxupquote{Data\_for\_localMaxima}} (set to \sphinxcode{\sphinxupquote{None}} by default). After second
step, Peaksearch results can be purged from peaks already present in a
file as an optional argument \sphinxcode{\sphinxupquote{Remove\_BlackListedPeaks\_fromfile}}.

\fvset{hllines={, ,}}%
\begin{sphinxVerbatim}[commandchars=\\\{\}]
\PYG{k+kn}{import} \PYG{n+nn}{os}
\PYG{n}{ti1}\PYG{o}{=} \PYG{n}{time}\PYG{o}{.}\PYG{n}{time}\PYG{p}{(}\PYG{p}{)}

\PYG{c+c1}{\PYGZsh{}blacklistedpeaksfile=os.path.join(folder,\PYGZsq{}Blacklist.dat\PYGZsq{})}
\PYG{n}{blacklistedpeaksfile} \PYG{o}{=} \PYG{k+kc}{None}

\PYG{n}{res}\PYG{o}{=}\PYG{n}{RMCCD}\PYG{o}{.}\PYG{n}{PeakSearch}\PYG{p}{(}\PYG{n}{fullpathimagefile}\PYG{p}{,}\PYG{n}{CCDLabel}\PYG{o}{=}\PYG{l+s+s1}{\PYGZsq{}}\PYG{l+s+s1}{MARCCD165}\PYG{l+s+s1}{\PYGZsq{}}\PYG{p}{,}
                     \PYG{n}{return\PYGZus{}histo}\PYG{o}{=}\PYG{l+m+mi}{0}\PYG{p}{,}\PYG{n}{local\PYGZus{}maxima\PYGZus{}search\PYGZus{}method}\PYG{o}{=}\PYG{l+m+mi}{0}\PYG{p}{,}
                     \PYG{n}{IntensityThreshold}\PYG{o}{=}\PYG{l+m+mi}{200}\PYG{p}{,}
                     \PYG{n}{boxsize}\PYG{o}{=}\PYG{l+m+mi}{5}\PYG{p}{,}
                     \PYG{n}{fit\PYGZus{}peaks\PYGZus{}gaussian}\PYG{o}{=}\PYG{l+m+mi}{1}\PYG{p}{,}
                     \PYG{n}{FitPixelDev}\PYG{o}{=}\PYG{l+m+mi}{10}\PYG{p}{,}
                     \PYG{n}{Data\PYGZus{}for\PYGZus{}localMaxima}\PYG{o}{=}\PYG{k+kc}{None}\PYG{p}{,}\PYG{c+c1}{\PYGZsh{}newdataimage,}
                     \PYG{n}{Remove\PYGZus{}BlackListedPeaks\PYGZus{}fromfile}\PYG{o}{=}\PYG{n}{blacklistedpeaksfile}\PYG{p}{)}
\PYG{n}{tps} \PYG{o}{=}\PYG{n}{time}\PYG{o}{.}\PYG{n}{time}\PYG{p}{(}\PYG{p}{)}
\PYG{n+nb}{print}\PYG{p}{(}\PYG{l+s+s2}{\PYGZdq{}}\PYG{l+s+s2}{peak search time}\PYG{l+s+s2}{\PYGZdq{}}\PYG{p}{,}\PYG{n}{tps}\PYG{o}{\PYGZhy{}}\PYG{n}{ti1}\PYG{p}{)}
\end{sphinxVerbatim}
\begin{sphinxalltt}
CCDLabel:  MARCCD165
nb of pixels (4194304,)
nb elements 4194304
framedim (2048, 2048)
framedim nb of elements 4194304
image from filename ../LaueImages/Ge\_blanc\_0000.mccd read!
Read Image. Execution time : 0.006 seconds
Data.shape for local maxima (2048, 2048)
Using simple intensity thresholding to detect local maxima (method 1/3)
len(peaklist) 82
Local maxima search. Execution time : 0.336 seconds
Keep 82 from 82 initial peaks (ready for peak positions and shape fitting)

\sphinxstylestrong{*************}
82 local maxima found

 Fitting of each local maxima

addImax False
nb elements 4194304
framedim (2048, 2048)
framedim nb of elements 4194304
framedim in readoneimage\_manycrops (2048, 2048)
fitting time for 82 peaks is : 0.2039
nb of results:  82
After fitting, 0/82 peaks have been rejected
 due to (final - initial position)\textgreater{} FitPixelDev = 10
0 spots have been rejected
 due to negative baseline
0 spots have been rejected
 due to much intensity
0 spots have been rejected
 due to weak intensity
0 spots have been rejected
 due to small peak size
0 spots have been rejected
 due to large peak size
ToTake \{0, 1, 2, 3, 4, 5, 6, 7, 8, 9, 10, 11, 12, 13, 14, 15, 16, 17, 18, 19, 20, 21, 22, 23, 24, 25, 26, 27, 28, 29, 30, 31, 32, 33, 34, 35, 36, 37, 38, 39, 40, 41, 42, 43, 44, 45, 46, 47, 48, 49, 50, 51, 52, 53, 54, 55, 56, 57, 58, 59, 60, 61, 62, 63, 64, 65, 66, 67, 68, 69, 70, 71, 72, 73, 74, 75, 76, 77, 78, 79, 80, 81\}
len(ToTake) 82

82 fitted peak(s)

Removing duplicates from fit

82 peaks found after removing duplicates (minimum intermaxima distance = 5)
peak search time 0.5514366626739502
\end{sphinxalltt}

\sphinxstylestrong{Spots properties}:

peak\_X, peak\_Y, peak\_I, peak\_fwaxmaj, peak\_fwaxmin,
peak\_inclination, Xdev, Ydev, peak\_bkg, Ipixmax,

Spots are sorted by intensity (according to the 2D gaussian fit)

\fvset{hllines={, ,}}%
\begin{sphinxVerbatim}[commandchars=\\\{\}]
\PYG{n}{peaklist}\PYG{o}{=}\PYG{n}{res}\PYG{p}{[}\PYG{l+m+mi}{0}\PYG{p}{]}
\PYG{n+nb}{print}\PYG{p}{(}\PYG{l+s+s1}{\PYGZsq{}}\PYG{l+s+s1}{Digital Spots properties for the 5 most intense spots}\PYG{l+s+s1}{\PYGZsq{}}\PYG{p}{)}
\PYG{n+nb}{print}\PYG{p}{(}\PYG{n}{peaklist}\PYG{p}{[}\PYG{p}{:}\PYG{l+m+mi}{6}\PYG{p}{]}\PYG{p}{)}
\end{sphinxVerbatim}

\fvset{hllines={, ,}}%
\begin{sphinxVerbatim}[commandchars=\\\{\}]
\PYG{n}{Digital} \PYG{n}{Spots} \PYG{n}{properties} \PYG{k}{for} \PYG{n}{the} \PYG{l+m+mi}{5} \PYG{n}{most} \PYG{n}{intense} \PYG{n}{spots}
\PYG{p}{[}\PYG{p}{[} \PYG{l+m+mf}{6.231334887002143e+02}  \PYG{l+m+mf}{1.657728161614024e+03}  \PYG{l+m+mf}{2.979937967247360e+04}
   \PYG{l+m+mf}{8.515577956136006e\PYGZhy{}01}  \PYG{l+m+mf}{7.511212178165599e\PYGZhy{}01}  \PYG{l+m+mf}{1.820409415704489e+01}
   \PYG{l+m+mf}{1.334887002143432e\PYGZhy{}01} \PYG{o}{\PYGZhy{}}\PYG{l+m+mf}{2.718383859764799e\PYGZhy{}01}  \PYG{l+m+mf}{2.966046444454810e+02}
   \PYG{l+m+mf}{2.000000000000000e+02}\PYG{p}{]}
 \PYG{p}{[} \PYG{l+m+mf}{1.244326205473488e+03}  \PYG{l+m+mf}{1.662150473603958e+03}  \PYG{l+m+mf}{2.242563070589073e+04}
   \PYG{l+m+mf}{6.977180389301006e\PYGZhy{}01}  \PYG{l+m+mf}{6.390759549880105e\PYGZhy{}01}  \PYG{l+m+mf}{1.293529253564775e+02}
   \PYG{l+m+mf}{3.262054734875619e\PYGZhy{}01}  \PYG{l+m+mf}{1.504736039580621e\PYGZhy{}01}  \PYG{l+m+mf}{1.998717405717028e+02}
   \PYG{l+m+mf}{2.000000000000000e+02}\PYG{p}{]}
 \PYG{p}{[} \PYG{l+m+mf}{9.330365915823824e+02}  \PYG{l+m+mf}{1.215440948340315e+03}  \PYG{l+m+mf}{2.219753607623998e+04}
   \PYG{l+m+mf}{7.846021988134166e\PYGZhy{}01}  \PYG{l+m+mf}{7.862341648303387e\PYGZhy{}01}  \PYG{l+m+mf}{3.246533552343026e+02}
   \PYG{l+m+mf}{3.659158238235705e\PYGZhy{}02}  \PYG{l+m+mf}{4.409483403153445e\PYGZhy{}01}  \PYG{l+m+mf}{2.110241433113940e+02}
   \PYG{l+m+mf}{2.000000000000000e+02}\PYG{p}{]}
 \PYG{p}{[} \PYG{l+m+mf}{5.852254505694141e+02}  \PYG{l+m+mf}{5.887990375606668e+02}  \PYG{l+m+mf}{9.528825561251553e+03}
   \PYG{l+m+mf}{7.791458064142315e\PYGZhy{}01}  \PYG{l+m+mf}{7.399755695034808e\PYGZhy{}01}  \PYG{l+m+mf}{1.068330480796285e+02}
   \PYG{l+m+mf}{2.254505694140789e\PYGZhy{}01} \PYG{o}{\PYGZhy{}}\PYG{l+m+mf}{2.009624393332388e\PYGZhy{}01}  \PYG{l+m+mf}{1.501265419627265e+02}
   \PYG{l+m+mf}{2.000000000000000e+02}\PYG{p}{]}
 \PYG{p}{[} \PYG{l+m+mf}{1.276607259246803e+03}  \PYG{l+m+mf}{6.002998208781320e+02}  \PYG{l+m+mf}{9.153971543092421e+03}
   \PYG{l+m+mf}{8.153699693677648e\PYGZhy{}01}  \PYG{l+m+mf}{8.035243472697837e\PYGZhy{}01}  \PYG{l+m+mf}{8.044575533084571e+01}
  \PYG{o}{\PYGZhy{}}\PYG{l+m+mf}{3.927407531973586e\PYGZhy{}01}  \PYG{l+m+mf}{2.998208781319818e\PYGZhy{}01}  \PYG{l+m+mf}{1.324821469609317e+02}
   \PYG{l+m+mf}{2.000000000000000e+02}\PYG{p}{]}
 \PYG{p}{[} \PYG{l+m+mf}{9.326643784727646e+02}  \PYG{l+m+mf}{7.500763813513167e+02}  \PYG{l+m+mf}{6.940078500922347e+03}
   \PYG{l+m+mf}{7.065049204848781e\PYGZhy{}01}  \PYG{l+m+mf}{7.482895847319173e\PYGZhy{}01}  \PYG{l+m+mf}{8.063512928783894e+00}
  \PYG{o}{\PYGZhy{}}\PYG{l+m+mf}{3.356215272353893e\PYGZhy{}01}  \PYG{l+m+mf}{7.638135131674062e\PYGZhy{}02}  \PYG{l+m+mf}{1.317606341369902e+02}
   \PYG{l+m+mf}{2.000000000000000e+02}\PYG{p}{]}\PYG{p}{]}
\end{sphinxVerbatim}

\fvset{hllines={, ,}}%
\begin{sphinxVerbatim}[commandchars=\\\{\}]
\PYG{n+nb}{print}\PYG{p}{(}\PYG{l+s+s1}{\PYGZsq{}}\PYG{l+s+s1}{X, Y pixel refinement positions for the first 5 spots}\PYG{l+s+s1}{\PYGZsq{}}\PYG{p}{)}
\PYG{n}{peaklist}\PYG{p}{[}\PYG{p}{:}\PYG{l+m+mi}{5}\PYG{p}{,}\PYG{p}{:}\PYG{l+m+mi}{2}\PYG{p}{]}
\end{sphinxVerbatim}

\fvset{hllines={, ,}}%
\begin{sphinxVerbatim}[commandchars=\\\{\}]
\PYG{n}{X}\PYG{p}{,} \PYG{n}{Y} \PYG{n}{pixel} \PYG{n}{refinement} \PYG{n}{positions} \PYG{k}{for} \PYG{n}{the} \PYG{n}{first} \PYG{l+m+mi}{5} \PYG{n}{spots}
\end{sphinxVerbatim}

\fvset{hllines={, ,}}%
\begin{sphinxVerbatim}[commandchars=\\\{\}]
\PYG{n}{array}\PYG{p}{(}\PYG{p}{[}\PYG{p}{[} \PYG{l+m+mf}{623.1334887002143}\PYG{p}{,} \PYG{l+m+mf}{1657.7281616140235}\PYG{p}{]}\PYG{p}{,}
       \PYG{p}{[}\PYG{l+m+mf}{1244.3262054734876}\PYG{p}{,} \PYG{l+m+mf}{1662.150473603958} \PYG{p}{]}\PYG{p}{,}
       \PYG{p}{[} \PYG{l+m+mf}{933.0365915823824}\PYG{p}{,} \PYG{l+m+mf}{1215.4409483403153}\PYG{p}{]}\PYG{p}{,}
       \PYG{p}{[} \PYG{l+m+mf}{585.2254505694141}\PYG{p}{,}  \PYG{l+m+mf}{588.7990375606668}\PYG{p}{]}\PYG{p}{,}
       \PYG{p}{[}\PYG{l+m+mf}{1276.6072592468026}\PYG{p}{,}  \PYG{l+m+mf}{600.299820878132} \PYG{p}{]}\PYG{p}{]}\PYG{p}{)}
\end{sphinxVerbatim}

\sphinxstyleemphasis{add markers to image}

\fvset{hllines={, ,}}%
\begin{sphinxVerbatim}[commandchars=\\\{\}]
\PYG{k}{if} \PYG{n+nb}{len}\PYG{p}{(}\PYG{n}{peaklist}\PYG{p}{)}\PYG{o}{\PYGZlt{}}\PYG{o}{=}\PYG{l+m+mi}{1}\PYG{p}{:} \PYG{k}{raise} \PYG{n+ne}{ValueError}

\PYG{c+c1}{\PYGZsh{}datatoplot=newdataimage}
\PYG{n}{datatoplot} \PYG{o}{=} \PYG{n}{dataimage}

\PYG{n}{fig}\PYG{p}{,} \PYG{n}{ax} \PYG{o}{=} \PYG{n}{plt}\PYG{o}{.}\PYG{n}{subplots}\PYG{p}{(}\PYG{p}{)}
\PYG{n}{ax}\PYG{o}{.}\PYG{n}{imshow}\PYG{p}{(}\PYG{n}{datatoplot}\PYG{p}{,}\PYG{n}{vmin}\PYG{o}{=}\PYG{l+m+mi}{0}\PYG{p}{,}\PYG{n}{vmax}\PYG{o}{=}\PYG{l+m+mi}{1000}\PYG{p}{,}\PYG{n}{cmap}\PYG{o}{=}\PYG{l+s+s1}{\PYGZsq{}}\PYG{l+s+s1}{hot}\PYG{l+s+s1}{\PYGZsq{}}\PYG{p}{)}

\PYG{k+kn}{from} \PYG{n+nn}{matplotlib}\PYG{n+nn}{.}\PYG{n+nn}{patches} \PYG{k}{import} \PYG{n}{Circle}

\PYG{n}{F}\PYG{o}{=}\PYG{n}{plt}\PYG{o}{.}\PYG{n}{gcf}\PYG{p}{(}\PYG{p}{)}
\PYG{n}{axes}\PYG{o}{=}\PYG{n}{F}\PYG{o}{.}\PYG{n}{gca}\PYG{p}{(}\PYG{p}{)}
\PYG{n}{F}\PYG{o}{.}\PYG{n}{get\PYGZus{}dpi}\PYG{p}{(}\PYG{p}{)}
\PYG{n}{defaultSize}\PYG{o}{=}\PYG{n}{F}\PYG{o}{.}\PYG{n}{get\PYGZus{}size\PYGZus{}inches}\PYG{p}{(}\PYG{p}{)}
\PYG{n}{F}\PYG{o}{.}\PYG{n}{set\PYGZus{}size\PYGZus{}inches}\PYG{p}{(}\PYG{n}{defaultSize}\PYG{o}{*}\PYG{l+m+mf}{1.5}\PYG{p}{)}

\PYG{c+c1}{\PYGZsh{} delete previous patches:}

\PYG{n}{axes}\PYG{o}{.}\PYG{n}{patches} \PYG{o}{=} \PYG{p}{[}\PYG{p}{]}

\PYG{c+c1}{\PYGZsh{} rebuild circular markers}
\PYG{n}{largehollowcircles} \PYG{o}{=} \PYG{p}{[}\PYG{p}{]}
\PYG{n}{smallredcircles} \PYG{o}{=} \PYG{p}{[}\PYG{p}{]}
\PYG{c+c1}{\PYGZsh{} correction only to fit peak position to the display}
\PYG{n}{offset\PYGZus{}convention} \PYG{o}{=} \PYG{n}{np}\PYG{o}{.}\PYG{n}{array}\PYG{p}{(}\PYG{p}{[}\PYG{l+m+mi}{1}\PYG{p}{,} \PYG{l+m+mi}{1}\PYG{p}{]}\PYG{p}{)}

\PYG{n}{XYlist} \PYG{o}{=} \PYG{n}{peaklist}\PYG{p}{[}\PYG{p}{:}\PYG{p}{,} \PYG{p}{:}\PYG{l+m+mi}{2}\PYG{p}{]} \PYG{o}{\PYGZhy{}} \PYG{n}{offset\PYGZus{}convention}

\PYG{k}{for} \PYG{n}{po} \PYG{o+ow}{in} \PYG{n}{XYlist}\PYG{p}{:}

    \PYG{n}{large\PYGZus{}circle} \PYG{o}{=} \PYG{n}{Circle}\PYG{p}{(}\PYG{n}{po}\PYG{p}{,} \PYG{l+m+mi}{7}\PYG{p}{,} \PYG{n}{fill}\PYG{o}{=}\PYG{k+kc}{False}\PYG{p}{,} \PYG{n}{color}\PYG{o}{=}\PYG{l+s+s1}{\PYGZsq{}}\PYG{l+s+s1}{b}\PYG{l+s+s1}{\PYGZsq{}}\PYG{p}{)}
    \PYG{n}{center\PYGZus{}circle} \PYG{o}{=} \PYG{n}{Circle}\PYG{p}{(}\PYG{n}{po}\PYG{p}{,} \PYG{o}{.}\PYG{l+m+mi}{5} \PYG{p}{,} \PYG{n}{fill}\PYG{o}{=}\PYG{k+kc}{True}\PYG{p}{,} \PYG{n}{color}\PYG{o}{=}\PYG{l+s+s1}{\PYGZsq{}}\PYG{l+s+s1}{r}\PYG{l+s+s1}{\PYGZsq{}}\PYG{p}{)}
    \PYG{n}{axes}\PYG{o}{.}\PYG{n}{add\PYGZus{}patch}\PYG{p}{(}\PYG{n}{large\PYGZus{}circle}\PYG{p}{)}
    \PYG{n}{axes}\PYG{o}{.}\PYG{n}{add\PYGZus{}patch}\PYG{p}{(}\PYG{n}{center\PYGZus{}circle}\PYG{p}{)}

    \PYG{n}{largehollowcircles}\PYG{o}{.}\PYG{n}{append}\PYG{p}{(}\PYG{n}{large\PYGZus{}circle}\PYG{p}{)}
    \PYG{n}{smallredcircles}\PYG{o}{.}\PYG{n}{append}\PYG{p}{(}\PYG{n}{center\PYGZus{}circle}\PYG{p}{)}
\end{sphinxVerbatim}

\fvset{hllines={, ,}}%
\begin{sphinxVerbatim}[commandchars=\\\{\}]
\PYG{o}{\PYGZlt{}}\PYG{n}{IPython}\PYG{o}{.}\PYG{n}{core}\PYG{o}{.}\PYG{n}{display}\PYG{o}{.}\PYG{n}{Javascript} \PYG{n+nb}{object}\PYG{o}{\PYGZgt{}}
\end{sphinxVerbatim}



\sphinxstylestrong{List of peaks props is written in a file with extension .dat, here the
variable is {}`{}`datfilename{}`{}`}

\fvset{hllines={, ,}}%
\begin{sphinxVerbatim}[commandchars=\\\{\}]
\PYG{k}{if} \PYG{n}{imageindex} \PYG{o+ow}{is} \PYG{o+ow}{not} \PYG{k+kc}{None}\PYG{p}{:}
    \PYG{n}{peaklistprefix}\PYG{o}{=}\PYG{n}{prefixfilename}\PYG{o}{+}\PYG{l+s+s1}{\PYGZsq{}}\PYG{l+s+s1}{cor\PYGZus{}}\PYG{l+s+si}{\PYGZpc{}04d}\PYG{l+s+s1}{\PYGZsq{}}\PYG{o}{\PYGZpc{}}\PYG{n}{imageindex}
\PYG{k}{else}\PYG{p}{:}
    \PYG{n}{peaklistprefix}\PYG{o}{=}\PYG{n}{imagefilename}\PYG{o}{.}\PYG{n}{split}\PYG{p}{(}\PYG{l+s+s1}{\PYGZsq{}}\PYG{l+s+s1}{.}\PYG{l+s+s1}{\PYGZsq{}}\PYG{p}{)}\PYG{p}{[}\PYG{l+m+mi}{0}\PYG{p}{]}\PYG{o}{+}\PYG{l+s+s1}{\PYGZsq{}}\PYG{l+s+s1}{Notebook}\PYG{l+s+s1}{\PYGZsq{}}
\PYG{n+nb}{print}\PYG{p}{(}\PYG{l+s+s1}{\PYGZsq{}}\PYG{l+s+s1}{peaklist.shape}\PYG{l+s+s1}{\PYGZsq{}}\PYG{p}{,}\PYG{n}{peaklist}\PYG{o}{.}\PYG{n}{shape}\PYG{p}{)}
\PYG{n+nb}{print}\PYG{p}{(}\PYG{l+s+s2}{\PYGZdq{}}\PYG{l+s+s2}{fullpathimagefile}\PYG{l+s+s2}{\PYGZdq{}}\PYG{p}{,}\PYG{n}{fullpathimagefile}\PYG{p}{)}
\PYG{n+nb}{print}\PYG{p}{(}\PYG{l+s+s1}{\PYGZsq{}}\PYG{l+s+s1}{imagefolder}\PYG{l+s+s1}{\PYGZsq{}}\PYG{p}{,}\PYG{n}{imagefolder}\PYG{p}{)}
\PYG{n}{RMCCD}\PYG{o}{.}\PYG{n}{writepeaklist}\PYG{p}{(}\PYG{n}{peaklist}\PYG{p}{,}\PYG{n}{peaklistprefix}\PYG{p}{,}\PYG{n}{outputfolder}\PYG{o}{=}\PYG{n}{imagefolder}\PYG{p}{,}\PYG{n}{initialfilename}\PYG{o}{=}\PYG{n}{fullpathimagefile}\PYG{p}{)}

\PYG{n}{datfilename} \PYG{o}{=} \PYG{n}{peaklistprefix}\PYG{o}{+}\PYG{l+s+s1}{\PYGZsq{}}\PYG{l+s+s1}{.dat}\PYG{l+s+s1}{\PYGZsq{}}
\end{sphinxVerbatim}

\fvset{hllines={, ,}}%
\begin{sphinxVerbatim}[commandchars=\\\{\}]
\PYG{n}{peaklist}\PYG{o}{.}\PYG{n}{shape} \PYG{p}{(}\PYG{l+m+mi}{82}\PYG{p}{,} \PYG{l+m+mi}{10}\PYG{p}{)}
\PYG{n}{fullpathimagefile} \PYG{o}{.}\PYG{o}{.}\PYG{o}{/}\PYG{n}{LaueImages}\PYG{o}{/}\PYG{n}{Ge\PYGZus{}blanc\PYGZus{}0000}\PYG{o}{.}\PYG{n}{mccd}
\PYG{n}{imagefolder} \PYG{o}{.}\PYG{o}{.}\PYG{o}{/}\PYG{n}{LaueImages}
\PYG{n}{table} \PYG{n}{of} \PYG{l+m+mi}{82} \PYG{n}{peak}\PYG{p}{(}\PYG{n}{s}\PYG{p}{)} \PYG{k}{with} \PYG{l+m+mi}{10} \PYG{n}{columns} \PYG{n}{has} \PYG{n}{been} \PYG{n}{written} \PYG{o+ow}{in}
\PYG{o}{/}\PYG{n}{home}\PYG{o}{/}\PYG{n}{micha}\PYG{o}{/}\PYG{n}{LaueToolsPy3}\PYG{o}{/}\PYG{n}{LaueTools}\PYG{o}{/}\PYG{n}{LaueImages}\PYG{o}{/}\PYG{n}{Ge\PYGZus{}blanc\PYGZus{}0000Notebook}\PYG{o}{.}\PYG{n}{dat}
\end{sphinxVerbatim}


\subsubsection{Now indexing}
\label{\detokenize{Basic_PeakSearch_IndexRefine:now-indexing}}

\paragraph{geometry calibration parameters}
\label{\detokenize{Basic_PeakSearch_IndexRefine:geometry-calibration-parameters}}
Either you fill manually the dict of parameters or you read a file .det

\fvset{hllines={, ,}}%
\begin{sphinxVerbatim}[commandchars=\\\{\}]
\PYG{c+c1}{\PYGZsh{} detector geometry and parameters as read from Geblanc0000.det}
\PYG{n}{calibration\PYGZus{}parameters} \PYG{o}{=} \PYG{p}{[}\PYG{l+m+mf}{70.775}\PYG{p}{,} \PYG{l+m+mf}{941.74}\PYG{p}{,} \PYG{l+m+mf}{1082.57}\PYG{p}{,} \PYG{l+m+mf}{0.631}\PYG{p}{,} \PYG{o}{\PYGZhy{}}\PYG{l+m+mf}{0.681}\PYG{p}{]}
\PYG{n}{CCDCalibdict} \PYG{o}{=} \PYG{p}{\PYGZob{}}\PYG{p}{\PYGZcb{}}
\PYG{n}{CCDCalibdict}\PYG{p}{[}\PYG{l+s+s1}{\PYGZsq{}}\PYG{l+s+s1}{CCDCalibParameters}\PYG{l+s+s1}{\PYGZsq{}}\PYG{p}{]} \PYG{o}{=} \PYG{n}{calibration\PYGZus{}parameters}
\PYG{n}{CCDCalibdict}\PYG{p}{[}\PYG{l+s+s1}{\PYGZsq{}}\PYG{l+s+s1}{framedim}\PYG{l+s+s1}{\PYGZsq{}}\PYG{p}{]} \PYG{o}{=} \PYG{p}{(}\PYG{l+m+mi}{2048}\PYG{p}{,} \PYG{l+m+mi}{2048}\PYG{p}{)}
\PYG{n}{CCDCalibdict}\PYG{p}{[}\PYG{l+s+s1}{\PYGZsq{}}\PYG{l+s+s1}{detectordiameter}\PYG{l+s+s1}{\PYGZsq{}}\PYG{p}{]} \PYG{o}{=} \PYG{l+m+mf}{165.}
\PYG{n}{CCDCalibdict}\PYG{p}{[}\PYG{l+s+s1}{\PYGZsq{}}\PYG{l+s+s1}{kf\PYGZus{}direction}\PYG{l+s+s1}{\PYGZsq{}}\PYG{p}{]} \PYG{o}{=} \PYG{l+s+s1}{\PYGZsq{}}\PYG{l+s+s1}{Z\PYGZgt{}0}\PYG{l+s+s1}{\PYGZsq{}}
\PYG{n}{CCDCalibdict}\PYG{p}{[}\PYG{l+s+s1}{\PYGZsq{}}\PYG{l+s+s1}{xpixelsize}\PYG{l+s+s1}{\PYGZsq{}}\PYG{p}{]} \PYG{o}{=} \PYG{l+m+mf}{0.07914}

\PYG{c+c1}{\PYGZsh{} CCDCalibdict can also be simply build by reading the proper .det file}
\PYG{n+nb}{print}\PYG{p}{(}\PYG{l+s+s2}{\PYGZdq{}}\PYG{l+s+s2}{reading geometry calibration file}\PYG{l+s+s2}{\PYGZdq{}}\PYG{p}{)}
\PYG{n}{CCDCalibdict}\PYG{o}{=}\PYG{n}{IOLT}\PYG{o}{.}\PYG{n}{readCalib\PYGZus{}det\PYGZus{}file}\PYG{p}{(}\PYG{n}{os}\PYG{o}{.}\PYG{n}{path}\PYG{o}{.}\PYG{n}{join}\PYG{p}{(}\PYG{n}{imagefolder}\PYG{p}{,}\PYG{l+s+s1}{\PYGZsq{}}\PYG{l+s+s1}{Geblanc0000.det}\PYG{l+s+s1}{\PYGZsq{}}\PYG{p}{)}\PYG{p}{)}
\PYG{n}{CCDCalibdict}\PYG{p}{[}\PYG{l+s+s1}{\PYGZsq{}}\PYG{l+s+s1}{kf\PYGZus{}direction}\PYG{l+s+s1}{\PYGZsq{}}\PYG{p}{]} \PYG{o}{=} \PYG{l+s+s1}{\PYGZsq{}}\PYG{l+s+s1}{Z\PYGZgt{}0}\PYG{l+s+s1}{\PYGZsq{}}
\end{sphinxVerbatim}

\fvset{hllines={, ,}}%
\begin{sphinxVerbatim}[commandchars=\\\{\}]
\PYG{n}{reading} \PYG{n}{geometry} \PYG{n}{calibration} \PYG{n}{file}
\PYG{n}{calib} \PYG{o}{=}  \PYG{p}{[} \PYG{l+m+mf}{7.07760e+01}  \PYG{l+m+mf}{9.41760e+02}  \PYG{l+m+mf}{1.08244e+03}  \PYG{l+m+mf}{6.29000e\PYGZhy{}01} \PYG{o}{\PYGZhy{}}\PYG{l+m+mf}{6.85000e\PYGZhy{}01}
  \PYG{l+m+mf}{7.91400e\PYGZhy{}02}  \PYG{l+m+mf}{2.04800e+03}  \PYG{l+m+mf}{2.04800e+03}\PYG{p}{]}
\PYG{n}{matrix} \PYG{o}{=}  \PYG{p}{[} \PYG{l+m+mf}{0.995829} \PYG{o}{\PYGZhy{}}\PYG{l+m+mf}{0.071471} \PYG{o}{\PYGZhy{}}\PYG{l+m+mf}{0.056709}  \PYG{l+m+mf}{0.012247}  \PYG{l+m+mf}{0.720654} \PYG{o}{\PYGZhy{}}\PYG{l+m+mf}{0.693187}  \PYG{l+m+mf}{0.09041}
  \PYG{l+m+mf}{0.689602}  \PYG{l+m+mf}{0.718523}\PYG{p}{]}
\end{sphinxVerbatim}

\sphinxstylestrong{creation of a .cor file containing accurate scattering angles thanks
to detector geometry parameters}

Only list of spots with scattering angles can be indexed. In LaueTools
.dat file contains only X, Y pixel positions, .cor file contains in
addition 2theta and chi scattering angles, and .fit file in addition
indexed results properties (such as h, k, l, energy, grain index …)

\fvset{hllines={, ,}}%
\begin{sphinxVerbatim}[commandchars=\\\{\}]
\PYG{k+kn}{import} \PYG{n+nn}{LaueTools}\PYG{n+nn}{.}\PYG{n+nn}{LaueGeometry} \PYG{k}{as} \PYG{n+nn}{LTGeo}
\PYG{n}{LTGeo}\PYG{o}{.}\PYG{n}{convert2corfile}\PYG{p}{(}\PYG{n}{datfilename}\PYG{p}{,}
                         \PYG{n}{calibration\PYGZus{}parameters}\PYG{p}{,}
                         \PYG{n}{dirname\PYGZus{}in}\PYG{o}{=}\PYG{n}{imagefolder}\PYG{p}{,}
                        \PYG{n}{dirname\PYGZus{}out}\PYG{o}{=}\PYG{n}{imagefolder}\PYG{p}{,}
                        \PYG{n}{CCDCalibdict}\PYG{o}{=}\PYG{n}{CCDCalibdict}\PYG{p}{)}
\PYG{n}{corfilename} \PYG{o}{=} \PYG{n}{datfilename}\PYG{o}{.}\PYG{n}{split}\PYG{p}{(}\PYG{l+s+s1}{\PYGZsq{}}\PYG{l+s+s1}{.}\PYG{l+s+s1}{\PYGZsq{}}\PYG{p}{)}\PYG{p}{[}\PYG{l+m+mi}{0}\PYG{p}{]} \PYG{o}{+} \PYG{l+s+s1}{\PYGZsq{}}\PYG{l+s+s1}{.cor}\PYG{l+s+s1}{\PYGZsq{}}
\PYG{n}{fullpathcorfile} \PYG{o}{=} \PYG{n}{os}\PYG{o}{.}\PYG{n}{path}\PYG{o}{.}\PYG{n}{join}\PYG{p}{(}\PYG{n}{imagefolder}\PYG{p}{,}\PYG{n}{corfilename}\PYG{p}{)}
\end{sphinxVerbatim}
\begin{sphinxalltt}
Entering CrystalParameters \sphinxstylestrong{**}---\sphinxstylestrong{***********************}


nb of spots and columns in .dat file (82, 3)
file :../LaueImages/Ge\_blanc\_0000Notebook.dat
containing 82 peaks
(2theta chi X Y I) written in ../LaueImages/Ge\_blanc\_0000Notebook.cor
\end{sphinxalltt}


\subparagraph{create instance of an objet spotsset class}
\label{\detokenize{Basic_PeakSearch_IndexRefine:create-instance-of-an-objet-spotsset-class}}
\fvset{hllines={, ,}}%
\begin{sphinxVerbatim}[commandchars=\\\{\}]
\PYG{k+kn}{import} \PYG{n+nn}{LaueTools}\PYG{n+nn}{.}\PYG{n+nn}{indexingSpotsSet} \PYG{k}{as} \PYG{n+nn}{ISS}
\PYG{n}{DataSet} \PYG{o}{=} \PYG{n}{ISS}\PYG{o}{.}\PYG{n}{spotsset}\PYG{p}{(}\PYG{p}{)}

\PYG{n}{DataSet}\PYG{o}{.}\PYG{n}{importdatafromfile}\PYG{p}{(}\PYG{n}{fullpathcorfile}\PYG{p}{)}
\end{sphinxVerbatim}

\fvset{hllines={, ,}}%
\begin{sphinxVerbatim}[commandchars=\\\{\}]
Cython compiled module for fast computation of Laue spots is not installed!
Cython compiled \PYGZsq{}angulardist\PYGZsq{} module for fast computation of angular distance is not installed!
Using default module
Cython compiled module for fast computation of angular distance is not installed!
module Image / PIL is not installed
CCDcalib in readfile\PYGZus{}cor \PYGZob{}\PYGZsq{}dd\PYGZsq{}: 70.776, \PYGZsq{}xcen\PYGZsq{}: 941.76, \PYGZsq{}ycen\PYGZsq{}: 1082.44, \PYGZsq{}xbet\PYGZsq{}: 0.629, \PYGZsq{}xgam\PYGZsq{}: \PYGZhy{}0.685, \PYGZsq{}xpixelsize\PYGZsq{}: 0.07914, \PYGZsq{}ypixelsize\PYGZsq{}: 0.07914, \PYGZsq{}CCDLabel\PYGZsq{}: \PYGZsq{}MARCCD165\PYGZsq{}, \PYGZsq{}framedim\PYGZsq{}: [2048.0, 2048.0], \PYGZsq{}detectordiameter\PYGZsq{}: 162.07872, \PYGZsq{}kf\PYGZus{}direction\PYGZsq{}: \PYGZsq{}Z\PYGZgt{}0\PYGZsq{}, \PYGZsq{}pixelsize\PYGZsq{}: 0.07914\PYGZcb{}
CCD Detector parameters read from .cor file
CCDcalibdict \PYGZob{}\PYGZsq{}dd\PYGZsq{}: 70.776, \PYGZsq{}xcen\PYGZsq{}: 941.76, \PYGZsq{}ycen\PYGZsq{}: 1082.44, \PYGZsq{}xbet\PYGZsq{}: 0.629, \PYGZsq{}xgam\PYGZsq{}: \PYGZhy{}0.685, \PYGZsq{}xpixelsize\PYGZsq{}: 0.07914, \PYGZsq{}ypixelsize\PYGZsq{}: 0.07914, \PYGZsq{}CCDLabel\PYGZsq{}: \PYGZsq{}MARCCD165\PYGZsq{}, \PYGZsq{}framedim\PYGZsq{}: [2048.0, 2048.0], \PYGZsq{}detectordiameter\PYGZsq{}: 162.07872, \PYGZsq{}kf\PYGZus{}direction\PYGZsq{}: \PYGZsq{}Z\PYGZgt{}0\PYGZsq{}, \PYGZsq{}pixelsize\PYGZsq{}: 0.07914\PYGZcb{}
\end{sphinxVerbatim}

\fvset{hllines={, ,}}%
\begin{sphinxVerbatim}[commandchars=\\\{\}]
\PYG{k+kc}{True}
\end{sphinxVerbatim}

\fvset{hllines={, ,}}%
\begin{sphinxVerbatim}[commandchars=\\\{\}]
\PYG{n}{DataSet}\PYG{o}{.}\PYG{n}{getUnIndexedSpotsallData}\PYG{p}{(}\PYG{p}{)}\PYG{p}{[}\PYG{p}{:}\PYG{l+m+mi}{3}\PYG{p}{]}
\end{sphinxVerbatim}

\fvset{hllines={, ,}}%
\begin{sphinxVerbatim}[commandchars=\\\{\}]
\PYG{n}{array}\PYG{p}{(}\PYG{p}{[}\PYG{p}{[} \PYG{l+m+mf}{0.0000000e+00}\PYG{p}{,}  \PYG{l+m+mf}{5.8426915e+01}\PYG{p}{,}  \PYG{l+m+mf}{2.0130035e+01}\PYG{p}{,}  \PYG{l+m+mf}{6.2313000e+02}\PYG{p}{,}
         \PYG{l+m+mf}{1.6577300e+03}\PYG{p}{,}  \PYG{l+m+mf}{2.9799380e+04}\PYG{p}{]}\PYG{p}{,}
       \PYG{p}{[} \PYG{l+m+mf}{1.0000000e+00}\PYG{p}{,}  \PYG{l+m+mf}{5.7634672e+01}\PYG{p}{,} \PYG{o}{\PYGZhy{}}\PYG{l+m+mf}{1.8415523e+01}\PYG{p}{,}  \PYG{l+m+mf}{1.2443300e+03}\PYG{p}{,}
         \PYG{l+m+mf}{1.6621500e+03}\PYG{p}{,}  \PYG{l+m+mf}{2.2425630e+04}\PYG{p}{]}\PYG{p}{,}
       \PYG{p}{[} \PYG{l+m+mf}{2.0000000e+00}\PYG{p}{,}  \PYG{l+m+mf}{8.0919846e+01}\PYG{p}{,}  \PYG{l+m+mf}{6.6158100e\PYGZhy{}01}\PYG{p}{,}  \PYG{l+m+mf}{9.3304000e+02}\PYG{p}{,}
         \PYG{l+m+mf}{1.2154400e+03}\PYG{p}{,}  \PYG{l+m+mf}{2.2197540e+04}\PYG{p}{]}\PYG{p}{]}\PYG{p}{)}
\end{sphinxVerbatim}

\sphinxstylestrong{*Set parameters for indexation: Ge, maximum energy*}

All materials are listed in dict\_LaueTools.py in dict\_Materials. You
can edit/modify the module (then a restart of the kernel is necessary)

\fvset{hllines={, ,}}%
\begin{sphinxVerbatim}[commandchars=\\\{\}]
\PYG{n}{emin}\PYG{o}{=}\PYG{l+m+mi}{5}
\PYG{c+c1}{\PYGZsh{} emax can be lowered for large unit cell indexation (but greater than BM32 highest energy is meaningless)}
\PYG{n}{emax}\PYG{o}{=}\PYG{l+m+mi}{22}
\PYG{c+c1}{\PYGZsh{} key of materials}
\PYG{n}{key\PYGZus{}material}\PYG{o}{=}\PYG{l+s+s1}{\PYGZsq{}}\PYG{l+s+s1}{Ge}\PYG{l+s+s1}{\PYGZsq{}}

\PYG{n}{dict\PYGZus{}indexrefine} \PYG{o}{=} \PYG{p}{\PYGZob{}}\PYG{c+c1}{\PYGZsh{} recognition angle parameters from two sets A and B}
                   \PYG{l+s+s1}{\PYGZsq{}}\PYG{l+s+s1}{AngleTolLUT}\PYG{l+s+s1}{\PYGZsq{}}\PYG{p}{:} \PYG{l+m+mf}{0.5}\PYG{p}{,}
                   \PYG{l+s+s1}{\PYGZsq{}}\PYG{l+s+s1}{nlutmax}\PYG{l+s+s1}{\PYGZsq{}}\PYG{p}{:}\PYG{l+m+mi}{3}\PYG{p}{,}
                   \PYG{l+s+s1}{\PYGZsq{}}\PYG{l+s+s1}{central spots indices}\PYG{l+s+s1}{\PYGZsq{}}\PYG{p}{:} \PYG{p}{[}\PYG{l+m+mi}{0}\PYG{p}{,}\PYG{l+m+mi}{1}\PYG{p}{,}\PYG{l+m+mi}{2}\PYG{p}{,}\PYG{l+m+mi}{3}\PYG{p}{,}\PYG{l+m+mi}{4}\PYG{p}{]}\PYG{p}{,}  \PYG{c+c1}{\PYGZsh{} spots set A}
                   \PYG{l+s+s1}{\PYGZsq{}}\PYG{l+s+s1}{NBMAXPROBED}\PYG{l+s+s1}{\PYGZsq{}}\PYG{p}{:} \PYG{l+m+mi}{10}\PYG{p}{,}  \PYG{c+c1}{\PYGZsh{} spots set B}
                   \PYG{l+s+s1}{\PYGZsq{}}\PYG{l+s+s1}{MATCHINGRATE\PYGZus{}ANGLE\PYGZus{}TOL}\PYG{l+s+s1}{\PYGZsq{}}\PYG{p}{:} \PYG{l+m+mf}{0.2}\PYG{p}{,}
                \PYG{c+c1}{\PYGZsh{} refinement parameters (loop over narrower matching angles)}
                   \PYG{l+s+s1}{\PYGZsq{}}\PYG{l+s+s1}{list matching tol angles}\PYG{l+s+s1}{\PYGZsq{}}\PYG{p}{:}\PYG{p}{[}\PYG{l+m+mf}{0.5}\PYG{p}{,}\PYG{l+m+mf}{0.2}\PYG{p}{,}\PYG{l+m+mf}{0.1}\PYG{p}{]}\PYG{p}{,}

                \PYG{c+c1}{\PYGZsh{} minor parameters}
                \PYG{l+s+s1}{\PYGZsq{}}\PYG{l+s+s1}{MATCHINGRATE\PYGZus{}THRESHOLD\PYGZus{}IAL}\PYG{l+s+s1}{\PYGZsq{}}\PYG{p}{:} \PYG{l+m+mi}{100}\PYG{p}{,}
                   \PYG{l+s+s1}{\PYGZsq{}}\PYG{l+s+s1}{UseIntensityWeights}\PYG{l+s+s1}{\PYGZsq{}}\PYG{p}{:} \PYG{k+kc}{False}\PYG{p}{,}
                   \PYG{l+s+s1}{\PYGZsq{}}\PYG{l+s+s1}{nbSpotsToIndex}\PYG{l+s+s1}{\PYGZsq{}}\PYG{p}{:}\PYG{l+m+mi}{10000}\PYG{p}{,}
                   \PYG{l+s+s1}{\PYGZsq{}}\PYG{l+s+s1}{MinimumNumberMatches}\PYG{l+s+s1}{\PYGZsq{}}\PYG{p}{:} \PYG{l+m+mi}{3}\PYG{p}{,}
                   \PYG{l+s+s1}{\PYGZsq{}}\PYG{l+s+s1}{MinimumMatchingRate}\PYG{l+s+s1}{\PYGZsq{}}\PYG{p}{:}\PYG{l+m+mi}{3}
                   \PYG{p}{\PYGZcb{}}

\PYG{c+c1}{\PYGZsh{}}
\PYG{n}{grainindex}\PYG{o}{=}\PYG{l+m+mi}{0}
\PYG{n}{DataSet} \PYG{o}{=} \PYG{n}{ISS}\PYG{o}{.}\PYG{n}{spotsset}\PYG{p}{(}\PYG{p}{)}

\PYG{n}{DataSet}\PYG{o}{.}\PYG{n}{pixelsize} \PYG{o}{=} \PYG{n}{CCDCalibdict}\PYG{p}{[}\PYG{l+s+s1}{\PYGZsq{}}\PYG{l+s+s1}{xpixelsize}\PYG{l+s+s1}{\PYGZsq{}}\PYG{p}{]}
\PYG{n}{DataSet}\PYG{o}{.}\PYG{n}{dim} \PYG{o}{=} \PYG{n}{CCDCalibdict}\PYG{p}{[}\PYG{l+s+s1}{\PYGZsq{}}\PYG{l+s+s1}{framedim}\PYG{l+s+s1}{\PYGZsq{}}\PYG{p}{]}
\PYG{n}{DataSet}\PYG{o}{.}\PYG{n}{detectordiameter} \PYG{o}{=} \PYG{n}{CCDCalibdict}\PYG{p}{[}\PYG{l+s+s1}{\PYGZsq{}}\PYG{l+s+s1}{detectordiameter}\PYG{l+s+s1}{\PYGZsq{}}\PYG{p}{]}
\PYG{n}{DataSet}\PYG{o}{.}\PYG{n}{kf\PYGZus{}direction} \PYG{o}{=} \PYG{n}{CCDCalibdict}\PYG{p}{[}\PYG{l+s+s1}{\PYGZsq{}}\PYG{l+s+s1}{kf\PYGZus{}direction}\PYG{l+s+s1}{\PYGZsq{}}\PYG{p}{]}
\PYG{n}{DataSet}\PYG{o}{.}\PYG{n}{key\PYGZus{}material} \PYG{o}{=} \PYG{n}{key\PYGZus{}material}
\PYG{n}{DataSet}\PYG{o}{.}\PYG{n}{emin} \PYG{o}{=} \PYG{n}{emin}
\PYG{n}{DataSet}\PYG{o}{.}\PYG{n}{emax} \PYG{o}{=} \PYG{n}{emax}
\end{sphinxVerbatim}

\sphinxstylestrong{Before launching the indexation procedure you may want to check a
solution found elsewhere or sometimes ago. Then fill {}`{}`previousResults{}`{}`
as shown below}

\fvset{hllines={, ,}}%
\begin{sphinxVerbatim}[commandchars=\\\{\}]
\PYG{c+c1}{\PYGZsh{}CheckFirstThisMatrix=np.array([[\PYGZhy{}0.44486058225058 ,  0.098996190230096 ,\PYGZhy{}0.897868909077371],[\PYGZhy{}0.883970521873963,0.1130536332378 , 0.462465547362675],}
\PYG{c+c1}{\PYGZsh{} [ 0.143878606007886, 0.993706753289519 , 0.035064809225047]])}

\PYG{c+c1}{\PYGZsh{} nb of matrices, list of matrices to check, dummy parameter, dummy parameter}
\PYG{c+c1}{\PYGZsh{}previousResults = 1,[CheckFirstThisMatrix],50,50}


\PYG{n}{previousResults} \PYG{o}{=} \PYG{k+kc}{None}
\end{sphinxVerbatim}

\sphinxstylestrong{Then launch indexation by specifying some arguments of the method
{}`{}`IndexSpotsSet{}`{}`:}

\fvset{hllines={, ,}}%
\begin{sphinxVerbatim}[commandchars=\\\{\}]
\PYG{o}{\PYGZhy{}} \PYG{n}{nbGrainstoFind}\PYG{p}{:} \PYG{n}{nb} \PYG{n}{of} \PYG{n}{grains} \PYG{n}{of} \PYG{n}{this} \PYG{n}{material} \PYG{n}{you} \PYG{n}{want} \PYG{n}{to} \PYG{n}{find}
\PYG{o}{\PYGZhy{}} \PYG{n}{set\PYGZus{}central\PYGZus{}spots\PYGZus{}hkl}\PYG{p}{:} \PYG{n}{imposed} \PYG{n}{miller} \PYG{n}{indices} \PYG{p}{[}\PYG{n}{h}\PYG{p}{,}\PYG{n}{k}\PYG{p}{,}\PYG{n}{l}\PYG{p}{]} \PYG{n}{of} \PYG{n}{central} \PYG{n}{spots} \PYG{p}{(}\PYG{n+nb}{set} \PYG{n}{A} \PYG{n}{of} \PYG{n}{spots}\PYG{p}{)}  \PYG{k}{else} \PYG{p}{:} \PYG{k+kc}{None}
\PYG{o}{.}\PYG{o}{.}\PYG{o}{.}
\end{sphinxVerbatim}

\fvset{hllines={, ,}}%
\begin{sphinxVerbatim}[commandchars=\\\{\}]
\PYG{n}{t0} \PYG{o}{=}\PYG{n}{time}\PYG{o}{.}\PYG{n}{time}\PYG{p}{(}\PYG{p}{)}

\PYG{n}{DataSet}\PYG{o}{.}\PYG{n}{IndexSpotsSet}\PYG{p}{(}\PYG{n}{fullpathcorfile}\PYG{p}{,} \PYG{n}{key\PYGZus{}material}\PYG{p}{,} \PYG{n}{emin}\PYG{p}{,} \PYG{n}{emax}\PYG{p}{,} \PYG{n}{dict\PYGZus{}indexrefine}\PYG{p}{,} \PYG{k+kc}{None}\PYG{p}{,}
                         \PYG{n}{use\PYGZus{}file}\PYG{o}{=}\PYG{l+m+mi}{1}\PYG{p}{,} \PYG{c+c1}{\PYGZsh{} read .cor file and reset also spots properties dictionary}
                         \PYG{n}{IMM}\PYG{o}{=}\PYG{k+kc}{False}\PYG{p}{,}\PYG{n}{LUT}\PYG{o}{=}\PYG{k+kc}{None}\PYG{p}{,}\PYG{n}{n\PYGZus{}LUT}\PYG{o}{=}\PYG{n}{dict\PYGZus{}indexrefine}\PYG{p}{[}\PYG{l+s+s1}{\PYGZsq{}}\PYG{l+s+s1}{nlutmax}\PYG{l+s+s1}{\PYGZsq{}}\PYG{p}{]}\PYG{p}{,}\PYG{n}{angletol\PYGZus{}list}\PYG{o}{=}\PYG{n}{dict\PYGZus{}indexrefine}\PYG{p}{[}\PYG{l+s+s1}{\PYGZsq{}}\PYG{l+s+s1}{list matching tol angles}\PYG{l+s+s1}{\PYGZsq{}}\PYG{p}{]}\PYG{p}{,}
                        \PYG{n}{nbGrainstoFind}\PYG{o}{=}\PYG{l+m+mi}{1}\PYG{p}{,}  \PYG{c+c1}{\PYGZsh{} nb of grains of the same material in this case}
                        \PYG{n}{set\PYGZus{}central\PYGZus{}spots\PYGZus{}hkl}\PYG{o}{=}\PYG{p}{[}\PYG{l+m+mi}{0}\PYG{p}{,}\PYG{l+m+mi}{1}\PYG{p}{,}\PYG{l+m+mi}{1}\PYG{p}{]}\PYG{p}{,}  \PYG{c+c1}{\PYGZsh{} set hkl of spots of set A}
                        \PYG{n}{MatchingRate\PYGZus{}List}\PYG{o}{=}\PYG{p}{[}\PYG{l+m+mi}{10}\PYG{p}{,} \PYG{l+m+mi}{10}\PYG{p}{,} \PYG{l+m+mi}{10}\PYG{p}{,}\PYG{l+m+mi}{10}\PYG{p}{,}\PYG{l+m+mi}{10}\PYG{p}{,}\PYG{l+m+mi}{10}\PYG{p}{,}\PYG{l+m+mi}{10}\PYG{p}{,}\PYG{l+m+mi}{10}\PYG{p}{]}\PYG{p}{,}  \PYG{c+c1}{\PYGZsh{} minimum matching rate figure to keep on looping for refinement}
                        \PYG{n}{verbose}\PYG{o}{=}\PYG{l+m+mi}{0}\PYG{p}{,}
                        \PYG{n}{previousResults}\PYG{o}{=}\PYG{n}{previousResults}\PYG{p}{,} \PYG{c+c1}{\PYGZsh{} check before the orientation if not None}
                        \PYG{n}{corfilename}\PYG{o}{=}\PYG{n}{corfilename}\PYG{p}{)}

\PYG{c+c1}{\PYGZsh{} write unindexed spots list in a .cor file}
\PYG{n}{DataSet}\PYG{o}{.}\PYG{n}{writecorFile\PYGZus{}unindexedSpots}\PYG{p}{(}\PYG{n}{corfilename}\PYG{o}{=}\PYG{n}{corfilename}\PYG{p}{,}
                                                \PYG{n}{dirname}\PYG{o}{=}\PYG{n}{imagefolder}\PYG{p}{,}
                                                \PYG{n}{filename\PYGZus{}nbdigits}\PYG{o}{=}\PYG{l+m+mi}{4}\PYG{p}{)}

\PYG{c+c1}{\PYGZsh{} write .fit file of indexed spots belonging to grain \PYGZsh{}0}
\PYG{n}{DataSet}\PYG{o}{.}\PYG{n}{writeFitFile}\PYG{p}{(}\PYG{l+m+mi}{0}\PYG{p}{,}\PYG{n}{corfilename}\PYG{o}{=}\PYG{n}{corfilename}\PYG{p}{,}\PYG{n}{dirname}\PYG{o}{=}\PYG{n}{imagefolder}\PYG{p}{)}

\PYG{n}{tf} \PYG{o}{=} \PYG{n}{time}\PYG{o}{.}\PYG{n}{time}\PYG{p}{(}\PYG{p}{)}\PYG{o}{\PYGZhy{}}\PYG{n}{t0}
\end{sphinxVerbatim}
\begin{sphinxalltt}
CCDcalib in readfile\_cor \{'dd': 70.776, 'xcen': 941.76, 'ycen': 1082.44, 'xbet': 0.629, 'xgam': -0.685, 'xpixelsize': 0.07914, 'ypixelsize': 0.07914, 'CCDLabel': 'MARCCD165', 'framedim': {[}2048.0, 2048.0{]}, 'detectordiameter': 162.07872, 'kf\_direction': 'Z\textgreater{}0', 'pixelsize': 0.07914\}
CCD Detector parameters read from .cor file
CCDcalibdict \{'dd': 70.776, 'xcen': 941.76, 'ycen': 1082.44, 'xbet': 0.629, 'xgam': -0.685, 'xpixelsize': 0.07914, 'ypixelsize': 0.07914, 'CCDLabel': 'MARCCD165', 'framedim': {[}2048.0, 2048.0{]}, 'detectordiameter': 162.07872, 'kf\_direction': 'Z\textgreater{}0', 'pixelsize': 0.07914\}
self.pixelsize in IndexSpotsSet 0.07914
ResolutionAngstromLUT in IndexSpotsSet False

 Remaining nb of spots to index for grain \#0 : 82


 \sphinxstylestrong{**}
start to index grain \#0 of Material: Ge

\sphinxstylestrong{**}

providing new set of matrices Using Angles LUT template matching
nbspots 82
NBMAXPROBED 10
nbspots 82
set\_central\_spots\_hkl {[}0, 1, 1{]}
Central set of exp. spotDistances from spot\_index\_central\_list probed
self.absolute\_index {[} 0  1  2  3  4  5  6  7  8  9 10 11 12 13 14 15 16 17 18 19 20 21 22 23
 24 25 26 27 28 29 30 31 32 33 34 35 36 37 38 39 40 41 42 43 44 45 46 47
 48 49 50 51 52 53 54 55 56 57 58 59 60 61 62 63 64 65 66 67 68 69 70 71
 72 73 74 75 76 77 78 79 80 81{]}
spot\_index\_central\_list {[}0, 1, 2, 3, 4{]}
{[}0 1 2 3 4{]}
LUT is None when entering getOrientMatrices()
set\_central\_spots\_hkl {[}0, 1, 1{]}
set\_central\_spots\_hkl is not None in getOrientMatrices()
set\_central\_spots\_hkl {[}0 1 1{]}
set\_central\_spots\_hkl.shape (3,)
case: 1a
set\_central\_spots\_hkl\_list {[}{[}0 1 1{]}
 {[}0 1 1{]}
 {[}0 1 1{]}
 {[}0 1 1{]}
 {[}0 1 1{]}{]}
cubicSymmetry True
LUT\_tol\_angle 0.5
\sphinxstyleemphasis{---***}------------------------------------------------*
Calculating all possible matrices from exp spot \#0 and the 9 other(s)
hkl in getOrientMatrices {[}0 1 1{]} \textless{}class 'numpy.ndarray'\textgreater{}
using LUTspecific
LUTspecific is None for k\_centspot\_index 0 in getOrientMatrices()
hkl1 in matrices\_from\_onespot\_hkl() {[}0 1 1{]}
Computing hkl2 list for specific or cubic LUT in matrices\_from\_onespot\_hkl()
Calculating LUT in PlanePairs\_from2sets()
Looking up planes pairs in LUT from exp. spots (0, 2):
Looking up planes pairs in LUT from exp. spots (0, 6):
Looking up planes pairs in LUT from exp. spots (0, 9):
calculating matching rates of solutions for exp. spots {[}0, 2{]}
calculating matching rates of solutions for exp. spots {[}0, 6{]}
calculating matching rates of solutions for exp. spots {[}0, 9{]}


\sphinxstyleemphasis{---***}------------------------------------------------*
Calculating all possible matrices from exp spot \#1 and the 9 other(s)
hkl in getOrientMatrices {[}0 1 1{]} \textless{}class 'numpy.ndarray'\textgreater{}
using LUTspecific
LUTspecific is not None for k\_centspot\_index 1 in getOrientMatrices()
hkl1 in matrices\_from\_onespot\_hkl() {[}0 1 1{]}
Using specific LUT in matrices\_from\_onespot\_hkl()
Looking up planes pairs in LUT from exp. spots (1, 2):
Looking up planes pairs in LUT from exp. spots (1, 7):
Looking up planes pairs in LUT from exp. spots (1, 9):
calculating matching rates of solutions for exp. spots {[}1, 2{]}
calculating matching rates of solutions for exp. spots {[}1, 7{]}
calculating matching rates of solutions for exp. spots {[}1, 9{]}


\sphinxstyleemphasis{---***}------------------------------------------------*
Calculating all possible matrices from exp spot \#2 and the 9 other(s)
hkl in getOrientMatrices {[}0 1 1{]} \textless{}class 'numpy.ndarray'\textgreater{}
using LUTspecific
LUTspecific is not None for k\_centspot\_index 2 in getOrientMatrices()
hkl1 in matrices\_from\_onespot\_hkl() {[}0 1 1{]}
Using specific LUT in matrices\_from\_onespot\_hkl()
Looking up planes pairs in LUT from exp. spots (2, 0):
Looking up planes pairs in LUT from exp. spots (2, 1):
Looking up planes pairs in LUT from exp. spots (2, 9):
calculating matching rates of solutions for exp. spots {[}2, 0{]}
calculating matching rates of solutions for exp. spots {[}2, 1{]}
calculating matching rates of solutions for exp. spots {[}2, 9{]}


\sphinxstyleemphasis{---***}------------------------------------------------*
Calculating all possible matrices from exp spot \#3 and the 9 other(s)
hkl in getOrientMatrices {[}0 1 1{]} \textless{}class 'numpy.ndarray'\textgreater{}
using LUTspecific
LUTspecific is not None for k\_centspot\_index 3 in getOrientMatrices()
hkl1 in matrices\_from\_onespot\_hkl() {[}0 1 1{]}
Using specific LUT in matrices\_from\_onespot\_hkl()
Looking up planes pairs in LUT from exp. spots (3, 7):
Looking up planes pairs in LUT from exp. spots (3, 8):
Looking up planes pairs in LUT from exp. spots (3, 9):
calculating matching rates of solutions for exp. spots {[}3, 7{]}
calculating matching rates of solutions for exp. spots {[}3, 8{]}
calculating matching rates of solutions for exp. spots {[}3, 9{]}


\sphinxstyleemphasis{---***}------------------------------------------------*
Calculating all possible matrices from exp spot \#4 and the 9 other(s)
hkl in getOrientMatrices {[}0 1 1{]} \textless{}class 'numpy.ndarray'\textgreater{}
using LUTspecific
LUTspecific is not None for k\_centspot\_index 4 in getOrientMatrices()
hkl1 in matrices\_from\_onespot\_hkl() {[}0 1 1{]}
Using specific LUT in matrices\_from\_onespot\_hkl()
Looking up planes pairs in LUT from exp. spots (4, 6):
Looking up planes pairs in LUT from exp. spots (4, 8):
Looking up planes pairs in LUT from exp. spots (4, 9):
calculating matching rates of solutions for exp. spots {[}4, 6{]}
calculating matching rates of solutions for exp. spots {[}4, 8{]}
calculating matching rates of solutions for exp. spots {[}4, 9{]}



-----------------------------------------
results:
matrix:                                         matching results
{[} 0.071442298339536 -0.056791122889477 -0.995826674863109{]}        res: {[}94.0, 135.0{]} 0.005 69.63
{[}-0.720597705663885 -0.693247631822884 -0.012110638459969{]}        spot indices {[}0 9{]}
{[}-0.689608632382347  0.718518569419943 -0.090393581312322{]}        planes {[}{[}-2.0, 1.0, 1.0{]}, {[}0.0, 1.0, 1.0{]}{]}

{[} 0.071284911945937 -0.056791915954001 -0.995837908301915{]}        res: {[}93.0, 134.0{]} 0.006 69.40
{[}-0.720581964957976 -0.693265307713712 -0.012035152592104{]}        spot indices {[}1 9{]}
{[}-0.68968586701208   0.718453369664079 -0.09032253573791 {]}        planes {[}{[}-1.0, 2.0, 1.0{]}, {[}0.0, 1.0, 1.0{]}{]}

{[}-0.071331310042019 -0.995834915200135 -0.056786141584281{]}        res: {[}93.0, 134.0{]} 0.005 69.40
{[} 0.720561755804718 -0.012058289359476 -0.693285910522741{]}        spot indices {[}2 9{]}
{[} 0.689652960030445 -0.090345399841628  0.718482082900264{]}        planes {[}{[}1.0, 1.0, 1.0{]}, {[}0.0, 1.0, 1.0{]}{]}

{[}-0.071391558888759 -0.995830258190095 -0.056792096214888{]}        res: {[}94.0, 135.0{]} 0.006 69.63
{[} 0.720671448854468 -0.012140732995851 -0.693170444701969{]}        spot indices {[}3 9{]}
{[} 0.689588625514858 -0.090412962995124  0.718535332243983{]}        planes {[}{[}2.0, 3.0, 1.0{]}, {[}0.0, 1.0, 1.0{]}{]}

{[} 0.071340651481789 -0.056823338834159 -0.995832124210648{]}        res: {[}94.0, 135.0{]} 0.005 69.63
{[}-0.720605825766681 -0.693235936569283 -0.012295532523216{]}        spot indices {[}4 9{]}
{[}-0.689563524109094  0.718557247109645 -0.090430242974657{]}        planes {[}{[}-1.0, 2.0, 3.0{]}, {[}0.0, 1.0, 1.0{]}{]}

Number of matrices found (nb\_sol):  5
set\_central\_spots\_hkl in FindOrientMatrices {[}0, 1, 1{]}
Merging matrices
keep\_only\_equivalent = False
sorting according to rank
rank {[}0 4 3 2 1{]}

-----------------------------------------
results:
matrix:                                         matching results
{[} 0.071442298339536 -0.056791122889477 -0.995826674863109{]}        res: {[} 94. 135.{]} 0.005 69.63
{[}-0.720597705663885 -0.693247631822884 -0.012110638459969{]}
{[}-0.689608632382347  0.718518569419943 -0.090393581312322{]}

{[} 0.071340651481789 -0.056823338834159 -0.995832124210648{]}        res: {[} 94. 135.{]} 0.005 69.63
{[}-0.720605825766681 -0.693235936569283 -0.012295532523216{]}
{[}-0.689563524109094  0.718557247109645 -0.090430242974657{]}

{[}-0.071391558888759 -0.995830258190095 -0.056792096214888{]}        res: {[} 94. 135.{]} 0.006 69.63
{[} 0.720671448854468 -0.012140732995851 -0.693170444701969{]}
{[} 0.689588625514858 -0.090412962995124  0.718535332243983{]}

{[}-0.071331310042019 -0.995834915200135 -0.056786141584281{]}        res: {[} 93. 134.{]} 0.005 69.40
{[} 0.720561755804718 -0.012058289359476 -0.693285910522741{]}
{[} 0.689652960030445 -0.090345399841628  0.718482082900264{]}

{[} 0.071284911945937 -0.056791915954001 -0.995837908301915{]}        res: {[} 93. 134.{]} 0.006 69.40
{[}-0.720581964957976 -0.693265307713712 -0.012035152592104{]}
{[}-0.68968586701208   0.718453369664079 -0.09032253573791 {]}

Nb of potential orientation matrice(s) UB found: 5
{[}{[}{[} 0.071442298339536 -0.056791122889477 -0.995826674863109{]}
  {[}-0.720597705663885 -0.693247631822884 -0.012110638459969{]}
  {[}-0.689608632382347  0.718518569419943 -0.090393581312322{]}{]}

 {[}{[} 0.071340651481789 -0.056823338834159 -0.995832124210648{]}
  {[}-0.720605825766681 -0.693235936569283 -0.012295532523216{]}
  {[}-0.689563524109094  0.718557247109645 -0.090430242974657{]}{]}

 {[}{[}-0.071391558888759 -0.995830258190095 -0.056792096214888{]}
  {[} 0.720671448854468 -0.012140732995851 -0.693170444701969{]}
  {[} 0.689588625514858 -0.090412962995124  0.718535332243983{]}{]}

 {[}{[}-0.071331310042019 -0.995834915200135 -0.056786141584281{]}
  {[} 0.720561755804718 -0.012058289359476 -0.693285910522741{]}
  {[} 0.689652960030445 -0.090345399841628  0.718482082900264{]}{]}

 {[}{[} 0.071284911945937 -0.056791915954001 -0.995837908301915{]}
  {[}-0.720581964957976 -0.693265307713712 -0.012035152592104{]}
  {[}-0.68968586701208   0.718453369664079 -0.09032253573791 {]}{]}{]}
Nb of potential UBs  5

Working with a new stack of orientation matrices
MATCHINGRATE\_THRESHOLD\_IAL= 100.0
has not been reached! All potential solutions have been calculated
taking the first one only.
bestUB object \textless{}LaueTools.indexingSpotsSet.OrientMatrix object at 0x7f7c954f5e80\textgreater{}


---------------refining grain orientation and strain \#0-----------------


 refining grain \#0 step -----0

bestUB \textless{}LaueTools.indexingSpotsSet.OrientMatrix object at 0x7f7c954f5e80\textgreater{}
True it is an OrientMatrix object
Orientation \textless{}LaueTools.indexingSpotsSet.OrientMatrix object at 0x7f7c954f5e80\textgreater{}
matrix {[}{[}-0.071391558888759 -0.995830258190095 -0.056792096214888{]}
 {[} 0.720671448854468 -0.012140732995851 -0.693170444701969{]}
 {[} 0.689588625514858 -0.090412962995124  0.718535332243983{]}{]}
\sphinxstylestrong{*nb of selected spots in AssignHKL***} 82
UBOrientMatrix {[}{[}-0.071391558888759 -0.995830258190095 -0.056792096214888{]}
 {[} 0.720671448854468 -0.012140732995851 -0.693170444701969{]}
 {[} 0.689588625514858 -0.090412962995124  0.718535332243983{]}{]}
For angular tolerance 0.50 deg
Nb of pairs found / nb total of expected spots: 81/147
Matching Rate : 55.10
Nb missing reflections: 66

grain \#0 : 81 links to simulated spots have been found
\sphinxstylestrong{*********mean pixel deviation    0.2522039400422887     ******}
Initial residues {[}0.191356096913678 0.158479888122211 0.125984922997516 0.007464331088751
 0.220920883122328 0.065197410024489 0.405464259132319 0.079769518806149
 0.309835172580193 0.024815180122634 0.146635771529913 0.197926454567734
 0.253815574123594 0.241517985294699 0.301875442673439 0.217498144625921
 0.186026257638361 0.152430964466482 0.022468745875909 0.372498387808433
 0.225884815274198 0.155682523936061 0.308213213363587 0.354423361607117
 0.237793437184287 0.344146246502948 0.117700835663451 0.22732103372742
 0.263538267437741 0.133994037769124 0.091015982918167 0.367309714380722
 0.359174426753832 0.281533512444384 0.191021625928391 0.219461033259323
 0.371983339466526 0.3512796731268   0.298580209240117 0.447936020775024
 0.160438308161376 0.631208433750478 0.420060120050684 0.195238104695171
 0.051118832992816 0.159003375870547 0.354123955360538 0.049380652521924
 0.301744705672337 0.127112320459672 0.082920786835417 0.19281838475986
 0.130182524243209 0.332360152782496 0.533160923855596 0.276782907418236
 0.125265672564509 0.184320173657227 0.238789408490181 0.149666955002009
 0.473603697641706 0.235878500572685 0.425250385266374 0.445829965130009
 0.255853120078437 0.271987274130697 0.298711159306184 0.310382741777609
 0.228936459666657 0.374245425300233 0.100039587285176 0.096572087547537
 0.196098380129156 0.140612883827292 0.338936919946618 0.526244208385607
 0.190627233723994 0.629487817359844 0.233333060530866 0.316852495355672
 0.61336433904477 {]}
---------------------------------------------------



\sphinxstylestrong{***********************}
first error with initial values of: {[}'b/a', 'c/a', 'a12', 'a13', 'a23', 'theta1', 'theta2', 'theta3'{]}

\sphinxstylestrong{***********************}

\sphinxstylestrong{*********mean pixel deviation    0.2522039400422887     ******}


\sphinxstylestrong{***********************}
Fitting parameters:   {[}'b/a', 'c/a', 'a12', 'a13', 'a23', 'theta1', 'theta2', 'theta3'{]}

\sphinxstylestrong{***********************}

With initial values {[}1. 1. 0. 0. 0. 0. 0. 0.{]}
code results 1
nb iterations 189
mesg Both actual and predicted relative reductions in the sum of squares
  are at most 0.000000
strain\_sol {[} 9.999862096544356e-01  9.999941276097474e-01 -7.144700897145182e-06
  5.728542645493929e-05  1.242229382006964e-05  2.977984275077042e-05
 -1.953472667313407e-03  6.919002586889341e-03{]}


 \sphinxstylestrong{**********}  End of Fitting  -  Final errors  \sphinxstylestrong{**************}


\sphinxstylestrong{*********mean pixel deviation    0.18215488925149526     ******}
devstrain, lattice\_parameter\_direct\_strain {[}{[}-4.815998025634964e-06  5.250997004746657e-06 -2.870284279887629e-05{]}
 {[} 5.250997004746657e-06  7.352907498721824e-06 -9.199263307200638e-06{]}
 {[}-2.870284279887629e-05 -9.199263307200638e-06 -2.536909473086861e-06{]}{]} {[} 5.657509728355469  5.657578574250457  5.657522951317707
 90.0010541881893   90.00328909016345  89.99939828842365 {]}
For comparison: a,b,c are rescaled with respect to the reference value of a = 5.657500 Angstroms
lattice\_parameter\_direct\_strain {[} 5.6575             5.657568845776604  5.6575132229395
 90.0010541881893   90.00328909016345  89.99939828842365 {]}
devstrain1, lattice\_parameter\_direct\_strain1 {[}{[}-4.815998025634964e-06  5.250997004746657e-06 -2.870284279887629e-05{]}
 {[} 5.250997004746657e-06  7.352907498721824e-06 -9.199263307200638e-06{]}
 {[}-2.870284279887629e-05 -9.199263307200638e-06 -2.536909473086861e-06{]}{]} {[} 5.6575             5.657568845776604  5.6575132229395
 90.0010541881893   90.00328909016345  89.99939828842365 {]}
new UBs matrix in q= UBs G (s for strain)
strain\_direct {[}{[} 1.719550237533340e-06  5.250997004746657e-06 -2.870284279887629e-05{]}
 {[} 5.250997004746657e-06  1.388845576189013e-05 -9.199263307200638e-06{]}
 {[}-2.870284279887629e-05 -9.199263307200638e-06  3.998638790081444e-06{]}{]}
deviatoric strain {[}{[}-4.815998025634964e-06  5.250997004746657e-06 -2.870284279887629e-05{]}
 {[} 5.250997004746657e-06  7.352907498721824e-06 -9.199263307200638e-06{]}
 {[}-2.870284279887629e-05 -9.199263307200638e-06 -2.536909473086861e-06{]}{]}
new UBs matrix in q= UBs G (s for strain)
strain\_direct {[}{[} 1.719550237533340e-06  5.250997004746657e-06 -2.870284279887629e-05{]}
 {[} 5.250997004746657e-06  1.388845576189013e-05 -9.199263307200638e-06{]}
 {[}-2.870284279887629e-05 -9.199263307200638e-06  3.998638790081444e-06{]}{]}
deviatoric strain {[}{[}-4.815998025634964e-06  5.250997004746657e-06 -2.870284279887629e-05{]}
 {[} 5.250997004746657e-06  7.352907498721824e-06 -9.199263307200638e-06{]}
 {[}-2.870284279887629e-05 -9.199263307200638e-06 -2.536909473086861e-06{]}{]}
For comparison: a,b,c are rescaled with respect to the reference value of a = 5.657500 Angstroms
lattice\_parameter\_direct\_strain {[} 5.6575             5.657568845776604  5.6575132229395
 90.0010541881893   90.00328909016345  89.99939828842365 {]}
final lattice\_parameters {[} 5.6575             5.657568845776604  5.6575132229395
 90.0010541881893   90.00328909016345  89.99939828842365 {]}
UB and strain refinement completed
True it is an OrientMatrix object
Orientation \textless{}LaueTools.indexingSpotsSet.OrientMatrix object at 0x7f7c851edeb8\textgreater{}
matrix {[}{[}-0.071478224945242 -0.995814588672313 -0.056724144676328{]}
 {[} 0.720639310822985 -0.012262885888099 -0.693156594892675{]}
 {[} 0.689613233174427 -0.090416539541802  0.718545890295648{]}{]}
\sphinxstylestrong{*nb of selected spots in AssignHKL***} 82
UBOrientMatrix {[}{[}-0.071478224945242 -0.995814588672313 -0.056724144676328{]}
 {[} 0.720639310822985 -0.012262885888099 -0.693156594892675{]}
 {[} 0.689613233174427 -0.090416539541802  0.718545890295648{]}{]}
For angular tolerance 0.50 deg
Nb of pairs found / nb total of expected spots: 81/147
Matching Rate : 55.10
Nb missing reflections: 66

grain \#0 : 81 links to simulated spots have been found
GoodRefinement condition is  True
nb\_updates 81 compared to 6


 refining grain \#0 step -----1

bestUB \textless{}LaueTools.indexingSpotsSet.OrientMatrix object at 0x7f7c954f5e80\textgreater{}
True it is an OrientMatrix object
Orientation \textless{}LaueTools.indexingSpotsSet.OrientMatrix object at 0x7f7c954f5e80\textgreater{}
matrix {[}{[}-0.071391558888759 -0.995830258190095 -0.056792096214888{]}
 {[} 0.720671448854468 -0.012140732995851 -0.693170444701969{]}
 {[} 0.689588625514858 -0.090412962995124  0.718535332243983{]}{]}
\sphinxstylestrong{*nb of selected spots in AssignHKL***} 82
UBOrientMatrix {[}{[}-0.071391558888759 -0.995830258190095 -0.056792096214888{]}
 {[} 0.720671448854468 -0.012140732995851 -0.693170444701969{]}
 {[} 0.689588625514858 -0.090412962995124  0.718535332243983{]}{]}
For angular tolerance 0.20 deg
Nb of pairs found / nb total of expected spots: 81/147
Matching Rate : 55.10
Nb missing reflections: 66

grain \#0 : 81 links to simulated spots have been found
\sphinxstylestrong{*********mean pixel deviation    0.2522039400422887     ******}
Initial residues {[}0.191356096913678 0.158479888122211 0.125984922997516 0.007464331088751
 0.220920883122328 0.065197410024489 0.405464259132319 0.079769518806149
 0.309835172580193 0.024815180122634 0.146635771529913 0.197926454567734
 0.253815574123594 0.241517985294699 0.301875442673439 0.217498144625921
 0.186026257638361 0.152430964466482 0.022468745875909 0.372498387808433
 0.225884815274198 0.155682523936061 0.308213213363587 0.354423361607117
 0.237793437184287 0.344146246502948 0.117700835663451 0.22732103372742
 0.263538267437741 0.133994037769124 0.091015982918167 0.367309714380722
 0.359174426753832 0.281533512444384 0.191021625928391 0.219461033259323
 0.371983339466526 0.3512796731268   0.298580209240117 0.447936020775024
 0.160438308161376 0.631208433750478 0.420060120050684 0.195238104695171
 0.051118832992816 0.159003375870547 0.354123955360538 0.049380652521924
 0.301744705672337 0.127112320459672 0.082920786835417 0.19281838475986
 0.130182524243209 0.332360152782496 0.533160923855596 0.276782907418236
 0.125265672564509 0.184320173657227 0.238789408490181 0.149666955002009
 0.473603697641706 0.235878500572685 0.425250385266374 0.445829965130009
 0.255853120078437 0.271987274130697 0.298711159306184 0.310382741777609
 0.228936459666657 0.374245425300233 0.100039587285176 0.096572087547537
 0.196098380129156 0.140612883827292 0.338936919946618 0.526244208385607
 0.190627233723994 0.629487817359844 0.233333060530866 0.316852495355672
 0.61336433904477 {]}
---------------------------------------------------



\sphinxstylestrong{***********************}
first error with initial values of: {[}'b/a', 'c/a', 'a12', 'a13', 'a23', 'theta1', 'theta2', 'theta3'{]}

\sphinxstylestrong{***********************}

\sphinxstylestrong{*********mean pixel deviation    0.2522039400422887     ******}


\sphinxstylestrong{***********************}
Fitting parameters:   {[}'b/a', 'c/a', 'a12', 'a13', 'a23', 'theta1', 'theta2', 'theta3'{]}

\sphinxstylestrong{***********************}

With initial values {[}1. 1. 0. 0. 0. 0. 0. 0.{]}
code results 1
nb iterations 189
mesg Both actual and predicted relative reductions in the sum of squares
  are at most 0.000000
strain\_sol {[} 9.999862096544356e-01  9.999941276097474e-01 -7.144700897145182e-06
  5.728542645493929e-05  1.242229382006964e-05  2.977984275077042e-05
 -1.953472667313407e-03  6.919002586889341e-03{]}


 \sphinxstylestrong{**********}  End of Fitting  -  Final errors  \sphinxstylestrong{**************}


\sphinxstylestrong{*********mean pixel deviation    0.18215488925149526     ******}
devstrain, lattice\_parameter\_direct\_strain {[}{[}-4.815998025634964e-06  5.250997004746657e-06 -2.870284279887629e-05{]}
 {[} 5.250997004746657e-06  7.352907498721824e-06 -9.199263307200638e-06{]}
 {[}-2.870284279887629e-05 -9.199263307200638e-06 -2.536909473086861e-06{]}{]} {[} 5.657509728355469  5.657578574250457  5.657522951317707
 90.0010541881893   90.00328909016345  89.99939828842365 {]}
For comparison: a,b,c are rescaled with respect to the reference value of a = 5.657500 Angstroms
lattice\_parameter\_direct\_strain {[} 5.6575             5.657568845776604  5.6575132229395
 90.0010541881893   90.00328909016345  89.99939828842365 {]}
devstrain1, lattice\_parameter\_direct\_strain1 {[}{[}-4.815998025634964e-06  5.250997004746657e-06 -2.870284279887629e-05{]}
 {[} 5.250997004746657e-06  7.352907498721824e-06 -9.199263307200638e-06{]}
 {[}-2.870284279887629e-05 -9.199263307200638e-06 -2.536909473086861e-06{]}{]} {[} 5.6575             5.657568845776604  5.6575132229395
 90.0010541881893   90.00328909016345  89.99939828842365 {]}
new UBs matrix in q= UBs G (s for strain)
strain\_direct {[}{[} 1.719550237533340e-06  5.250997004746657e-06 -2.870284279887629e-05{]}
 {[} 5.250997004746657e-06  1.388845576189013e-05 -9.199263307200638e-06{]}
 {[}-2.870284279887629e-05 -9.199263307200638e-06  3.998638790081444e-06{]}{]}
deviatoric strain {[}{[}-4.815998025634964e-06  5.250997004746657e-06 -2.870284279887629e-05{]}
 {[} 5.250997004746657e-06  7.352907498721824e-06 -9.199263307200638e-06{]}
 {[}-2.870284279887629e-05 -9.199263307200638e-06 -2.536909473086861e-06{]}{]}
new UBs matrix in q= UBs G (s for strain)
strain\_direct {[}{[} 1.719550237533340e-06  5.250997004746657e-06 -2.870284279887629e-05{]}
 {[} 5.250997004746657e-06  1.388845576189013e-05 -9.199263307200638e-06{]}
 {[}-2.870284279887629e-05 -9.199263307200638e-06  3.998638790081444e-06{]}{]}
deviatoric strain {[}{[}-4.815998025634964e-06  5.250997004746657e-06 -2.870284279887629e-05{]}
 {[} 5.250997004746657e-06  7.352907498721824e-06 -9.199263307200638e-06{]}
 {[}-2.870284279887629e-05 -9.199263307200638e-06 -2.536909473086861e-06{]}{]}
For comparison: a,b,c are rescaled with respect to the reference value of a = 5.657500 Angstroms
lattice\_parameter\_direct\_strain {[} 5.6575             5.657568845776604  5.6575132229395
 90.0010541881893   90.00328909016345  89.99939828842365 {]}
final lattice\_parameters {[} 5.6575             5.657568845776604  5.6575132229395
 90.0010541881893   90.00328909016345  89.99939828842365 {]}
UB and strain refinement completed
True it is an OrientMatrix object
Orientation \textless{}LaueTools.indexingSpotsSet.OrientMatrix object at 0x7f7c84338898\textgreater{}
matrix {[}{[}-0.071478224945242 -0.995814588672313 -0.056724144676328{]}
 {[} 0.720639310822985 -0.012262885888099 -0.693156594892675{]}
 {[} 0.689613233174427 -0.090416539541802  0.718545890295648{]}{]}
\sphinxstylestrong{*nb of selected spots in AssignHKL***} 82
UBOrientMatrix {[}{[}-0.071478224945242 -0.995814588672313 -0.056724144676328{]}
 {[} 0.720639310822985 -0.012262885888099 -0.693156594892675{]}
 {[} 0.689613233174427 -0.090416539541802  0.718545890295648{]}{]}
For angular tolerance 0.20 deg
Nb of pairs found / nb total of expected spots: 81/147
Matching Rate : 55.10
Nb missing reflections: 66

grain \#0 : 81 links to simulated spots have been found
GoodRefinement condition is  True
nb\_updates 81 compared to 6


 refining grain \#0 step -----2

bestUB \textless{}LaueTools.indexingSpotsSet.OrientMatrix object at 0x7f7c954f5e80\textgreater{}
True it is an OrientMatrix object
Orientation \textless{}LaueTools.indexingSpotsSet.OrientMatrix object at 0x7f7c954f5e80\textgreater{}
matrix {[}{[}-0.071391558888759 -0.995830258190095 -0.056792096214888{]}
 {[} 0.720671448854468 -0.012140732995851 -0.693170444701969{]}
 {[} 0.689588625514858 -0.090412962995124  0.718535332243983{]}{]}
\sphinxstylestrong{*nb of selected spots in AssignHKL***} 82
UBOrientMatrix {[}{[}-0.071391558888759 -0.995830258190095 -0.056792096214888{]}
 {[} 0.720671448854468 -0.012140732995851 -0.693170444701969{]}
 {[} 0.689588625514858 -0.090412962995124  0.718535332243983{]}{]}
For angular tolerance 0.10 deg
Nb of pairs found / nb total of expected spots: 81/147
Matching Rate : 55.10
Nb missing reflections: 66

grain \#0 : 81 links to simulated spots have been found
\sphinxstylestrong{*********mean pixel deviation    0.2522039400422887     ******}
Initial residues {[}0.191356096913678 0.158479888122211 0.125984922997516 0.007464331088751
 0.220920883122328 0.065197410024489 0.405464259132319 0.079769518806149
 0.309835172580193 0.024815180122634 0.146635771529913 0.197926454567734
 0.253815574123594 0.241517985294699 0.301875442673439 0.217498144625921
 0.186026257638361 0.152430964466482 0.022468745875909 0.372498387808433
 0.225884815274198 0.155682523936061 0.308213213363587 0.354423361607117
 0.237793437184287 0.344146246502948 0.117700835663451 0.22732103372742
 0.263538267437741 0.133994037769124 0.091015982918167 0.367309714380722
 0.359174426753832 0.281533512444384 0.191021625928391 0.219461033259323
 0.371983339466526 0.3512796731268   0.298580209240117 0.447936020775024
 0.160438308161376 0.631208433750478 0.420060120050684 0.195238104695171
 0.051118832992816 0.159003375870547 0.354123955360538 0.049380652521924
 0.301744705672337 0.127112320459672 0.082920786835417 0.19281838475986
 0.130182524243209 0.332360152782496 0.533160923855596 0.276782907418236
 0.125265672564509 0.184320173657227 0.238789408490181 0.149666955002009
 0.473603697641706 0.235878500572685 0.425250385266374 0.445829965130009
 0.255853120078437 0.271987274130697 0.298711159306184 0.310382741777609
 0.228936459666657 0.374245425300233 0.100039587285176 0.096572087547537
 0.196098380129156 0.140612883827292 0.338936919946618 0.526244208385607
 0.190627233723994 0.629487817359844 0.233333060530866 0.316852495355672
 0.61336433904477 {]}
---------------------------------------------------



\sphinxstylestrong{***********************}
first error with initial values of: {[}'b/a', 'c/a', 'a12', 'a13', 'a23', 'theta1', 'theta2', 'theta3'{]}

\sphinxstylestrong{***********************}

\sphinxstylestrong{*********mean pixel deviation    0.2522039400422887     ******}


\sphinxstylestrong{***********************}
Fitting parameters:   {[}'b/a', 'c/a', 'a12', 'a13', 'a23', 'theta1', 'theta2', 'theta3'{]}

\sphinxstylestrong{***********************}

With initial values {[}1. 1. 0. 0. 0. 0. 0. 0.{]}
code results 1
nb iterations 189
mesg Both actual and predicted relative reductions in the sum of squares
  are at most 0.000000
strain\_sol {[} 9.999862096544356e-01  9.999941276097474e-01 -7.144700897145182e-06
  5.728542645493929e-05  1.242229382006964e-05  2.977984275077042e-05
 -1.953472667313407e-03  6.919002586889341e-03{]}


 \sphinxstylestrong{**********}  End of Fitting  -  Final errors  \sphinxstylestrong{**************}


\sphinxstylestrong{*********mean pixel deviation    0.18215488925149526     ******}
devstrain, lattice\_parameter\_direct\_strain {[}{[}-4.815998025634964e-06  5.250997004746657e-06 -2.870284279887629e-05{]}
 {[} 5.250997004746657e-06  7.352907498721824e-06 -9.199263307200638e-06{]}
 {[}-2.870284279887629e-05 -9.199263307200638e-06 -2.536909473086861e-06{]}{]} {[} 5.657509728355469  5.657578574250457  5.657522951317707
 90.0010541881893   90.00328909016345  89.99939828842365 {]}
For comparison: a,b,c are rescaled with respect to the reference value of a = 5.657500 Angstroms
lattice\_parameter\_direct\_strain {[} 5.6575             5.657568845776604  5.6575132229395
 90.0010541881893   90.00328909016345  89.99939828842365 {]}
devstrain1, lattice\_parameter\_direct\_strain1 {[}{[}-4.815998025634964e-06  5.250997004746657e-06 -2.870284279887629e-05{]}
 {[} 5.250997004746657e-06  7.352907498721824e-06 -9.199263307200638e-06{]}
 {[}-2.870284279887629e-05 -9.199263307200638e-06 -2.536909473086861e-06{]}{]} {[} 5.6575             5.657568845776604  5.6575132229395
 90.0010541881893   90.00328909016345  89.99939828842365 {]}
new UBs matrix in q= UBs G (s for strain)
strain\_direct {[}{[} 1.719550237533340e-06  5.250997004746657e-06 -2.870284279887629e-05{]}
 {[} 5.250997004746657e-06  1.388845576189013e-05 -9.199263307200638e-06{]}
 {[}-2.870284279887629e-05 -9.199263307200638e-06  3.998638790081444e-06{]}{]}
deviatoric strain {[}{[}-4.815998025634964e-06  5.250997004746657e-06 -2.870284279887629e-05{]}
 {[} 5.250997004746657e-06  7.352907498721824e-06 -9.199263307200638e-06{]}
 {[}-2.870284279887629e-05 -9.199263307200638e-06 -2.536909473086861e-06{]}{]}
new UBs matrix in q= UBs G (s for strain)
strain\_direct {[}{[} 1.719550237533340e-06  5.250997004746657e-06 -2.870284279887629e-05{]}
 {[} 5.250997004746657e-06  1.388845576189013e-05 -9.199263307200638e-06{]}
 {[}-2.870284279887629e-05 -9.199263307200638e-06  3.998638790081444e-06{]}{]}
deviatoric strain {[}{[}-4.815998025634964e-06  5.250997004746657e-06 -2.870284279887629e-05{]}
 {[} 5.250997004746657e-06  7.352907498721824e-06 -9.199263307200638e-06{]}
 {[}-2.870284279887629e-05 -9.199263307200638e-06 -2.536909473086861e-06{]}{]}
For comparison: a,b,c are rescaled with respect to the reference value of a = 5.657500 Angstroms
lattice\_parameter\_direct\_strain {[} 5.6575             5.657568845776604  5.6575132229395
 90.0010541881893   90.00328909016345  89.99939828842365 {]}
final lattice\_parameters {[} 5.6575             5.657568845776604  5.6575132229395
 90.0010541881893   90.00328909016345  89.99939828842365 {]}
UB and strain refinement completed
True it is an OrientMatrix object
Orientation \textless{}LaueTools.indexingSpotsSet.OrientMatrix object at 0x7f7c94485da0\textgreater{}
matrix {[}{[}-0.071478224945242 -0.995814588672313 -0.056724144676328{]}
 {[} 0.720639310822985 -0.012262885888099 -0.693156594892675{]}
 {[} 0.689613233174427 -0.090416539541802  0.718545890295648{]}{]}
\sphinxstylestrong{*nb of selected spots in AssignHKL***} 81
UBOrientMatrix {[}{[}-0.071478224945242 -0.995814588672313 -0.056724144676328{]}
 {[} 0.720639310822985 -0.012262885888099 -0.693156594892675{]}
 {[} 0.689613233174427 -0.090416539541802  0.718545890295648{]}{]}
For angular tolerance 0.10 deg
Nb of pairs found / nb total of expected spots: 81/147
Matching Rate : 55.10
Nb missing reflections: 66

grain \#0 : 81 links to simulated spots have been found
GoodRefinement condition is  True
nb\_updates 81 compared to 6

---------------------------------------------
indexing completed for grain \#0 with matching rate 55.10
---------------------------------------------

writing fit file -------------------------
for grainindex= 0
self.dict\_grain\_matrix{[}grain\_index{]} {[}{[}-0.071478224945242 -0.995814588672313 -0.056724144676328{]}
 {[} 0.720639310822985 -0.012262885888099 -0.693156594892675{]}
 {[} 0.689613233174427 -0.090416539541802  0.718545890295648{]}{]}
self.refinedUBmatrix {[}{[}-0.071478224945242 -0.995814588672313 -0.056724144676328{]}
 {[} 0.720639310822985 -0.012262885888099 -0.693156594892675{]}
 {[} 0.689613233174427 -0.090416539541802  0.718545890295648{]}{]}
new UBs matrix in q= UBs G (s for strain)
strain\_direct {[}{[} 1.719550237533340e-06  5.250997004746657e-06 -2.870284279887629e-05{]}
 {[} 5.250997004746657e-06  1.388845576189013e-05 -9.199263307200638e-06{]}
 {[}-2.870284279887629e-05 -9.199263307200638e-06  3.998638790081444e-06{]}{]}
deviatoric strain {[}{[}-4.815998025634964e-06  5.250997004746657e-06 -2.870284279887629e-05{]}
 {[} 5.250997004746657e-06  7.352907498721824e-06 -9.199263307200638e-06{]}
 {[}-2.870284279887629e-05 -9.199263307200638e-06 -2.536909473086861e-06{]}{]}
new UBs matrix in q= UBs G (s for strain)
strain\_direct {[}{[} 1.719550237533340e-06  5.250997004746657e-06 -2.870284279887629e-05{]}
 {[} 5.250997004746657e-06  1.388845576189013e-05 -9.199263307200638e-06{]}
 {[}-2.870284279887629e-05 -9.199263307200638e-06  3.998638790081444e-06{]}{]}
deviatoric strain {[}{[}-4.815998025634964e-06  5.250997004746657e-06 -2.870284279887629e-05{]}
 {[} 5.250997004746657e-06  7.352907498721824e-06 -9.199263307200638e-06{]}
 {[}-2.870284279887629e-05 -9.199263307200638e-06 -2.536909473086861e-06{]}{]}
For comparison: a,b,c are rescaled with respect to the reference value of a = 5.657500 Angstroms
lattice\_parameter\_direct\_strain {[} 5.6575             5.657568845776604  5.6575132229395
 90.0010541881893   90.00328909016345  89.99939828842365 {]}
final lattice\_parameters {[} 5.6575             5.657568845776604  5.6575132229395
 90.0010541881893   90.00328909016345  89.99939828842365 {]}
File : Ge\_blanc\_0000Notebook\_g0.fit written in /home/micha/LaueToolsPy3/LaueTools/notebooks
Experimental experimental spots indices which are not indexed {[}{]}
Missing reflections grainindex is -100 for indexed grainindex 0
within angular tolerance 0.500

 Remaining nb of spots to index for grain \#1 : 1

81 spots have been indexed over 82
indexing rate is --- : 98.8 percents
indexation of ../LaueImages/Ge\_blanc\_0000Notebook.cor is completed
for the 1 grain(s) that has(ve) been indexed as requested
Leaving Index and Refine procedures...
Saving unindexed  fit file: ../LaueImages/Ge\_blanc\_0000Notebook\_unindexed.cor
File : ../LaueImages/Ge\_blanc\_0000Notebook\_g0.fit written in /home/micha/LaueToolsPy3/LaueTools/notebooks
\end{sphinxalltt}

\fvset{hllines={, ,}}%
\begin{sphinxVerbatim}[commandchars=\\\{\}]
\PYG{n+nb}{print}\PYG{p}{(}\PYG{l+s+s1}{\PYGZsq{}}\PYG{l+s+s1}{Indexation time }\PYG{l+s+si}{\PYGZpc{}.3f}\PYG{l+s+s1}{ second(s) }\PYG{l+s+se}{\PYGZbs{}n}\PYG{l+s+se}{\PYGZbs{}n}\PYG{l+s+s1}{\PYGZsq{}}\PYG{o}{\PYGZpc{}}\PYG{n}{tf}\PYG{p}{)}
\PYG{n+nb}{print}\PYG{p}{(}\PYG{l+s+s1}{\PYGZsq{}}\PYG{l+s+s1}{Spots properties of the 10 first spots that have been indexed (sorted by intensity)}\PYG{l+s+s1}{\PYGZsq{}}\PYG{p}{)}
\PYG{n+nb}{print}\PYG{p}{(}\PYG{l+s+s1}{\PYGZsq{}}\PYG{l+s+s1}{\PYGZsh{}spot 2theta chi X, Y intensity h k l energy}\PYG{l+s+s1}{\PYGZsq{}}\PYG{p}{)}
\PYG{n+nb}{print}\PYG{p}{(}\PYG{n}{DataSet}\PYG{o}{.}\PYG{n}{getSpotsFamilyallData}\PYG{p}{(}\PYG{l+m+mi}{0}\PYG{p}{)}\PYG{p}{[}\PYG{p}{:}\PYG{l+m+mi}{10}\PYG{p}{]}\PYG{p}{)}
\end{sphinxVerbatim}

\fvset{hllines={, ,}}%
\begin{sphinxVerbatim}[commandchars=\\\{\}]
\PYG{n}{Indexation} \PYG{n}{time} \PYG{l+m+mf}{2.487} \PYG{n}{second}\PYG{p}{(}\PYG{n}{s}\PYG{p}{)}


\PYG{n}{Spots} \PYG{n}{properties} \PYG{n}{of} \PYG{n}{the} \PYG{l+m+mi}{10} \PYG{n}{first} \PYG{n}{spots} \PYG{n}{that} \PYG{n}{have} \PYG{n}{been} \PYG{n}{indexed} \PYG{p}{(}\PYG{n+nb}{sorted} \PYG{n}{by} \PYG{n}{intensity}\PYG{p}{)}
\PYG{c+c1}{\PYGZsh{}spot 2theta chi X, Y intensity h k l energy}
\PYG{p}{[}\PYG{p}{[} \PYG{l+m+mf}{0.000000000000000e+00}  \PYG{l+m+mf}{5.842691500000000e+01}  \PYG{l+m+mf}{2.013003500000000e+01}
   \PYG{l+m+mf}{6.231300000000000e+02}  \PYG{l+m+mf}{1.657730000000000e+03}  \PYG{l+m+mf}{2.979938000000000e+04}
   \PYG{l+m+mf}{4.000000000000000e+00}  \PYG{l+m+mf}{2.000000000000000e+00}  \PYG{l+m+mf}{2.000000000000000e+00}
   \PYG{l+m+mf}{1.099874517758171e+01}\PYG{p}{]}
 \PYG{p}{[} \PYG{l+m+mf}{1.000000000000000e+00}  \PYG{l+m+mf}{5.763467200000000e+01} \PYG{o}{\PYGZhy{}}\PYG{l+m+mf}{1.841552300000000e+01}
   \PYG{l+m+mf}{1.244330000000000e+03}  \PYG{l+m+mf}{1.662150000000000e+03}  \PYG{l+m+mf}{2.242563000000000e+04}
   \PYG{l+m+mf}{2.000000000000000e+00}  \PYG{l+m+mf}{2.000000000000000e+00}  \PYG{l+m+mf}{4.000000000000000e+00}
   \PYG{l+m+mf}{1.113626245226060e+01}\PYG{p}{]}
 \PYG{p}{[} \PYG{l+m+mf}{2.000000000000000e+00}  \PYG{l+m+mf}{8.091984600000001e+01}  \PYG{l+m+mf}{6.615810000000000e\PYGZhy{}01}
   \PYG{l+m+mf}{9.330400000000000e+02}  \PYG{l+m+mf}{1.215440000000000e+03}  \PYG{l+m+mf}{2.219754000000000e+04}
   \PYG{l+m+mf}{3.000000000000000e+00}  \PYG{l+m+mf}{3.000000000000000e+00}  \PYG{l+m+mf}{3.000000000000000e+00}
   \PYG{l+m+mf}{8.773642495456929e+00}\PYG{p}{]}
 \PYG{p}{[} \PYG{l+m+mf}{3.000000000000000e+00}  \PYG{l+m+mf}{1.168117220000000e+02}  \PYG{l+m+mf}{2.128975200000000e+01}
   \PYG{l+m+mf}{5.852300000000000e+02}  \PYG{l+m+mf}{5.888000000000000e+02}  \PYG{l+m+mf}{9.528830000000000e+03}
   \PYG{l+m+mf}{4.000000000000000e+00}  \PYG{l+m+mf}{6.000000000000000e+00}  \PYG{l+m+mf}{2.000000000000000e+00}
   \PYG{l+m+mf}{9.626187155122841e+00}\PYG{p}{]}
 \PYG{p}{[} \PYG{l+m+mf}{4.000000000000000e+00}  \PYG{l+m+mf}{1.159585970000000e+02} \PYG{o}{\PYGZhy{}}\PYG{l+m+mf}{2.073868400000000e+01}
   \PYG{l+m+mf}{1.276610000000000e+03}  \PYG{l+m+mf}{6.003000000000000e+02}  \PYG{l+m+mf}{9.153969999999999e+03}
   \PYG{l+m+mf}{2.000000000000000e+00}  \PYG{l+m+mf}{6.000000000000000e+00}  \PYG{l+m+mf}{4.000000000000000e+00}
   \PYG{l+m+mf}{9.671066779127555e+00}\PYG{p}{]}
 \PYG{p}{[} \PYG{l+m+mf}{5.000000000000000e+00}  \PYG{l+m+mf}{1.097624120000000e+02}  \PYG{l+m+mf}{3.270820000000000e\PYGZhy{}01}
   \PYG{l+m+mf}{9.326600000000000e+02}  \PYG{l+m+mf}{7.500800000000000e+02}  \PYG{l+m+mf}{6.940080000000000e+03}
   \PYG{l+m+mf}{3.000000000000000e+00}  \PYG{l+m+mf}{5.000000000000000e+00}  \PYG{l+m+mf}{3.000000000000000e+00}
   \PYG{l+m+mf}{8.784305505382406e+00}\PYG{p}{]}
 \PYG{p}{[} \PYG{l+m+mf}{6.000000000000000e+00}  \PYG{l+m+mf}{9.664626300000000e+01} \PYG{o}{\PYGZhy{}}\PYG{l+m+mf}{3.623969200000000e+01}
   \PYG{l+m+mf}{1.596620000000000e+03}  \PYG{l+m+mf}{9.353600000000000e+02}  \PYG{l+m+mf}{6.136730000000000e+03}
   \PYG{l+m+mf}{1.000000000000000e+00}  \PYG{l+m+mf}{5.000000000000000e+00}  \PYG{l+m+mf}{5.000000000000000e+00}
   \PYG{l+m+mf}{1.047621498150968e+01}\PYG{p}{]}
 \PYG{p}{[} \PYG{l+m+mf}{7.000000000000000e+00}  \PYG{l+m+mf}{9.807526100000000e+01}  \PYG{l+m+mf}{3.749168800000000e+01}
   \PYG{l+m+mf}{2.523600000000000e+02}  \PYG{l+m+mf}{9.206000000000000e+02}  \PYG{l+m+mf}{5.635550000000000e+03}
   \PYG{l+m+mf}{5.000000000000000e+00}  \PYG{l+m+mf}{5.000000000000000e+00}  \PYG{l+m+mf}{1.000000000000000e+00}
   \PYG{l+m+mf}{1.036136430057910e+01}\PYG{p}{]}
 \PYG{p}{[} \PYG{l+m+mf}{8.000000000000000e+00}  \PYG{l+m+mf}{6.421410100000000e+01} \PYG{o}{\PYGZhy{}}\PYG{l+m+mf}{1.397936600000000e+01}
   \PYG{l+m+mf}{1.168360000000000e+03}  \PYG{l+m+mf}{1.512820000000000e+03}  \PYG{l+m+mf}{5.570320000000000e+03}
   \PYG{l+m+mf}{3.000000000000000e+00}  \PYG{l+m+mf}{3.000000000000000e+00}  \PYG{l+m+mf}{5.000000000000000e+00}
   \PYG{l+m+mf}{1.351813316448898e+01}\PYG{p}{]}
 \PYG{p}{[} \PYG{l+m+mf}{9.000000000000000e+00}  \PYG{l+m+mf}{9.620103100000000e+01} \PYG{o}{\PYGZhy{}}\PYG{l+m+mf}{4.831242900000000e+01}
   \PYG{l+m+mf}{1.945880000000000e+03}  \PYG{l+m+mf}{9.142200000000000e+02}  \PYG{l+m+mf}{5.317320000000000e+03}
   \PYG{l+m+mf}{0.000000000000000e+00}  \PYG{l+m+mf}{4.000000000000000e+00}  \PYG{l+m+mf}{4.000000000000000e+00}
   \PYG{l+m+mf}{8.328162135695869e+00}\PYG{p}{]}\PYG{p}{]}
\end{sphinxVerbatim}

DataSet is an object with many attributes and methods related to spots
properties (indexed or not, belonging to grains counted from zero). By
press Tab key after having typed DataSet. can show you infos about spots

\fvset{hllines={, ,}}%
\begin{sphinxVerbatim}[commandchars=\\\{\}]
\PYG{n}{DataSet}\PYG{o}{.}\PYG{n}{B0matrix}
\end{sphinxVerbatim}

\fvset{hllines={, ,}}%
\begin{sphinxVerbatim}[commandchars=\\\{\}]
\PYG{n}{array}\PYG{p}{(}\PYG{p}{[}\PYG{p}{[} \PYG{l+m+mf}{1.767565178965975e\PYGZhy{}01}\PYG{p}{,} \PYG{o}{\PYGZhy{}}\PYG{l+m+mf}{2.842461599074922e\PYGZhy{}17}\PYG{p}{,}
        \PYG{o}{\PYGZhy{}}\PYG{l+m+mf}{2.842461599074922e\PYGZhy{}17}\PYG{p}{]}\PYG{p}{,}
       \PYG{p}{[} \PYG{l+m+mf}{0.000000000000000e+00}\PYG{p}{,}  \PYG{l+m+mf}{1.767565178965975e\PYGZhy{}01}\PYG{p}{,}
        \PYG{o}{\PYGZhy{}}\PYG{l+m+mf}{1.082321519352500e\PYGZhy{}17}\PYG{p}{]}\PYG{p}{,}
       \PYG{p}{[} \PYG{l+m+mf}{0.000000000000000e+00}\PYG{p}{,}  \PYG{l+m+mf}{0.000000000000000e+00}\PYG{p}{,}
         \PYG{l+m+mf}{1.767565178965975e\PYGZhy{}01}\PYG{p}{]}\PYG{p}{]}\PYG{p}{)}
\end{sphinxVerbatim}


\section{Indexation of spots data set.}
\label{\detokenize{Indexation::doc}}\label{\detokenize{Indexation:indexation-of-spots-data-set}}
\sphinxstylestrong{Using Class spotsSet and play with spots considered for indexation and
refinement}


\subsection{This Notebook is a part of Tutorials on LaueTools Suite. Author:J.-S. Micha Date: July 2019}
\label{\detokenize{Indexation:this-notebook-is-a-part-of-tutorials-on-lauetools-suite-author-j-s-micha-date-july-2019}}
\fvset{hllines={, ,}}%
\begin{sphinxVerbatim}[commandchars=\\\{\}]
\PYG{o}{\PYGZpc{}}\PYG{n}{matplotlib} \PYG{n}{inline}

\PYG{k+kn}{import} \PYG{n+nn}{matplotlib}
\PYG{k+kn}{import} \PYG{n+nn}{numpy} \PYG{k}{as} \PYG{n+nn}{np}
\PYG{k+kn}{import} \PYG{n+nn}{matplotlib}\PYG{n+nn}{.}\PYG{n+nn}{pyplot} \PYG{k}{as} \PYG{n+nn}{plt}
\PYG{k+kn}{import} \PYG{n+nn}{time}\PYG{o}{,}\PYG{n+nn}{copy}\PYG{o}{,}\PYG{n+nn}{os}

\PYG{c+c1}{\PYGZsh{} third party LaueTools import}
\PYG{k+kn}{import} \PYG{n+nn}{LaueTools}\PYG{n+nn}{.}\PYG{n+nn}{readmccd} \PYG{k}{as} \PYG{n+nn}{RMCCD}
\PYG{k+kn}{import} \PYG{n+nn}{LaueTools}\PYG{n+nn}{.}\PYG{n+nn}{LaueGeometry} \PYG{k}{as} \PYG{n+nn}{F2TC}
\PYG{k+kn}{import} \PYG{n+nn}{LaueTools}\PYG{n+nn}{.}\PYG{n+nn}{indexingSpotsSet} \PYG{k}{as} \PYG{n+nn}{ISS}
\PYG{k+kn}{import} \PYG{n+nn}{LaueTools}\PYG{n+nn}{.}\PYG{n+nn}{IOLaueTools} \PYG{k}{as} \PYG{n+nn}{RWASCII}
\end{sphinxVerbatim}
\begin{sphinxalltt}
/home/micha/anaconda3/lib/python3.6/site-packages/h5py/\_\_init\_\_.py:36: FutureWarning: Conversion of the second argument of issubdtype from \sphinxtitleref{float} to \sphinxtitleref{np.floating} is deprecated. In future, it will be treated as \sphinxtitleref{np.float64 == np.dtype(float).type}.
  from .\_conv import register\_converters as \_register\_converters
\end{sphinxalltt}
\begin{sphinxalltt}
module Image / PIL is not installed
LaueToolsProjectFolder /home/micha/LaueToolsPy3/LaueTools
Cython compiled module 'gaussian2D' for fast computation is not installed!
module Image / PIL is not installed
Entering CrystalParameters \sphinxstylestrong{**}---\sphinxstylestrong{***********************}


Cython compiled module for fast computation of Laue spots is not installed!
Cython compiled 'angulardist' module for fast computation of angular distance is not installed!
Using default module
Cython compiled module for fast computation of angular distance is not installed!
module Image / PIL is not installed
\end{sphinxalltt}

refinement from guessed solutions with two materials (see script
IndexingTwinsSeries)


\subsubsection{Let’s take a simple example of a single Laue Pattern. From the peak search we get 83 spots}
\label{\detokenize{Indexation:let-s-take-a-simple-example-of-a-single-laue-pattern-from-the-peak-search-we-get-83-spots}}
\fvset{hllines={, ,}}%
\begin{sphinxVerbatim}[commandchars=\\\{\}]
\PYG{n}{folder}\PYG{o}{=} \PYG{l+s+s1}{\PYGZsq{}}\PYG{l+s+s1}{../Examples/Ge/}\PYG{l+s+s1}{\PYGZsq{}}
\PYG{n}{datfilename}\PYG{o}{=}\PYG{l+s+s1}{\PYGZsq{}}\PYG{l+s+s1}{Ge0001.dat}\PYG{l+s+s1}{\PYGZsq{}}
\end{sphinxVerbatim}

\fvset{hllines={, ,}}%
\begin{sphinxVerbatim}[commandchars=\\\{\}]
\PYG{n}{key\PYGZus{}material}\PYG{o}{=}\PYG{l+s+s1}{\PYGZsq{}}\PYG{l+s+s1}{Ge}\PYG{l+s+s1}{\PYGZsq{}}
\PYG{n}{emin}\PYG{p}{,} \PYG{n}{emax}\PYG{o}{=} \PYG{l+m+mi}{5}\PYG{p}{,}\PYG{l+m+mi}{23}
\end{sphinxVerbatim}

\fvset{hllines={, ,}}%
\begin{sphinxVerbatim}[commandchars=\\\{\}]
\PYG{c+c1}{\PYGZsh{} detector geometry and parameters as read from Ge0001.det}
\PYG{n}{calibration\PYGZus{}parameters} \PYG{o}{=} \PYG{p}{[}\PYG{l+m+mf}{69.179}\PYG{p}{,} \PYG{l+m+mf}{1050.81}\PYG{p}{,} \PYG{l+m+mf}{1115.59}\PYG{p}{,} \PYG{l+m+mf}{0.104}\PYG{p}{,} \PYG{o}{\PYGZhy{}}\PYG{l+m+mf}{0.273}\PYG{p}{]}
\PYG{n}{CCDCalibdict} \PYG{o}{=} \PYG{p}{\PYGZob{}}\PYG{p}{\PYGZcb{}}
\PYG{n}{CCDCalibdict}\PYG{p}{[}\PYG{l+s+s1}{\PYGZsq{}}\PYG{l+s+s1}{CCDCalibParameters}\PYG{l+s+s1}{\PYGZsq{}}\PYG{p}{]} \PYG{o}{=} \PYG{n}{calibration\PYGZus{}parameters}
\PYG{n}{CCDCalibdict}\PYG{p}{[}\PYG{l+s+s1}{\PYGZsq{}}\PYG{l+s+s1}{framedim}\PYG{l+s+s1}{\PYGZsq{}}\PYG{p}{]} \PYG{o}{=} \PYG{p}{(}\PYG{l+m+mi}{2048}\PYG{p}{,} \PYG{l+m+mi}{2048}\PYG{p}{)}
\PYG{n}{CCDCalibdict}\PYG{p}{[}\PYG{l+s+s1}{\PYGZsq{}}\PYG{l+s+s1}{detectordiameter}\PYG{l+s+s1}{\PYGZsq{}}\PYG{p}{]} \PYG{o}{=} \PYG{l+m+mf}{165.}
\PYG{n}{CCDCalibdict}\PYG{p}{[}\PYG{l+s+s1}{\PYGZsq{}}\PYG{l+s+s1}{kf\PYGZus{}direction}\PYG{l+s+s1}{\PYGZsq{}}\PYG{p}{]} \PYG{o}{=} \PYG{l+s+s1}{\PYGZsq{}}\PYG{l+s+s1}{Z\PYGZgt{}0}\PYG{l+s+s1}{\PYGZsq{}}
\PYG{n}{CCDCalibdict}\PYG{p}{[}\PYG{l+s+s1}{\PYGZsq{}}\PYG{l+s+s1}{xpixelsize}\PYG{l+s+s1}{\PYGZsq{}}\PYG{p}{]} \PYG{o}{=} \PYG{l+m+mf}{0.08057}
\PYG{c+c1}{\PYGZsh{} CCDCalibdict can also be simply build by reading the proper .det file}
\PYG{n+nb}{print}\PYG{p}{(}\PYG{l+s+s2}{\PYGZdq{}}\PYG{l+s+s2}{reading geometry calibration file}\PYG{l+s+s2}{\PYGZdq{}}\PYG{p}{)}
\PYG{n}{CCDCalibdict}\PYG{o}{=}\PYG{n}{RWASCII}\PYG{o}{.}\PYG{n}{readCalib\PYGZus{}det\PYGZus{}file}\PYG{p}{(}\PYG{n}{os}\PYG{o}{.}\PYG{n}{path}\PYG{o}{.}\PYG{n}{join}\PYG{p}{(}\PYG{n}{folder}\PYG{p}{,}\PYG{l+s+s1}{\PYGZsq{}}\PYG{l+s+s1}{Ge0001.det}\PYG{l+s+s1}{\PYGZsq{}}\PYG{p}{)}\PYG{p}{)}
\PYG{n}{CCDCalibdict}\PYG{p}{[}\PYG{l+s+s1}{\PYGZsq{}}\PYG{l+s+s1}{kf\PYGZus{}direction}\PYG{l+s+s1}{\PYGZsq{}}\PYG{p}{]} \PYG{o}{=} \PYG{l+s+s1}{\PYGZsq{}}\PYG{l+s+s1}{Z\PYGZgt{}0}\PYG{l+s+s1}{\PYGZsq{}}
\end{sphinxVerbatim}

\fvset{hllines={, ,}}%
\begin{sphinxVerbatim}[commandchars=\\\{\}]
\PYG{n}{reading} \PYG{n}{geometry} \PYG{n}{calibration} \PYG{n}{file}
\PYG{n}{calib} \PYG{o}{=}  \PYG{p}{[} \PYG{l+m+mf}{6.91790e+01}  \PYG{l+m+mf}{1.05081e+03}  \PYG{l+m+mf}{1.11559e+03}  \PYG{l+m+mf}{1.04000e\PYGZhy{}01} \PYG{o}{\PYGZhy{}}\PYG{l+m+mf}{2.73000e\PYGZhy{}01}
  \PYG{l+m+mf}{8.05700e\PYGZhy{}02}  \PYG{l+m+mf}{2.04800e+03}  \PYG{l+m+mf}{2.04800e+03}\PYG{p}{]}
\PYG{n}{matrix} \PYG{o}{=}  \PYG{p}{[}\PYG{o}{\PYGZhy{}}\PYG{l+m+mf}{0.211596}  \PYG{l+m+mf}{0.092178} \PYG{o}{\PYGZhy{}}\PYG{l+m+mf}{0.973001} \PYG{o}{\PYGZhy{}}\PYG{l+m+mf}{0.77574}   \PYG{l+m+mf}{0.589743}  \PYG{l+m+mf}{0.224568}  \PYG{l+m+mf}{0.594521}
  \PYG{l+m+mf}{0.802313} \PYG{o}{\PYGZhy{}}\PYG{l+m+mf}{0.053281}\PYG{p}{]}
\end{sphinxVerbatim}

Compute scattering angles from spots pixel positions and detector geometry. Write a .cor file from .dat including these new infos

\fvset{hllines={, ,}}%
\begin{sphinxVerbatim}[commandchars=\\\{\}]
\PYG{n}{F2TC}\PYG{o}{.}\PYG{n}{convert2corfile}\PYG{p}{(}\PYG{n}{datfilename}\PYG{p}{,}
                         \PYG{n}{calibration\PYGZus{}parameters}\PYG{p}{,}
                         \PYG{n}{dirname\PYGZus{}in}\PYG{o}{=}\PYG{n}{folder}\PYG{p}{,}
                        \PYG{n}{dirname\PYGZus{}out}\PYG{o}{=}\PYG{n}{folder}\PYG{p}{,}
                        \PYG{n}{CCDCalibdict}\PYG{o}{=}\PYG{n}{CCDCalibdict}\PYG{p}{)}
\PYG{n}{corfilename} \PYG{o}{=} \PYG{n}{datfilename}\PYG{o}{.}\PYG{n}{split}\PYG{p}{(}\PYG{l+s+s1}{\PYGZsq{}}\PYG{l+s+s1}{.}\PYG{l+s+s1}{\PYGZsq{}}\PYG{p}{)}\PYG{p}{[}\PYG{l+m+mi}{0}\PYG{p}{]} \PYG{o}{+} \PYG{l+s+s1}{\PYGZsq{}}\PYG{l+s+s1}{.cor}\PYG{l+s+s1}{\PYGZsq{}}
\PYG{n}{fullpathcorfile} \PYG{o}{=} \PYG{n}{os}\PYG{o}{.}\PYG{n}{path}\PYG{o}{.}\PYG{n}{join}\PYG{p}{(}\PYG{n}{folder}\PYG{p}{,}\PYG{n}{corfilename}\PYG{p}{)}
\end{sphinxVerbatim}

\fvset{hllines={, ,}}%
\begin{sphinxVerbatim}[commandchars=\\\{\}]
\PYG{n}{nb} \PYG{n}{of} \PYG{n}{spots} \PYG{o+ow}{and} \PYG{n}{columns} \PYG{o+ow}{in} \PYG{o}{.}\PYG{n}{dat} \PYG{n}{file} \PYG{p}{(}\PYG{l+m+mi}{83}\PYG{p}{,} \PYG{l+m+mi}{3}\PYG{p}{)}
\PYG{n}{file} \PYG{p}{:}\PYG{o}{.}\PYG{o}{.}\PYG{o}{/}\PYG{n}{Examples}\PYG{o}{/}\PYG{n}{Ge}\PYG{o}{/}\PYG{n}{Ge0001}\PYG{o}{.}\PYG{n}{dat}
\PYG{n}{containing} \PYG{l+m+mi}{83} \PYG{n}{peaks}
\PYG{p}{(}\PYG{l+m+mi}{2}\PYG{n}{theta} \PYG{n}{chi} \PYG{n}{X} \PYG{n}{Y} \PYG{n}{I}\PYG{p}{)} \PYG{n}{written} \PYG{o+ow}{in} \PYG{o}{.}\PYG{o}{.}\PYG{o}{/}\PYG{n}{Examples}\PYG{o}{/}\PYG{n}{Ge}\PYG{o}{/}\PYG{n}{Ge0001}\PYG{o}{.}\PYG{n}{cor}
\end{sphinxVerbatim}

Create an instance of the class spotset. Initialize spots properties to data contained in .cor file

\fvset{hllines={, ,}}%
\begin{sphinxVerbatim}[commandchars=\\\{\}]
\PYG{n}{DataSet} \PYG{o}{=} \PYG{n}{ISS}\PYG{o}{.}\PYG{n}{spotsset}\PYG{p}{(}\PYG{p}{)}

\PYG{n}{DataSet}\PYG{o}{.}\PYG{n}{importdatafromfile}\PYG{p}{(}\PYG{n}{fullpathcorfile}\PYG{p}{)}
\end{sphinxVerbatim}

\fvset{hllines={, ,}}%
\begin{sphinxVerbatim}[commandchars=\\\{\}]
\PYG{n}{CCDcalib} \PYG{o+ow}{in} \PYG{n}{readfile\PYGZus{}cor} \PYG{p}{\PYGZob{}}\PYG{l+s+s1}{\PYGZsq{}}\PYG{l+s+s1}{dd}\PYG{l+s+s1}{\PYGZsq{}}\PYG{p}{:} \PYG{l+m+mf}{69.179}\PYG{p}{,} \PYG{l+s+s1}{\PYGZsq{}}\PYG{l+s+s1}{xcen}\PYG{l+s+s1}{\PYGZsq{}}\PYG{p}{:} \PYG{l+m+mf}{1050.81}\PYG{p}{,} \PYG{l+s+s1}{\PYGZsq{}}\PYG{l+s+s1}{ycen}\PYG{l+s+s1}{\PYGZsq{}}\PYG{p}{:} \PYG{l+m+mf}{1115.59}\PYG{p}{,} \PYG{l+s+s1}{\PYGZsq{}}\PYG{l+s+s1}{xbet}\PYG{l+s+s1}{\PYGZsq{}}\PYG{p}{:} \PYG{l+m+mf}{0.104}\PYG{p}{,} \PYG{l+s+s1}{\PYGZsq{}}\PYG{l+s+s1}{xgam}\PYG{l+s+s1}{\PYGZsq{}}\PYG{p}{:} \PYG{o}{\PYGZhy{}}\PYG{l+m+mf}{0.273}\PYG{p}{,} \PYG{l+s+s1}{\PYGZsq{}}\PYG{l+s+s1}{xpixelsize}\PYG{l+s+s1}{\PYGZsq{}}\PYG{p}{:} \PYG{l+m+mf}{0.08057}\PYG{p}{,} \PYG{l+s+s1}{\PYGZsq{}}\PYG{l+s+s1}{ypixelsize}\PYG{l+s+s1}{\PYGZsq{}}\PYG{p}{:} \PYG{l+m+mf}{0.08057}\PYG{p}{,} \PYG{l+s+s1}{\PYGZsq{}}\PYG{l+s+s1}{CCDLabel}\PYG{l+s+s1}{\PYGZsq{}}\PYG{p}{:} \PYG{l+s+s1}{\PYGZsq{}}\PYG{l+s+s1}{sCMOS}\PYG{l+s+s1}{\PYGZsq{}}\PYG{p}{,} \PYG{l+s+s1}{\PYGZsq{}}\PYG{l+s+s1}{framedim}\PYG{l+s+s1}{\PYGZsq{}}\PYG{p}{:} \PYG{p}{[}\PYG{l+m+mf}{2048.0}\PYG{p}{,} \PYG{l+m+mf}{2048.0}\PYG{p}{]}\PYG{p}{,} \PYG{l+s+s1}{\PYGZsq{}}\PYG{l+s+s1}{detectordiameter}\PYG{l+s+s1}{\PYGZsq{}}\PYG{p}{:} \PYG{l+m+mf}{165.00736}\PYG{p}{,} \PYG{l+s+s1}{\PYGZsq{}}\PYG{l+s+s1}{kf\PYGZus{}direction}\PYG{l+s+s1}{\PYGZsq{}}\PYG{p}{:} \PYG{l+s+s1}{\PYGZsq{}}\PYG{l+s+s1}{Z\PYGZgt{}0}\PYG{l+s+s1}{\PYGZsq{}}\PYG{p}{,} \PYG{l+s+s1}{\PYGZsq{}}\PYG{l+s+s1}{pixelsize}\PYG{l+s+s1}{\PYGZsq{}}\PYG{p}{:} \PYG{l+m+mf}{0.08057}\PYG{p}{\PYGZcb{}}
\PYG{n}{CCD} \PYG{n}{Detector} \PYG{n}{parameters} \PYG{n}{read} \PYG{k+kn}{from} \PYG{n+nn}{.}\PYG{n+nn}{cor} \PYG{n}{file}
\PYG{n}{CCDcalibdict} \PYG{p}{\PYGZob{}}\PYG{l+s+s1}{\PYGZsq{}}\PYG{l+s+s1}{dd}\PYG{l+s+s1}{\PYGZsq{}}\PYG{p}{:} \PYG{l+m+mf}{69.179}\PYG{p}{,} \PYG{l+s+s1}{\PYGZsq{}}\PYG{l+s+s1}{xcen}\PYG{l+s+s1}{\PYGZsq{}}\PYG{p}{:} \PYG{l+m+mf}{1050.81}\PYG{p}{,} \PYG{l+s+s1}{\PYGZsq{}}\PYG{l+s+s1}{ycen}\PYG{l+s+s1}{\PYGZsq{}}\PYG{p}{:} \PYG{l+m+mf}{1115.59}\PYG{p}{,} \PYG{l+s+s1}{\PYGZsq{}}\PYG{l+s+s1}{xbet}\PYG{l+s+s1}{\PYGZsq{}}\PYG{p}{:} \PYG{l+m+mf}{0.104}\PYG{p}{,} \PYG{l+s+s1}{\PYGZsq{}}\PYG{l+s+s1}{xgam}\PYG{l+s+s1}{\PYGZsq{}}\PYG{p}{:} \PYG{o}{\PYGZhy{}}\PYG{l+m+mf}{0.273}\PYG{p}{,} \PYG{l+s+s1}{\PYGZsq{}}\PYG{l+s+s1}{xpixelsize}\PYG{l+s+s1}{\PYGZsq{}}\PYG{p}{:} \PYG{l+m+mf}{0.08057}\PYG{p}{,} \PYG{l+s+s1}{\PYGZsq{}}\PYG{l+s+s1}{ypixelsize}\PYG{l+s+s1}{\PYGZsq{}}\PYG{p}{:} \PYG{l+m+mf}{0.08057}\PYG{p}{,} \PYG{l+s+s1}{\PYGZsq{}}\PYG{l+s+s1}{CCDLabel}\PYG{l+s+s1}{\PYGZsq{}}\PYG{p}{:} \PYG{l+s+s1}{\PYGZsq{}}\PYG{l+s+s1}{sCMOS}\PYG{l+s+s1}{\PYGZsq{}}\PYG{p}{,} \PYG{l+s+s1}{\PYGZsq{}}\PYG{l+s+s1}{framedim}\PYG{l+s+s1}{\PYGZsq{}}\PYG{p}{:} \PYG{p}{[}\PYG{l+m+mf}{2048.0}\PYG{p}{,} \PYG{l+m+mf}{2048.0}\PYG{p}{]}\PYG{p}{,} \PYG{l+s+s1}{\PYGZsq{}}\PYG{l+s+s1}{detectordiameter}\PYG{l+s+s1}{\PYGZsq{}}\PYG{p}{:} \PYG{l+m+mf}{165.00736}\PYG{p}{,} \PYG{l+s+s1}{\PYGZsq{}}\PYG{l+s+s1}{kf\PYGZus{}direction}\PYG{l+s+s1}{\PYGZsq{}}\PYG{p}{:} \PYG{l+s+s1}{\PYGZsq{}}\PYG{l+s+s1}{Z\PYGZgt{}0}\PYG{l+s+s1}{\PYGZsq{}}\PYG{p}{,} \PYG{l+s+s1}{\PYGZsq{}}\PYG{l+s+s1}{pixelsize}\PYG{l+s+s1}{\PYGZsq{}}\PYG{p}{:} \PYG{l+m+mf}{0.08057}\PYG{p}{\PYGZcb{}}
\end{sphinxVerbatim}

\fvset{hllines={, ,}}%
\begin{sphinxVerbatim}[commandchars=\\\{\}]
\PYG{k+kc}{True}
\end{sphinxVerbatim}

Class methods and attributes rely on a dictionnary of spots properties. key = exprimental spot index, val = spots properties

\fvset{hllines={, ,}}%
\begin{sphinxVerbatim}[commandchars=\\\{\}]
\PYG{p}{[}\PYG{n}{DataSet}\PYG{o}{.}\PYG{n}{indexed\PYGZus{}spots\PYGZus{}dict}\PYG{p}{[}\PYG{n}{k}\PYG{p}{]} \PYG{k}{for} \PYG{n}{k} \PYG{o+ow}{in} \PYG{n+nb}{range}\PYG{p}{(}\PYG{l+m+mi}{10}\PYG{p}{)}\PYG{p}{]}
\end{sphinxVerbatim}

\fvset{hllines={, ,}}%
\begin{sphinxVerbatim}[commandchars=\\\{\}]
\PYG{p}{[}\PYG{p}{[}\PYG{l+m+mi}{0}\PYG{p}{,} \PYG{l+m+mf}{78.215821}\PYG{p}{,} \PYG{l+m+mf}{1.638153}\PYG{p}{,} \PYG{l+m+mf}{1027.11}\PYG{p}{,} \PYG{l+m+mf}{1293.28}\PYG{p}{,} \PYG{l+m+mf}{70931.27}\PYG{p}{,} \PYG{l+m+mi}{0}\PYG{p}{]}\PYG{p}{,}
 \PYG{p}{[}\PYG{l+m+mi}{1}\PYG{p}{,} \PYG{l+m+mf}{64.329767}\PYG{p}{,} \PYG{o}{\PYGZhy{}}\PYG{l+m+mf}{20.824155}\PYG{p}{,} \PYG{l+m+mf}{1379.17}\PYG{p}{,} \PYG{l+m+mf}{1553.58}\PYG{p}{,} \PYG{l+m+mf}{51933.84}\PYG{p}{,} \PYG{l+m+mi}{0}\PYG{p}{]}\PYG{p}{,}
 \PYG{p}{[}\PYG{l+m+mi}{2}\PYG{p}{,} \PYG{l+m+mf}{68.680451}\PYG{p}{,} \PYG{o}{\PYGZhy{}}\PYG{l+m+mf}{15.358122}\PYG{p}{,} \PYG{l+m+mf}{1288.11}\PYG{p}{,} \PYG{l+m+mf}{1460.16}\PYG{p}{,} \PYG{l+m+mf}{22795.07}\PYG{p}{,} \PYG{l+m+mi}{0}\PYG{p}{]}\PYG{p}{,}
 \PYG{p}{[}\PYG{l+m+mi}{3}\PYG{p}{,} \PYG{l+m+mf}{105.61498}\PYG{p}{,} \PYG{l+m+mf}{8.176187}\PYG{p}{,} \PYG{l+m+mf}{926.22}\PYG{p}{,} \PYG{l+m+mf}{872.06}\PYG{p}{,} \PYG{l+m+mf}{19489.69}\PYG{p}{,} \PYG{l+m+mi}{0}\PYG{p}{]}\PYG{p}{,}
 \PYG{p}{[}\PYG{l+m+mi}{4}\PYG{p}{,} \PYG{l+m+mf}{103.859791}\PYG{p}{,} \PYG{l+m+mf}{27.866566}\PYG{p}{,} \PYG{l+m+mf}{595.46}\PYG{p}{,} \PYG{l+m+mf}{876.44}\PYG{p}{,} \PYG{l+m+mf}{19058.79}\PYG{p}{,} \PYG{l+m+mi}{0}\PYG{p}{]}\PYG{p}{,}
 \PYG{p}{[}\PYG{l+m+mi}{5}\PYG{p}{,} \PYG{l+m+mf}{120.59561}\PYG{p}{,} \PYG{o}{\PYGZhy{}}\PYG{l+m+mf}{8.92066}\PYG{p}{,} \PYG{l+m+mf}{1183.27}\PYG{p}{,} \PYG{l+m+mf}{598.92}\PYG{p}{,} \PYG{l+m+mf}{17182.88}\PYG{p}{,} \PYG{l+m+mi}{0}\PYG{p}{]}\PYG{p}{,}
 \PYG{p}{[}\PYG{l+m+mi}{6}\PYG{p}{,} \PYG{l+m+mf}{60.359458}\PYG{p}{,} \PYG{l+m+mf}{26.483191}\PYG{p}{,} \PYG{l+m+mf}{626.12}\PYG{p}{,} \PYG{l+m+mf}{1661.28}\PYG{p}{,} \PYG{l+m+mf}{15825.39}\PYG{p}{,} \PYG{l+m+mi}{0}\PYG{p}{]}\PYG{p}{,}
 \PYG{p}{[}\PYG{l+m+mi}{7}\PYG{p}{,} \PYG{l+m+mf}{56.269853}\PYG{p}{,} \PYG{l+m+mf}{12.967153}\PYG{p}{,} \PYG{l+m+mf}{856.14}\PYG{p}{,} \PYG{l+m+mf}{1702.52}\PYG{p}{,} \PYG{l+m+mf}{15486.2}\PYG{p}{,} \PYG{l+m+mi}{0}\PYG{p}{]}\PYG{p}{,}
 \PYG{p}{[}\PYG{l+m+mi}{8}\PYG{p}{,} \PYG{l+m+mf}{82.072076}\PYG{p}{,} \PYG{o}{\PYGZhy{}}\PYG{l+m+mf}{35.89243}\PYG{p}{,} \PYG{l+m+mf}{1672.67}\PYG{p}{,} \PYG{l+m+mf}{1258.62}\PYG{p}{,} \PYG{l+m+mf}{13318.81}\PYG{p}{,} \PYG{l+m+mi}{0}\PYG{p}{]}\PYG{p}{,}
 \PYG{p}{[}\PYG{l+m+mi}{9}\PYG{p}{,} \PYG{l+m+mf}{83.349535}\PYG{p}{,} \PYG{o}{\PYGZhy{}}\PYG{l+m+mf}{27.458061}\PYG{p}{,} \PYG{l+m+mf}{1497.4}\PYG{p}{,} \PYG{l+m+mf}{1224.7}\PYG{p}{,} \PYG{l+m+mf}{13145.99}\PYG{p}{,} \PYG{l+m+mi}{0}\PYG{p}{]}\PYG{p}{]}
\end{sphinxVerbatim}

\fvset{hllines={, ,}}%
\begin{sphinxVerbatim}[commandchars=\\\{\}]
\PYG{n}{DataSet}\PYG{o}{.}\PYG{n}{getUnIndexedSpotsallData}\PYG{p}{(}\PYG{p}{)}\PYG{p}{[}\PYG{p}{:}\PYG{l+m+mi}{3}\PYG{p}{]}
\end{sphinxVerbatim}

\fvset{hllines={, ,}}%
\begin{sphinxVerbatim}[commandchars=\\\{\}]
\PYG{n}{array}\PYG{p}{(}\PYG{p}{[}\PYG{p}{[} \PYG{l+m+mf}{0.0000000e+00}\PYG{p}{,}  \PYG{l+m+mf}{7.8215821e+01}\PYG{p}{,}  \PYG{l+m+mf}{1.6381530e+00}\PYG{p}{,}  \PYG{l+m+mf}{1.0271100e+03}\PYG{p}{,}
         \PYG{l+m+mf}{1.2932800e+03}\PYG{p}{,}  \PYG{l+m+mf}{7.0931270e+04}\PYG{p}{]}\PYG{p}{,}
       \PYG{p}{[} \PYG{l+m+mf}{1.0000000e+00}\PYG{p}{,}  \PYG{l+m+mf}{6.4329767e+01}\PYG{p}{,} \PYG{o}{\PYGZhy{}}\PYG{l+m+mf}{2.0824155e+01}\PYG{p}{,}  \PYG{l+m+mf}{1.3791700e+03}\PYG{p}{,}
         \PYG{l+m+mf}{1.5535800e+03}\PYG{p}{,}  \PYG{l+m+mf}{5.1933840e+04}\PYG{p}{]}\PYG{p}{,}
       \PYG{p}{[} \PYG{l+m+mf}{2.0000000e+00}\PYG{p}{,}  \PYG{l+m+mf}{6.8680451e+01}\PYG{p}{,} \PYG{o}{\PYGZhy{}}\PYG{l+m+mf}{1.5358122e+01}\PYG{p}{,}  \PYG{l+m+mf}{1.2881100e+03}\PYG{p}{,}
         \PYG{l+m+mf}{1.4601600e+03}\PYG{p}{,}  \PYG{l+m+mf}{2.2795070e+04}\PYG{p}{]}\PYG{p}{]}\PYG{p}{)}
\end{sphinxVerbatim}

\fvset{hllines={, ,}}%
\begin{sphinxVerbatim}[commandchars=\\\{\}]
\PYG{n}{dict\PYGZus{}loop} \PYG{o}{=} \PYG{p}{\PYGZob{}}\PYG{l+s+s1}{\PYGZsq{}}\PYG{l+s+s1}{MATCHINGRATE\PYGZus{}THRESHOLD\PYGZus{}IAL}\PYG{l+s+s1}{\PYGZsq{}}\PYG{p}{:} \PYG{l+m+mi}{100}\PYG{p}{,}
                   \PYG{l+s+s1}{\PYGZsq{}}\PYG{l+s+s1}{MATCHINGRATE\PYGZus{}ANGLE\PYGZus{}TOL}\PYG{l+s+s1}{\PYGZsq{}}\PYG{p}{:} \PYG{l+m+mf}{0.2}\PYG{p}{,}
                   \PYG{l+s+s1}{\PYGZsq{}}\PYG{l+s+s1}{NBMAXPROBED}\PYG{l+s+s1}{\PYGZsq{}}\PYG{p}{:} \PYG{l+m+mi}{6}\PYG{p}{,}
                   \PYG{l+s+s1}{\PYGZsq{}}\PYG{l+s+s1}{central spots indices}\PYG{l+s+s1}{\PYGZsq{}}\PYG{p}{:} \PYG{p}{[}\PYG{l+m+mi}{0}\PYG{p}{,}\PYG{p}{]}\PYG{p}{,}
                   \PYG{l+s+s1}{\PYGZsq{}}\PYG{l+s+s1}{AngleTolLUT}\PYG{l+s+s1}{\PYGZsq{}}\PYG{p}{:} \PYG{l+m+mf}{0.5}\PYG{p}{,}
                   \PYG{l+s+s1}{\PYGZsq{}}\PYG{l+s+s1}{UseIntensityWeights}\PYG{l+s+s1}{\PYGZsq{}}\PYG{p}{:} \PYG{k+kc}{False}\PYG{p}{,}
                   \PYG{l+s+s1}{\PYGZsq{}}\PYG{l+s+s1}{nbSpotsToIndex}\PYG{l+s+s1}{\PYGZsq{}}\PYG{p}{:}\PYG{l+m+mi}{10000}\PYG{p}{,}
                   \PYG{l+s+s1}{\PYGZsq{}}\PYG{l+s+s1}{list matching tol angles}\PYG{l+s+s1}{\PYGZsq{}}\PYG{p}{:}\PYG{p}{[}\PYG{l+m+mf}{0.5}\PYG{p}{,}\PYG{l+m+mf}{0.5}\PYG{p}{,}\PYG{l+m+mf}{0.2}\PYG{p}{,}\PYG{l+m+mf}{0.2}\PYG{p}{]}\PYG{p}{,}
                   \PYG{l+s+s1}{\PYGZsq{}}\PYG{l+s+s1}{nlutmax}\PYG{l+s+s1}{\PYGZsq{}}\PYG{p}{:}\PYG{l+m+mi}{3}\PYG{p}{,}
                   \PYG{l+s+s1}{\PYGZsq{}}\PYG{l+s+s1}{MinimumNumberMatches}\PYG{l+s+s1}{\PYGZsq{}}\PYG{p}{:} \PYG{l+m+mi}{3}\PYG{p}{,}
                   \PYG{l+s+s1}{\PYGZsq{}}\PYG{l+s+s1}{MinimumMatchingRate}\PYG{l+s+s1}{\PYGZsq{}}\PYG{p}{:}\PYG{l+m+mi}{3}
                   \PYG{p}{\PYGZcb{}}
\PYG{n}{grainindex}\PYG{o}{=}\PYG{l+m+mi}{0}
\PYG{n}{DataSet} \PYG{o}{=} \PYG{n}{ISS}\PYG{o}{.}\PYG{n}{spotsset}\PYG{p}{(}\PYG{p}{)}

\PYG{n}{DataSet}\PYG{o}{.}\PYG{n}{pixelsize} \PYG{o}{=} \PYG{n}{CCDCalibdict}\PYG{p}{[}\PYG{l+s+s1}{\PYGZsq{}}\PYG{l+s+s1}{xpixelsize}\PYG{l+s+s1}{\PYGZsq{}}\PYG{p}{]}
\PYG{n}{DataSet}\PYG{o}{.}\PYG{n}{dim} \PYG{o}{=} \PYG{n}{CCDCalibdict}\PYG{p}{[}\PYG{l+s+s1}{\PYGZsq{}}\PYG{l+s+s1}{framedim}\PYG{l+s+s1}{\PYGZsq{}}\PYG{p}{]}
\PYG{n}{DataSet}\PYG{o}{.}\PYG{n}{detectordiameter} \PYG{o}{=} \PYG{n}{CCDCalibdict}\PYG{p}{[}\PYG{l+s+s1}{\PYGZsq{}}\PYG{l+s+s1}{detectordiameter}\PYG{l+s+s1}{\PYGZsq{}}\PYG{p}{]}
\PYG{n}{DataSet}\PYG{o}{.}\PYG{n}{kf\PYGZus{}direction} \PYG{o}{=} \PYG{n}{CCDCalibdict}\PYG{p}{[}\PYG{l+s+s1}{\PYGZsq{}}\PYG{l+s+s1}{kf\PYGZus{}direction}\PYG{l+s+s1}{\PYGZsq{}}\PYG{p}{]}
\PYG{n}{DataSet}\PYG{o}{.}\PYG{n}{key\PYGZus{}material} \PYG{o}{=} \PYG{n}{key\PYGZus{}material}
\PYG{n}{DataSet}\PYG{o}{.}\PYG{n}{emin} \PYG{o}{=} \PYG{n}{emin}
\PYG{n}{DataSet}\PYG{o}{.}\PYG{n}{emax} \PYG{o}{=} \PYG{n}{emax}
\end{sphinxVerbatim}


\paragraph{Normally we read all spots data from a .cor file}
\label{\detokenize{Indexation:normally-we-read-all-spots-data-from-a-cor-file}}
\fvset{hllines={, ,}}%
\begin{sphinxVerbatim}[commandchars=\\\{\}]
\PYG{n}{DataSet}\PYG{o}{.}\PYG{n}{importdatafromfile}\PYG{p}{(}\PYG{n}{fullpathcorfile}\PYG{p}{)}
\PYG{n}{DataSet}\PYG{o}{.}\PYG{n}{emin}
\end{sphinxVerbatim}

\fvset{hllines={, ,}}%
\begin{sphinxVerbatim}[commandchars=\\\{\}]
\PYG{n}{CCDcalib} \PYG{o+ow}{in} \PYG{n}{readfile\PYGZus{}cor} \PYG{p}{\PYGZob{}}\PYG{l+s+s1}{\PYGZsq{}}\PYG{l+s+s1}{dd}\PYG{l+s+s1}{\PYGZsq{}}\PYG{p}{:} \PYG{l+m+mf}{69.179}\PYG{p}{,} \PYG{l+s+s1}{\PYGZsq{}}\PYG{l+s+s1}{xcen}\PYG{l+s+s1}{\PYGZsq{}}\PYG{p}{:} \PYG{l+m+mf}{1050.81}\PYG{p}{,} \PYG{l+s+s1}{\PYGZsq{}}\PYG{l+s+s1}{ycen}\PYG{l+s+s1}{\PYGZsq{}}\PYG{p}{:} \PYG{l+m+mf}{1115.59}\PYG{p}{,} \PYG{l+s+s1}{\PYGZsq{}}\PYG{l+s+s1}{xbet}\PYG{l+s+s1}{\PYGZsq{}}\PYG{p}{:} \PYG{l+m+mf}{0.104}\PYG{p}{,} \PYG{l+s+s1}{\PYGZsq{}}\PYG{l+s+s1}{xgam}\PYG{l+s+s1}{\PYGZsq{}}\PYG{p}{:} \PYG{o}{\PYGZhy{}}\PYG{l+m+mf}{0.273}\PYG{p}{,} \PYG{l+s+s1}{\PYGZsq{}}\PYG{l+s+s1}{xpixelsize}\PYG{l+s+s1}{\PYGZsq{}}\PYG{p}{:} \PYG{l+m+mf}{0.08057}\PYG{p}{,} \PYG{l+s+s1}{\PYGZsq{}}\PYG{l+s+s1}{ypixelsize}\PYG{l+s+s1}{\PYGZsq{}}\PYG{p}{:} \PYG{l+m+mf}{0.08057}\PYG{p}{,} \PYG{l+s+s1}{\PYGZsq{}}\PYG{l+s+s1}{CCDLabel}\PYG{l+s+s1}{\PYGZsq{}}\PYG{p}{:} \PYG{l+s+s1}{\PYGZsq{}}\PYG{l+s+s1}{sCMOS}\PYG{l+s+s1}{\PYGZsq{}}\PYG{p}{,} \PYG{l+s+s1}{\PYGZsq{}}\PYG{l+s+s1}{framedim}\PYG{l+s+s1}{\PYGZsq{}}\PYG{p}{:} \PYG{p}{[}\PYG{l+m+mf}{2048.0}\PYG{p}{,} \PYG{l+m+mf}{2048.0}\PYG{p}{]}\PYG{p}{,} \PYG{l+s+s1}{\PYGZsq{}}\PYG{l+s+s1}{detectordiameter}\PYG{l+s+s1}{\PYGZsq{}}\PYG{p}{:} \PYG{l+m+mf}{165.00736}\PYG{p}{,} \PYG{l+s+s1}{\PYGZsq{}}\PYG{l+s+s1}{kf\PYGZus{}direction}\PYG{l+s+s1}{\PYGZsq{}}\PYG{p}{:} \PYG{l+s+s1}{\PYGZsq{}}\PYG{l+s+s1}{Z\PYGZgt{}0}\PYG{l+s+s1}{\PYGZsq{}}\PYG{p}{,} \PYG{l+s+s1}{\PYGZsq{}}\PYG{l+s+s1}{pixelsize}\PYG{l+s+s1}{\PYGZsq{}}\PYG{p}{:} \PYG{l+m+mf}{0.08057}\PYG{p}{\PYGZcb{}}
\PYG{n}{CCD} \PYG{n}{Detector} \PYG{n}{parameters} \PYG{n}{read} \PYG{k+kn}{from} \PYG{n+nn}{.}\PYG{n+nn}{cor} \PYG{n}{file}
\PYG{n}{CCDcalibdict} \PYG{p}{\PYGZob{}}\PYG{l+s+s1}{\PYGZsq{}}\PYG{l+s+s1}{dd}\PYG{l+s+s1}{\PYGZsq{}}\PYG{p}{:} \PYG{l+m+mf}{69.179}\PYG{p}{,} \PYG{l+s+s1}{\PYGZsq{}}\PYG{l+s+s1}{xcen}\PYG{l+s+s1}{\PYGZsq{}}\PYG{p}{:} \PYG{l+m+mf}{1050.81}\PYG{p}{,} \PYG{l+s+s1}{\PYGZsq{}}\PYG{l+s+s1}{ycen}\PYG{l+s+s1}{\PYGZsq{}}\PYG{p}{:} \PYG{l+m+mf}{1115.59}\PYG{p}{,} \PYG{l+s+s1}{\PYGZsq{}}\PYG{l+s+s1}{xbet}\PYG{l+s+s1}{\PYGZsq{}}\PYG{p}{:} \PYG{l+m+mf}{0.104}\PYG{p}{,} \PYG{l+s+s1}{\PYGZsq{}}\PYG{l+s+s1}{xgam}\PYG{l+s+s1}{\PYGZsq{}}\PYG{p}{:} \PYG{o}{\PYGZhy{}}\PYG{l+m+mf}{0.273}\PYG{p}{,} \PYG{l+s+s1}{\PYGZsq{}}\PYG{l+s+s1}{xpixelsize}\PYG{l+s+s1}{\PYGZsq{}}\PYG{p}{:} \PYG{l+m+mf}{0.08057}\PYG{p}{,} \PYG{l+s+s1}{\PYGZsq{}}\PYG{l+s+s1}{ypixelsize}\PYG{l+s+s1}{\PYGZsq{}}\PYG{p}{:} \PYG{l+m+mf}{0.08057}\PYG{p}{,} \PYG{l+s+s1}{\PYGZsq{}}\PYG{l+s+s1}{CCDLabel}\PYG{l+s+s1}{\PYGZsq{}}\PYG{p}{:} \PYG{l+s+s1}{\PYGZsq{}}\PYG{l+s+s1}{sCMOS}\PYG{l+s+s1}{\PYGZsq{}}\PYG{p}{,} \PYG{l+s+s1}{\PYGZsq{}}\PYG{l+s+s1}{framedim}\PYG{l+s+s1}{\PYGZsq{}}\PYG{p}{:} \PYG{p}{[}\PYG{l+m+mf}{2048.0}\PYG{p}{,} \PYG{l+m+mf}{2048.0}\PYG{p}{]}\PYG{p}{,} \PYG{l+s+s1}{\PYGZsq{}}\PYG{l+s+s1}{detectordiameter}\PYG{l+s+s1}{\PYGZsq{}}\PYG{p}{:} \PYG{l+m+mf}{165.00736}\PYG{p}{,} \PYG{l+s+s1}{\PYGZsq{}}\PYG{l+s+s1}{kf\PYGZus{}direction}\PYG{l+s+s1}{\PYGZsq{}}\PYG{p}{:} \PYG{l+s+s1}{\PYGZsq{}}\PYG{l+s+s1}{Z\PYGZgt{}0}\PYG{l+s+s1}{\PYGZsq{}}\PYG{p}{,} \PYG{l+s+s1}{\PYGZsq{}}\PYG{l+s+s1}{pixelsize}\PYG{l+s+s1}{\PYGZsq{}}\PYG{p}{:} \PYG{l+m+mf}{0.08057}\PYG{p}{\PYGZcb{}}
\end{sphinxVerbatim}

\fvset{hllines={, ,}}%
\begin{sphinxVerbatim}[commandchars=\\\{\}]
\PYG{l+m+mi}{5}
\end{sphinxVerbatim}


\paragraph{but we can import a custom list of spots. For example, starting from spots a the previous .cor file}
\label{\detokenize{Indexation:but-we-can-import-a-custom-list-of-spots-for-example-starting-from-spots-a-the-previous-cor-file}}
\fvset{hllines={, ,}}%
\begin{sphinxVerbatim}[commandchars=\\\{\}]
\PYG{n}{Gespots} \PYG{o}{=} \PYG{n}{RWASCII}\PYG{o}{.}\PYG{n}{readfile\PYGZus{}cor}\PYG{p}{(}\PYG{n}{fullpathcorfile}\PYG{p}{)}\PYG{p}{[}\PYG{l+m+mi}{0}\PYG{p}{]}
\end{sphinxVerbatim}

\fvset{hllines={, ,}}%
\begin{sphinxVerbatim}[commandchars=\\\{\}]
\PYG{n}{CCDcalib} \PYG{o+ow}{in} \PYG{n}{readfile\PYGZus{}cor} \PYG{p}{\PYGZob{}}\PYG{l+s+s1}{\PYGZsq{}}\PYG{l+s+s1}{dd}\PYG{l+s+s1}{\PYGZsq{}}\PYG{p}{:} \PYG{l+m+mf}{69.179}\PYG{p}{,} \PYG{l+s+s1}{\PYGZsq{}}\PYG{l+s+s1}{xcen}\PYG{l+s+s1}{\PYGZsq{}}\PYG{p}{:} \PYG{l+m+mf}{1050.81}\PYG{p}{,} \PYG{l+s+s1}{\PYGZsq{}}\PYG{l+s+s1}{ycen}\PYG{l+s+s1}{\PYGZsq{}}\PYG{p}{:} \PYG{l+m+mf}{1115.59}\PYG{p}{,} \PYG{l+s+s1}{\PYGZsq{}}\PYG{l+s+s1}{xbet}\PYG{l+s+s1}{\PYGZsq{}}\PYG{p}{:} \PYG{l+m+mf}{0.104}\PYG{p}{,} \PYG{l+s+s1}{\PYGZsq{}}\PYG{l+s+s1}{xgam}\PYG{l+s+s1}{\PYGZsq{}}\PYG{p}{:} \PYG{o}{\PYGZhy{}}\PYG{l+m+mf}{0.273}\PYG{p}{,} \PYG{l+s+s1}{\PYGZsq{}}\PYG{l+s+s1}{xpixelsize}\PYG{l+s+s1}{\PYGZsq{}}\PYG{p}{:} \PYG{l+m+mf}{0.08057}\PYG{p}{,} \PYG{l+s+s1}{\PYGZsq{}}\PYG{l+s+s1}{ypixelsize}\PYG{l+s+s1}{\PYGZsq{}}\PYG{p}{:} \PYG{l+m+mf}{0.08057}\PYG{p}{,} \PYG{l+s+s1}{\PYGZsq{}}\PYG{l+s+s1}{CCDLabel}\PYG{l+s+s1}{\PYGZsq{}}\PYG{p}{:} \PYG{l+s+s1}{\PYGZsq{}}\PYG{l+s+s1}{sCMOS}\PYG{l+s+s1}{\PYGZsq{}}\PYG{p}{,} \PYG{l+s+s1}{\PYGZsq{}}\PYG{l+s+s1}{framedim}\PYG{l+s+s1}{\PYGZsq{}}\PYG{p}{:} \PYG{p}{[}\PYG{l+m+mf}{2048.0}\PYG{p}{,} \PYG{l+m+mf}{2048.0}\PYG{p}{]}\PYG{p}{,} \PYG{l+s+s1}{\PYGZsq{}}\PYG{l+s+s1}{detectordiameter}\PYG{l+s+s1}{\PYGZsq{}}\PYG{p}{:} \PYG{l+m+mf}{165.00736}\PYG{p}{,} \PYG{l+s+s1}{\PYGZsq{}}\PYG{l+s+s1}{kf\PYGZus{}direction}\PYG{l+s+s1}{\PYGZsq{}}\PYG{p}{:} \PYG{l+s+s1}{\PYGZsq{}}\PYG{l+s+s1}{Z\PYGZgt{}0}\PYG{l+s+s1}{\PYGZsq{}}\PYG{p}{,} \PYG{l+s+s1}{\PYGZsq{}}\PYG{l+s+s1}{pixelsize}\PYG{l+s+s1}{\PYGZsq{}}\PYG{p}{:} \PYG{l+m+mf}{0.08057}\PYG{p}{\PYGZcb{}}
\PYG{n}{CCD} \PYG{n}{Detector} \PYG{n}{parameters} \PYG{n}{read} \PYG{k+kn}{from} \PYG{n+nn}{.}\PYG{n+nn}{cor} \PYG{n}{file}
\end{sphinxVerbatim}

\fvset{hllines={, ,}}%
\begin{sphinxVerbatim}[commandchars=\\\{\}]
\PYG{c+c1}{\PYGZsh{} 2theta chi X, Y Intensity of the first 7 spots}
\PYG{n}{Gespots}\PYG{p}{[}\PYG{p}{:}\PYG{l+m+mi}{7}\PYG{p}{,}\PYG{p}{:}\PYG{l+m+mi}{5}\PYG{p}{]}
\end{sphinxVerbatim}

\fvset{hllines={, ,}}%
\begin{sphinxVerbatim}[commandchars=\\\{\}]
\PYG{n}{array}\PYG{p}{(}\PYG{p}{[}\PYG{p}{[} \PYG{l+m+mf}{7.82158210e+01}\PYG{p}{,}  \PYG{l+m+mf}{1.63815300e+00}\PYG{p}{,}  \PYG{l+m+mf}{1.02711000e+03}\PYG{p}{,}
         \PYG{l+m+mf}{1.29328000e+03}\PYG{p}{,}  \PYG{l+m+mf}{7.09312700e+04}\PYG{p}{]}\PYG{p}{,}
       \PYG{p}{[} \PYG{l+m+mf}{6.43297670e+01}\PYG{p}{,} \PYG{o}{\PYGZhy{}}\PYG{l+m+mf}{2.08241550e+01}\PYG{p}{,}  \PYG{l+m+mf}{1.37917000e+03}\PYG{p}{,}
         \PYG{l+m+mf}{1.55358000e+03}\PYG{p}{,}  \PYG{l+m+mf}{5.19338400e+04}\PYG{p}{]}\PYG{p}{,}
       \PYG{p}{[} \PYG{l+m+mf}{6.86804510e+01}\PYG{p}{,} \PYG{o}{\PYGZhy{}}\PYG{l+m+mf}{1.53581220e+01}\PYG{p}{,}  \PYG{l+m+mf}{1.28811000e+03}\PYG{p}{,}
         \PYG{l+m+mf}{1.46016000e+03}\PYG{p}{,}  \PYG{l+m+mf}{2.27950700e+04}\PYG{p}{]}\PYG{p}{,}
       \PYG{p}{[} \PYG{l+m+mf}{1.05614980e+02}\PYG{p}{,}  \PYG{l+m+mf}{8.17618700e+00}\PYG{p}{,}  \PYG{l+m+mf}{9.26220000e+02}\PYG{p}{,}
         \PYG{l+m+mf}{8.72060000e+02}\PYG{p}{,}  \PYG{l+m+mf}{1.94896900e+04}\PYG{p}{]}\PYG{p}{,}
       \PYG{p}{[} \PYG{l+m+mf}{1.03859791e+02}\PYG{p}{,}  \PYG{l+m+mf}{2.78665660e+01}\PYG{p}{,}  \PYG{l+m+mf}{5.95460000e+02}\PYG{p}{,}
         \PYG{l+m+mf}{8.76440000e+02}\PYG{p}{,}  \PYG{l+m+mf}{1.90587900e+04}\PYG{p}{]}\PYG{p}{,}
       \PYG{p}{[} \PYG{l+m+mf}{1.20595610e+02}\PYG{p}{,} \PYG{o}{\PYGZhy{}}\PYG{l+m+mf}{8.92066000e+00}\PYG{p}{,}  \PYG{l+m+mf}{1.18327000e+03}\PYG{p}{,}
         \PYG{l+m+mf}{5.98920000e+02}\PYG{p}{,}  \PYG{l+m+mf}{1.71828800e+04}\PYG{p}{]}\PYG{p}{,}
       \PYG{p}{[} \PYG{l+m+mf}{6.03594580e+01}\PYG{p}{,}  \PYG{l+m+mf}{2.64831910e+01}\PYG{p}{,}  \PYG{l+m+mf}{6.26120000e+02}\PYG{p}{,}
         \PYG{l+m+mf}{1.66128000e+03}\PYG{p}{,}  \PYG{l+m+mf}{1.58253900e+04}\PYG{p}{]}\PYG{p}{]}\PYG{p}{)}
\end{sphinxVerbatim}

\fvset{hllines={, ,}}%
\begin{sphinxVerbatim}[commandchars=\\\{\}]
\PYG{n}{tth}\PYG{p}{,}\PYG{n}{chi}\PYG{p}{,}\PYG{n}{X}\PYG{p}{,}\PYG{n}{Y}\PYG{p}{,}\PYG{n}{I}\PYG{o}{=}\PYG{n}{Gespots}\PYG{p}{[}\PYG{p}{:}\PYG{p}{,}\PYG{p}{:}\PYG{l+m+mi}{5}\PYG{p}{]}\PYG{o}{.}\PYG{n}{T}
\PYG{n}{exp\PYGZus{}data\PYGZus{}all}\PYG{o}{=}\PYG{n}{np}\PYG{o}{.}\PYG{n}{array}\PYG{p}{(}\PYG{p}{[}\PYG{n}{tth}\PYG{p}{,}\PYG{n}{chi}\PYG{p}{,}\PYG{n}{I}\PYG{p}{,}\PYG{n}{X}\PYG{p}{,}\PYG{n}{Y}\PYG{p}{]}\PYG{p}{)}
\PYG{n}{exp\PYGZus{}data\PYGZus{}all}\PYG{o}{.}\PYG{n}{shape}
\end{sphinxVerbatim}

\fvset{hllines={, ,}}%
\begin{sphinxVerbatim}[commandchars=\\\{\}]
\PYG{p}{(}\PYG{l+m+mi}{5}\PYG{p}{,} \PYG{l+m+mi}{83}\PYG{p}{)}
\end{sphinxVerbatim}

\fvset{hllines={, ,}}%
\begin{sphinxVerbatim}[commandchars=\\\{\}]
\PYG{c+c1}{\PYGZsh{}select some exp spots from absolute index   (6,0,2,30,9,8,20,10,5,1,7,14)}
\PYG{n}{tth\PYGZus{}e}\PYG{p}{,}\PYG{n}{chi\PYGZus{}e}\PYG{p}{,}\PYG{n}{X\PYGZus{}e}\PYG{p}{,}\PYG{n}{Y\PYGZus{}e}\PYG{p}{,}\PYG{n}{I\PYGZus{}e} \PYG{o}{=} \PYG{p}{(}\PYG{n}{np}\PYG{o}{.}\PYG{n}{take}\PYG{p}{(}\PYG{n}{Gespots}\PYG{p}{[}\PYG{p}{:}\PYG{p}{,}\PYG{p}{:}\PYG{l+m+mi}{5}\PYG{p}{]}\PYG{p}{,}\PYG{p}{(}\PYG{l+m+mi}{6}\PYG{p}{,}\PYG{l+m+mi}{0}\PYG{p}{,}\PYG{l+m+mi}{2}\PYG{p}{,}\PYG{l+m+mi}{30}\PYG{p}{,}\PYG{l+m+mi}{9}\PYG{p}{,}\PYG{l+m+mi}{8}\PYG{p}{,}\PYG{l+m+mi}{20}\PYG{p}{,}\PYG{l+m+mi}{10}\PYG{p}{,}\PYG{l+m+mi}{5}\PYG{p}{,}\PYG{l+m+mi}{1}\PYG{p}{,}\PYG{l+m+mi}{7}\PYG{p}{,}\PYG{l+m+mi}{14}\PYG{p}{)}\PYG{p}{,}\PYG{n}{axis}\PYG{o}{=}\PYG{l+m+mi}{0}\PYG{p}{)}\PYG{p}{)}\PYG{o}{.}\PYG{n}{T}
\PYG{n}{exp\PYGZus{}data}\PYG{o}{=}\PYG{n}{np}\PYG{o}{.}\PYG{n}{array}\PYG{p}{(}\PYG{p}{[}\PYG{n}{tth\PYGZus{}e}\PYG{p}{,}\PYG{n}{chi\PYGZus{}e}\PYG{p}{,}\PYG{n}{I\PYGZus{}e}\PYG{p}{,}\PYG{n}{X\PYGZus{}e}\PYG{p}{,}\PYG{n}{Y\PYGZus{}e}\PYG{p}{]}\PYG{p}{)}
\end{sphinxVerbatim}


\subsubsection{spots data must be imported as an array of 5 elements: 2theta, chi, Intensity, pixelX, pixelY}
\label{\detokenize{Indexation:spots-data-must-be-imported-as-an-array-of-5-elements-2theta-chi-intensity-pixelx-pixely}}
\fvset{hllines={, ,}}%
\begin{sphinxVerbatim}[commandchars=\\\{\}]
\PYG{n}{DataSet}\PYG{o}{.}\PYG{n}{importdata}\PYG{p}{(}\PYG{n}{exp\PYGZus{}data}\PYG{p}{)}
\PYG{n}{DataSet}\PYG{o}{.}\PYG{n}{detectorparameters} \PYG{o}{=} \PYG{n}{calibration\PYGZus{}parameters}
\PYG{n}{DataSet}\PYG{o}{.}\PYG{n}{nbspots} \PYG{o}{=} \PYG{n+nb}{len}\PYG{p}{(}\PYG{n}{exp\PYGZus{}data}\PYG{p}{[}\PYG{l+m+mi}{0}\PYG{p}{]}\PYG{p}{)}
\PYG{n}{DataSet}\PYG{o}{.}\PYG{n}{filename} \PYG{o}{=} \PYG{l+s+s1}{\PYGZsq{}}\PYG{l+s+s1}{short\PYGZus{}}\PYG{l+s+s1}{\PYGZsq{}}\PYG{o}{+}\PYG{n}{corfilename}
\PYG{c+c1}{\PYGZsh{}DataSet.setSelectedExpSpotsData(0)}
\PYG{n}{DataSet}\PYG{o}{.}\PYG{n}{getSelectedExpSpotsData}\PYG{p}{(}\PYG{l+m+mi}{0}\PYG{p}{)}
\end{sphinxVerbatim}

\fvset{hllines={, ,}}%
\begin{sphinxVerbatim}[commandchars=\\\{\}]
\PYG{p}{(}\PYG{n}{array}\PYG{p}{(}\PYG{p}{[}\PYG{p}{[} \PYG{l+m+mf}{6.03594580e+01}\PYG{p}{,}  \PYG{l+m+mf}{7.82158210e+01}\PYG{p}{,}  \PYG{l+m+mf}{6.86804510e+01}\PYG{p}{,}
          \PYG{l+m+mf}{1.08452917e+02}\PYG{p}{,}  \PYG{l+m+mf}{8.33495350e+01}\PYG{p}{,}  \PYG{l+m+mf}{8.20720760e+01}\PYG{p}{,}
          \PYG{l+m+mf}{8.17712570e+01}\PYG{p}{,}  \PYG{l+m+mf}{9.17982210e+01}\PYG{p}{,}  \PYG{l+m+mf}{1.20595610e+02}\PYG{p}{,}
          \PYG{l+m+mf}{6.43297670e+01}\PYG{p}{,}  \PYG{l+m+mf}{5.62698530e+01}\PYG{p}{,}  \PYG{l+m+mf}{1.14942090e+02}\PYG{p}{]}\PYG{p}{,}
        \PYG{p}{[} \PYG{l+m+mf}{2.64831910e+01}\PYG{p}{,}  \PYG{l+m+mf}{1.63815300e+00}\PYG{p}{,} \PYG{o}{\PYGZhy{}}\PYG{l+m+mf}{1.53581220e+01}\PYG{p}{,}
          \PYG{l+m+mf}{3.77494610e+01}\PYG{p}{,} \PYG{o}{\PYGZhy{}}\PYG{l+m+mf}{2.74580610e+01}\PYG{p}{,} \PYG{o}{\PYGZhy{}}\PYG{l+m+mf}{3.58924300e+01}\PYG{p}{,}
          \PYG{l+m+mf}{3.03824700e+01}\PYG{p}{,} \PYG{o}{\PYGZhy{}}\PYG{l+m+mf}{8.30994100e+00}\PYG{p}{,} \PYG{o}{\PYGZhy{}}\PYG{l+m+mf}{8.92066000e+00}\PYG{p}{,}
         \PYG{o}{\PYGZhy{}}\PYG{l+m+mf}{2.08241550e+01}\PYG{p}{,}  \PYG{l+m+mf}{1.29671530e+01}\PYG{p}{,}  \PYG{l+m+mf}{1.15295700e+01}\PYG{p}{]}\PYG{p}{,}
        \PYG{p}{[} \PYG{l+m+mf}{1.58253900e+04}\PYG{p}{,}  \PYG{l+m+mf}{7.09312700e+04}\PYG{p}{,}  \PYG{l+m+mf}{2.27950700e+04}\PYG{p}{,}
          \PYG{l+m+mf}{4.40061000e+03}\PYG{p}{,}  \PYG{l+m+mf}{1.31459900e+04}\PYG{p}{,}  \PYG{l+m+mf}{1.33188100e+04}\PYG{p}{,}
          \PYG{l+m+mf}{6.13787000e+03}\PYG{p}{,}  \PYG{l+m+mf}{1.17999300e+04}\PYG{p}{,}  \PYG{l+m+mf}{1.71828800e+04}\PYG{p}{,}
          \PYG{l+m+mf}{5.19338400e+04}\PYG{p}{,}  \PYG{l+m+mf}{1.54862000e+04}\PYG{p}{,}  \PYG{l+m+mf}{1.00105200e+04}\PYG{p}{]}\PYG{p}{]}\PYG{p}{)}\PYG{p}{,}
 \PYG{n}{array}\PYG{p}{(}\PYG{p}{[} \PYG{l+m+mi}{0}\PYG{p}{,}  \PYG{l+m+mi}{1}\PYG{p}{,}  \PYG{l+m+mi}{2}\PYG{p}{,}  \PYG{l+m+mi}{3}\PYG{p}{,}  \PYG{l+m+mi}{4}\PYG{p}{,}  \PYG{l+m+mi}{5}\PYG{p}{,}  \PYG{l+m+mi}{6}\PYG{p}{,}  \PYG{l+m+mi}{7}\PYG{p}{,}  \PYG{l+m+mi}{8}\PYG{p}{,}  \PYG{l+m+mi}{9}\PYG{p}{,} \PYG{l+m+mi}{10}\PYG{p}{,} \PYG{l+m+mi}{11}\PYG{p}{]}\PYG{p}{)}\PYG{p}{)}
\end{sphinxVerbatim}


\subsection{core function to index a set of spots}
\label{\detokenize{Indexation:core-function-to-index-a-set-of-spots}}
by defaut DataSet.getUnIndexedSpotsallData() is called

if use\_file = 0, then current non indexed exp. spots will be considered
for indexation

if use\_file = 1, reimport data from file and reset also spots
properties dictionary (i.e. with status unindexed)

\fvset{hllines={, ,}}%
\begin{sphinxVerbatim}[commandchars=\\\{\}]
\PYG{n}{DataSet}\PYG{o}{.}\PYG{n}{IndexSpotsSet}\PYG{p}{(}\PYG{n}{fullpathcorfile}\PYG{p}{,} \PYG{n}{key\PYGZus{}material}\PYG{p}{,} \PYG{n}{emin}\PYG{p}{,} \PYG{n}{emax}\PYG{p}{,} \PYG{n}{dict\PYGZus{}loop}\PYG{p}{,} \PYG{k+kc}{None}\PYG{p}{,}
                         \PYG{n}{use\PYGZus{}file}\PYG{o}{=}\PYG{l+m+mi}{0}\PYG{p}{,} \PYG{c+c1}{\PYGZsh{} if 1, reimport data from file and reset also spots properties dictionary}
                         \PYG{n}{IMM}\PYG{o}{=}\PYG{k+kc}{False}\PYG{p}{,}\PYG{n}{LUT}\PYG{o}{=}\PYG{k+kc}{None}\PYG{p}{,}\PYG{n}{n\PYGZus{}LUT}\PYG{o}{=}\PYG{n}{dict\PYGZus{}loop}\PYG{p}{[}\PYG{l+s+s1}{\PYGZsq{}}\PYG{l+s+s1}{nlutmax}\PYG{l+s+s1}{\PYGZsq{}}\PYG{p}{]}\PYG{p}{,}\PYG{n}{angletol\PYGZus{}list}\PYG{o}{=}\PYG{n}{dict\PYGZus{}loop}\PYG{p}{[}\PYG{l+s+s1}{\PYGZsq{}}\PYG{l+s+s1}{list matching tol angles}\PYG{l+s+s1}{\PYGZsq{}}\PYG{p}{]}\PYG{p}{,}
                        \PYG{n}{nbGrainstoFind}\PYG{o}{=}\PYG{l+m+mi}{1}\PYG{p}{,}
                      \PYG{n}{starting\PYGZus{}grainindex}\PYG{o}{=}\PYG{l+m+mi}{0}\PYG{p}{,}
                      \PYG{n}{MatchingRate\PYGZus{}List}\PYG{o}{=}\PYG{p}{[}\PYG{l+m+mi}{1}\PYG{p}{,} \PYG{l+m+mi}{1}\PYG{p}{,} \PYG{l+m+mi}{1}\PYG{p}{,}\PYG{l+m+mi}{1}\PYG{p}{,}\PYG{l+m+mi}{1}\PYG{p}{,}\PYG{l+m+mi}{1}\PYG{p}{,}\PYG{l+m+mi}{1}\PYG{p}{,}\PYG{l+m+mi}{1}\PYG{p}{]}\PYG{p}{,}
                        \PYG{n}{verbose}\PYG{o}{=}\PYG{l+m+mi}{0}\PYG{p}{,} \PYG{n}{previousResults}\PYG{o}{=}\PYG{k+kc}{None}\PYG{p}{,}
                        \PYG{n}{corfilename}\PYG{o}{=}\PYG{n}{corfilename}\PYG{p}{)}
\end{sphinxVerbatim}
\begin{sphinxalltt}
self.pixelsize in IndexSpotsSet 0.08057
ResolutionAngstromLUT in IndexSpotsSet False

 Remaining nb of spots to index for grain \#0 : 12


 \sphinxstylestrong{**}
start to index grain \#0 of Material: Ge

\sphinxstylestrong{**}

providing new set of matrices Using Angles LUT template matching
nbspots 12
NBMAXPROBED 6
nbspots 12
set\_central\_spots\_hkl None
Computing LUT from material data
Compute LUT for indexing Ge spots in LauePattern
Build angles LUT with latticeparameters
{[} 5.657499999999999  5.657499999999999  5.657499999999999
 90.                90.                90.               {]}
and n=3
MaxRadiusHKL False
cubicSymmetry True
Central set of exp. spotDistances from spot\_index\_central\_list probed
self.absolute\_index {[} 0  1  2  3  4  5  6  7  8  9 10 11{]}
spot\_index\_central\_list {[}0{]}
{[}0{]}
LUT is not None when entering getOrientMatrices()
set\_central\_spots\_hkl None
set\_central\_spots\_hkl\_list {[}None{]}
cubicSymmetry True
LUT\_tol\_angle 0.5
\sphinxstyleemphasis{---***}------------------------------------------------*
Calculating all possible matrices from exp spot \#0 and the 5 other(s)
hkl in getOrientMatrices None \textless{}class 'NoneType'\textgreater{}
using LUTcubic
LUTcubic is None for k\_centspot\_index 0 in getOrientMatrices()
hkl1 in matrices\_from\_onespot\_hkl() {[}{[}1 0 0{]}
 {[}1 1 0{]}
 {[}1 1 1{]}
 {[}2 1 0{]}
 {[}2 1 1{]}
 {[}2 2 1{]}
 {[}3 1 0{]}
 {[}3 1 1{]}
 {[}3 2 1{]}
 {[}3 2 2{]}
 {[}3 3 1{]}
 {[}3 3 2{]}{]}
Computing hkl2 list for specific or cubic LUT in matrices\_from\_onespot\_hkl()
Calculating LUT in PlanePairs\_from2sets()
Looking up planes pairs in LUT from exp. spots (0, 1):
Looking up planes pairs in LUT from exp. spots (0, 2):
Looking up planes pairs in LUT from exp. spots (0, 3):
Looking up planes pairs in LUT from exp. spots (0, 4):
Looking up planes pairs in LUT from exp. spots (0, 5):
calculating matching rates of solutions for exp. spots {[}0, 1{]}
calculating matching rates of solutions for exp. spots {[}0, 2{]}
calculating matching rates of solutions for exp. spots {[}0, 3{]}
calculating matching rates of solutions for exp. spots {[}0, 4{]}


return best matrix and matching scores for the one central\_spot

-----------------------------------------
results:
matrix:                                         matching results
{[}-0.211852735694566  0.092255643652867 -0.972937466948891{]}        res: {[}20.0, 162.0{]} 0.014 12.35
{[}-0.775856536468367  0.58951816141498   0.22475536073965 {]}        spot indices {[}0 1{]}
{[} 0.594300563948835  0.802473664571131 -0.053318452339475{]}        planes {[}{[}1.0, 3.0, 2.0{]}, {[}1.0, 1.0, 1.0{]}{]}

Number of matrices found (nb\_sol):  1
set\_central\_spots\_hkl in FindOrientMatrices None

-----------------------------------------
results:
matrix:                                         matching results
{[}-0.211852735694566  0.092255643652867 -0.972937466948891{]}        res: {[}20.0, 162.0{]} 0.014 12.35
{[}-0.775856536468367  0.58951816141498   0.22475536073965 {]}        spot indices {[}0 1{]}
{[} 0.594300563948835  0.802473664571131 -0.053318452339475{]}        planes {[}{[}1.0, 3.0, 2.0{]}, {[}1.0, 1.0, 1.0{]}{]}

Nb of potential orientation matrice(s) UB found: 1
{[}array({[}{[}-0.211852735694566,  0.092255643652867, -0.972937466948891{]},
       {[}-0.775856536468367,  0.58951816141498 ,  0.22475536073965 {]},
       {[} 0.594300563948835,  0.802473664571131, -0.053318452339475{]}{]}){]}
Nb of potential UBs  1

Working with a new stack of orientation matrices
MATCHINGRATE\_THRESHOLD\_IAL= 100.0
has not been reached! All potential solutions have been calculated
taking the first one only.
bestUB object \textless{}LaueTools.indexingSpotsSet.OrientMatrix object at 0x7fb8ec7ddc50\textgreater{}


---------------refining grain orientation and strain \#0-----------------


 refining grain \#0 step -----0

bestUB \textless{}LaueTools.indexingSpotsSet.OrientMatrix object at 0x7fb8ec7ddc50\textgreater{}
True it is an OrientMatrix object
Orientation \textless{}LaueTools.indexingSpotsSet.OrientMatrix object at 0x7fb8ec7ddc50\textgreater{}
matrix {[}{[}-0.211852735694566  0.092255643652867 -0.972937466948891{]}
 {[}-0.775856536468367  0.58951816141498   0.22475536073965 {]}
 {[} 0.594300563948835  0.802473664571131 -0.053318452339475{]}{]}
\sphinxstylestrong{*nb of selected spots in AssignHKL***} 12
UBOrientMatrix {[}{[}-0.211852735694566  0.092255643652867 -0.972937466948891{]}
 {[}-0.775856536468367  0.58951816141498   0.22475536073965 {]}
 {[} 0.594300563948835  0.802473664571131 -0.053318452339475{]}{]}
For angular tolerance 0.50 deg
Nb of pairs found / nb total of expected spots: 12/176
Matching Rate : 6.82
Nb missing reflections: 164

grain \#0 : 12 links to simulated spots have been found
\sphinxstylestrong{*********mean pixel deviation    0.560750282710606     ******}
Initial residues {[}0.053680370172309 0.013739858524874 0.921977335411896 0.403270956234836
 0.919825854310187 0.785969463406447 0.565019172757509 1.127873079813964
 0.363514793614926 0.412635402450867 0.711008521607465 0.450488584221994{]}
---------------------------------------------------



\sphinxstylestrong{***********************}
first error with initial values of: {[}'b/a', 'c/a', 'a12', 'a13', 'a23', 'theta1', 'theta2', 'theta3'{]}

\sphinxstylestrong{***********************}

\sphinxstylestrong{*********mean pixel deviation    0.560750282710606     ******}


\sphinxstylestrong{***********************}
Fitting parameters:   {[}'b/a', 'c/a', 'a12', 'a13', 'a23', 'theta1', 'theta2', 'theta3'{]}

\sphinxstylestrong{***********************}

With initial values {[}1. 1. 0. 0. 0. 0. 0. 0.{]}
code results 1
nb iterations 1767
mesg Both actual and predicted relative reductions in the sum of squares
  are at most 0.000000
strain\_sol {[} 1.001128981010799e+00  9.993806401299155e-01  8.449040381845989e-04
 -8.486913595751131e-04  3.520626401662759e-04 -2.714612741167435e-02
  3.054889720130747e-02  5.311773668297186e-02{]}


 \sphinxstylestrong{**********}  End of Fitting  -  Final errors  \sphinxstylestrong{**************}


\sphinxstylestrong{*********mean pixel deviation    0.3158807195732847     ******}
devstrain, lattice\_parameter\_direct\_strain {[}{[} 0.000159671589578 -0.000413843247724  0.00039782267455 {]}
 {[}-0.000413843247724 -0.00095830941256  -0.000151781897557{]}
 {[} 0.00039782267455  -0.000151781897557  0.000798637822982{]}{]} {[} 5.657424223234738  5.651101185787196  5.661104877237695
 90.01741959362538  89.9544419036642   90.04747664598754 {]}
For comparison: a,b,c are rescaled with respect to the reference value of a = 5.657500 Angstroms
lattice\_parameter\_direct\_strain {[} 5.657499999999999  5.651176877860324  5.661180703302433
 90.01741959362538  89.9544419036642   90.04747664598754 {]}
devstrain1, lattice\_parameter\_direct\_strain1 {[}{[} 0.000159671589578 -0.000413843247724  0.00039782267455 {]}
 {[}-0.000413843247724 -0.00095830941256  -0.000151781897557{]}
 {[} 0.00039782267455  -0.000151781897557  0.000798637822982{]}{]} {[} 5.657499999999999  5.651176877860324  5.661180703302433
 90.01741959362538  89.9544419036642   90.04747664598754 {]}
new UBs matrix in q= UBs G (s for strain)
strain\_direct {[}{[}-1.339403716515974e-05 -4.138432477238585e-04  3.978226745502020e-04{]}
 {[}-4.138432477238585e-04 -1.131375039302607e-03 -1.517818975573709e-04{]}
 {[} 3.978226745502020e-04 -1.517818975573709e-04  6.255721962393768e-04{]}{]}
deviatoric strain {[}{[} 0.000159671589578 -0.000413843247724  0.00039782267455 {]}
 {[}-0.000413843247724 -0.00095830941256  -0.000151781897557{]}
 {[} 0.00039782267455  -0.000151781897557  0.000798637822982{]}{]}
new UBs matrix in q= UBs G (s for strain)
strain\_direct {[}{[}-1.339403716515974e-05 -4.138432477238585e-04  3.978226745502020e-04{]}
 {[}-4.138432477238585e-04 -1.131375039302607e-03 -1.517818975573709e-04{]}
 {[} 3.978226745502020e-04 -1.517818975573709e-04  6.255721962393768e-04{]}{]}
deviatoric strain {[}{[} 0.000159671589578 -0.000413843247724  0.00039782267455 {]}
 {[}-0.000413843247724 -0.00095830941256  -0.000151781897557{]}
 {[} 0.00039782267455  -0.000151781897557  0.000798637822982{]}{]}
For comparison: a,b,c are rescaled with respect to the reference value of a = 5.657500 Angstroms
lattice\_parameter\_direct\_strain {[} 5.657499999999999  5.651176877860324  5.661180703302433
 90.01741959362538  89.9544419036642   90.04747664598754 {]}
final lattice\_parameters {[} 5.657499999999999  5.651176877860324  5.661180703302433
 90.01741959362538  89.9544419036642   90.04747664598754 {]}
UB and strain refinement completed
True it is an OrientMatrix object
Orientation \textless{}LaueTools.indexingSpotsSet.OrientMatrix object at 0x7fb8c98baa20\textgreater{}
matrix {[}{[}-0.211415207301911  0.091252946262469 -0.97230572627369 {]}
 {[}-0.77573594328912   0.590041596352251  0.224552051799525{]}
 {[} 0.594613709497768  0.803610914885283 -0.054088075690982{]}{]}
\sphinxstylestrong{*nb of selected spots in AssignHKL***} 12
UBOrientMatrix {[}{[}-0.211415207301911  0.091252946262469 -0.97230572627369 {]}
 {[}-0.77573594328912   0.590041596352251  0.224552051799525{]}
 {[} 0.594613709497768  0.803610914885283 -0.054088075690982{]}{]}
For angular tolerance 0.50 deg
Nb of pairs found / nb total of expected spots: 12/177
Matching Rate : 6.78
Nb missing reflections: 165

grain \#0 : 12 links to simulated spots have been found
GoodRefinement condition is  True
nb\_updates 12 compared to 6


 refining grain \#0 step -----1

bestUB \textless{}LaueTools.indexingSpotsSet.OrientMatrix object at 0x7fb8ec7ddc50\textgreater{}
True it is an OrientMatrix object
Orientation \textless{}LaueTools.indexingSpotsSet.OrientMatrix object at 0x7fb8ec7ddc50\textgreater{}
matrix {[}{[}-0.211852735694566  0.092255643652867 -0.972937466948891{]}
 {[}-0.775856536468367  0.58951816141498   0.22475536073965 {]}
 {[} 0.594300563948835  0.802473664571131 -0.053318452339475{]}{]}
\sphinxstylestrong{*nb of selected spots in AssignHKL***} 12
UBOrientMatrix {[}{[}-0.211852735694566  0.092255643652867 -0.972937466948891{]}
 {[}-0.775856536468367  0.58951816141498   0.22475536073965 {]}
 {[} 0.594300563948835  0.802473664571131 -0.053318452339475{]}{]}
For angular tolerance 0.50 deg
Nb of pairs found / nb total of expected spots: 12/176
Matching Rate : 6.82
Nb missing reflections: 164

grain \#0 : 12 links to simulated spots have been found
\sphinxstylestrong{*********mean pixel deviation    0.560750282710606     ******}
Initial residues {[}0.053680370172309 0.013739858524874 0.921977335411896 0.403270956234836
 0.919825854310187 0.785969463406447 0.565019172757509 1.127873079813964
 0.363514793614926 0.412635402450867 0.711008521607465 0.450488584221994{]}
---------------------------------------------------



\sphinxstylestrong{***********************}
first error with initial values of: {[}'b/a', 'c/a', 'a12', 'a13', 'a23', 'theta1', 'theta2', 'theta3'{]}

\sphinxstylestrong{***********************}

\sphinxstylestrong{*********mean pixel deviation    0.560750282710606     ******}


\sphinxstylestrong{***********************}
Fitting parameters:   {[}'b/a', 'c/a', 'a12', 'a13', 'a23', 'theta1', 'theta2', 'theta3'{]}

\sphinxstylestrong{***********************}

With initial values {[}1. 1. 0. 0. 0. 0. 0. 0.{]}
code results 1
nb iterations 1767
mesg Both actual and predicted relative reductions in the sum of squares
  are at most 0.000000
strain\_sol {[} 1.001128981010799e+00  9.993806401299155e-01  8.449040381845989e-04
 -8.486913595751131e-04  3.520626401662759e-04 -2.714612741167435e-02
  3.054889720130747e-02  5.311773668297186e-02{]}


 \sphinxstylestrong{**********}  End of Fitting  -  Final errors  \sphinxstylestrong{**************}


\sphinxstylestrong{*********mean pixel deviation    0.3158807195732847     ******}
devstrain, lattice\_parameter\_direct\_strain {[}{[} 0.000159671589578 -0.000413843247724  0.00039782267455 {]}
 {[}-0.000413843247724 -0.00095830941256  -0.000151781897557{]}
 {[} 0.00039782267455  -0.000151781897557  0.000798637822982{]}{]} {[} 5.657424223234738  5.651101185787196  5.661104877237695
 90.01741959362538  89.9544419036642   90.04747664598754 {]}
For comparison: a,b,c are rescaled with respect to the reference value of a = 5.657500 Angstroms
lattice\_parameter\_direct\_strain {[} 5.657499999999999  5.651176877860324  5.661180703302433
 90.01741959362538  89.9544419036642   90.04747664598754 {]}
devstrain1, lattice\_parameter\_direct\_strain1 {[}{[} 0.000159671589578 -0.000413843247724  0.00039782267455 {]}
 {[}-0.000413843247724 -0.00095830941256  -0.000151781897557{]}
 {[} 0.00039782267455  -0.000151781897557  0.000798637822982{]}{]} {[} 5.657499999999999  5.651176877860324  5.661180703302433
 90.01741959362538  89.9544419036642   90.04747664598754 {]}
new UBs matrix in q= UBs G (s for strain)
strain\_direct {[}{[}-1.339403716515974e-05 -4.138432477238585e-04  3.978226745502020e-04{]}
 {[}-4.138432477238585e-04 -1.131375039302607e-03 -1.517818975573709e-04{]}
 {[} 3.978226745502020e-04 -1.517818975573709e-04  6.255721962393768e-04{]}{]}
deviatoric strain {[}{[} 0.000159671589578 -0.000413843247724  0.00039782267455 {]}
 {[}-0.000413843247724 -0.00095830941256  -0.000151781897557{]}
 {[} 0.00039782267455  -0.000151781897557  0.000798637822982{]}{]}
new UBs matrix in q= UBs G (s for strain)
strain\_direct {[}{[}-1.339403716515974e-05 -4.138432477238585e-04  3.978226745502020e-04{]}
 {[}-4.138432477238585e-04 -1.131375039302607e-03 -1.517818975573709e-04{]}
 {[} 3.978226745502020e-04 -1.517818975573709e-04  6.255721962393768e-04{]}{]}
deviatoric strain {[}{[} 0.000159671589578 -0.000413843247724  0.00039782267455 {]}
 {[}-0.000413843247724 -0.00095830941256  -0.000151781897557{]}
 {[} 0.00039782267455  -0.000151781897557  0.000798637822982{]}{]}
For comparison: a,b,c are rescaled with respect to the reference value of a = 5.657500 Angstroms
lattice\_parameter\_direct\_strain {[} 5.657499999999999  5.651176877860324  5.661180703302433
 90.01741959362538  89.9544419036642   90.04747664598754 {]}
final lattice\_parameters {[} 5.657499999999999  5.651176877860324  5.661180703302433
 90.01741959362538  89.9544419036642   90.04747664598754 {]}
UB and strain refinement completed
True it is an OrientMatrix object
Orientation \textless{}LaueTools.indexingSpotsSet.OrientMatrix object at 0x7fb8c98ba9b0\textgreater{}
matrix {[}{[}-0.211415207301911  0.091252946262469 -0.97230572627369 {]}
 {[}-0.77573594328912   0.590041596352251  0.224552051799525{]}
 {[} 0.594613709497768  0.803610914885283 -0.054088075690982{]}{]}
\sphinxstylestrong{*nb of selected spots in AssignHKL***} 12
UBOrientMatrix {[}{[}-0.211415207301911  0.091252946262469 -0.97230572627369 {]}
 {[}-0.77573594328912   0.590041596352251  0.224552051799525{]}
 {[} 0.594613709497768  0.803610914885283 -0.054088075690982{]}{]}
For angular tolerance 0.50 deg
Nb of pairs found / nb total of expected spots: 12/177
Matching Rate : 6.78
Nb missing reflections: 165

grain \#0 : 12 links to simulated spots have been found
GoodRefinement condition is  True
nb\_updates 12 compared to 6


 refining grain \#0 step -----2

bestUB \textless{}LaueTools.indexingSpotsSet.OrientMatrix object at 0x7fb8ec7ddc50\textgreater{}
True it is an OrientMatrix object
Orientation \textless{}LaueTools.indexingSpotsSet.OrientMatrix object at 0x7fb8ec7ddc50\textgreater{}
matrix {[}{[}-0.211852735694566  0.092255643652867 -0.972937466948891{]}
 {[}-0.775856536468367  0.58951816141498   0.22475536073965 {]}
 {[} 0.594300563948835  0.802473664571131 -0.053318452339475{]}{]}
\sphinxstylestrong{*nb of selected spots in AssignHKL***} 12
UBOrientMatrix {[}{[}-0.211852735694566  0.092255643652867 -0.972937466948891{]}
 {[}-0.775856536468367  0.58951816141498   0.22475536073965 {]}
 {[} 0.594300563948835  0.802473664571131 -0.053318452339475{]}{]}
For angular tolerance 0.20 deg
Nb of pairs found / nb total of expected spots: 12/176
Matching Rate : 6.82
Nb missing reflections: 164

grain \#0 : 12 links to simulated spots have been found
\sphinxstylestrong{*********mean pixel deviation    0.560750282710606     ******}
Initial residues {[}0.053680370172309 0.013739858524874 0.921977335411896 0.403270956234836
 0.919825854310187 0.785969463406447 0.565019172757509 1.127873079813964
 0.363514793614926 0.412635402450867 0.711008521607465 0.450488584221994{]}
---------------------------------------------------



\sphinxstylestrong{***********************}
first error with initial values of: {[}'b/a', 'c/a', 'a12', 'a13', 'a23', 'theta1', 'theta2', 'theta3'{]}

\sphinxstylestrong{***********************}

\sphinxstylestrong{*********mean pixel deviation    0.560750282710606     ******}


\sphinxstylestrong{***********************}
Fitting parameters:   {[}'b/a', 'c/a', 'a12', 'a13', 'a23', 'theta1', 'theta2', 'theta3'{]}

\sphinxstylestrong{***********************}

With initial values {[}1. 1. 0. 0. 0. 0. 0. 0.{]}
code results 1
nb iterations 1767
mesg Both actual and predicted relative reductions in the sum of squares
  are at most 0.000000
strain\_sol {[} 1.001128981010799e+00  9.993806401299155e-01  8.449040381845989e-04
 -8.486913595751131e-04  3.520626401662759e-04 -2.714612741167435e-02
  3.054889720130747e-02  5.311773668297186e-02{]}


 \sphinxstylestrong{**********}  End of Fitting  -  Final errors  \sphinxstylestrong{**************}


\sphinxstylestrong{*********mean pixel deviation    0.3158807195732847     ******}
devstrain, lattice\_parameter\_direct\_strain {[}{[} 0.000159671589578 -0.000413843247724  0.00039782267455 {]}
 {[}-0.000413843247724 -0.00095830941256  -0.000151781897557{]}
 {[} 0.00039782267455  -0.000151781897557  0.000798637822982{]}{]} {[} 5.657424223234738  5.651101185787196  5.661104877237695
 90.01741959362538  89.9544419036642   90.04747664598754 {]}
For comparison: a,b,c are rescaled with respect to the reference value of a = 5.657500 Angstroms
lattice\_parameter\_direct\_strain {[} 5.657499999999999  5.651176877860324  5.661180703302433
 90.01741959362538  89.9544419036642   90.04747664598754 {]}
devstrain1, lattice\_parameter\_direct\_strain1 {[}{[} 0.000159671589578 -0.000413843247724  0.00039782267455 {]}
 {[}-0.000413843247724 -0.00095830941256  -0.000151781897557{]}
 {[} 0.00039782267455  -0.000151781897557  0.000798637822982{]}{]} {[} 5.657499999999999  5.651176877860324  5.661180703302433
 90.01741959362538  89.9544419036642   90.04747664598754 {]}
new UBs matrix in q= UBs G (s for strain)
strain\_direct {[}{[}-1.339403716515974e-05 -4.138432477238585e-04  3.978226745502020e-04{]}
 {[}-4.138432477238585e-04 -1.131375039302607e-03 -1.517818975573709e-04{]}
 {[} 3.978226745502020e-04 -1.517818975573709e-04  6.255721962393768e-04{]}{]}
deviatoric strain {[}{[} 0.000159671589578 -0.000413843247724  0.00039782267455 {]}
 {[}-0.000413843247724 -0.00095830941256  -0.000151781897557{]}
 {[} 0.00039782267455  -0.000151781897557  0.000798637822982{]}{]}
new UBs matrix in q= UBs G (s for strain)
strain\_direct {[}{[}-1.339403716515974e-05 -4.138432477238585e-04  3.978226745502020e-04{]}
 {[}-4.138432477238585e-04 -1.131375039302607e-03 -1.517818975573709e-04{]}
 {[} 3.978226745502020e-04 -1.517818975573709e-04  6.255721962393768e-04{]}{]}
deviatoric strain {[}{[} 0.000159671589578 -0.000413843247724  0.00039782267455 {]}
 {[}-0.000413843247724 -0.00095830941256  -0.000151781897557{]}
 {[} 0.00039782267455  -0.000151781897557  0.000798637822982{]}{]}
For comparison: a,b,c are rescaled with respect to the reference value of a = 5.657500 Angstroms
lattice\_parameter\_direct\_strain {[} 5.657499999999999  5.651176877860324  5.661180703302433
 90.01741959362538  89.9544419036642   90.04747664598754 {]}
final lattice\_parameters {[} 5.657499999999999  5.651176877860324  5.661180703302433
 90.01741959362538  89.9544419036642   90.04747664598754 {]}
UB and strain refinement completed
True it is an OrientMatrix object
Orientation \textless{}LaueTools.indexingSpotsSet.OrientMatrix object at 0x7fb8cf573a90\textgreater{}
matrix {[}{[}-0.211415207301911  0.091252946262469 -0.97230572627369 {]}
 {[}-0.77573594328912   0.590041596352251  0.224552051799525{]}
 {[} 0.594613709497768  0.803610914885283 -0.054088075690982{]}{]}
\sphinxstylestrong{*nb of selected spots in AssignHKL***} 12
UBOrientMatrix {[}{[}-0.211415207301911  0.091252946262469 -0.97230572627369 {]}
 {[}-0.77573594328912   0.590041596352251  0.224552051799525{]}
 {[} 0.594613709497768  0.803610914885283 -0.054088075690982{]}{]}
For angular tolerance 0.20 deg
Nb of pairs found / nb total of expected spots: 12/177
Matching Rate : 6.78
Nb missing reflections: 165

grain \#0 : 12 links to simulated spots have been found
GoodRefinement condition is  True
nb\_updates 12 compared to 6


 refining grain \#0 step -----3

bestUB \textless{}LaueTools.indexingSpotsSet.OrientMatrix object at 0x7fb8ec7ddc50\textgreater{}
True it is an OrientMatrix object
Orientation \textless{}LaueTools.indexingSpotsSet.OrientMatrix object at 0x7fb8ec7ddc50\textgreater{}
matrix {[}{[}-0.211852735694566  0.092255643652867 -0.972937466948891{]}
 {[}-0.775856536468367  0.58951816141498   0.22475536073965 {]}
 {[} 0.594300563948835  0.802473664571131 -0.053318452339475{]}{]}
\sphinxstylestrong{*nb of selected spots in AssignHKL***} 12
UBOrientMatrix {[}{[}-0.211852735694566  0.092255643652867 -0.972937466948891{]}
 {[}-0.775856536468367  0.58951816141498   0.22475536073965 {]}
 {[} 0.594300563948835  0.802473664571131 -0.053318452339475{]}{]}
For angular tolerance 0.20 deg
Nb of pairs found / nb total of expected spots: 12/176
Matching Rate : 6.82
Nb missing reflections: 164

grain \#0 : 12 links to simulated spots have been found
\sphinxstylestrong{*********mean pixel deviation    0.560750282710606     ******}
Initial residues {[}0.053680370172309 0.013739858524874 0.921977335411896 0.403270956234836
 0.919825854310187 0.785969463406447 0.565019172757509 1.127873079813964
 0.363514793614926 0.412635402450867 0.711008521607465 0.450488584221994{]}
---------------------------------------------------



\sphinxstylestrong{***********************}
first error with initial values of: {[}'b/a', 'c/a', 'a12', 'a13', 'a23', 'theta1', 'theta2', 'theta3'{]}

\sphinxstylestrong{***********************}

\sphinxstylestrong{*********mean pixel deviation    0.560750282710606     ******}


\sphinxstylestrong{***********************}
Fitting parameters:   {[}'b/a', 'c/a', 'a12', 'a13', 'a23', 'theta1', 'theta2', 'theta3'{]}

\sphinxstylestrong{***********************}

With initial values {[}1. 1. 0. 0. 0. 0. 0. 0.{]}
code results 1
nb iterations 1767
mesg Both actual and predicted relative reductions in the sum of squares
  are at most 0.000000
strain\_sol {[} 1.001128981010799e+00  9.993806401299155e-01  8.449040381845989e-04
 -8.486913595751131e-04  3.520626401662759e-04 -2.714612741167435e-02
  3.054889720130747e-02  5.311773668297186e-02{]}


 \sphinxstylestrong{**********}  End of Fitting  -  Final errors  \sphinxstylestrong{**************}


\sphinxstylestrong{*********mean pixel deviation    0.3158807195732847     ******}
devstrain, lattice\_parameter\_direct\_strain {[}{[} 0.000159671589578 -0.000413843247724  0.00039782267455 {]}
 {[}-0.000413843247724 -0.00095830941256  -0.000151781897557{]}
 {[} 0.00039782267455  -0.000151781897557  0.000798637822982{]}{]} {[} 5.657424223234738  5.651101185787196  5.661104877237695
 90.01741959362538  89.9544419036642   90.04747664598754 {]}
For comparison: a,b,c are rescaled with respect to the reference value of a = 5.657500 Angstroms
lattice\_parameter\_direct\_strain {[} 5.657499999999999  5.651176877860324  5.661180703302433
 90.01741959362538  89.9544419036642   90.04747664598754 {]}
devstrain1, lattice\_parameter\_direct\_strain1 {[}{[} 0.000159671589578 -0.000413843247724  0.00039782267455 {]}
 {[}-0.000413843247724 -0.00095830941256  -0.000151781897557{]}
 {[} 0.00039782267455  -0.000151781897557  0.000798637822982{]}{]} {[} 5.657499999999999  5.651176877860324  5.661180703302433
 90.01741959362538  89.9544419036642   90.04747664598754 {]}
new UBs matrix in q= UBs G (s for strain)
strain\_direct {[}{[}-1.339403716515974e-05 -4.138432477238585e-04  3.978226745502020e-04{]}
 {[}-4.138432477238585e-04 -1.131375039302607e-03 -1.517818975573709e-04{]}
 {[} 3.978226745502020e-04 -1.517818975573709e-04  6.255721962393768e-04{]}{]}
deviatoric strain {[}{[} 0.000159671589578 -0.000413843247724  0.00039782267455 {]}
 {[}-0.000413843247724 -0.00095830941256  -0.000151781897557{]}
 {[} 0.00039782267455  -0.000151781897557  0.000798637822982{]}{]}
new UBs matrix in q= UBs G (s for strain)
strain\_direct {[}{[}-1.339403716515974e-05 -4.138432477238585e-04  3.978226745502020e-04{]}
 {[}-4.138432477238585e-04 -1.131375039302607e-03 -1.517818975573709e-04{]}
 {[} 3.978226745502020e-04 -1.517818975573709e-04  6.255721962393768e-04{]}{]}
deviatoric strain {[}{[} 0.000159671589578 -0.000413843247724  0.00039782267455 {]}
 {[}-0.000413843247724 -0.00095830941256  -0.000151781897557{]}
 {[} 0.00039782267455  -0.000151781897557  0.000798637822982{]}{]}
For comparison: a,b,c are rescaled with respect to the reference value of a = 5.657500 Angstroms
lattice\_parameter\_direct\_strain {[} 5.657499999999999  5.651176877860324  5.661180703302433
 90.01741959362538  89.9544419036642   90.04747664598754 {]}
final lattice\_parameters {[} 5.657499999999999  5.651176877860324  5.661180703302433
 90.01741959362538  89.9544419036642   90.04747664598754 {]}
UB and strain refinement completed
True it is an OrientMatrix object
Orientation \textless{}LaueTools.indexingSpotsSet.OrientMatrix object at 0x7fb8ec7bc128\textgreater{}
matrix {[}{[}-0.211415207301911  0.091252946262469 -0.97230572627369 {]}
 {[}-0.77573594328912   0.590041596352251  0.224552051799525{]}
 {[} 0.594613709497768  0.803610914885283 -0.054088075690982{]}{]}
\sphinxstylestrong{*nb of selected spots in AssignHKL***} 12
UBOrientMatrix {[}{[}-0.211415207301911  0.091252946262469 -0.97230572627369 {]}
 {[}-0.77573594328912   0.590041596352251  0.224552051799525{]}
 {[} 0.594613709497768  0.803610914885283 -0.054088075690982{]}{]}
For angular tolerance 0.20 deg
Nb of pairs found / nb total of expected spots: 12/177
Matching Rate : 6.78
Nb missing reflections: 165

grain \#0 : 12 links to simulated spots have been found
GoodRefinement condition is  True
nb\_updates 12 compared to 6

---------------------------------------------
indexing completed for grain \#0 with matching rate 6.78
---------------------------------------------

transform matrix to matrix with lowest Euler Angles
start
 {[}{[}-0.211415207301911  0.091252946262469 -0.97230572627369 {]}
 {[}-0.77573594328912   0.590041596352251  0.224552051799525{]}
 {[} 0.594613709497768  0.803610914885283 -0.054088075690982{]}{]}
final
 {[}{[} 0.97230572627369   0.211415207301911  0.091252946262469{]}
 {[}-0.224552051799525  0.77573594328912   0.590041596352251{]}
 {[} 0.054088075690982 -0.594613709497768  0.803610914885283{]}{]}
hkl {[}{[}2. 6. 4.{]}
 {[}3. 3. 3.{]}
 {[}5. 3. 3.{]}
 {[}1. 3. 5.{]}
 {[}6. 2. 4.{]}
 {[}5. 1. 3.{]}
 {[}1. 3. 3.{]}
 {[}6. 4. 6.{]}
 {[}4. 2. 6.{]}
 {[}4. 2. 2.{]}
 {[}3. 5. 3.{]}
 {[}2. 2. 4.{]}{]}
new hkl (min euler angles) {[}{[}-4. -2.  6.{]}
 {[}-3. -3.  3.{]}
 {[}-3. -5.  3.{]}
 {[}-5. -1.  3.{]}
 {[}-4. -6.  2.{]}
 {[}-3. -5.  1.{]}
 {[}-3. -1.  3.{]}
 {[}-6. -6.  4.{]}
 {[}-6. -4.  2.{]}
 {[}-2. -4.  2.{]}
 {[}-3. -3.  5.{]}
 {[}-4. -2.  2.{]}{]}
UB before {[}{[}-0.211415207301911  0.091252946262469 -0.97230572627369 {]}
 {[}-0.77573594328912   0.590041596352251  0.224552051799525{]}
 {[} 0.594613709497768  0.803610914885283 -0.054088075690982{]}{]}
new UB (min euler angles) {[}{[} 0.97230572627369   0.211415207301911  0.091252946262469{]}
 {[}-0.224552051799525  0.77573594328912   0.590041596352251{]}
 {[} 0.054088075690982 -0.594613709497768  0.803610914885283{]}{]}
writing fit file -------------------------
for grainindex= 0
self.dict\_grain\_matrix{[}grain\_index{]} {[}{[} 0.97230572627369   0.211415207301911  0.091252946262469{]}
 {[}-0.224552051799525  0.77573594328912   0.590041596352251{]}
 {[} 0.054088075690982 -0.594613709497768  0.803610914885283{]}{]}
self.refinedUBmatrix {[}{[}-0.211415207301911  0.091252946262469 -0.97230572627369 {]}
 {[}-0.77573594328912   0.590041596352251  0.224552051799525{]}
 {[} 0.594613709497768  0.803610914885283 -0.054088075690982{]}{]}
new UBs matrix in q= UBs G (s for strain)
strain\_direct {[}{[}-1.339403716515974e-05 -4.138432477238585e-04  3.978226745502020e-04{]}
 {[}-4.138432477238585e-04 -1.131375039302607e-03 -1.517818975573709e-04{]}
 {[} 3.978226745502020e-04 -1.517818975573709e-04  6.255721962393768e-04{]}{]}
deviatoric strain {[}{[} 0.000159671589578 -0.000413843247724  0.00039782267455 {]}
 {[}-0.000413843247724 -0.00095830941256  -0.000151781897557{]}
 {[} 0.00039782267455  -0.000151781897557  0.000798637822982{]}{]}
new UBs matrix in q= UBs G (s for strain)
strain\_direct {[}{[}-1.339403716515974e-05 -4.138432477238585e-04  3.978226745502020e-04{]}
 {[}-4.138432477238585e-04 -1.131375039302607e-03 -1.517818975573709e-04{]}
 {[} 3.978226745502020e-04 -1.517818975573709e-04  6.255721962393768e-04{]}{]}
deviatoric strain {[}{[} 0.000159671589578 -0.000413843247724  0.00039782267455 {]}
 {[}-0.000413843247724 -0.00095830941256  -0.000151781897557{]}
 {[} 0.00039782267455  -0.000151781897557  0.000798637822982{]}{]}
For comparison: a,b,c are rescaled with respect to the reference value of a = 5.657500 Angstroms
lattice\_parameter\_direct\_strain {[} 5.657499999999999  5.651176877860324  5.661180703302433
 90.01741959362538  89.9544419036642   90.04747664598754 {]}
final lattice\_parameters {[} 5.657499999999999  5.651176877860324  5.661180703302433
 90.01741959362538  89.9544419036642   90.04747664598754 {]}
File : Ge0001\_g0.fit written in /home/micha/LaueToolsPy3/LaueTools/notebooks
Experimental experimental spots indices which are not indexed {[}{]}
Missing reflections grainindex is -100 for indexed grainindex 0
within angular tolerance 0.500

 Remaining nb of spots to index for grain \#1 : 0

12 spots have been indexed over 12
indexing rate is --- : 100.0 percents
indexation of short\_Ge0001.cor is completed
for the 1 grain(s) that has(ve) been indexed as requested
Leaving Index and Refine procedures...
\end{sphinxalltt}

\fvset{hllines={, ,}}%
\begin{sphinxVerbatim}[commandchars=\\\{\}]
\PYG{n}{index\PYGZus{}grain\PYGZus{}retrieve}\PYG{o}{=}\PYG{l+m+mi}{0}
\PYG{n+nb}{print}\PYG{p}{(}\PYG{l+s+s2}{\PYGZdq{}}\PYG{l+s+s2}{number of indexed spots}\PYG{l+s+s2}{\PYGZdq{}}\PYG{p}{,} \PYG{n+nb}{len}\PYG{p}{(}\PYG{n}{DataSet}\PYG{o}{.}\PYG{n}{getallIndexedSpotsallData}\PYG{p}{(}\PYG{p}{)}\PYG{p}{[}\PYG{n}{index\PYGZus{}grain\PYGZus{}retrieve}\PYG{p}{]}\PYG{p}{)}\PYG{p}{)}
\end{sphinxVerbatim}

\fvset{hllines={, ,}}%
\begin{sphinxVerbatim}[commandchars=\\\{\}]
\PYG{n}{number} \PYG{n}{of} \PYG{n}{indexed} \PYG{n}{spots} \PYG{l+m+mi}{12}
\end{sphinxVerbatim}


\subsection{Results of indexation can be found in attributes or through methods}
\label{\detokenize{Indexation:results-of-indexation-can-be-found-in-attributes-or-through-methods}}
\fvset{hllines={, ,}}%
\begin{sphinxVerbatim}[commandchars=\\\{\}]
\PYG{n}{spotsdata}\PYG{o}{=}\PYG{n}{DataSet}\PYG{o}{.}\PYG{n}{getSummaryallData}\PYG{p}{(}\PYG{p}{)}
\PYG{n+nb}{print}\PYG{p}{(}\PYG{l+s+s2}{\PYGZdq{}}\PYG{l+s+s2}{first 2 indexed spots properties}\PYG{l+s+se}{\PYGZbs{}n}\PYG{l+s+s2}{\PYGZdq{}}\PYG{p}{)}
\PYG{n+nb}{print}\PYG{p}{(}\PYG{l+s+s1}{\PYGZsq{}}\PYG{l+s+s1}{\PYGZsh{}spot  \PYGZsh{}grain 2theta chi X Y I h k l Energy}\PYG{l+s+s1}{\PYGZsq{}}\PYG{p}{)}
\PYG{n+nb}{print}\PYG{p}{(}\PYG{n}{spotsdata}\PYG{p}{[}\PYG{p}{:}\PYG{l+m+mi}{2}\PYG{p}{]}\PYG{p}{)}
\end{sphinxVerbatim}

\fvset{hllines={, ,}}%
\begin{sphinxVerbatim}[commandchars=\\\{\}]
\PYG{n}{first} \PYG{l+m+mi}{2} \PYG{n}{indexed} \PYG{n}{spots} \PYG{n}{properties}

\PYG{c+c1}{\PYGZsh{}spot  \PYGZsh{}grain 2theta chi X Y I h k l Energy}
\PYG{p}{[}\PYG{p}{[} \PYG{l+m+mf}{0.000000000000000e+00}  \PYG{l+m+mf}{0.000000000000000e+00}  \PYG{l+m+mf}{6.035945800000000e+01}
   \PYG{l+m+mf}{2.648319100000000e+01}  \PYG{l+m+mf}{6.261200000000000e+02}  \PYG{l+m+mf}{1.661280000000000e+03}
   \PYG{l+m+mf}{1.582539000000000e+04} \PYG{o}{\PYGZhy{}}\PYG{l+m+mf}{4.000000000000000e+00} \PYG{o}{\PYGZhy{}}\PYG{l+m+mf}{2.000000000000000e+00}
   \PYG{l+m+mf}{6.000000000000000e+00}  \PYG{l+m+mf}{1.632366988464985e+01}\PYG{p}{]}
 \PYG{p}{[} \PYG{l+m+mf}{1.000000000000000e+00}  \PYG{l+m+mf}{0.000000000000000e+00}  \PYG{l+m+mf}{7.821582100000001e+01}
   \PYG{l+m+mf}{1.638153000000000e+00}  \PYG{l+m+mf}{1.027110000000000e+03}  \PYG{l+m+mf}{1.293280000000000e+03}
   \PYG{l+m+mf}{7.093127000000000e+04} \PYG{o}{\PYGZhy{}}\PYG{l+m+mf}{3.000000000000000e+00} \PYG{o}{\PYGZhy{}}\PYG{l+m+mf}{3.000000000000000e+00}
   \PYG{l+m+mf}{3.000000000000000e+00}  \PYG{l+m+mf}{9.031851548397370e+00}\PYG{p}{]}\PYG{p}{]}
\end{sphinxVerbatim}

\fvset{hllines={, ,}}%
\begin{sphinxVerbatim}[commandchars=\\\{\}]
\PYG{n+nb}{print}\PYG{p}{(}\PYG{l+s+s1}{\PYGZsq{}}\PYG{l+s+s1}{\PYGZsh{}grain  : [Npairs = Nb pairs with tolerance angle }\PYG{l+s+si}{\PYGZpc{}.4f}\PYG{l+s+s1}{, 100*Npairs/Ndirections theo.]}\PYG{l+s+s1}{\PYGZsq{}}\PYG{o}{\PYGZpc{}}\PYG{n}{dict\PYGZus{}loop}\PYG{p}{[}\PYG{l+s+s1}{\PYGZsq{}}\PYG{l+s+s1}{list matching tol angles}\PYG{l+s+s1}{\PYGZsq{}}\PYG{p}{]}\PYG{p}{[}\PYG{o}{\PYGZhy{}}\PYG{l+m+mi}{1}\PYG{p}{]}\PYG{p}{)}
\PYG{n}{DataSet}\PYG{o}{.}\PYG{n}{dict\PYGZus{}grain\PYGZus{}matching\PYGZus{}rate}
\end{sphinxVerbatim}

\fvset{hllines={, ,}}%
\begin{sphinxVerbatim}[commandchars=\\\{\}]
\PYG{c+c1}{\PYGZsh{}grain  : [Npairs = Nb pairs with tolerance angle 0.2000, 100*Npairs/Ndirections theo.]}
\end{sphinxVerbatim}

\fvset{hllines={, ,}}%
\begin{sphinxVerbatim}[commandchars=\\\{\}]
\PYG{p}{\PYGZob{}}\PYG{l+m+mi}{0}\PYG{p}{:} \PYG{p}{[}\PYG{l+m+mi}{12}\PYG{p}{,} \PYG{l+m+mf}{6.779661016949152}\PYG{p}{]}\PYG{p}{\PYGZcb{}}
\end{sphinxVerbatim}

\fvset{hllines={, ,}}%
\begin{sphinxVerbatim}[commandchars=\\\{\}]
\PYG{n+nb}{print}\PYG{p}{(}\PYG{l+s+s2}{\PYGZdq{}}\PYG{l+s+s2}{\PYGZsh{}grain  : deviatoric strain}\PYG{l+s+s2}{\PYGZdq{}}\PYG{p}{)}
\PYG{n}{DataSet}\PYG{o}{.}\PYG{n}{dict\PYGZus{}grain\PYGZus{}devstrain}
\end{sphinxVerbatim}

\fvset{hllines={, ,}}%
\begin{sphinxVerbatim}[commandchars=\\\{\}]
\PYG{c+c1}{\PYGZsh{}grain  : deviatoric strain}
\end{sphinxVerbatim}

\fvset{hllines={, ,}}%
\begin{sphinxVerbatim}[commandchars=\\\{\}]
\PYG{p}{\PYGZob{}}\PYG{l+m+mi}{0}\PYG{p}{:} \PYG{n}{array}\PYG{p}{(}\PYG{p}{[}\PYG{p}{[} \PYG{l+m+mf}{0.000159671589578}\PYG{p}{,} \PYG{o}{\PYGZhy{}}\PYG{l+m+mf}{0.000413843247724}\PYG{p}{,}  \PYG{l+m+mf}{0.00039782267455} \PYG{p}{]}\PYG{p}{,}
        \PYG{p}{[}\PYG{o}{\PYGZhy{}}\PYG{l+m+mf}{0.000413843247724}\PYG{p}{,} \PYG{o}{\PYGZhy{}}\PYG{l+m+mf}{0.00095830941256} \PYG{p}{,} \PYG{o}{\PYGZhy{}}\PYG{l+m+mf}{0.000151781897557}\PYG{p}{]}\PYG{p}{,}
        \PYG{p}{[} \PYG{l+m+mf}{0.00039782267455} \PYG{p}{,} \PYG{o}{\PYGZhy{}}\PYG{l+m+mf}{0.000151781897557}\PYG{p}{,}  \PYG{l+m+mf}{0.000798637822982}\PYG{p}{]}\PYG{p}{]}\PYG{p}{)}\PYG{p}{\PYGZcb{}}
\end{sphinxVerbatim}

\fvset{hllines={, ,}}%
\begin{sphinxVerbatim}[commandchars=\\\{\}]
\PYG{c+c1}{\PYGZsh{}RefinedUB= DataSet.dict\PYGZus{}grain\PYGZus{}matrix}
\PYG{n+nb}{print}\PYG{p}{(}\PYG{l+s+s2}{\PYGZdq{}}\PYG{l+s+s2}{\PYGZsh{}grain  : refined UB matrix}\PYG{l+s+s2}{\PYGZdq{}}\PYG{p}{)}
\PYG{n}{DataSet}\PYG{o}{.}\PYG{n}{dict\PYGZus{}grain\PYGZus{}matrix}
\end{sphinxVerbatim}

\fvset{hllines={, ,}}%
\begin{sphinxVerbatim}[commandchars=\\\{\}]
\PYG{c+c1}{\PYGZsh{}grain  : refined UB matrix}
\end{sphinxVerbatim}

\fvset{hllines={, ,}}%
\begin{sphinxVerbatim}[commandchars=\\\{\}]
\PYG{p}{\PYGZob{}}\PYG{l+m+mi}{0}\PYG{p}{:} \PYG{n}{array}\PYG{p}{(}\PYG{p}{[}\PYG{p}{[} \PYG{l+m+mf}{0.97230572627369} \PYG{p}{,}  \PYG{l+m+mf}{0.211415207301911}\PYG{p}{,}  \PYG{l+m+mf}{0.091252946262469}\PYG{p}{]}\PYG{p}{,}
        \PYG{p}{[}\PYG{o}{\PYGZhy{}}\PYG{l+m+mf}{0.224552051799525}\PYG{p}{,}  \PYG{l+m+mf}{0.77573594328912} \PYG{p}{,}  \PYG{l+m+mf}{0.590041596352251}\PYG{p}{]}\PYG{p}{,}
        \PYG{p}{[} \PYG{l+m+mf}{0.054088075690982}\PYG{p}{,} \PYG{o}{\PYGZhy{}}\PYG{l+m+mf}{0.594613709497768}\PYG{p}{,}  \PYG{l+m+mf}{0.803610914885283}\PYG{p}{]}\PYG{p}{]}\PYG{p}{)}\PYG{p}{\PYGZcb{}}
\end{sphinxVerbatim}

\fvset{hllines={, ,}}%
\begin{sphinxVerbatim}[commandchars=\\\{\}]
\PYG{n+nb}{print}\PYG{p}{(}\PYG{p}{[}\PYG{n}{DataSet}\PYG{o}{.}\PYG{n}{indexed\PYGZus{}spots\PYGZus{}dict}\PYG{p}{[}\PYG{n}{k}\PYG{p}{]} \PYG{k}{for} \PYG{n}{k} \PYG{o+ow}{in} \PYG{n+nb}{range}\PYG{p}{(}\PYG{l+m+mi}{10}\PYG{p}{)}\PYG{p}{]}\PYG{p}{)}
\end{sphinxVerbatim}

\fvset{hllines={, ,}}%
\begin{sphinxVerbatim}[commandchars=\\\{\}]
\PYG{p}{[}\PYG{p}{[}\PYG{l+m+mi}{0}\PYG{p}{,} \PYG{l+m+mf}{60.359458}\PYG{p}{,} \PYG{l+m+mf}{26.483191}\PYG{p}{,} \PYG{l+m+mf}{626.12}\PYG{p}{,} \PYG{l+m+mf}{1661.28}\PYG{p}{,} \PYG{l+m+mf}{15825.39}\PYG{p}{,} \PYG{n}{array}\PYG{p}{(}\PYG{p}{[}\PYG{o}{\PYGZhy{}}\PYG{l+m+mf}{4.}\PYG{p}{,} \PYG{o}{\PYGZhy{}}\PYG{l+m+mf}{2.}\PYG{p}{,}  \PYG{l+m+mf}{6.}\PYG{p}{]}\PYG{p}{)}\PYG{p}{,} \PYG{l+m+mf}{16.323669884649853}\PYG{p}{,} \PYG{l+m+mi}{0}\PYG{p}{,} \PYG{l+m+mi}{1}\PYG{p}{]}\PYG{p}{,} \PYG{p}{[}\PYG{l+m+mi}{1}\PYG{p}{,} \PYG{l+m+mf}{78.215821}\PYG{p}{,} \PYG{l+m+mf}{1.638153}\PYG{p}{,} \PYG{l+m+mf}{1027.11}\PYG{p}{,} \PYG{l+m+mf}{1293.28}\PYG{p}{,} \PYG{l+m+mf}{70931.27}\PYG{p}{,} \PYG{n}{array}\PYG{p}{(}\PYG{p}{[}\PYG{o}{\PYGZhy{}}\PYG{l+m+mf}{3.}\PYG{p}{,} \PYG{o}{\PYGZhy{}}\PYG{l+m+mf}{3.}\PYG{p}{,}  \PYG{l+m+mf}{3.}\PYG{p}{]}\PYG{p}{)}\PYG{p}{,} \PYG{l+m+mf}{9.03185154839737}\PYG{p}{,} \PYG{l+m+mi}{0}\PYG{p}{,} \PYG{l+m+mi}{1}\PYG{p}{]}\PYG{p}{,} \PYG{p}{[}\PYG{l+m+mi}{2}\PYG{p}{,} \PYG{l+m+mf}{68.680451}\PYG{p}{,} \PYG{o}{\PYGZhy{}}\PYG{l+m+mf}{15.358122}\PYG{p}{,} \PYG{l+m+mf}{1288.11}\PYG{p}{,} \PYG{l+m+mf}{1460.16}\PYG{p}{,} \PYG{l+m+mf}{22795.07}\PYG{p}{,} \PYG{n}{array}\PYG{p}{(}\PYG{p}{[}\PYG{o}{\PYGZhy{}}\PYG{l+m+mf}{3.}\PYG{p}{,} \PYG{o}{\PYGZhy{}}\PYG{l+m+mf}{5.}\PYG{p}{,}  \PYG{l+m+mf}{3.}\PYG{p}{]}\PYG{p}{)}\PYG{p}{,} \PYG{l+m+mf}{12.73794897550435}\PYG{p}{,} \PYG{l+m+mi}{0}\PYG{p}{,} \PYG{l+m+mi}{1}\PYG{p}{]}\PYG{p}{,} \PYG{p}{[}\PYG{l+m+mi}{3}\PYG{p}{,} \PYG{l+m+mf}{108.452917}\PYG{p}{,} \PYG{l+m+mf}{37.749461}\PYG{p}{,} \PYG{l+m+mf}{383.77}\PYG{p}{,} \PYG{l+m+mf}{754.58}\PYG{p}{,} \PYG{l+m+mf}{4400.61}\PYG{p}{,} \PYG{n}{array}\PYG{p}{(}\PYG{p}{[}\PYG{o}{\PYGZhy{}}\PYG{l+m+mf}{5.}\PYG{p}{,} \PYG{o}{\PYGZhy{}}\PYG{l+m+mf}{1.}\PYG{p}{,}  \PYG{l+m+mf}{3.}\PYG{p}{]}\PYG{p}{)}\PYG{p}{,} \PYG{l+m+mf}{7.989738078276174}\PYG{p}{,} \PYG{l+m+mi}{0}\PYG{p}{,} \PYG{l+m+mi}{1}\PYG{p}{]}\PYG{p}{,} \PYG{p}{[}\PYG{l+m+mi}{4}\PYG{p}{,} \PYG{l+m+mf}{83.349535}\PYG{p}{,} \PYG{o}{\PYGZhy{}}\PYG{l+m+mf}{27.458061}\PYG{p}{,} \PYG{l+m+mf}{1497.4}\PYG{p}{,} \PYG{l+m+mf}{1224.7}\PYG{p}{,} \PYG{l+m+mf}{13145.99}\PYG{p}{,} \PYG{n}{array}\PYG{p}{(}\PYG{p}{[}\PYG{o}{\PYGZhy{}}\PYG{l+m+mf}{4.}\PYG{p}{,} \PYG{o}{\PYGZhy{}}\PYG{l+m+mf}{6.}\PYG{p}{,}  \PYG{l+m+mf}{2.}\PYG{p}{]}\PYG{p}{)}\PYG{p}{,} \PYG{l+m+mf}{12.327936233758937}\PYG{p}{,} \PYG{l+m+mi}{0}\PYG{p}{,} \PYG{l+m+mi}{1}\PYG{p}{]}\PYG{p}{,} \PYG{p}{[}\PYG{l+m+mi}{5}\PYG{p}{,} \PYG{l+m+mf}{82.072076}\PYG{p}{,} \PYG{o}{\PYGZhy{}}\PYG{l+m+mf}{35.89243}\PYG{p}{,} \PYG{l+m+mf}{1672.67}\PYG{p}{,} \PYG{l+m+mf}{1258.62}\PYG{p}{,} \PYG{l+m+mf}{13318.81}\PYG{p}{,} \PYG{n}{array}\PYG{p}{(}\PYG{p}{[}\PYG{o}{\PYGZhy{}}\PYG{l+m+mf}{3.}\PYG{p}{,} \PYG{o}{\PYGZhy{}}\PYG{l+m+mf}{5.}\PYG{p}{,}  \PYG{l+m+mf}{1.}\PYG{p}{]}\PYG{p}{)}\PYG{p}{,} \PYG{l+m+mf}{9.870774398536632}\PYG{p}{,} \PYG{l+m+mi}{0}\PYG{p}{,} \PYG{l+m+mi}{1}\PYG{p}{]}\PYG{p}{,} \PYG{p}{[}\PYG{l+m+mi}{6}\PYG{p}{,} \PYG{l+m+mf}{81.771257}\PYG{p}{,} \PYG{l+m+mf}{30.38247}\PYG{p}{,} \PYG{l+m+mf}{548.25}\PYG{p}{,} \PYG{l+m+mf}{1260.32}\PYG{p}{,} \PYG{l+m+mf}{6137.87}\PYG{p}{,} \PYG{n}{array}\PYG{p}{(}\PYG{p}{[}\PYG{o}{\PYGZhy{}}\PYG{l+m+mf}{3.}\PYG{p}{,} \PYG{o}{\PYGZhy{}}\PYG{l+m+mf}{1.}\PYG{p}{,}  \PYG{l+m+mf}{3.}\PYG{p}{]}\PYG{p}{)}\PYG{p}{,} \PYG{l+m+mf}{7.2986772558942405}\PYG{p}{,} \PYG{l+m+mi}{0}\PYG{p}{,} \PYG{l+m+mi}{1}\PYG{p}{]}\PYG{p}{,} \PYG{p}{[}\PYG{l+m+mi}{7}\PYG{p}{,} \PYG{l+m+mf}{91.798221}\PYG{p}{,} \PYG{o}{\PYGZhy{}}\PYG{l+m+mf}{8.309941}\PYG{p}{,} \PYG{l+m+mf}{1176.09}\PYG{p}{,} \PYG{l+m+mf}{1086.19}\PYG{p}{,} \PYG{l+m+mf}{11799.93}\PYG{p}{,} \PYG{n}{array}\PYG{p}{(}\PYG{p}{[}\PYG{o}{\PYGZhy{}}\PYG{l+m+mf}{6.}\PYG{p}{,} \PYG{o}{\PYGZhy{}}\PYG{l+m+mf}{6.}\PYG{p}{,}  \PYG{l+m+mf}{4.}\PYG{p}{]}\PYG{p}{)}\PYG{p}{,} \PYG{l+m+mf}{14.309911950813975}\PYG{p}{,} \PYG{l+m+mi}{0}\PYG{p}{,} \PYG{l+m+mi}{1}\PYG{p}{]}\PYG{p}{,} \PYG{p}{[}\PYG{l+m+mi}{8}\PYG{p}{,} \PYG{l+m+mf}{120.59561}\PYG{p}{,} \PYG{o}{\PYGZhy{}}\PYG{l+m+mf}{8.92066}\PYG{p}{,} \PYG{l+m+mf}{1183.27}\PYG{p}{,} \PYG{l+m+mf}{598.92}\PYG{p}{,} \PYG{l+m+mf}{17182.88}\PYG{p}{,} \PYG{n}{array}\PYG{p}{(}\PYG{p}{[}\PYG{o}{\PYGZhy{}}\PYG{l+m+mf}{6.}\PYG{p}{,} \PYG{o}{\PYGZhy{}}\PYG{l+m+mf}{4.}\PYG{p}{,}  \PYG{l+m+mf}{2.}\PYG{p}{]}\PYG{p}{)}\PYG{p}{,} \PYG{l+m+mf}{9.435284238042113}\PYG{p}{,} \PYG{l+m+mi}{0}\PYG{p}{,} \PYG{l+m+mi}{1}\PYG{p}{]}\PYG{p}{,} \PYG{p}{[}\PYG{l+m+mi}{9}\PYG{p}{,} \PYG{l+m+mf}{64.329767}\PYG{p}{,} \PYG{o}{\PYGZhy{}}\PYG{l+m+mf}{20.824155}\PYG{p}{,} \PYG{l+m+mf}{1379.17}\PYG{p}{,} \PYG{l+m+mf}{1553.58}\PYG{p}{,} \PYG{l+m+mf}{51933.84}\PYG{p}{,} \PYG{n}{array}\PYG{p}{(}\PYG{p}{[}\PYG{o}{\PYGZhy{}}\PYG{l+m+mf}{2.}\PYG{p}{,} \PYG{o}{\PYGZhy{}}\PYG{l+m+mf}{4.}\PYG{p}{,}  \PYG{l+m+mf}{2.}\PYG{p}{]}\PYG{p}{)}\PYG{p}{,} \PYG{l+m+mf}{10.087272916663885}\PYG{p}{,} \PYG{l+m+mi}{0}\PYG{p}{,} \PYG{l+m+mi}{1}\PYG{p}{]}\PYG{p}{]}
\end{sphinxVerbatim}


\chapter{LaueTools Modules}
\label{\detokenize{LaueToolsModules::doc}}\label{\detokenize{LaueToolsModules:lauetools-modules}}

\section{Browse Modules and Functions}
\label{\detokenize{LaueToolsModules:browse-modules-and-functions}}\begin{itemize}
\item {} 
\DUrole{xref,std,std-ref}{genindex}

\item {} 
\DUrole{xref,std,std-ref}{modindex}

\end{itemize}


\section{Modules for Laue Pattern Simulation}
\label{\detokenize{LaueToolsModules:modules-for-laue-pattern-simulation}}
The following modules are used to compute Laue pattern from grain (or crystal) structural parameters and 2D plane detector geometry:
\begin{itemize}
\item {} 
\sphinxtitleref{CrystalParameters.py} defines structural parameters describing the crystal. It includes orientation matrix and strain operators.

\item {} 
\sphinxtitleref{lauecore.py} contains the core procedures to compute all Laue spots properties.

\item {} 
\sphinxtitleref{LaueGeometry.py} handles the 2D plane geometry set by the detector position and orientation with respect to sample and incoming direction.

\item {} 
\sphinxtitleref{multigrainsSimulator.py} allows to simulate an assembly of grains, some of them according to a distribution of grains. This module is called by the graphical user interface (LaueSimulatorGUI) which provides all arguments in an intuitive way.

\end{itemize}


\subsection{CrystalParameters}
\label{\detokenize{Simulation_Module::doc}}\label{\detokenize{Simulation_Module:crystalparameters}}\label{\detokenize{Simulation_Module:simulation}}\label{\detokenize{Simulation_Module:module-LaueTools.CrystalParameters}}\index{LaueTools.CrystalParameters (module)}
This module belong to LaueTools package. It gathers procedures to define crystal
lattice parameters and strain calculations

Main authors are JS Micha, O. Robach, S. Tardif June 2019
\index{GrainParameter\_from\_Material() (in module LaueTools.CrystalParameters)}

\begin{fulllineitems}
\phantomsection\label{\detokenize{Simulation_Module:LaueTools.CrystalParameters.GrainParameter_from_Material}}\pysiglinewithargsret{\sphinxcode{\sphinxupquote{LaueTools.CrystalParameters.}}\sphinxbfcode{\sphinxupquote{GrainParameter\_from\_Material}}}{\emph{key\_material, dictmaterials=\{'Al': {[}'Al', {[}4.05, 4.05, 4.05, 90, 90, 90{]}, 'fcc'{]}, 'Al2Cu': {[}'Al2Cu', {[}6.063, 6.063, 4.872, 90, 90, 90{]}, 'no'{]}, 'Al2O3': {[}'Al2O3', {[}4.785, 4.785, 12.991, 90, 90, 120{]}, 'Al2O3'{]}, 'Al2O3\_all': {[}'Al2O3\_all', {[}4.785, 4.785, 12.991, 90, 90, 120{]}, 'no'{]}, 'AlN': {[}'AlN', {[}3.11, 3.11, 4.98, 90.0, 90.0, 120.0{]}, 'wurtzite'{]}, 'Au': {[}'Au', {[}4.078, 4.078, 4.078, 90, 90, 90{]}, 'fcc'{]}, 'CCDL1949': {[}'CCDL1949', {[}9.89, 17.85, 5.31, 90, 107.5, 90{]}, 'h+k=2n'{]}, 'CdHgTe': {[}'CdHgTe', {[}6.46678, 6.46678, 6.46678, 90, 90, 90{]}, 'dia'{]}, 'CdHgTe\_fcc': {[}'CdHgTe\_fcc', {[}6.46678, 6.46678, 6.46678, 90, 90, 90{]}, 'fcc'{]}, 'CdTe': {[}'CdTe', {[}6.487, 6.487, 6.487, 90, 90, 90{]}, 'fcc'{]}, 'CdTeDiagB': {[}'CdTeDiagB', {[}4.5721, 7.9191, 11.1993, 90, 90, 90{]}, 'no'{]}, 'Crocidolite': {[}'Crocidolite', {[}9.811, 18.013, 5.326, 90, 103.68, 90{]}, 'no'{]}, 'Crocidolite\_2': {[}'Crocidolite\_2', {[}9.76, 17.93, 5.35, 90, 103.6, 90{]}, 'no'{]}, 'Crocidolite\_2\_72deg': {[}'Crocidolite\_2', {[}9.76, 17.93, 5.35, 90, 76.4, 90{]}, 'no'{]}, 'Crocidolite\_small': {[}'Crocidolite\_small', {[}3.2533333333333334, 5.976666666666667, 1.7833333333333332, 90, 103.6, 90{]}, 'no'{]}, 'Crocidolite\_whittaker\_1949': {[}'Crocidolite\_whittaker\_1949', {[}9.89, 17.85, 5.31, 90, 107.5, 90{]}, 'no'{]}, 'Cu': {[}'Cu', {[}3.6, 3.6, 3.6, 90, 90, 90{]}, 'fcc'{]}, 'Cu6Sn5\_monoclinic': {[}'Cu6Sn5\_monoclinic', {[}11.02, 7.28, 9.827, 90, 98.84, 90{]}, 'no'{]}, 'Cu6Sn5\_tetra': {[}'Cu6Sn5\_tetra', {[}3.608, 3.608, 5.037, 90, 90, 90{]}, 'no'{]}, 'DIA': {[}'DIA', {[}5.0, 5.0, 5.0, 90, 90, 90{]}, 'dia'{]}, 'DIAs': {[}'DIAs', {[}3.56683, 3.56683, 3.56683, 90, 90, 90{]}, 'dia'{]}, 'DarinaMolecule': {[}'DarinaMolecule', {[}9.4254, 13.5004, 13.8241, 61.83, 84.555, 75.231{]}, 'no'{]}, 'FCC': {[}'FCC', {[}5.0, 5.0, 5.0, 90, 90, 90{]}, 'fcc'{]}, 'Fe': {[}'Fe', {[}2.856, 2.856, 2.856, 90, 90, 90{]}, 'bcc'{]}, 'Fe2Ta': {[}'Fe2Ta', {[}4.83, 4.83, 0.788, 90, 90, 120{]}, 'no'{]}, 'FeAl': {[}'FeAl', {[}5.871, 5.871, 5.871, 90, 90, 90{]}, 'fcc'{]}, 'GaAs': {[}'GaAs', {[}5.65325, 5.65325, 5.65325, 90, 90, 90{]}, 'dia'{]}, 'GaN': {[}'GaN', {[}3.189, 3.189, 5.185, 90, 90, 120{]}, 'wurtzite'{]}, 'GaN\_all': {[}'GaN\_all', {[}3.189, 3.189, 5.185, 90, 90, 120{]}, 'no'{]}, 'Ge': {[}'Ge', {[}5.6575, 5.6575, 5.6575, 90, 90, 90{]}, 'dia'{]}, 'Ge\_compressedhydro': {[}'Ge\_compressedhydro', {[}5.64, 5.64, 5.64, 90, 90, 90.0{]}, 'dia'{]}, 'Ge\_s': {[}'Ge\_s', {[}5.6575, 5.6575, 5.6575, 90, 90, 89.5{]}, 'dia'{]}, 'Getest': {[}'Getest', {[}5.6575, 5.6575, 5.6574, 90, 90, 90{]}, 'dia'{]}, 'Hematite': {[}'Hematite', {[}5.03459, 5.03459, 13.7533, 90, 90, 120{]}, 'no'{]}, 'In': {[}'In', {[}3.2517, 3.2517, 4.9459, 90, 90, 90{]}, 'h+k+l=2n'{]}, 'InGaN': {[}'InGaN', {[}3.3609999999999998, 3.3609999999999998, 5.439, 90, 90, 120{]}, 'wurtzite'{]}, 'InN': {[}'InN', {[}3.533, 3.533, 5.693, 90, 90, 120{]}, 'wurtzite'{]}, 'Magnetite': {[}'Magnetite', {[}8.391, 8.391, 8.391, 90, 90, 90{]}, 'dia'{]}, 'Magnetite\_fcc': {[}'Magnetite\_fcc', {[}8.391, 8.391, 8.391, 90, 90, 90{]}, 'fcc'{]}, 'Magnetite\_sc': {[}'Magnetite\_sc', {[}8.391, 8.391, 8.391, 90, 90, 90{]}, 'no'{]}, 'Nd45': {[}'Nd45', {[}5.4884, 5.4884, 5.4884, 90, 90, 90{]}, 'fcc'{]}, 'Ni': {[}'Ni', {[}3.5238, 3.5238, 3.5238, 90, 90, 90{]}, 'fcc'{]}, 'NiO': {[}'NiO', {[}2.96, 2.96, 7.23, 90, 90, 120{]}, 'no'{]}, 'NiTi': {[}'NiTi', {[}3.5506, 3.5506, 3.5506, 90, 90, 90{]}, 'fcc'{]}, 'SC': {[}'SC', {[}1.0, 1.0, 1.0, 90, 90, 90{]}, 'no'{]}, 'SC5': {[}'SC5', {[}5.0, 5.0, 5.0, 90, 90, 90{]}, 'no'{]}, 'SC7': {[}'SC7', {[}7.0, 7.0, 7.0, 90, 90, 90{]}, 'no'{]}, 'Sb': {[}'Sb', {[}4.3, 4.3, 11.3, 90, 90, 120{]}, 'no'{]}, 'Si': {[}'Si', {[}5.4309, 5.4309, 5.4309, 90, 90, 90{]}, 'dia'{]}, 'Sn': {[}'Sn', {[}5.83, 5.83, 3.18, 90, 90, 90{]}, 'h+k+l=2n'{]}, 'Ti': {[}'Ti', {[}2.95, 2.95, 4.68, 90, 90, 120{]}, 'no'{]}, 'Ti2AlN': {[}'Ti2AlN', {[}2.989, 2.989, 13.624, 90, 90, 120{]}, 'Ti2AlN'{]}, 'Ti2AlN\_w': {[}'Ti2AlN\_w', {[}2.989, 2.989, 13.624, 90, 90, 120{]}, 'wurtzite'{]}, 'Ti\_beta': {[}'Ti\_beta', {[}3.2587, 3.2587, 3.2587, 90, 90, 90{]}, 'bcc'{]}, 'Ti\_omega': {[}'Ti\_omega', {[}4.6085, 4.6085, 2.8221, 90, 90, 120{]}, 'no'{]}, 'Ti\_s': {[}'Ti\_s', {[}3.0, 3.0, 4.7, 90.5, 89.5, 120.5{]}, 'no'{]}, 'UO2': {[}'UO2', {[}5.47, 5.47, 5.47, 90, 90, 90{]}, 'fcc'{]}, 'VO2M1': {[}'VO2M1', {[}5.75175, 4.52596, 5.38326, 90.0, 122.6148, 90.0{]}, 'VO2\_mono'{]}, 'VO2M2': {[}'VO2M2', {[}4.5546, 4.5546, 2.8514, 90.0, 90, 90.0{]}, 'no'{]}, 'VO2R': {[}'VO2R', {[}4.5546, 4.5546, 2.8514, 90.0, 90, 90.0{]}, 'rutile'{]}, 'W': {[}'W', {[}3.1652, 3.1652, 3.1652, 90, 90, 90{]}, 'bcc'{]}, 'YAG': {[}'YAG', {[}9.2, 9.2, 9.2, 90, 90, 90{]}, 'no'{]}, 'ZnO': {[}'ZnO', {[}3.252, 3.252, 5.213, 90, 90, 120{]}, 'wurtzite'{]}, 'ZrO2': {[}'ZrO2', {[}5.1505, 5.2116, 5.3173, 90, 99.23, 90{]}, 'VO2\_mono'{]}, 'ZrO2Y2O3': {[}'ZrO2Y2O3', {[}5.1378, 5.1378, 5.1378, 90, 90, 90{]}, 'fcc'{]}, 'alphaQuartz': {[}'alphaQuartz', {[}4.9, 4.9, 5.4, 90, 90, 120{]}, 'no'{]}, 'betaQuartznew': {[}'betaQuartznew', {[}4.9, 4.9, 6.685, 90, 90, 120{]}, 'no'{]}, 'bigpro': {[}'bigpro', {[}112.0, 112.0, 136.0, 90, 90, 90{]}, 'no'{]}, 'dummy': {[}'dummy', {[}4.0, 8.0, 2.0, 90, 90, 90{]}, 'no'{]}, 'ferrydrite': {[}'ferrydrite', {[}2.96, 2.96, 9.4, 90, 90, 120{]}, 'no'{]}, 'hexagonal': {[}'hexagonal', {[}1.0, 1.0, 3.0, 90, 90, 120.0{]}, 'no'{]}, 'inputB': {[}'inputB', {[}1.0, 1.0, 1.0, 90, 90, 90{]}, 'no'{]}, 'quartz\_alpha': {[}'quartz\_alpha', {[}4.913, 4.913, 5.404, 90, 90, 120{]}, 'no'{]}, 'smallpro': {[}'smallpro', {[}20.0, 4.8, 49.0, 90, 90, 90{]}, 'no'{]}, 'test\_reference': {[}'test\_reference', {[}3.2, 4.5, 5.2, 83, 92.0, 122{]}, 'wurtzite'{]}, 'test\_solution': {[}'test\_solution', {[}3.252, 4.48, 5.213, 83.2569, 92.125478, 122.364{]}, 'wurtzite'{]}, 'testindex': {[}'testindex', {[}2.0, 1.0, 4.0, 90, 90, 90{]}, 'no'{]}, 'testindex2': {[}'testindex2', {[}2.0, 1.0, 4.0, 75, 90, 120{]}, 'no'{]}\}}}{}
create grain parameters list for the Laue pattern simulation

Can handle material defined in dictionary by four elements instead of 6 lattice parameters
\begin{quote}\begin{description}
\item[{Parameters}] \leavevmode
\sphinxstyleliteralstrong{\sphinxupquote{key\_material}} (\sphinxstyleliteralemphasis{\sphinxupquote{string}}) \textendash{} material or structure label

\item[{Returns}] \leavevmode
grain (4 elements list),  contains\_U (boolean)

\end{description}\end{quote}

\end{fulllineitems}

\index{Prepare\_Grain() (in module LaueTools.CrystalParameters)}

\begin{fulllineitems}
\phantomsection\label{\detokenize{Simulation_Module:LaueTools.CrystalParameters.Prepare_Grain}}\pysiglinewithargsret{\sphinxcode{\sphinxupquote{LaueTools.CrystalParameters.}}\sphinxbfcode{\sphinxupquote{Prepare\_Grain}}}{\emph{key\_material, OrientMatrix, force\_extinction=None, dictmaterials=\{'Al': {[}'Al', {[}4.05, 4.05, 4.05, 90, 90, 90{]}, 'fcc'{]}, 'Al2Cu': {[}'Al2Cu', {[}6.063, 6.063, 4.872, 90, 90, 90{]}, 'no'{]}, 'Al2O3': {[}'Al2O3', {[}4.785, 4.785, 12.991, 90, 90, 120{]}, 'Al2O3'{]}, 'Al2O3\_all': {[}'Al2O3\_all', {[}4.785, 4.785, 12.991, 90, 90, 120{]}, 'no'{]}, 'AlN': {[}'AlN', {[}3.11, 3.11, 4.98, 90.0, 90.0, 120.0{]}, 'wurtzite'{]}, 'Au': {[}'Au', {[}4.078, 4.078, 4.078, 90, 90, 90{]}, 'fcc'{]}, 'CCDL1949': {[}'CCDL1949', {[}9.89, 17.85, 5.31, 90, 107.5, 90{]}, 'h+k=2n'{]}, 'CdHgTe': {[}'CdHgTe', {[}6.46678, 6.46678, 6.46678, 90, 90, 90{]}, 'dia'{]}, 'CdHgTe\_fcc': {[}'CdHgTe\_fcc', {[}6.46678, 6.46678, 6.46678, 90, 90, 90{]}, 'fcc'{]}, 'CdTe': {[}'CdTe', {[}6.487, 6.487, 6.487, 90, 90, 90{]}, 'fcc'{]}, 'CdTeDiagB': {[}'CdTeDiagB', {[}4.5721, 7.9191, 11.1993, 90, 90, 90{]}, 'no'{]}, 'Crocidolite': {[}'Crocidolite', {[}9.811, 18.013, 5.326, 90, 103.68, 90{]}, 'no'{]}, 'Crocidolite\_2': {[}'Crocidolite\_2', {[}9.76, 17.93, 5.35, 90, 103.6, 90{]}, 'no'{]}, 'Crocidolite\_2\_72deg': {[}'Crocidolite\_2', {[}9.76, 17.93, 5.35, 90, 76.4, 90{]}, 'no'{]}, 'Crocidolite\_small': {[}'Crocidolite\_small', {[}3.2533333333333334, 5.976666666666667, 1.7833333333333332, 90, 103.6, 90{]}, 'no'{]}, 'Crocidolite\_whittaker\_1949': {[}'Crocidolite\_whittaker\_1949', {[}9.89, 17.85, 5.31, 90, 107.5, 90{]}, 'no'{]}, 'Cu': {[}'Cu', {[}3.6, 3.6, 3.6, 90, 90, 90{]}, 'fcc'{]}, 'Cu6Sn5\_monoclinic': {[}'Cu6Sn5\_monoclinic', {[}11.02, 7.28, 9.827, 90, 98.84, 90{]}, 'no'{]}, 'Cu6Sn5\_tetra': {[}'Cu6Sn5\_tetra', {[}3.608, 3.608, 5.037, 90, 90, 90{]}, 'no'{]}, 'DIA': {[}'DIA', {[}5.0, 5.0, 5.0, 90, 90, 90{]}, 'dia'{]}, 'DIAs': {[}'DIAs', {[}3.56683, 3.56683, 3.56683, 90, 90, 90{]}, 'dia'{]}, 'DarinaMolecule': {[}'DarinaMolecule', {[}9.4254, 13.5004, 13.8241, 61.83, 84.555, 75.231{]}, 'no'{]}, 'FCC': {[}'FCC', {[}5.0, 5.0, 5.0, 90, 90, 90{]}, 'fcc'{]}, 'Fe': {[}'Fe', {[}2.856, 2.856, 2.856, 90, 90, 90{]}, 'bcc'{]}, 'Fe2Ta': {[}'Fe2Ta', {[}4.83, 4.83, 0.788, 90, 90, 120{]}, 'no'{]}, 'FeAl': {[}'FeAl', {[}5.871, 5.871, 5.871, 90, 90, 90{]}, 'fcc'{]}, 'GaAs': {[}'GaAs', {[}5.65325, 5.65325, 5.65325, 90, 90, 90{]}, 'dia'{]}, 'GaN': {[}'GaN', {[}3.189, 3.189, 5.185, 90, 90, 120{]}, 'wurtzite'{]}, 'GaN\_all': {[}'GaN\_all', {[}3.189, 3.189, 5.185, 90, 90, 120{]}, 'no'{]}, 'Ge': {[}'Ge', {[}5.6575, 5.6575, 5.6575, 90, 90, 90{]}, 'dia'{]}, 'Ge\_compressedhydro': {[}'Ge\_compressedhydro', {[}5.64, 5.64, 5.64, 90, 90, 90.0{]}, 'dia'{]}, 'Ge\_s': {[}'Ge\_s', {[}5.6575, 5.6575, 5.6575, 90, 90, 89.5{]}, 'dia'{]}, 'Getest': {[}'Getest', {[}5.6575, 5.6575, 5.6574, 90, 90, 90{]}, 'dia'{]}, 'Hematite': {[}'Hematite', {[}5.03459, 5.03459, 13.7533, 90, 90, 120{]}, 'no'{]}, 'In': {[}'In', {[}3.2517, 3.2517, 4.9459, 90, 90, 90{]}, 'h+k+l=2n'{]}, 'InGaN': {[}'InGaN', {[}3.3609999999999998, 3.3609999999999998, 5.439, 90, 90, 120{]}, 'wurtzite'{]}, 'InN': {[}'InN', {[}3.533, 3.533, 5.693, 90, 90, 120{]}, 'wurtzite'{]}, 'Magnetite': {[}'Magnetite', {[}8.391, 8.391, 8.391, 90, 90, 90{]}, 'dia'{]}, 'Magnetite\_fcc': {[}'Magnetite\_fcc', {[}8.391, 8.391, 8.391, 90, 90, 90{]}, 'fcc'{]}, 'Magnetite\_sc': {[}'Magnetite\_sc', {[}8.391, 8.391, 8.391, 90, 90, 90{]}, 'no'{]}, 'Nd45': {[}'Nd45', {[}5.4884, 5.4884, 5.4884, 90, 90, 90{]}, 'fcc'{]}, 'Ni': {[}'Ni', {[}3.5238, 3.5238, 3.5238, 90, 90, 90{]}, 'fcc'{]}, 'NiO': {[}'NiO', {[}2.96, 2.96, 7.23, 90, 90, 120{]}, 'no'{]}, 'NiTi': {[}'NiTi', {[}3.5506, 3.5506, 3.5506, 90, 90, 90{]}, 'fcc'{]}, 'SC': {[}'SC', {[}1.0, 1.0, 1.0, 90, 90, 90{]}, 'no'{]}, 'SC5': {[}'SC5', {[}5.0, 5.0, 5.0, 90, 90, 90{]}, 'no'{]}, 'SC7': {[}'SC7', {[}7.0, 7.0, 7.0, 90, 90, 90{]}, 'no'{]}, 'Sb': {[}'Sb', {[}4.3, 4.3, 11.3, 90, 90, 120{]}, 'no'{]}, 'Si': {[}'Si', {[}5.4309, 5.4309, 5.4309, 90, 90, 90{]}, 'dia'{]}, 'Sn': {[}'Sn', {[}5.83, 5.83, 3.18, 90, 90, 90{]}, 'h+k+l=2n'{]}, 'Ti': {[}'Ti', {[}2.95, 2.95, 4.68, 90, 90, 120{]}, 'no'{]}, 'Ti2AlN': {[}'Ti2AlN', {[}2.989, 2.989, 13.624, 90, 90, 120{]}, 'Ti2AlN'{]}, 'Ti2AlN\_w': {[}'Ti2AlN\_w', {[}2.989, 2.989, 13.624, 90, 90, 120{]}, 'wurtzite'{]}, 'Ti\_beta': {[}'Ti\_beta', {[}3.2587, 3.2587, 3.2587, 90, 90, 90{]}, 'bcc'{]}, 'Ti\_omega': {[}'Ti\_omega', {[}4.6085, 4.6085, 2.8221, 90, 90, 120{]}, 'no'{]}, 'Ti\_s': {[}'Ti\_s', {[}3.0, 3.0, 4.7, 90.5, 89.5, 120.5{]}, 'no'{]}, 'UO2': {[}'UO2', {[}5.47, 5.47, 5.47, 90, 90, 90{]}, 'fcc'{]}, 'VO2M1': {[}'VO2M1', {[}5.75175, 4.52596, 5.38326, 90.0, 122.6148, 90.0{]}, 'VO2\_mono'{]}, 'VO2M2': {[}'VO2M2', {[}4.5546, 4.5546, 2.8514, 90.0, 90, 90.0{]}, 'no'{]}, 'VO2R': {[}'VO2R', {[}4.5546, 4.5546, 2.8514, 90.0, 90, 90.0{]}, 'rutile'{]}, 'W': {[}'W', {[}3.1652, 3.1652, 3.1652, 90, 90, 90{]}, 'bcc'{]}, 'YAG': {[}'YAG', {[}9.2, 9.2, 9.2, 90, 90, 90{]}, 'no'{]}, 'ZnO': {[}'ZnO', {[}3.252, 3.252, 5.213, 90, 90, 120{]}, 'wurtzite'{]}, 'ZrO2': {[}'ZrO2', {[}5.1505, 5.2116, 5.3173, 90, 99.23, 90{]}, 'VO2\_mono'{]}, 'ZrO2Y2O3': {[}'ZrO2Y2O3', {[}5.1378, 5.1378, 5.1378, 90, 90, 90{]}, 'fcc'{]}, 'alphaQuartz': {[}'alphaQuartz', {[}4.9, 4.9, 5.4, 90, 90, 120{]}, 'no'{]}, 'betaQuartznew': {[}'betaQuartznew', {[}4.9, 4.9, 6.685, 90, 90, 120{]}, 'no'{]}, 'bigpro': {[}'bigpro', {[}112.0, 112.0, 136.0, 90, 90, 90{]}, 'no'{]}, 'dummy': {[}'dummy', {[}4.0, 8.0, 2.0, 90, 90, 90{]}, 'no'{]}, 'ferrydrite': {[}'ferrydrite', {[}2.96, 2.96, 9.4, 90, 90, 120{]}, 'no'{]}, 'hexagonal': {[}'hexagonal', {[}1.0, 1.0, 3.0, 90, 90, 120.0{]}, 'no'{]}, 'inputB': {[}'inputB', {[}1.0, 1.0, 1.0, 90, 90, 90{]}, 'no'{]}, 'quartz\_alpha': {[}'quartz\_alpha', {[}4.913, 4.913, 5.404, 90, 90, 120{]}, 'no'{]}, 'smallpro': {[}'smallpro', {[}20.0, 4.8, 49.0, 90, 90, 90{]}, 'no'{]}, 'test\_reference': {[}'test\_reference', {[}3.2, 4.5, 5.2, 83, 92.0, 122{]}, 'wurtzite'{]}, 'test\_solution': {[}'test\_solution', {[}3.252, 4.48, 5.213, 83.2569, 92.125478, 122.364{]}, 'wurtzite'{]}, 'testindex': {[}'testindex', {[}2.0, 1.0, 4.0, 90, 90, 90{]}, 'no'{]}, 'testindex2': {[}'testindex2', {[}2.0, 1.0, 4.0, 75, 90, 120{]}, 'no'{]}\}}}{}
Constructor of the grain (crystal) parameters for Laue pattern simulation

if in key\_material definition (see dict\_Materials) orient matrix is missing
(i.e. only lattice parameter are input)
\begin{quote}\begin{description}
\item[{Parameters}] \leavevmode
\sphinxstyleliteralstrong{\sphinxupquote{key\_material}} (\sphinxhref{https://docs.python.org/3/library/stdtypes.html\#str}{\sphinxstyleliteralemphasis{\sphinxupquote{str}}}) \textendash{} material label

\end{description}\end{quote}

then list parameter will consider the provided value of the optional
OrientMatrix argument
\begin{quote}\begin{description}
\item[{Parameters}] \leavevmode
\sphinxstyleliteralstrong{\sphinxupquote{force\_extinction}} (\sphinxhref{https://docs.python.org/3/library/stdtypes.html\#str}{\sphinxstyleliteralemphasis{\sphinxupquote{str}}}) \textendash{} None, use default extinction rules,
otherwise use other extinction correspondoing to the label

\end{description}\end{quote}

\end{fulllineitems}

\index{AngleBetweenNormals() (in module LaueTools.CrystalParameters)}

\begin{fulllineitems}
\phantomsection\label{\detokenize{Simulation_Module:LaueTools.CrystalParameters.AngleBetweenNormals}}\pysiglinewithargsret{\sphinxcode{\sphinxupquote{LaueTools.CrystalParameters.}}\sphinxbfcode{\sphinxupquote{AngleBetweenNormals}}}{\emph{HKL1s}, \emph{HKL2s}, \emph{Gstar}}{}
compute pairwise angles (in degrees) between reflections or lattice plane normals
of two sets according to unit cell metrics Gstar
\begin{quote}\begin{description}
\item[{Parameters}] \leavevmode\begin{itemize}
\item {} 
\sphinxstyleliteralstrong{\sphinxupquote{HKL1s}} \textendash{} list of {[}H1,K1,L1{]}

\item {} 
\sphinxstyleliteralstrong{\sphinxupquote{HKL2s}} \textendash{} list of {[}H2,K2,L2{]}

\item {} 
\sphinxstyleliteralstrong{\sphinxupquote{Gstar}} \textendash{} 3*3 matrix corresponding to reciprocal metric tensor of unit cell (as provided by Gstar\_from\_directlatticeparams())

\end{itemize}

\item[{Returns}] \leavevmode
array of pairwise angles between reflections

\end{description}\end{quote}

\end{fulllineitems}

\index{FilterHarmonics\_2() (in module LaueTools.CrystalParameters)}

\begin{fulllineitems}
\phantomsection\label{\detokenize{Simulation_Module:LaueTools.CrystalParameters.FilterHarmonics_2}}\pysiglinewithargsret{\sphinxcode{\sphinxupquote{LaueTools.CrystalParameters.}}\sphinxbfcode{\sphinxupquote{FilterHarmonics\_2}}}{\emph{hkl}, \emph{return\_indices\_toremove=0}}{}
keep only hkl 3d vectors that are representative of direction nh,nk,nl
for any h,k,l signed integers

It removes only parallel vector but KEEPs antiparallel vectors (vec , -n vec) with n\textgreater{}0
\begin{quote}\begin{description}
\item[{Parameters}] \leavevmode\begin{itemize}
\item {} 
\sphinxstyleliteralstrong{\sphinxupquote{hkl}} \textendash{} array of 3d hkl indices

\item {} 
\sphinxstyleliteralstrong{\sphinxupquote{return\_indices\_toremove}} \textendash{} 1, returns indices of element in hkl that have been removed

\end{itemize}

\end{description}\end{quote}

\end{fulllineitems}

\index{calc\_B\_RR() (in module LaueTools.CrystalParameters)}

\begin{fulllineitems}
\phantomsection\label{\detokenize{Simulation_Module:LaueTools.CrystalParameters.calc_B_RR}}\pysiglinewithargsret{\sphinxcode{\sphinxupquote{LaueTools.CrystalParameters.}}\sphinxbfcode{\sphinxupquote{calc\_B\_RR}}}{\emph{latticeparameters}, \emph{directspace=1}, \emph{setvolume=False}}{}~\begin{itemize}
\item {} 
Calculate B0 matrix (columns = vectors a*,b*,c*) from direct (real) space lattice parameters (directspace=1)

\item {} 
Calculate a matrix (columns = vectors a,b,c) from direct (real) space lattice parameters (directspace=0)

\end{itemize}

\(\boldsymbol q_{ortho}=B_0 {\bf G^*}\) where \({\bf G^*}=h{\bf a^*}+k{\bf b^*}+l{\bf c^*}\)
\begin{quote}\begin{description}
\item[{Parameters}] \leavevmode\begin{itemize}
\item {} 
\sphinxstyleliteralstrong{\sphinxupquote{latticeparameters}} \textendash{} \begin{itemize}
\item {} 
{[}a,b,c, alpha, beta, gamma{]}    (angles are in degrees) if directspace=1

\item {} 
{[}a*,b*,c*, alpha*, beta*, gamma*{]} (angles are in degrees) if directspace=0

\end{itemize}


\item {} 
\sphinxstyleliteralstrong{\sphinxupquote{directspace}} \textendash{} \begin{itemize}
\item {} \begin{description}
\item[{1 (default) converts  (reciprocal) direct lattice parameters}] \leavevmode
to (direct) reciprocal space calculates “B” matrix in the reciprocal space of input latticeparameters

\end{description}

\item {} \begin{description}
\item[{0  converts  (reciprocal) direct lattice parameters to (reciprocal) direct space}] \leavevmode
calculates “B” matrix in same space of  input latticeparameters

\end{description}

\end{itemize}


\item {} 
\sphinxstyleliteralstrong{\sphinxupquote{setvolume}} \textendash{} \begin{itemize}
\item {} 
False, sets direct unit cell volume to the true volume from lattice parameters

\item {} 
1,      sets direct unit cell volume to 1

\item {} 
’a**3’,  sets direct unit cell volume to a**3

\item {} 
’b**3’, sets direct unit cell volume to b**3

\item {} 
’c**3’,  sets direct unit cell volume to c**3

\end{itemize}


\end{itemize}

\item[{Returns}] \leavevmode
B Matrix (triangular up) from  crystal (reciprocal space) frame to orthonormal frame matrix

\item[{Return type}] \leavevmode
numpy array

\end{description}\end{quote}
\begin{description}
\item[{B matrix is used in q=U B G* formula or}] \leavevmode
as B0  in q= (UB) B0 G*

\end{description}

after Busing Levy, Acta Crysta 22 (1967), p 457
\begin{equation*}
\begin{split}\left( \begin{matrix}
a^*  & b^*\cos \gamma^* & c^*\cos beta^*\\
0  & b^*\sin \gamma^* &-c^*\sin \beta^*\cos \alpha\\
0 &  0    &      c^*\sin \beta^*\sin \alpha\\
        \end{matrix} \right)\end{split}
\end{equation*}
with
\begin{equation*}
\begin{split}\cos(\alpha)=(\cos \beta^*\cos \gamma^*-\cos \alpha^*)/(\sin \beta^*\sin \gamma^*)\end{split}
\end{equation*}
and
\begin{equation*}
\begin{split}c^* \sin \beta^* \sin \alpha = 1/c\end{split}
\end{equation*}
\end{fulllineitems}

\index{DeviatoricStrain\_LatticeParams() (in module LaueTools.CrystalParameters)}

\begin{fulllineitems}
\phantomsection\label{\detokenize{Simulation_Module:LaueTools.CrystalParameters.DeviatoricStrain_LatticeParams}}\pysiglinewithargsret{\sphinxcode{\sphinxupquote{LaueTools.CrystalParameters.}}\sphinxbfcode{\sphinxupquote{DeviatoricStrain\_LatticeParams}}}{\emph{newUBmat}, \emph{latticeparams}, \emph{constantlength='a'}}{}
Computes deviatoric strain and new direct (real) lattice parameters
from matrix newUBmat (rotation and deformation)
considering that one lattice length is chosen to be constant

Zero strain corresponds to reference state of input \sphinxtitleref{lattice parameters}
\begin{quote}\begin{description}
\item[{Parameters}] \leavevmode\begin{itemize}
\item {} 
\sphinxstyleliteralstrong{\sphinxupquote{newUBmat}} \textendash{} (3x3) matrix operator including rotation and deformation

\item {} 
\sphinxstyleliteralstrong{\sphinxupquote{latticeparams}} \textendash{} 6 lattice parameters  {[}a,b,c,:math:\sphinxtitleref{alpha, beta, gamma}{]} in Angstrom and degrees

\item {} 
\sphinxstyleliteralstrong{\sphinxupquote{constantlength}} \textendash{} ‘a’,’b’, or ‘c’ to set one length according to the value in \sphinxtitleref{latticeparams}

\end{itemize}

\item[{Returns}] \leavevmode
\begin{itemize}
\item {} 
3x3 deviatoric strain tensor)

\item {} 
lattice\_parameter\_direct\_strain (direct (real) lattice parameters)

\end{itemize}


\item[{Return type}] \leavevmode
3x3 numpy array, 6 elements list

\end{description}\end{quote}

\begin{sphinxadmonition}{note}{Note:}\begin{itemize}
\item {} 
q = newUBmat . B0 . G*  where B0 (triangular up matrix) comes from lattice parameters input.

\item {} 
equivalently, q = UBstar\_s . G*

\end{itemize}
\end{sphinxadmonition}

\end{fulllineitems}

\index{VolumeCell() (in module LaueTools.CrystalParameters)}

\begin{fulllineitems}
\phantomsection\label{\detokenize{Simulation_Module:LaueTools.CrystalParameters.VolumeCell}}\pysiglinewithargsret{\sphinxcode{\sphinxupquote{LaueTools.CrystalParameters.}}\sphinxbfcode{\sphinxupquote{VolumeCell}}}{\emph{latticeparameters}}{}
Computes unit cell volume from lattice parameters (either real or reciprocal)
\begin{quote}\begin{description}
\item[{Parameters}] \leavevmode
\sphinxstyleliteralstrong{\sphinxupquote{latticeparameters}} \textendash{} 6 lattice parameters

\item[{Returns}] \leavevmode
scalar volume

\end{description}\end{quote}

\end{fulllineitems}

\index{VolumeCell() (in module LaueTools.CrystalParameters)}

\begin{fulllineitems}
\pysiglinewithargsret{\sphinxcode{\sphinxupquote{LaueTools.CrystalParameters.}}\sphinxbfcode{\sphinxupquote{VolumeCell}}}{\emph{latticeparameters}}{}
Computes unit cell volume from lattice parameters (either real or reciprocal)
\begin{quote}\begin{description}
\item[{Parameters}] \leavevmode
\sphinxstyleliteralstrong{\sphinxupquote{latticeparameters}} \textendash{} 6 lattice parameters

\item[{Returns}] \leavevmode
scalar volume

\end{description}\end{quote}

\end{fulllineitems}

\index{matrix\_to\_rlat() (in module LaueTools.CrystalParameters)}

\begin{fulllineitems}
\phantomsection\label{\detokenize{Simulation_Module:LaueTools.CrystalParameters.matrix_to_rlat}}\pysiglinewithargsret{\sphinxcode{\sphinxupquote{LaueTools.CrystalParameters.}}\sphinxbfcode{\sphinxupquote{matrix\_to\_rlat}}}{\emph{mat}, \emph{angles\_in\_deg=1}}{}
Returns RECIPROCAL lattice parameters of the unit cell a*,b*,c* in columns of \sphinxtitleref{mat}
\begin{quote}\begin{description}
\item[{Parameters}] \leavevmode
\sphinxstyleliteralstrong{\sphinxupquote{mat}} \textendash{} matrix where columns are respectively a*,b*,c* coordinates in orthonormal frame

\item[{Returns}] \leavevmode
{[}a*,b*,c*, alpha*, beta*, gamma*{]} (angles are in degrees)

\end{description}\end{quote}

\begin{sphinxadmonition}{note}{Note:}
Reciprocal lattice parameters are contained in UB matrix : q =  mat G*
\end{sphinxadmonition}

\end{fulllineitems}



\subsection{Laue Pattern Simulation}
\label{\detokenize{Simulation_Module:laue-pattern-simulation}}\label{\detokenize{Simulation_Module:module-LaueTools.lauecore}}\index{LaueTools.lauecore (module)}
Core module to compute Laue Pattern in various geometry

Main author is J. S. Micha:   micha {[}at{]} esrf {[}dot{]} fr

version July 2019
from LaueTools package hosted in

\sphinxurl{http://sourceforge.net/projects/lauetools/}

or

\sphinxurl{https://gitlab.esrf.fr/micha/lauetools}
\index{Quicklist() (in module LaueTools.lauecore)}

\begin{fulllineitems}
\phantomsection\label{\detokenize{Simulation_Module:LaueTools.lauecore.Quicklist}}\pysiglinewithargsret{\sphinxcode{\sphinxupquote{LaueTools.lauecore.}}\sphinxbfcode{\sphinxupquote{Quicklist}}}{\emph{OrientMatrix}, \emph{ReciprocBasisVectors}, \emph{listRSnorm}, \emph{lambdamin}, \emph{verbose=0}}{}
return 6 indices min and max boundary values for each Miller index h, k, l
to be contained in the largest Ewald Sphere.
\begin{quote}\begin{description}
\item[{Parameters}] \leavevmode\begin{itemize}
\item {} 
\sphinxstyleliteralstrong{\sphinxupquote{OrientMatrix}} \textendash{} orientation matrix (3*3 matrix)

\item {} 
\sphinxstyleliteralstrong{\sphinxupquote{ReciprocBasisVectors}} \textendash{} list of the three vectors a*,b*,c* in the lab frame
before rotation with OrientMatrix

\item {} 
\sphinxstyleliteralstrong{\sphinxupquote{listRSnorm}} \textendash{} : list of the three reciprocal space lengthes of a*,b*,c*

\item {} 
\sphinxstyleliteralstrong{\sphinxupquote{lambdamin}} \textendash{} : lambdamin (in Angstrom) corresponding to energy max

\end{itemize}

\item[{Returns}] \leavevmode
{[}{[}hmin,hmax{]},{[}kmin,kmax{]},{[}lmin,lmax{]}{]}

\end{description}\end{quote}

\end{fulllineitems}

\index{genHKL\_np() (in module LaueTools.lauecore)}

\begin{fulllineitems}
\phantomsection\label{\detokenize{Simulation_Module:LaueTools.lauecore.genHKL_np}}\pysiglinewithargsret{\sphinxcode{\sphinxupquote{LaueTools.lauecore.}}\sphinxbfcode{\sphinxupquote{genHKL\_np}}}{\emph{listn}, \emph{Extinc}}{}
Generate all Miller indices hkl from indices limits given  by listn
and taking into account for systematic exctinctions
\begin{quote}\begin{description}
\item[{Parameters}] \leavevmode\begin{itemize}
\item {} 
\sphinxstyleliteralstrong{\sphinxupquote{listn}} (\sphinxstyleliteralemphasis{\sphinxupquote{{[}}}\sphinxstyleliteralemphasis{\sphinxupquote{{[}}}\sphinxstyleliteralemphasis{\sphinxupquote{hmin}}\sphinxstyleliteralemphasis{\sphinxupquote{,}}\sphinxstyleliteralemphasis{\sphinxupquote{hmax}}\sphinxstyleliteralemphasis{\sphinxupquote{{]}}}\sphinxstyleliteralemphasis{\sphinxupquote{,}}\sphinxstyleliteralemphasis{\sphinxupquote{{[}}}\sphinxstyleliteralemphasis{\sphinxupquote{kmin}}\sphinxstyleliteralemphasis{\sphinxupquote{,}}\sphinxstyleliteralemphasis{\sphinxupquote{kmax}}\sphinxstyleliteralemphasis{\sphinxupquote{{]}}}\sphinxstyleliteralemphasis{\sphinxupquote{,}}\sphinxstyleliteralemphasis{\sphinxupquote{{[}}}\sphinxstyleliteralemphasis{\sphinxupquote{lmin}}\sphinxstyleliteralemphasis{\sphinxupquote{,}}\sphinxstyleliteralemphasis{\sphinxupquote{lmax}}\sphinxstyleliteralemphasis{\sphinxupquote{{]}}}\sphinxstyleliteralemphasis{\sphinxupquote{{]}}}) \textendash{} Miller indices limits (warning: these lists are used in python range (last index is excluded))

\item {} 
\sphinxstyleliteralstrong{\sphinxupquote{Extinc}} (\sphinxstyleliteralemphasis{\sphinxupquote{string}}) \textendash{} label corresponding to systematic exctinction
rules on h k and l miller indics such as (‘fcc’, ‘bcc’, ‘dia’, …) or ‘no’ for any rules

\end{itemize}

\item[{Returns}] \leavevmode
array of {[}h,k,l{]}

\end{description}\end{quote}

\begin{sphinxadmonition}{note}{Note:}
node {[}0,0,0{]} is excluded
\end{sphinxadmonition}

\end{fulllineitems}

\index{getLaueSpots() (in module LaueTools.lauecore)}

\begin{fulllineitems}
\phantomsection\label{\detokenize{Simulation_Module:LaueTools.lauecore.getLaueSpots}}\pysiglinewithargsret{\sphinxcode{\sphinxupquote{LaueTools.lauecore.}}\sphinxbfcode{\sphinxupquote{getLaueSpots}}}{\emph{wavelmin}, \emph{wavelmax}, \emph{crystalsParams}, \emph{linestowrite}, \emph{kf\_direction='Z\textgreater{}0'}, \emph{OpeningAngleCollection=22.0}, \emph{fastcompute=0}, \emph{ResolutionAngstrom=False}, \emph{fileOK=1}, \emph{verbose=1}, \emph{dictmaterials=None}}{}
Compute Qxyz vectors and corresponding HKL miller indices for nodes in recicprocal space that can be measured
for the given detection geometry and energy bandpass configuration.
\begin{quote}\begin{description}
\item[{Parameters}] \leavevmode\begin{itemize}
\item {} 
\sphinxstyleliteralstrong{\sphinxupquote{wavelmin}} \textendash{} smallest wavelength in Angstrom

\item {} 
\sphinxstyleliteralstrong{\sphinxupquote{wavelmax}} \textendash{} largest wavelength in Angstrom

\item {} 
\sphinxstyleliteralstrong{\sphinxupquote{crystalsParams}} \textendash{} 
list of \sphinxstyleemphasis{SingleCrystalParams}, each of them being a list
of 4 elements for crystal orientation and strain properties:
\begin{itemize}
\item {} \begin{description}
\item[{{[}0{]}(array): is the B matrix a*,b*,c* vectors are expressed in column}] \leavevmode
in LaueTools frame in reciprocal angstrom units

\end{description}

\item {} 
{[}1{]}(str): peak Extinction rules (‘no’,’fcc’,’dia’, etc…)

\item {} 
{[}2{]}(array): orientation matrix

\item {} 
{[}3{]}(str): key for material element

\end{itemize}


\item {} 
\sphinxstyleliteralstrong{\sphinxupquote{kf\_direction}} \textendash{} 
string defining the average geometry, mean value of exit scattered vector:
‘Z\textgreater{}0’   top spots

’Y\textgreater{}0’   one side spots (towards hutch door)

’Y\textless{}0’   other side spots

’X\textgreater{}0’   transmission spots

’X\textless{}0’   backreflection spots


\item {} 
\sphinxstyleliteralstrong{\sphinxupquote{fastcompute}} \textendash{} \begin{itemize}
\item {} 
1, compute reciprocal space (RS) vector BUT NOT the Miller indices

\item {} 
0, returns both RS vectors (normalised) and Miller indices

\end{itemize}


\item {} 
\sphinxstyleliteralstrong{\sphinxupquote{ResolutionAngstrom}} \textendash{} \begin{itemize}
\item {} 
scalar, smallest interplanar distance ordered in crystal in angstrom.

\item {} 
None, all reflections will be calculated that can be time-consuming for large unit cell

\end{itemize}


\item {} 
\sphinxstyleliteralstrong{\sphinxupquote{linestowrite}} \textendash{} list of {[}string{]} that can be write in file or display in
stdout. Example: {[}{[}“”{]}{]} or {[}{[}“\sphinxstylestrong{******}”{]},{[}“lauetools”{]}{]}

\end{itemize}

\item[{Returns}] \leavevmode
\begin{itemize}
\item {} 
list of {[}Qx,Qy,Qz{]}s for each grain, list of {[}H,K,L{]}s for each grain (fastcompute = 0)

\item {} 
list of {[}Qx,Qy,Qz{]}s for each grain, None  (fastcompute = 1)

\end{itemize}


\end{description}\end{quote}

\begin{sphinxadmonition}{caution}{Caution:}
This method doesn’t create spot instances.

This is done in filterLaueSpots with fastcompute = 0
\end{sphinxadmonition}

\begin{sphinxadmonition}{caution}{Caution:}
finer selection of nodes : on camera , without harmonics can be
done later with filterLaueSpots()
\end{sphinxadmonition}

\begin{sphinxadmonition}{note}{Note:}
lauetools laboratory frame is in this case:
x// ki (center of ewald sphere has neagtive x component)
z perp to x and belonging to the plane defined by x and dd vectors
(where dd vector is the smallest vector joining sample impact point and points on CCD plane)
y is perpendicular to x and z
\end{sphinxadmonition}

\end{fulllineitems}

\index{create\_spot() (in module LaueTools.lauecore)}

\begin{fulllineitems}
\phantomsection\label{\detokenize{Simulation_Module:LaueTools.lauecore.create_spot}}\pysiglinewithargsret{\sphinxcode{\sphinxupquote{LaueTools.lauecore.}}\sphinxbfcode{\sphinxupquote{create\_spot}}}{\emph{pos\_vec}, \emph{miller}, \emph{detectordistance}, \emph{allattributes=False}, \emph{pixelsize=0.08056640625}, \emph{dim=(2048}, \emph{2048)}}{}
From reciprocal space position and 3 miller indices
create a spot instance (on top camera geometry)
\begin{quote}\begin{description}
\item[{Parameters}] \leavevmode\begin{itemize}
\item {} 
\sphinxstyleliteralstrong{\sphinxupquote{pos\_vec}} (\sphinxstyleliteralemphasis{\sphinxupquote{list of 3 float}}) \textendash{} 3D vector

\item {} 
\sphinxstyleliteralstrong{\sphinxupquote{miller}} \textendash{} list of 3 miller indices

\item {} 
\sphinxstyleliteralstrong{\sphinxupquote{detectordistance}} \textendash{} approximate distance detector sample (to compute complementary spots attributes)

\item {} 
\sphinxstyleliteralstrong{\sphinxupquote{allattributes}} \textendash{} False or 0  not to compute complementary spot attributes

\item {} 
\sphinxstyleliteralstrong{\sphinxupquote{allattributes}} \textendash{} boolean

\end{itemize}

\item[{Returns}] \leavevmode
spot instance

\end{description}\end{quote}

\begin{sphinxadmonition}{note}{Note:}
spot.Qxyz is a vector expressed in lauetools frame
\end{sphinxadmonition}

X along x-ray and Z towards CCD when CCD on top and y towards experimental hutch door

\end{fulllineitems}

\index{create\_spot\_np() (in module LaueTools.lauecore)}

\begin{fulllineitems}
\phantomsection\label{\detokenize{Simulation_Module:LaueTools.lauecore.create_spot_np}}\pysiglinewithargsret{\sphinxcode{\sphinxupquote{LaueTools.lauecore.}}\sphinxbfcode{\sphinxupquote{create\_spot\_np}}}{\emph{Qxyz}, \emph{miller}, \emph{detectordistance}, \emph{allattributes=False}, \emph{pixelsize=0.08056640625}, \emph{dim=(2048}, \emph{2048)}}{}
From reciprocal space position and 3 miller indices
create a spot instance (on top camera geometry)
\begin{quote}\begin{description}
\item[{Parameters}] \leavevmode\begin{itemize}
\item {} 
\sphinxstyleliteralstrong{\sphinxupquote{pos\_vec}} (\sphinxstyleliteralemphasis{\sphinxupquote{list of 3 float}}) \textendash{} 3D vector

\item {} 
\sphinxstyleliteralstrong{\sphinxupquote{miller}} \textendash{} list of 3 miller indices

\item {} 
\sphinxstyleliteralstrong{\sphinxupquote{detectordistance}} \textendash{} approximate distance detector sample (to compute complementary spots attributes)

\item {} 
\sphinxstyleliteralstrong{\sphinxupquote{allattributes}} \textendash{} False or 0  not to compute complementary spot attributes

\item {} 
\sphinxstyleliteralstrong{\sphinxupquote{allattributes}} \textendash{} boolean

\end{itemize}

\item[{Returns}] \leavevmode
spot instance

\end{description}\end{quote}

\begin{sphinxadmonition}{note}{Note:}
spot.Qxyz is a vector expressed in lauetools frame
\end{sphinxadmonition}

X along x-ray and Z towards CCD when CCD on top and y towards experimental hutch door

\end{fulllineitems}

\index{filterLaueSpots() (in module LaueTools.lauecore)}

\begin{fulllineitems}
\phantomsection\label{\detokenize{Simulation_Module:LaueTools.lauecore.filterLaueSpots}}\pysiglinewithargsret{\sphinxcode{\sphinxupquote{LaueTools.lauecore.}}\sphinxbfcode{\sphinxupquote{filterLaueSpots}}}{\emph{vec\_and\_indices, HarmonicsRemoval=1, fastcompute=0, kf\_direction='Z\textgreater{}0', fileOK=0, detectordistance=70.0, detectordiameter=165.0, pixelsize=0.08056640625, dim=(2048, 2048), linestowrite={[}{[}''{]}{]}}}{}
Calculates list of grains spots on camera and without harmonics
and on CCD camera from {[}{[}spots grain 0{]},{[}spots grain 1{]},etc{]} =\textgreater{}
returns {[}{[}spots grain 0{]},{[}spots grain 1{]},etc{]} w / o harmonics and on camera  CCD
\begin{quote}\begin{description}
\item[{Parameters}] \leavevmode\begin{itemize}
\item {} 
\sphinxstyleliteralstrong{\sphinxupquote{vec\_and\_indices}} \textendash{} 
list of elements corresponding to 1 grain, each element is composed by
* {[}0{]} array of vector
\begin{itemize}
\item {} 
{[}1{]} array of indices

\end{itemize}


\item {} 
\sphinxstyleliteralstrong{\sphinxupquote{HarmonicsRemoval}} \textendash{} 1, removes harmonics according to their miller indices
(only for fastcompute = 0)

\item {} 
\sphinxstyleliteralstrong{\sphinxupquote{fastcompute}} \textendash{} \begin{itemize}
\item {} \begin{description}
\item[{1, outputs a list for each grain of 2theta spots and a list for each grain of chi spots}] \leavevmode
(HARMONICS spots are still HERE!)

\end{description}

\item {} 
0, outputs list for each grain of spots with

\end{itemize}


\item {} 
\sphinxstyleliteralstrong{\sphinxupquote{kf\_direction}} (\sphinxstyleliteralemphasis{\sphinxupquote{string}}) \textendash{} label for detection geometry (CCD plane with respect to the incoming beam and sample)

\end{itemize}

\item[{Returns}] \leavevmode
\begin{itemize}
\item {} 
list of spot instances if fastcompute=0

\item {} 
2theta, chi          if fastcompute=1

\end{itemize}


\end{description}\end{quote}

\begin{sphinxadmonition}{note}{Note:}\begin{itemize}
\item {} 
USED IMPORTANTLY in lauecore.SimulateResults  lauecore.SimulateLaue

\item {} 
USED in matchingrate.AngularResidues

\item {} 
USED in ParametricLaueSimulator.dosimulation\_parametric

\item {} 
USED in AutoindexationGUI.OnSimulate\_S3, DetectorCalibration.Reckon\_2pts, and others

\end{itemize}
\end{sphinxadmonition}

\begin{sphinxadmonition}{note}{\label{Simulation_Module:index-0}Todo:}
add dim in create\_spot in various geometries
\end{sphinxadmonition}

\end{fulllineitems}

\index{get2ThetaChi\_geometry() (in module LaueTools.lauecore)}

\begin{fulllineitems}
\phantomsection\label{\detokenize{Simulation_Module:LaueTools.lauecore.get2ThetaChi_geometry}}\pysiglinewithargsret{\sphinxcode{\sphinxupquote{LaueTools.lauecore.}}\sphinxbfcode{\sphinxupquote{get2ThetaChi\_geometry}}}{\emph{oncam\_vec}, \emph{oncam\_HKL}, \emph{detectordistance=70.0}, \emph{pixelsize=0.08056640625}, \emph{dim=(2048}, \emph{2048)}, \emph{kf\_direction='Z\textgreater{}0'}}{}
computes list of spots instances from oncam\_vec (q 3D vectors)
and oncam\_HKL (miller indices 3D vectors)
\begin{quote}\begin{description}
\item[{Parameters}] \leavevmode\begin{itemize}
\item {} 
\sphinxstyleliteralstrong{\sphinxupquote{oncam\_vec}} (\sphinxstyleliteralemphasis{\sphinxupquote{array with 3D elements}}\sphinxstyleliteralemphasis{\sphinxupquote{ (}}\sphinxstyleliteralemphasis{\sphinxupquote{shape =}}\sphinxstyleliteralemphasis{\sphinxupquote{ (}}\sphinxstyleliteralemphasis{\sphinxupquote{n}}\sphinxstyleliteralemphasis{\sphinxupquote{,}}\sphinxstyleliteralemphasis{\sphinxupquote{3}}\sphinxstyleliteralemphasis{\sphinxupquote{)}}\sphinxstyleliteralemphasis{\sphinxupquote{)}}) \textendash{} q vectors {[}qx,qy,qz{]} (corresponding to kf collected on camera)

\item {} 
\sphinxstyleliteralstrong{\sphinxupquote{dim}} (\sphinxhref{https://docs.python.org/3/library/stdtypes.html\#list}{\sphinxstyleliteralemphasis{\sphinxupquote{list}}}\sphinxstyleliteralemphasis{\sphinxupquote{ or }}\sphinxstyleliteralemphasis{\sphinxupquote{tuple of 2 integers}}) \textendash{} CCD frame dimensions (nb pixels, nb pixels)

\item {} 
\sphinxstyleliteralstrong{\sphinxupquote{detectordistance}} \textendash{} approximate distance detector sample

\item {} 
\sphinxstyleliteralstrong{\sphinxupquote{detectordistance}} \textendash{} float or integer

\item {} 
\sphinxstyleliteralstrong{\sphinxupquote{pixelsize}} (\sphinxhref{https://docs.python.org/3/library/functions.html\#float}{\sphinxstyleliteralemphasis{\sphinxupquote{float}}}) \textendash{} pixel size in mm

\end{itemize}

\item[{Param}] \leavevmode
kf\_direction : label for detection geometry
(CCD plane with respect to
the incoming beam and sample)

\item[{Type}] \leavevmode
kf\_direction: string

\item[{Returns}] \leavevmode
list of spot instances

\end{description}\end{quote}

\begin{sphinxadmonition}{note}{Note:}
USED in lauecore.filterLaueSpots
\end{sphinxadmonition}

\begin{sphinxadmonition}{note}{\label{Simulation_Module:index-1}Todo:}\begin{itemize}
\item {} 
to be replaced by something else not using spot class

\item {} 
put this function in LaueGeometry module ?

\end{itemize}
\end{sphinxadmonition}

\end{fulllineitems}

\index{calcSpots\_fromHKLlist() (in module LaueTools.lauecore)}

\begin{fulllineitems}
\phantomsection\label{\detokenize{Simulation_Module:LaueTools.lauecore.calcSpots_fromHKLlist}}\pysiglinewithargsret{\sphinxcode{\sphinxupquote{LaueTools.lauecore.}}\sphinxbfcode{\sphinxupquote{calcSpots\_fromHKLlist}}}{\emph{UB}, \emph{B0}, \emph{HKL}, \emph{dictCCD}}{}
computes all Laue Spots properties on 2D detector from a list of hkl
(given structure by B0 matrix, orientation by UB matrix, and detector geometry by dictCCD)
\begin{quote}\begin{description}
\item[{Parameters}] \leavevmode\begin{itemize}
\item {} 
\sphinxstyleliteralstrong{\sphinxupquote{UB}} (\sphinxstyleliteralemphasis{\sphinxupquote{3x3 array}}\sphinxstyleliteralemphasis{\sphinxupquote{ (or }}\sphinxhref{https://docs.python.org/3/library/stdtypes.html\#list}{\sphinxstyleliteralemphasis{\sphinxupquote{list}}}\sphinxstyleliteralemphasis{\sphinxupquote{)}}) \textendash{} orientation matrix (rotation -and if any- strain)

\item {} 
\sphinxstyleliteralstrong{\sphinxupquote{B0}} (\sphinxstyleliteralemphasis{\sphinxupquote{3x3 array}}\sphinxstyleliteralemphasis{\sphinxupquote{ (or }}\sphinxhref{https://docs.python.org/3/library/stdtypes.html\#list}{\sphinxstyleliteralemphasis{\sphinxupquote{list}}}\sphinxstyleliteralemphasis{\sphinxupquote{)}}) \textendash{} initial a*,b*,c* reciprocal unit cell basis vector in Lauetools frame (x// ki))

\item {} 
\sphinxstyleliteralstrong{\sphinxupquote{HKL}} (\sphinxstyleliteralemphasis{\sphinxupquote{array with shape =}}\sphinxstyleliteralemphasis{\sphinxupquote{ (}}\sphinxstyleliteralemphasis{\sphinxupquote{n}}\sphinxstyleliteralemphasis{\sphinxupquote{,}}\sphinxstyleliteralemphasis{\sphinxupquote{3}}\sphinxstyleliteralemphasis{\sphinxupquote{)}}) \textendash{} array of Miller indices

\item {} 
\sphinxstyleliteralstrong{\sphinxupquote{dictCCD}} \textendash{} dictionnary of CCD properties (with key ‘CCDparam’, ‘pixelsize’,’dim’)
for ‘ccdparam’ 5 CCD calibration parameters {[}dd,xcen,ycen,xbet,xgam{]}, pixelsize in mm, and (dim1, dim2)

\item {} 
\sphinxstyleliteralstrong{\sphinxupquote{dictCCD}} \textendash{} dict object

\end{itemize}

\item[{Returns}] \leavevmode
list of arrays H, K, L, Qx, Qy, Qz, X, Y, twthe, chi, Energy

\end{description}\end{quote}

Fundamental equation
\({\bf q} = UB*B0 * {\bf G^*}\)
with \({\bf G^*} = h{\bf a^*}+k{\bf b^*}+l{\bf c^*}\)

\begin{sphinxadmonition}{note}{Note:}
USED in DetectorCalibration.OnWriteResults, and PlotRefineGUI.onWriteFitFile
\end{sphinxadmonition}

\end{fulllineitems}

\index{SimulateLaue() (in module LaueTools.lauecore)}

\begin{fulllineitems}
\phantomsection\label{\detokenize{Simulation_Module:LaueTools.lauecore.SimulateLaue}}\pysiglinewithargsret{\sphinxcode{\sphinxupquote{LaueTools.lauecore.}}\sphinxbfcode{\sphinxupquote{SimulateLaue}}}{\emph{grain, emin, emax, detectorparameters, kf\_direction='Z\textgreater{}0', ResolutionAngstrom=False, removeharmonics=0, pixelsize=0.08056640625, dim=(2048, 2048), detectordiameter=None, force\_extinction=None, dictmaterials=\{'Al': {[}'Al', {[}4.05, 4.05, 4.05, 90, 90, 90{]}, 'fcc'{]}, 'Al2Cu': {[}'Al2Cu', {[}6.063, 6.063, 4.872, 90, 90, 90{]}, 'no'{]}, 'Al2O3': {[}'Al2O3', {[}4.785, 4.785, 12.991, 90, 90, 120{]}, 'Al2O3'{]}, 'Al2O3\_all': {[}'Al2O3\_all', {[}4.785, 4.785, 12.991, 90, 90, 120{]}, 'no'{]}, 'AlN': {[}'AlN', {[}3.11, 3.11, 4.98, 90.0, 90.0, 120.0{]}, 'wurtzite'{]}, 'Au': {[}'Au', {[}4.078, 4.078, 4.078, 90, 90, 90{]}, 'fcc'{]}, 'CCDL1949': {[}'CCDL1949', {[}9.89, 17.85, 5.31, 90, 107.5, 90{]}, 'h+k=2n'{]}, 'CdHgTe': {[}'CdHgTe', {[}6.46678, 6.46678, 6.46678, 90, 90, 90{]}, 'dia'{]}, 'CdHgTe\_fcc': {[}'CdHgTe\_fcc', {[}6.46678, 6.46678, 6.46678, 90, 90, 90{]}, 'fcc'{]}, 'CdTe': {[}'CdTe', {[}6.487, 6.487, 6.487, 90, 90, 90{]}, 'fcc'{]}, 'CdTeDiagB': {[}'CdTeDiagB', {[}4.5721, 7.9191, 11.1993, 90, 90, 90{]}, 'no'{]}, 'Crocidolite': {[}'Crocidolite', {[}9.811, 18.013, 5.326, 90, 103.68, 90{]}, 'no'{]}, 'Crocidolite\_2': {[}'Crocidolite\_2', {[}9.76, 17.93, 5.35, 90, 103.6, 90{]}, 'no'{]}, 'Crocidolite\_2\_72deg': {[}'Crocidolite\_2', {[}9.76, 17.93, 5.35, 90, 76.4, 90{]}, 'no'{]}, 'Crocidolite\_small': {[}'Crocidolite\_small', {[}3.2533333333333334, 5.976666666666667, 1.7833333333333332, 90, 103.6, 90{]}, 'no'{]}, 'Crocidolite\_whittaker\_1949': {[}'Crocidolite\_whittaker\_1949', {[}9.89, 17.85, 5.31, 90, 107.5, 90{]}, 'no'{]}, 'Cu': {[}'Cu', {[}3.6, 3.6, 3.6, 90, 90, 90{]}, 'fcc'{]}, 'Cu6Sn5\_monoclinic': {[}'Cu6Sn5\_monoclinic', {[}11.02, 7.28, 9.827, 90, 98.84, 90{]}, 'no'{]}, 'Cu6Sn5\_tetra': {[}'Cu6Sn5\_tetra', {[}3.608, 3.608, 5.037, 90, 90, 90{]}, 'no'{]}, 'DIA': {[}'DIA', {[}5.0, 5.0, 5.0, 90, 90, 90{]}, 'dia'{]}, 'DIAs': {[}'DIAs', {[}3.56683, 3.56683, 3.56683, 90, 90, 90{]}, 'dia'{]}, 'DarinaMolecule': {[}'DarinaMolecule', {[}9.4254, 13.5004, 13.8241, 61.83, 84.555, 75.231{]}, 'no'{]}, 'FCC': {[}'FCC', {[}5.0, 5.0, 5.0, 90, 90, 90{]}, 'fcc'{]}, 'Fe': {[}'Fe', {[}2.856, 2.856, 2.856, 90, 90, 90{]}, 'bcc'{]}, 'Fe2Ta': {[}'Fe2Ta', {[}4.83, 4.83, 0.788, 90, 90, 120{]}, 'no'{]}, 'FeAl': {[}'FeAl', {[}5.871, 5.871, 5.871, 90, 90, 90{]}, 'fcc'{]}, 'GaAs': {[}'GaAs', {[}5.65325, 5.65325, 5.65325, 90, 90, 90{]}, 'dia'{]}, 'GaN': {[}'GaN', {[}3.189, 3.189, 5.185, 90, 90, 120{]}, 'wurtzite'{]}, 'GaN\_all': {[}'GaN\_all', {[}3.189, 3.189, 5.185, 90, 90, 120{]}, 'no'{]}, 'Ge': {[}'Ge', {[}5.6575, 5.6575, 5.6575, 90, 90, 90{]}, 'dia'{]}, 'Ge\_compressedhydro': {[}'Ge\_compressedhydro', {[}5.64, 5.64, 5.64, 90, 90, 90.0{]}, 'dia'{]}, 'Ge\_s': {[}'Ge\_s', {[}5.6575, 5.6575, 5.6575, 90, 90, 89.5{]}, 'dia'{]}, 'Getest': {[}'Getest', {[}5.6575, 5.6575, 5.6574, 90, 90, 90{]}, 'dia'{]}, 'Hematite': {[}'Hematite', {[}5.03459, 5.03459, 13.7533, 90, 90, 120{]}, 'no'{]}, 'In': {[}'In', {[}3.2517, 3.2517, 4.9459, 90, 90, 90{]}, 'h+k+l=2n'{]}, 'InGaN': {[}'InGaN', {[}3.3609999999999998, 3.3609999999999998, 5.439, 90, 90, 120{]}, 'wurtzite'{]}, 'InN': {[}'InN', {[}3.533, 3.533, 5.693, 90, 90, 120{]}, 'wurtzite'{]}, 'Magnetite': {[}'Magnetite', {[}8.391, 8.391, 8.391, 90, 90, 90{]}, 'dia'{]}, 'Magnetite\_fcc': {[}'Magnetite\_fcc', {[}8.391, 8.391, 8.391, 90, 90, 90{]}, 'fcc'{]}, 'Magnetite\_sc': {[}'Magnetite\_sc', {[}8.391, 8.391, 8.391, 90, 90, 90{]}, 'no'{]}, 'Nd45': {[}'Nd45', {[}5.4884, 5.4884, 5.4884, 90, 90, 90{]}, 'fcc'{]}, 'Ni': {[}'Ni', {[}3.5238, 3.5238, 3.5238, 90, 90, 90{]}, 'fcc'{]}, 'NiO': {[}'NiO', {[}2.96, 2.96, 7.23, 90, 90, 120{]}, 'no'{]}, 'NiTi': {[}'NiTi', {[}3.5506, 3.5506, 3.5506, 90, 90, 90{]}, 'fcc'{]}, 'SC': {[}'SC', {[}1.0, 1.0, 1.0, 90, 90, 90{]}, 'no'{]}, 'SC5': {[}'SC5', {[}5.0, 5.0, 5.0, 90, 90, 90{]}, 'no'{]}, 'SC7': {[}'SC7', {[}7.0, 7.0, 7.0, 90, 90, 90{]}, 'no'{]}, 'Sb': {[}'Sb', {[}4.3, 4.3, 11.3, 90, 90, 120{]}, 'no'{]}, 'Si': {[}'Si', {[}5.4309, 5.4309, 5.4309, 90, 90, 90{]}, 'dia'{]}, 'Sn': {[}'Sn', {[}5.83, 5.83, 3.18, 90, 90, 90{]}, 'h+k+l=2n'{]}, 'Ti': {[}'Ti', {[}2.95, 2.95, 4.68, 90, 90, 120{]}, 'no'{]}, 'Ti2AlN': {[}'Ti2AlN', {[}2.989, 2.989, 13.624, 90, 90, 120{]}, 'Ti2AlN'{]}, 'Ti2AlN\_w': {[}'Ti2AlN\_w', {[}2.989, 2.989, 13.624, 90, 90, 120{]}, 'wurtzite'{]}, 'Ti\_beta': {[}'Ti\_beta', {[}3.2587, 3.2587, 3.2587, 90, 90, 90{]}, 'bcc'{]}, 'Ti\_omega': {[}'Ti\_omega', {[}4.6085, 4.6085, 2.8221, 90, 90, 120{]}, 'no'{]}, 'Ti\_s': {[}'Ti\_s', {[}3.0, 3.0, 4.7, 90.5, 89.5, 120.5{]}, 'no'{]}, 'UO2': {[}'UO2', {[}5.47, 5.47, 5.47, 90, 90, 90{]}, 'fcc'{]}, 'VO2M1': {[}'VO2M1', {[}5.75175, 4.52596, 5.38326, 90.0, 122.6148, 90.0{]}, 'VO2\_mono'{]}, 'VO2M2': {[}'VO2M2', {[}4.5546, 4.5546, 2.8514, 90.0, 90, 90.0{]}, 'no'{]}, 'VO2R': {[}'VO2R', {[}4.5546, 4.5546, 2.8514, 90.0, 90, 90.0{]}, 'rutile'{]}, 'W': {[}'W', {[}3.1652, 3.1652, 3.1652, 90, 90, 90{]}, 'bcc'{]}, 'YAG': {[}'YAG', {[}9.2, 9.2, 9.2, 90, 90, 90{]}, 'no'{]}, 'ZnO': {[}'ZnO', {[}3.252, 3.252, 5.213, 90, 90, 120{]}, 'wurtzite'{]}, 'ZrO2': {[}'ZrO2', {[}5.1505, 5.2116, 5.3173, 90, 99.23, 90{]}, 'VO2\_mono'{]}, 'ZrO2Y2O3': {[}'ZrO2Y2O3', {[}5.1378, 5.1378, 5.1378, 90, 90, 90{]}, 'fcc'{]}, 'alphaQuartz': {[}'alphaQuartz', {[}4.9, 4.9, 5.4, 90, 90, 120{]}, 'no'{]}, 'betaQuartznew': {[}'betaQuartznew', {[}4.9, 4.9, 6.685, 90, 90, 120{]}, 'no'{]}, 'bigpro': {[}'bigpro', {[}112.0, 112.0, 136.0, 90, 90, 90{]}, 'no'{]}, 'dummy': {[}'dummy', {[}4.0, 8.0, 2.0, 90, 90, 90{]}, 'no'{]}, 'ferrydrite': {[}'ferrydrite', {[}2.96, 2.96, 9.4, 90, 90, 120{]}, 'no'{]}, 'hexagonal': {[}'hexagonal', {[}1.0, 1.0, 3.0, 90, 90, 120.0{]}, 'no'{]}, 'inputB': {[}'inputB', {[}1.0, 1.0, 1.0, 90, 90, 90{]}, 'no'{]}, 'quartz\_alpha': {[}'quartz\_alpha', {[}4.913, 4.913, 5.404, 90, 90, 120{]}, 'no'{]}, 'smallpro': {[}'smallpro', {[}20.0, 4.8, 49.0, 90, 90, 90{]}, 'no'{]}, 'test\_reference': {[}'test\_reference', {[}3.2, 4.5, 5.2, 83, 92.0, 122{]}, 'wurtzite'{]}, 'test\_solution': {[}'test\_solution', {[}3.252, 4.48, 5.213, 83.2569, 92.125478, 122.364{]}, 'wurtzite'{]}, 'testindex': {[}'testindex', {[}2.0, 1.0, 4.0, 90, 90, 90{]}, 'no'{]}, 'testindex2': {[}'testindex2', {[}2.0, 1.0, 4.0, 75, 90, 120{]}, 'no'{]}\}}}{}~\begin{description}
\item[{Computes Laue Pattern spots positions, scattering angles, miller indices}] \leavevmode
for a SINGLE grain or Xtal

\end{description}
\begin{quote}\begin{description}
\item[{Parameters}] \leavevmode\begin{itemize}
\item {} 
\sphinxstyleliteralstrong{\sphinxupquote{grain}} \textendash{} crystal parameters made of a 4 elements list

\item {} 
\sphinxstyleliteralstrong{\sphinxupquote{emin}} \textendash{} minimum bandpass energy (keV)

\item {} 
\sphinxstyleliteralstrong{\sphinxupquote{emax}} \textendash{} maximum bandpass energy (keV)

\item {} 
\sphinxstyleliteralstrong{\sphinxupquote{removeharmonics}} \textendash{} \begin{itemize}
\item {} \begin{description}
\item[{1, removes harmonics spots and keep fondamental spots (or reciprocal direction)}] \leavevmode
(with lowest Miller indices)

\end{description}

\item {} 
0 keep all spots (including harmonics)

\end{itemize}


\end{itemize}

\item[{Returns}] \leavevmode
single grain data: Twicetheta, Chi, Miller\_ind, posx, posy, Energy

\end{description}\end{quote}

\begin{sphinxadmonition}{note}{\label{Simulation_Module:index-2}Todo:}
To update to accept kf\_direction not only in reflection geometry
\end{sphinxadmonition}

\begin{sphinxadmonition}{note}{Note:}
USED in detectorCalibration…simulate\_theo  for non routine geometry (ie except Z\textgreater{}0 (reflection top) X\textgreater{}0 (transmission)
\end{sphinxadmonition}

\end{fulllineitems}

\index{SimulateLaue\_full\_np() (in module LaueTools.lauecore)}

\begin{fulllineitems}
\phantomsection\label{\detokenize{Simulation_Module:LaueTools.lauecore.SimulateLaue_full_np}}\pysiglinewithargsret{\sphinxcode{\sphinxupquote{LaueTools.lauecore.}}\sphinxbfcode{\sphinxupquote{SimulateLaue\_full\_np}}}{\emph{grain, emin, emax, detectorparameters, kf\_direction='Z\textgreater{}0', ResolutionAngstrom=False, removeharmonics=0, pixelsize=0.08056640625, dim=(2048, 2048), detectordiameter=None, force\_extinction=None, dictmaterials=\{'Al': {[}'Al', {[}4.05, 4.05, 4.05, 90, 90, 90{]}, 'fcc'{]}, 'Al2Cu': {[}'Al2Cu', {[}6.063, 6.063, 4.872, 90, 90, 90{]}, 'no'{]}, 'Al2O3': {[}'Al2O3', {[}4.785, 4.785, 12.991, 90, 90, 120{]}, 'Al2O3'{]}, 'Al2O3\_all': {[}'Al2O3\_all', {[}4.785, 4.785, 12.991, 90, 90, 120{]}, 'no'{]}, 'AlN': {[}'AlN', {[}3.11, 3.11, 4.98, 90.0, 90.0, 120.0{]}, 'wurtzite'{]}, 'Au': {[}'Au', {[}4.078, 4.078, 4.078, 90, 90, 90{]}, 'fcc'{]}, 'CCDL1949': {[}'CCDL1949', {[}9.89, 17.85, 5.31, 90, 107.5, 90{]}, 'h+k=2n'{]}, 'CdHgTe': {[}'CdHgTe', {[}6.46678, 6.46678, 6.46678, 90, 90, 90{]}, 'dia'{]}, 'CdHgTe\_fcc': {[}'CdHgTe\_fcc', {[}6.46678, 6.46678, 6.46678, 90, 90, 90{]}, 'fcc'{]}, 'CdTe': {[}'CdTe', {[}6.487, 6.487, 6.487, 90, 90, 90{]}, 'fcc'{]}, 'CdTeDiagB': {[}'CdTeDiagB', {[}4.5721, 7.9191, 11.1993, 90, 90, 90{]}, 'no'{]}, 'Crocidolite': {[}'Crocidolite', {[}9.811, 18.013, 5.326, 90, 103.68, 90{]}, 'no'{]}, 'Crocidolite\_2': {[}'Crocidolite\_2', {[}9.76, 17.93, 5.35, 90, 103.6, 90{]}, 'no'{]}, 'Crocidolite\_2\_72deg': {[}'Crocidolite\_2', {[}9.76, 17.93, 5.35, 90, 76.4, 90{]}, 'no'{]}, 'Crocidolite\_small': {[}'Crocidolite\_small', {[}3.2533333333333334, 5.976666666666667, 1.7833333333333332, 90, 103.6, 90{]}, 'no'{]}, 'Crocidolite\_whittaker\_1949': {[}'Crocidolite\_whittaker\_1949', {[}9.89, 17.85, 5.31, 90, 107.5, 90{]}, 'no'{]}, 'Cu': {[}'Cu', {[}3.6, 3.6, 3.6, 90, 90, 90{]}, 'fcc'{]}, 'Cu6Sn5\_monoclinic': {[}'Cu6Sn5\_monoclinic', {[}11.02, 7.28, 9.827, 90, 98.84, 90{]}, 'no'{]}, 'Cu6Sn5\_tetra': {[}'Cu6Sn5\_tetra', {[}3.608, 3.608, 5.037, 90, 90, 90{]}, 'no'{]}, 'DIA': {[}'DIA', {[}5.0, 5.0, 5.0, 90, 90, 90{]}, 'dia'{]}, 'DIAs': {[}'DIAs', {[}3.56683, 3.56683, 3.56683, 90, 90, 90{]}, 'dia'{]}, 'DarinaMolecule': {[}'DarinaMolecule', {[}9.4254, 13.5004, 13.8241, 61.83, 84.555, 75.231{]}, 'no'{]}, 'FCC': {[}'FCC', {[}5.0, 5.0, 5.0, 90, 90, 90{]}, 'fcc'{]}, 'Fe': {[}'Fe', {[}2.856, 2.856, 2.856, 90, 90, 90{]}, 'bcc'{]}, 'Fe2Ta': {[}'Fe2Ta', {[}4.83, 4.83, 0.788, 90, 90, 120{]}, 'no'{]}, 'FeAl': {[}'FeAl', {[}5.871, 5.871, 5.871, 90, 90, 90{]}, 'fcc'{]}, 'GaAs': {[}'GaAs', {[}5.65325, 5.65325, 5.65325, 90, 90, 90{]}, 'dia'{]}, 'GaN': {[}'GaN', {[}3.189, 3.189, 5.185, 90, 90, 120{]}, 'wurtzite'{]}, 'GaN\_all': {[}'GaN\_all', {[}3.189, 3.189, 5.185, 90, 90, 120{]}, 'no'{]}, 'Ge': {[}'Ge', {[}5.6575, 5.6575, 5.6575, 90, 90, 90{]}, 'dia'{]}, 'Ge\_compressedhydro': {[}'Ge\_compressedhydro', {[}5.64, 5.64, 5.64, 90, 90, 90.0{]}, 'dia'{]}, 'Ge\_s': {[}'Ge\_s', {[}5.6575, 5.6575, 5.6575, 90, 90, 89.5{]}, 'dia'{]}, 'Getest': {[}'Getest', {[}5.6575, 5.6575, 5.6574, 90, 90, 90{]}, 'dia'{]}, 'Hematite': {[}'Hematite', {[}5.03459, 5.03459, 13.7533, 90, 90, 120{]}, 'no'{]}, 'In': {[}'In', {[}3.2517, 3.2517, 4.9459, 90, 90, 90{]}, 'h+k+l=2n'{]}, 'InGaN': {[}'InGaN', {[}3.3609999999999998, 3.3609999999999998, 5.439, 90, 90, 120{]}, 'wurtzite'{]}, 'InN': {[}'InN', {[}3.533, 3.533, 5.693, 90, 90, 120{]}, 'wurtzite'{]}, 'Magnetite': {[}'Magnetite', {[}8.391, 8.391, 8.391, 90, 90, 90{]}, 'dia'{]}, 'Magnetite\_fcc': {[}'Magnetite\_fcc', {[}8.391, 8.391, 8.391, 90, 90, 90{]}, 'fcc'{]}, 'Magnetite\_sc': {[}'Magnetite\_sc', {[}8.391, 8.391, 8.391, 90, 90, 90{]}, 'no'{]}, 'Nd45': {[}'Nd45', {[}5.4884, 5.4884, 5.4884, 90, 90, 90{]}, 'fcc'{]}, 'Ni': {[}'Ni', {[}3.5238, 3.5238, 3.5238, 90, 90, 90{]}, 'fcc'{]}, 'NiO': {[}'NiO', {[}2.96, 2.96, 7.23, 90, 90, 120{]}, 'no'{]}, 'NiTi': {[}'NiTi', {[}3.5506, 3.5506, 3.5506, 90, 90, 90{]}, 'fcc'{]}, 'SC': {[}'SC', {[}1.0, 1.0, 1.0, 90, 90, 90{]}, 'no'{]}, 'SC5': {[}'SC5', {[}5.0, 5.0, 5.0, 90, 90, 90{]}, 'no'{]}, 'SC7': {[}'SC7', {[}7.0, 7.0, 7.0, 90, 90, 90{]}, 'no'{]}, 'Sb': {[}'Sb', {[}4.3, 4.3, 11.3, 90, 90, 120{]}, 'no'{]}, 'Si': {[}'Si', {[}5.4309, 5.4309, 5.4309, 90, 90, 90{]}, 'dia'{]}, 'Sn': {[}'Sn', {[}5.83, 5.83, 3.18, 90, 90, 90{]}, 'h+k+l=2n'{]}, 'Ti': {[}'Ti', {[}2.95, 2.95, 4.68, 90, 90, 120{]}, 'no'{]}, 'Ti2AlN': {[}'Ti2AlN', {[}2.989, 2.989, 13.624, 90, 90, 120{]}, 'Ti2AlN'{]}, 'Ti2AlN\_w': {[}'Ti2AlN\_w', {[}2.989, 2.989, 13.624, 90, 90, 120{]}, 'wurtzite'{]}, 'Ti\_beta': {[}'Ti\_beta', {[}3.2587, 3.2587, 3.2587, 90, 90, 90{]}, 'bcc'{]}, 'Ti\_omega': {[}'Ti\_omega', {[}4.6085, 4.6085, 2.8221, 90, 90, 120{]}, 'no'{]}, 'Ti\_s': {[}'Ti\_s', {[}3.0, 3.0, 4.7, 90.5, 89.5, 120.5{]}, 'no'{]}, 'UO2': {[}'UO2', {[}5.47, 5.47, 5.47, 90, 90, 90{]}, 'fcc'{]}, 'VO2M1': {[}'VO2M1', {[}5.75175, 4.52596, 5.38326, 90.0, 122.6148, 90.0{]}, 'VO2\_mono'{]}, 'VO2M2': {[}'VO2M2', {[}4.5546, 4.5546, 2.8514, 90.0, 90, 90.0{]}, 'no'{]}, 'VO2R': {[}'VO2R', {[}4.5546, 4.5546, 2.8514, 90.0, 90, 90.0{]}, 'rutile'{]}, 'W': {[}'W', {[}3.1652, 3.1652, 3.1652, 90, 90, 90{]}, 'bcc'{]}, 'YAG': {[}'YAG', {[}9.2, 9.2, 9.2, 90, 90, 90{]}, 'no'{]}, 'ZnO': {[}'ZnO', {[}3.252, 3.252, 5.213, 90, 90, 120{]}, 'wurtzite'{]}, 'ZrO2': {[}'ZrO2', {[}5.1505, 5.2116, 5.3173, 90, 99.23, 90{]}, 'VO2\_mono'{]}, 'ZrO2Y2O3': {[}'ZrO2Y2O3', {[}5.1378, 5.1378, 5.1378, 90, 90, 90{]}, 'fcc'{]}, 'alphaQuartz': {[}'alphaQuartz', {[}4.9, 4.9, 5.4, 90, 90, 120{]}, 'no'{]}, 'betaQuartznew': {[}'betaQuartznew', {[}4.9, 4.9, 6.685, 90, 90, 120{]}, 'no'{]}, 'bigpro': {[}'bigpro', {[}112.0, 112.0, 136.0, 90, 90, 90{]}, 'no'{]}, 'dummy': {[}'dummy', {[}4.0, 8.0, 2.0, 90, 90, 90{]}, 'no'{]}, 'ferrydrite': {[}'ferrydrite', {[}2.96, 2.96, 9.4, 90, 90, 120{]}, 'no'{]}, 'hexagonal': {[}'hexagonal', {[}1.0, 1.0, 3.0, 90, 90, 120.0{]}, 'no'{]}, 'inputB': {[}'inputB', {[}1.0, 1.0, 1.0, 90, 90, 90{]}, 'no'{]}, 'quartz\_alpha': {[}'quartz\_alpha', {[}4.913, 4.913, 5.404, 90, 90, 120{]}, 'no'{]}, 'smallpro': {[}'smallpro', {[}20.0, 4.8, 49.0, 90, 90, 90{]}, 'no'{]}, 'test\_reference': {[}'test\_reference', {[}3.2, 4.5, 5.2, 83, 92.0, 122{]}, 'wurtzite'{]}, 'test\_solution': {[}'test\_solution', {[}3.252, 4.48, 5.213, 83.2569, 92.125478, 122.364{]}, 'wurtzite'{]}, 'testindex': {[}'testindex', {[}2.0, 1.0, 4.0, 90, 90, 90{]}, 'no'{]}, 'testindex2': {[}'testindex2', {[}2.0, 1.0, 4.0, 75, 90, 120{]}, 'no'{]}\}}}{}~\begin{description}
\item[{Compute Laue Pattern spots positions, scattering angles, miller indices}] \leavevmode
for a SINGLE grain or Xtal using numpy vectorization

\end{description}
\begin{quote}\begin{description}
\item[{Parameters}] \leavevmode\begin{itemize}
\item {} 
\sphinxstyleliteralstrong{\sphinxupquote{grain}} \textendash{} crystal parameters in a 4 elements list

\item {} 
\sphinxstyleliteralstrong{\sphinxupquote{emin}} \textendash{} minimum bandpass energy (keV)

\item {} 
\sphinxstyleliteralstrong{\sphinxupquote{emax}} \textendash{} maximum bandpass energy (keV)

\item {} 
\sphinxstyleliteralstrong{\sphinxupquote{removeharmonics}} \textendash{} 1, remove harmonics spots and keep fondamental spots
(with lowest Miller indices)

\end{itemize}

\item[{Returns}] \leavevmode
single grain data: Twicetheta, Chi, Miller\_ind, posx, posy, Energy

\end{description}\end{quote}

\begin{sphinxadmonition}{note}{\label{Simulation_Module:index-3}Todo:}
update to accept kf\_direction not only in reflection geometry
\end{sphinxadmonition}

\begin{sphinxadmonition}{note}{Note:}
USED in detectorCalibration…simulate\_theo for routine geometry Z\textgreater{}0 (reflection top) X\textgreater{}0 (transmission)
\end{sphinxadmonition}

\end{fulllineitems}

\index{SimulateResult() (in module LaueTools.lauecore)}

\begin{fulllineitems}
\phantomsection\label{\detokenize{Simulation_Module:LaueTools.lauecore.SimulateResult}}\pysiglinewithargsret{\sphinxcode{\sphinxupquote{LaueTools.lauecore.}}\sphinxbfcode{\sphinxupquote{SimulateResult}}}{\emph{grain, emin, emax, simulparameters, fastcompute=1, ResolutionAngstrom=False, dictmaterials=\{'Al': {[}'Al', {[}4.05, 4.05, 4.05, 90, 90, 90{]}, 'fcc'{]}, 'Al2Cu': {[}'Al2Cu', {[}6.063, 6.063, 4.872, 90, 90, 90{]}, 'no'{]}, 'Al2O3': {[}'Al2O3', {[}4.785, 4.785, 12.991, 90, 90, 120{]}, 'Al2O3'{]}, 'Al2O3\_all': {[}'Al2O3\_all', {[}4.785, 4.785, 12.991, 90, 90, 120{]}, 'no'{]}, 'AlN': {[}'AlN', {[}3.11, 3.11, 4.98, 90.0, 90.0, 120.0{]}, 'wurtzite'{]}, 'Au': {[}'Au', {[}4.078, 4.078, 4.078, 90, 90, 90{]}, 'fcc'{]}, 'CCDL1949': {[}'CCDL1949', {[}9.89, 17.85, 5.31, 90, 107.5, 90{]}, 'h+k=2n'{]}, 'CdHgTe': {[}'CdHgTe', {[}6.46678, 6.46678, 6.46678, 90, 90, 90{]}, 'dia'{]}, 'CdHgTe\_fcc': {[}'CdHgTe\_fcc', {[}6.46678, 6.46678, 6.46678, 90, 90, 90{]}, 'fcc'{]}, 'CdTe': {[}'CdTe', {[}6.487, 6.487, 6.487, 90, 90, 90{]}, 'fcc'{]}, 'CdTeDiagB': {[}'CdTeDiagB', {[}4.5721, 7.9191, 11.1993, 90, 90, 90{]}, 'no'{]}, 'Crocidolite': {[}'Crocidolite', {[}9.811, 18.013, 5.326, 90, 103.68, 90{]}, 'no'{]}, 'Crocidolite\_2': {[}'Crocidolite\_2', {[}9.76, 17.93, 5.35, 90, 103.6, 90{]}, 'no'{]}, 'Crocidolite\_2\_72deg': {[}'Crocidolite\_2', {[}9.76, 17.93, 5.35, 90, 76.4, 90{]}, 'no'{]}, 'Crocidolite\_small': {[}'Crocidolite\_small', {[}3.2533333333333334, 5.976666666666667, 1.7833333333333332, 90, 103.6, 90{]}, 'no'{]}, 'Crocidolite\_whittaker\_1949': {[}'Crocidolite\_whittaker\_1949', {[}9.89, 17.85, 5.31, 90, 107.5, 90{]}, 'no'{]}, 'Cu': {[}'Cu', {[}3.6, 3.6, 3.6, 90, 90, 90{]}, 'fcc'{]}, 'Cu6Sn5\_monoclinic': {[}'Cu6Sn5\_monoclinic', {[}11.02, 7.28, 9.827, 90, 98.84, 90{]}, 'no'{]}, 'Cu6Sn5\_tetra': {[}'Cu6Sn5\_tetra', {[}3.608, 3.608, 5.037, 90, 90, 90{]}, 'no'{]}, 'DIA': {[}'DIA', {[}5.0, 5.0, 5.0, 90, 90, 90{]}, 'dia'{]}, 'DIAs': {[}'DIAs', {[}3.56683, 3.56683, 3.56683, 90, 90, 90{]}, 'dia'{]}, 'DarinaMolecule': {[}'DarinaMolecule', {[}9.4254, 13.5004, 13.8241, 61.83, 84.555, 75.231{]}, 'no'{]}, 'FCC': {[}'FCC', {[}5.0, 5.0, 5.0, 90, 90, 90{]}, 'fcc'{]}, 'Fe': {[}'Fe', {[}2.856, 2.856, 2.856, 90, 90, 90{]}, 'bcc'{]}, 'Fe2Ta': {[}'Fe2Ta', {[}4.83, 4.83, 0.788, 90, 90, 120{]}, 'no'{]}, 'FeAl': {[}'FeAl', {[}5.871, 5.871, 5.871, 90, 90, 90{]}, 'fcc'{]}, 'GaAs': {[}'GaAs', {[}5.65325, 5.65325, 5.65325, 90, 90, 90{]}, 'dia'{]}, 'GaN': {[}'GaN', {[}3.189, 3.189, 5.185, 90, 90, 120{]}, 'wurtzite'{]}, 'GaN\_all': {[}'GaN\_all', {[}3.189, 3.189, 5.185, 90, 90, 120{]}, 'no'{]}, 'Ge': {[}'Ge', {[}5.6575, 5.6575, 5.6575, 90, 90, 90{]}, 'dia'{]}, 'Ge\_compressedhydro': {[}'Ge\_compressedhydro', {[}5.64, 5.64, 5.64, 90, 90, 90.0{]}, 'dia'{]}, 'Ge\_s': {[}'Ge\_s', {[}5.6575, 5.6575, 5.6575, 90, 90, 89.5{]}, 'dia'{]}, 'Getest': {[}'Getest', {[}5.6575, 5.6575, 5.6574, 90, 90, 90{]}, 'dia'{]}, 'Hematite': {[}'Hematite', {[}5.03459, 5.03459, 13.7533, 90, 90, 120{]}, 'no'{]}, 'In': {[}'In', {[}3.2517, 3.2517, 4.9459, 90, 90, 90{]}, 'h+k+l=2n'{]}, 'InGaN': {[}'InGaN', {[}3.3609999999999998, 3.3609999999999998, 5.439, 90, 90, 120{]}, 'wurtzite'{]}, 'InN': {[}'InN', {[}3.533, 3.533, 5.693, 90, 90, 120{]}, 'wurtzite'{]}, 'Magnetite': {[}'Magnetite', {[}8.391, 8.391, 8.391, 90, 90, 90{]}, 'dia'{]}, 'Magnetite\_fcc': {[}'Magnetite\_fcc', {[}8.391, 8.391, 8.391, 90, 90, 90{]}, 'fcc'{]}, 'Magnetite\_sc': {[}'Magnetite\_sc', {[}8.391, 8.391, 8.391, 90, 90, 90{]}, 'no'{]}, 'Nd45': {[}'Nd45', {[}5.4884, 5.4884, 5.4884, 90, 90, 90{]}, 'fcc'{]}, 'Ni': {[}'Ni', {[}3.5238, 3.5238, 3.5238, 90, 90, 90{]}, 'fcc'{]}, 'NiO': {[}'NiO', {[}2.96, 2.96, 7.23, 90, 90, 120{]}, 'no'{]}, 'NiTi': {[}'NiTi', {[}3.5506, 3.5506, 3.5506, 90, 90, 90{]}, 'fcc'{]}, 'SC': {[}'SC', {[}1.0, 1.0, 1.0, 90, 90, 90{]}, 'no'{]}, 'SC5': {[}'SC5', {[}5.0, 5.0, 5.0, 90, 90, 90{]}, 'no'{]}, 'SC7': {[}'SC7', {[}7.0, 7.0, 7.0, 90, 90, 90{]}, 'no'{]}, 'Sb': {[}'Sb', {[}4.3, 4.3, 11.3, 90, 90, 120{]}, 'no'{]}, 'Si': {[}'Si', {[}5.4309, 5.4309, 5.4309, 90, 90, 90{]}, 'dia'{]}, 'Sn': {[}'Sn', {[}5.83, 5.83, 3.18, 90, 90, 90{]}, 'h+k+l=2n'{]}, 'Ti': {[}'Ti', {[}2.95, 2.95, 4.68, 90, 90, 120{]}, 'no'{]}, 'Ti2AlN': {[}'Ti2AlN', {[}2.989, 2.989, 13.624, 90, 90, 120{]}, 'Ti2AlN'{]}, 'Ti2AlN\_w': {[}'Ti2AlN\_w', {[}2.989, 2.989, 13.624, 90, 90, 120{]}, 'wurtzite'{]}, 'Ti\_beta': {[}'Ti\_beta', {[}3.2587, 3.2587, 3.2587, 90, 90, 90{]}, 'bcc'{]}, 'Ti\_omega': {[}'Ti\_omega', {[}4.6085, 4.6085, 2.8221, 90, 90, 120{]}, 'no'{]}, 'Ti\_s': {[}'Ti\_s', {[}3.0, 3.0, 4.7, 90.5, 89.5, 120.5{]}, 'no'{]}, 'UO2': {[}'UO2', {[}5.47, 5.47, 5.47, 90, 90, 90{]}, 'fcc'{]}, 'VO2M1': {[}'VO2M1', {[}5.75175, 4.52596, 5.38326, 90.0, 122.6148, 90.0{]}, 'VO2\_mono'{]}, 'VO2M2': {[}'VO2M2', {[}4.5546, 4.5546, 2.8514, 90.0, 90, 90.0{]}, 'no'{]}, 'VO2R': {[}'VO2R', {[}4.5546, 4.5546, 2.8514, 90.0, 90, 90.0{]}, 'rutile'{]}, 'W': {[}'W', {[}3.1652, 3.1652, 3.1652, 90, 90, 90{]}, 'bcc'{]}, 'YAG': {[}'YAG', {[}9.2, 9.2, 9.2, 90, 90, 90{]}, 'no'{]}, 'ZnO': {[}'ZnO', {[}3.252, 3.252, 5.213, 90, 90, 120{]}, 'wurtzite'{]}, 'ZrO2': {[}'ZrO2', {[}5.1505, 5.2116, 5.3173, 90, 99.23, 90{]}, 'VO2\_mono'{]}, 'ZrO2Y2O3': {[}'ZrO2Y2O3', {[}5.1378, 5.1378, 5.1378, 90, 90, 90{]}, 'fcc'{]}, 'alphaQuartz': {[}'alphaQuartz', {[}4.9, 4.9, 5.4, 90, 90, 120{]}, 'no'{]}, 'betaQuartznew': {[}'betaQuartznew', {[}4.9, 4.9, 6.685, 90, 90, 120{]}, 'no'{]}, 'bigpro': {[}'bigpro', {[}112.0, 112.0, 136.0, 90, 90, 90{]}, 'no'{]}, 'dummy': {[}'dummy', {[}4.0, 8.0, 2.0, 90, 90, 90{]}, 'no'{]}, 'ferrydrite': {[}'ferrydrite', {[}2.96, 2.96, 9.4, 90, 90, 120{]}, 'no'{]}, 'hexagonal': {[}'hexagonal', {[}1.0, 1.0, 3.0, 90, 90, 120.0{]}, 'no'{]}, 'inputB': {[}'inputB', {[}1.0, 1.0, 1.0, 90, 90, 90{]}, 'no'{]}, 'quartz\_alpha': {[}'quartz\_alpha', {[}4.913, 4.913, 5.404, 90, 90, 120{]}, 'no'{]}, 'smallpro': {[}'smallpro', {[}20.0, 4.8, 49.0, 90, 90, 90{]}, 'no'{]}, 'test\_reference': {[}'test\_reference', {[}3.2, 4.5, 5.2, 83, 92.0, 122{]}, 'wurtzite'{]}, 'test\_solution': {[}'test\_solution', {[}3.252, 4.48, 5.213, 83.2569, 92.125478, 122.364{]}, 'wurtzite'{]}, 'testindex': {[}'testindex', {[}2.0, 1.0, 4.0, 90, 90, 90{]}, 'no'{]}, 'testindex2': {[}'testindex2', {[}2.0, 1.0, 4.0, 75, 90, 120{]}, 'no'{]}\}}}{}
Simulates 2theta chi of Laue Pattern spots for ONE SINGLE grain
\begin{quote}\begin{description}
\item[{Parameters}] \leavevmode\begin{itemize}
\item {} 
\sphinxstyleliteralstrong{\sphinxupquote{grain}} \textendash{} crystal parameters in a 4 elements list

\item {} 
\sphinxstyleliteralstrong{\sphinxupquote{emin}} \textendash{} minimum bandpass energy (keV)

\item {} 
\sphinxstyleliteralstrong{\sphinxupquote{emax}} \textendash{} maximum bandpass energy (keV)

\end{itemize}

\item[{Returns}] \leavevmode
2theta, chi

\end{description}\end{quote}

\begin{sphinxadmonition}{warning}{Warning:}
Need of approximate detector distance and diameter to restrict simulation to a limited solid angle
\end{sphinxadmonition}

\begin{sphinxadmonition}{note}{Note:}\begin{itemize}
\item {} 
USED: in AutoindexationGUI.OnStart, LaueToolsGUI.OnCheckOrientationMatrix

\item {} 
USED also IndexingImageMatching, lauecore.SimulateLaue\_merge

\end{itemize}
\end{sphinxadmonition}

\end{fulllineitems}



\subsection{2D Detection Geometry}
\label{\detokenize{Simulation_Module:module-LaueTools.LaueGeometry}}\label{\detokenize{Simulation_Module:d-detection-geometry}}\index{LaueTools.LaueGeometry (module)}
Module of lauetools project to compute Laue spots position on CCD camera.
It handles detection and source geometry.

\begin{sphinxadmonition}{warning}{Warning:}
The frame (LT2) considered in this package (with y axis parallel to the incoming beam) in not the LaueTools frame (for which x is parallel to the incoming beam)
\end{sphinxadmonition}

JS Micha June 2019
\begin{itemize}
\item {} \begin{description}
\item[{Vectors Definitions}] \leavevmode\begin{itemize}
\item {} 
\sphinxstylestrong{q} momentum transfer vector from resp. incoming and outgoing wave vector \sphinxstylestrong{ki} and \sphinxstylestrong{kf}, \(q=kf-ki\)

\item {} 
When a Laue spot exists, \sphinxstylestrong{q} is equal to the one node of the reciprocal lattice given by \sphinxstylestrong{G*} vector

\item {} \begin{description}
\item[{\sphinxstylestrong{G*} is perpendicular to atomic planes defined by the three Miller indices h,k,l}] \leavevmode
such as \sphinxstylestrong{G***=h**a*} + k**b*** +l**c*** where \sphinxstylestrong{a*}, \sphinxstylestrong{b*}, and \sphinxstylestrong{c*} are the unit cell lattice basis vectors.

\end{description}

\item {} 
\sphinxstylestrong{kf}: scattered beam vector whose corresponding unit vector is \sphinxstylestrong{uf}

\item {} 
\sphinxstylestrong{ki} incoming beam vector, \sphinxstylestrong{ui} corresponding unit vector

\end{itemize}

\end{description}

\item {} \begin{description}
\item[{Laboratory Frame LT2}] \leavevmode\begin{itemize}
\item {} 
I: origin

\item {} 
z vertical up perpendicular to CCD plane (top camera geometry)

\item {} 
y along X-ray horizontal

\item {} 
x towards wall behind horizontal

\item {} 
O: origin of pixel CCD frame in detecting plane

\item {} 
\sphinxstylestrong{j} // \sphinxstylestrong{ui} incoming beam unit vector

\item {} \begin{description}
\item[{z axis is defined by the CCD camera position. z axis is perpendicular to CCD plane}] \leavevmode
such as IO belongs to the plane Iyz

\end{description}

\item {} 
bet: angle between \sphinxstylestrong{IO} and \sphinxstylestrong{k}

\item {} \begin{description}
\item[{\sphinxstylestrong{i**= **j**\textasciicircum{}**k} (when entering the BM32 hutch \sphinxstylestrong{i} is approximately towards the wall}] \leavevmode
(in CCD on top geometry and beam coming from the right)

\end{description}

\item {} 
M: point lying in CCD plane corresponding to Laue spot

\item {} 
\sphinxstylestrong{uf} is the unit vector relative to vector \sphinxstylestrong{IM}

\end{itemize}

\end{description}

\end{itemize}

\sphinxstylestrong{kf} is also a vector collinear to \sphinxstylestrong{IM} with a length of R=1/wavelength=E/12.398 {[}keV{]}
with wavelength and Energy of the corresponding bragg’s reflections.

I is the point from which calibration parameters (CCD position) are deduced (from a perfectly known crystal structure Laue pattern)
Iprime is an other source of emission (posI or offset in functions)

\(2 \theta\) is the scattering angle between \sphinxstylestrong{ui} and \sphinxstylestrong{uf}, i.e.

\(\cos(2 \theta)=u_i.u_f\)
\begin{align*}\!\begin{aligned}
{\bf k_f} = ( -\sin 2 \theta \sin \chi, \cos 2\theta, \sin 2\theta \cos \chi)\\
{\bf k_i} = (0, 1, 0)\\
\end{aligned}\end{align*}
Energy= 12.398*  q**2/(2* \sphinxstylestrong{q}.**ui**)=12.398 * q**2/ (-2 sin theta)
\begin{description}
\item[{\sphinxstyleemphasis{Calibration parameters (CCD position and detection geometry)}}] \leavevmode\begin{itemize}
\item {} 
calib: list of the 5 calibration parameters {[}dd,xcen,ycen,xbet,xgam{]}

\item {} 
dd: norm of \sphinxstylestrong{IO}  {[}mm{]}

\item {} \begin{description}
\item[{xcen,ycen {[}pixel unit{]}: pixels values in CCD frame of point O with respect to Oprime where}] \leavevmode
Oprime is the origin of CCD pixels frame (at a corner of the CCD array)

\end{description}

\item {} 
xbet: angle between \sphinxstylestrong{IO} and \sphinxstylestrong{k} {[}degree{]}

\item {} \begin{description}
\item[{xgam: azimutal rotation angle around z axis. Angle between CCD array axes}] \leavevmode
and (\sphinxstylestrong{i},**j**) after rotation by xbet {[}degree{]}.

\end{description}

\end{itemize}

\end{description}

\sphinxstyleemphasis{sample frame}

Origin is I and unit frame vectors (\sphinxstylestrong{is}, \sphinxstylestrong{js}, \sphinxstylestrong{ks}) are derived
from absolute frame by the rotation (axis= - \sphinxstylestrong{i}, angle= wo) where wo is the angle between \sphinxstylestrong{js} and \sphinxstylestrong{j}
\index{calc\_uflab() (in module LaueTools.LaueGeometry)}

\begin{fulllineitems}
\phantomsection\label{\detokenize{Simulation_Module:LaueTools.LaueGeometry.calc_uflab}}\pysiglinewithargsret{\sphinxcode{\sphinxupquote{LaueTools.LaueGeometry.}}\sphinxbfcode{\sphinxupquote{calc\_uflab}}}{\emph{xcam}, \emph{ycam}, \emph{detectorplaneparameters}, \emph{offset=0}, \emph{returnAngles=1}, \emph{verbose=0}, \emph{pixelsize=0.08056640625}, \emph{rectpix=0}, \emph{kf\_direction='Z\textgreater{}0'}}{}
Computes unit vector \({\bf u_f}=\frac{\bf k_f}{\|k_f\|}\) in laboratory frame of scattered beam \(k_f\)
(angle scattering angles 2theta and chi) from X, Y pixel Laue spot position

Unit vector uf correspond to normalized kf vector: q = kf - ki
from lists of X and Y Laue spots pixels positions on detector
\begin{quote}\begin{description}
\item[{Parameters}] \leavevmode\begin{itemize}
\item {} 
\sphinxstyleliteralstrong{\sphinxupquote{xcam}} (\sphinxstyleliteralemphasis{\sphinxupquote{list of floats}}) \textendash{} list of pixel X position

\item {} 
\sphinxstyleliteralstrong{\sphinxupquote{ycam}} (\sphinxstyleliteralemphasis{\sphinxupquote{list of floats}}) \textendash{} list of pixel Y position

\item {} 
\sphinxstyleliteralstrong{\sphinxupquote{detectorplaneparameters}} \textendash{} list of 5 calibration parameters

\item {} 
\sphinxstyleliteralstrong{\sphinxupquote{offset}} \textendash{} float, offset in position along incoming beam of source of scattered rays
if positive: offset in sample depth
units: mm

\end{itemize}

\item[{Returns}] \leavevmode
\begin{itemize}
\item {} 
if returnAngles=1   : twicetheta, chi   \sphinxstyleemphasis{(default)}

\item {} 
if returnAngles!=1  : uflab, IMlab

\end{itemize}


\end{description}\end{quote}

\end{fulllineitems}

\index{calc\_uflab\_trans() (in module LaueTools.LaueGeometry)}

\begin{fulllineitems}
\phantomsection\label{\detokenize{Simulation_Module:LaueTools.LaueGeometry.calc_uflab_trans}}\pysiglinewithargsret{\sphinxcode{\sphinxupquote{LaueTools.LaueGeometry.}}\sphinxbfcode{\sphinxupquote{calc\_uflab\_trans}}}{\emph{xcam}, \emph{ycam}, \emph{calib}, \emph{returnAngles=1}, \emph{verbose=0}, \emph{pixelsize=0.08056640625}, \emph{rectpix=0}}{}
compute \(2 \theta\) and \(\chi\) scattering angles or \sphinxstylestrong{uf} and \sphinxstylestrong{kf} vectors
from lists of X and Y Laue spots positions
in TRANSMISSION geometry
\begin{quote}\begin{description}
\item[{Parameters}] \leavevmode\begin{itemize}
\item {} 
\sphinxstyleliteralstrong{\sphinxupquote{xcam}} (\sphinxstyleliteralemphasis{\sphinxupquote{list of floats}}) \textendash{} list of pixel X position

\item {} 
\sphinxstyleliteralstrong{\sphinxupquote{ycam}} (\sphinxstyleliteralemphasis{\sphinxupquote{list of floats}}) \textendash{} list of pixel Y position

\item {} 
\sphinxstyleliteralstrong{\sphinxupquote{calib}} \textendash{} list of 5 calibration parameters

\end{itemize}

\item[{Returns}] \leavevmode
\begin{itemize}
\item {} 
if returnAngles=1   : twicetheta, chi   \sphinxstyleemphasis{(default)}

\item {} 
if returnAngles!=1  : uflab, IMlab

\end{itemize}


\end{description}\end{quote}

\# TODO: add offset like in reflection geometry

\end{fulllineitems}

\index{calc\_xycam() (in module LaueTools.LaueGeometry)}

\begin{fulllineitems}
\phantomsection\label{\detokenize{Simulation_Module:LaueTools.LaueGeometry.calc_xycam}}\pysiglinewithargsret{\sphinxcode{\sphinxupquote{LaueTools.LaueGeometry.}}\sphinxbfcode{\sphinxupquote{calc\_xycam}}}{\emph{uflab}, \emph{calib}, \emph{energy=0}, \emph{offset=None}, \emph{verbose=0}, \emph{returnIpM=False}, \emph{pixelsize=0.08056640625}, \emph{dim=(2048}, \emph{2048)}, \emph{rectpix=0}}{}
Computes Laue spots position x and y in pixels units in CCD frame
from unit scattered vector uf expressed in Lab. frame

computes coordinates of point M on CCD from point source and \sphinxstylestrong{uflab}.
Point Ip (source Iprime of x-ray scattered beams)
(for each Laue spot \sphinxstylestrong{uflab} is the unit vector of \sphinxstylestrong{IpM})
Point Ip is shifted by offset (if not None) from the default point I
(used to calibrate the CCD camera and 2theta chi determination)

th0 (theta in degrees)
Energy (energy in keV)
\begin{quote}\begin{description}
\item[{Parameters}] \leavevmode\begin{itemize}
\item {} 
\sphinxstyleliteralstrong{\sphinxupquote{uflab}} (\sphinxhref{https://docs.python.org/3/library/stdtypes.html\#list}{\sphinxstyleliteralemphasis{\sphinxupquote{list}}}\sphinxstyleliteralemphasis{\sphinxupquote{ or }}\sphinxstyleliteralemphasis{\sphinxupquote{array}}\sphinxstyleliteralemphasis{\sphinxupquote{ (}}\sphinxstyleliteralemphasis{\sphinxupquote{length must \textgreater{} 1}}\sphinxstyleliteralemphasis{\sphinxupquote{)}}) \textendash{} list or array of {[}qx,qy,qz{]} (q vector)

\item {} 
\sphinxstyleliteralstrong{\sphinxupquote{calib}} (\sphinxstyleliteralemphasis{\sphinxupquote{list of floats}}) \textendash{} list 5 detector calibration parameters

\item {} 
\sphinxstyleliteralstrong{\sphinxupquote{offset}} (\sphinxstyleliteralemphasis{\sphinxupquote{list of floats}}\sphinxstyleliteralemphasis{\sphinxupquote{ (}}\sphinxstyleliteralemphasis{\sphinxupquote{{[}}}\sphinxstyleliteralemphasis{\sphinxupquote{x}}\sphinxstyleliteralemphasis{\sphinxupquote{,}}\sphinxstyleliteralemphasis{\sphinxupquote{y}}\sphinxstyleliteralemphasis{\sphinxupquote{,}}\sphinxstyleliteralemphasis{\sphinxupquote{z}}\sphinxstyleliteralemphasis{\sphinxupquote{{]}}}\sphinxstyleliteralemphasis{\sphinxupquote{)}}) \textendash{} offset (in mm) in the scattering source (origin of Laue spots)
position with respect to the position which has been used
for the calibration of  the CCD detector plane. Offset is positive when in the same
direction as incident beam (i.e. in sample depth)
(incident beam direction remains constant)

\end{itemize}

\item[{Returns}] \leavevmode
\begin{itemize}
\item {} 
xcam: list of pixel X coordinates

\item {} 
ycam: list of pixel Y coordinates

\item {} 
theta: list half scattering angle “theta” (in degree)

\item {} 
optionally energy=1: add in output list of spot energies (in keV)

\item {} 
if returnIpM and offset not None: return list of vectors \sphinxstylestrong{IprimeM}

\end{itemize}


\end{description}\end{quote}

\end{fulllineitems}

\index{calc\_xycam\_transmission() (in module LaueTools.LaueGeometry)}

\begin{fulllineitems}
\phantomsection\label{\detokenize{Simulation_Module:LaueTools.LaueGeometry.calc_xycam_transmission}}\pysiglinewithargsret{\sphinxcode{\sphinxupquote{LaueTools.LaueGeometry.}}\sphinxbfcode{\sphinxupquote{calc\_xycam\_transmission}}}{\emph{uflab}, \emph{calib}, \emph{energy=0}, \emph{offset=None}, \emph{verbose=0}, \emph{returnIpM=False}, \emph{pixelsize=0.08056640625}, \emph{dim=(2048}, \emph{2048)}, \emph{rectpix=0}}{}
Computes Laue spots position x and y in pixels units (in CCD frame) from scattering vector q

As calc\_xycam() but in TRANSMISSION geometry

\end{fulllineitems}

\index{uflab\_from2thetachi() (in module LaueTools.LaueGeometry)}

\begin{fulllineitems}
\phantomsection\label{\detokenize{Simulation_Module:LaueTools.LaueGeometry.uflab_from2thetachi}}\pysiglinewithargsret{\sphinxcode{\sphinxupquote{LaueTools.LaueGeometry.}}\sphinxbfcode{\sphinxupquote{uflab\_from2thetachi}}}{\emph{twicetheta}, \emph{chi}, \emph{verbose=0}}{}
Computes \({\bf u_f}\) vectors coordinates in lauetools LT2 frame
from \({\bf k_f}\) scattering angles \(2 \theta\) and \(2 \chi\) angles
\begin{quote}\begin{description}
\item[{Parameters}] \leavevmode\begin{itemize}
\item {} 
\sphinxstyleliteralstrong{\sphinxupquote{twicetheta}} \textendash{} (list) \(2 \theta\) angle(s) ( in degree)

\item {} 
\sphinxstyleliteralstrong{\sphinxupquote{chi}} \textendash{} (list) \(2 \chi\) angle(s) ( in degree)

\end{itemize}

\item[{Returns}] \leavevmode
list of \sphinxtitleref{\{bf u\_f\}} =  {[}\(uf_x,uf_y,uf_z\){]}

\item[{Return type}] \leavevmode
\sphinxhref{https://docs.python.org/3/library/stdtypes.html\#list}{list}

\end{description}\end{quote}

\end{fulllineitems}

\index{from\_twchi\_to\_qunit() (in module LaueTools.LaueGeometry)}

\begin{fulllineitems}
\phantomsection\label{\detokenize{Simulation_Module:LaueTools.LaueGeometry.from_twchi_to_qunit}}\pysiglinewithargsret{\sphinxcode{\sphinxupquote{LaueTools.LaueGeometry.}}\sphinxbfcode{\sphinxupquote{from\_twchi\_to\_qunit}}}{\emph{Angles}}{}
from kf 2theta, chi to q unit in LaueTools frame (xx// ki) q=kf-ki
returns array = (all x’s, all y’s, all z’s)

Angles in degrees !!
Angles{[}0{]} 2theta deg values,
Angles{[}1{]} chi values in deg

this is the inverse function of from\_qunit\_to\_twchi(), useful to check it

\end{fulllineitems}

\index{from\_twchi\_to\_q() (in module LaueTools.LaueGeometry)}

\begin{fulllineitems}
\phantomsection\label{\detokenize{Simulation_Module:LaueTools.LaueGeometry.from_twchi_to_q}}\pysiglinewithargsret{\sphinxcode{\sphinxupquote{LaueTools.LaueGeometry.}}\sphinxbfcode{\sphinxupquote{from\_twchi\_to\_q}}}{\emph{Angles}}{}
From kf 2theta,chi to q (arbitrary lenght) in lab frame (xx// ki) q=kf-ki
returns array = (all qx’s, all qy’s, all qz’s)

Angles in degrees !!
Angles{[}0{]} 2theta deg values,
Angles{[}1{]} chi values in deg

\end{fulllineitems}

\index{from\_qunit\_to\_twchi() (in module LaueTools.LaueGeometry)}

\begin{fulllineitems}
\phantomsection\label{\detokenize{Simulation_Module:LaueTools.LaueGeometry.from_qunit_to_twchi}}\pysiglinewithargsret{\sphinxcode{\sphinxupquote{LaueTools.LaueGeometry.}}\sphinxbfcode{\sphinxupquote{from\_qunit\_to\_twchi}}}{\emph{arrayXYZ}, \emph{labXMAS=0}}{}
Returns 2theta chi from a q unit vector (defining a direction) expressed in LaueTools frame (\sphinxstylestrong{xx}// \sphinxstylestrong{ki}) \sphinxstylestrong{q=kf-ki}
\begin{equation*}
\begin{split}\left [ \begin{matrix}
-\sin \theta \\ \cos \theta \sin \chi \\ \cos \theta \cos \chi
\end{matrix}
\right ]\end{split}
\end{equation*}
\begin{sphinxadmonition}{note}{Note:}
in LaueTools frame
\begin{align*}\!\begin{aligned}
kf = \left [ \begin{matrix}
\cos 2\theta \\ \sin 2\theta \sin \chi \\ \sin 2\theta \cos \chi
\end{matrix}
\right ]\\
q = 2 \sin \theta \left [ \begin{matrix}
-\sin \theta \\ \cos \theta \sin \chi \\ \cos \theta \cos \chi
\end{matrix}
\right ]\\
\end{aligned}\end{align*}
In LT2 Frame   labXMAS=1
\begin{align*}\!\begin{aligned}
kf = \left [ \begin{matrix}
\sin 2\theta \sin \chi \\ \cos 2\theta \\ \sin 2\theta \cos \chi
\end{matrix}
\right ]\\
q = 2 \sin \theta \left [ \begin{matrix}
\cos \theta \sin \chi \\ -\sin \theta \\ \cos \theta \cos \chi
\end{matrix}
\right ]\\
\end{aligned}\end{align*}\end{sphinxadmonition}

\end{fulllineitems}

\index{qvector\_from\_xy\_E() (in module LaueTools.LaueGeometry)}

\begin{fulllineitems}
\phantomsection\label{\detokenize{Simulation_Module:LaueTools.LaueGeometry.qvector_from_xy_E}}\pysiglinewithargsret{\sphinxcode{\sphinxupquote{LaueTools.LaueGeometry.}}\sphinxbfcode{\sphinxupquote{qvector\_from\_xy\_E}}}{\emph{xcamList}, \emph{ycamList}, \emph{energy}, \emph{detectorplaneparameters}, \emph{pixelsize}}{}
Returns q vectors in Lauetools frame given x and y pixel positions on detector
for a given Energy (keV)
\begin{quote}\begin{description}
\item[{Parameters}] \leavevmode\begin{itemize}
\item {} 
\sphinxstyleliteralstrong{\sphinxupquote{xcamList}} \textendash{} list pixel x postions

\item {} 
\sphinxstyleliteralstrong{\sphinxupquote{ycamList}} \textendash{} list pixel y postions

\item {} 
\sphinxstyleliteralstrong{\sphinxupquote{energy}} \textendash{} list pf energies

\item {} 
\sphinxstyleliteralstrong{\sphinxupquote{detectorplaneparameters}} \textendash{} list of 5 calibration parameters

\item {} 
\sphinxstyleliteralstrong{\sphinxupquote{pixelsize}} \textendash{} pixel size in mm

\end{itemize}

\end{description}\end{quote}

\end{fulllineitems}

\index{unit\_q() (in module LaueTools.LaueGeometry)}

\begin{fulllineitems}
\phantomsection\label{\detokenize{Simulation_Module:LaueTools.LaueGeometry.unit_q}}\pysiglinewithargsret{\sphinxcode{\sphinxupquote{LaueTools.LaueGeometry.}}\sphinxbfcode{\sphinxupquote{unit\_q}}}{\emph{ttheta}, \emph{chi}, \emph{frame='lauetools'}, \emph{anglesample=40.0}}{}
Returns unit q vector from 2theta,chi coordinates
\begin{quote}\begin{description}
\item[{Parameters}] \leavevmode\begin{itemize}
\item {} 
\sphinxstyleliteralstrong{\sphinxupquote{ttheta}} \textendash{} list of 2theta angles (in degrees)

\item {} 
\sphinxstyleliteralstrong{\sphinxupquote{chi}} \textendash{} list of chi angles (in degrees)

\item {} 
\sphinxstyleliteralstrong{\sphinxupquote{anglesample}} \textendash{} incidence angle of beam to surface plane (degrees)

\item {} 
\sphinxstyleliteralstrong{\sphinxupquote{frame}} \textendash{} frame to express vectors in: ‘lauetools’ , ‘XMASlab’ (LT2 frame),’XMASsample’

\end{itemize}

\item[{Returns}] \leavevmode
list of 3D u\_f (unit vector of scattering transfer q)

\end{description}\end{quote}

\end{fulllineitems}

\index{Compute\_data2thetachi() (in module LaueTools.LaueGeometry)}

\begin{fulllineitems}
\phantomsection\label{\detokenize{Simulation_Module:LaueTools.LaueGeometry.Compute_data2thetachi}}\pysiglinewithargsret{\sphinxcode{\sphinxupquote{LaueTools.LaueGeometry.}}\sphinxbfcode{\sphinxupquote{Compute\_data2thetachi}}}{\emph{filename}, \emph{tuple\_column\_X\_Y\_I}, \emph{\_nblines\_headertoskip}, \emph{sorting\_intensity='yes'}, \emph{param=None}, \emph{kf\_direction='Z\textgreater{}0'}, \emph{verbose=1}, \emph{pixelsize=0.08056640625}, \emph{dim=(2048}, \emph{2048)}, \emph{saturation=0}, \emph{forceextension\_lines\_to\_extract=None}, \emph{col\_isbadspot=None}, \emph{alpha\_xray\_incidence\_correction=None}}{}
Converts spot positions x,y to scattering angles 2theta, chi from a list of peaks
\begin{quote}\begin{description}
\item[{Parameters}] \leavevmode\begin{itemize}
\item {} 
\sphinxstyleliteralstrong{\sphinxupquote{filename}} (\sphinxstyleliteralemphasis{\sphinxupquote{string}}) \textendash{} fullpath to peaks list ASCII file

\item {} 
\sphinxstyleliteralstrong{\sphinxupquote{tuple\_column\_X\_Y\_I}} (\sphinxstyleliteralemphasis{\sphinxupquote{3 elements}}) \textendash{} tuple with column indices of spots X, Y (pixels on CCD) and intensity

\item {} 
\sphinxstyleliteralstrong{\sphinxupquote{\_nblines\_headertoskip}} \textendash{} nb of line to skip before reading an array of data in ascii file

\item {} 
\sphinxstyleliteralstrong{\sphinxupquote{param}} \textendash{} list of CCD calibration parameters {[}det, xcen, ycen, xbet, xgam{]}

\item {} 
\sphinxstyleliteralstrong{\sphinxupquote{pixelsize}} \textendash{} pixelsize in mm

\item {} 
\sphinxstyleliteralstrong{\sphinxupquote{dim}} \textendash{} (nb pixels x, nb pixels y)

\item {} 
\sphinxstyleliteralstrong{\sphinxupquote{kf\_direction}} (\sphinxstyleliteralemphasis{\sphinxupquote{string}}) \textendash{} label of detection geometry (CCD position): ‘Z\textgreater{}0’,’X\textgreater{}0’,…

\item {} 
\sphinxstyleliteralstrong{\sphinxupquote{sorting\_intensity}} \textendash{} ‘yes’ sort spots list by decreasing intensity

\end{itemize}

\end{description}\end{quote}

saturation = 0 : do not read Ipixmax column of DAT file from LaueTools peaksearch
saturation \textgreater{} 0 : read Ipixmax column and create data\_sat list
data\_sat{[}i{]} = 1 if Ipixmax{[}i{]}\textgreater{} saturation, =0 otherwise
\begin{description}
\item[{Note: \_nblines\_headertoskip =0 for .pik file (no header at all)}] \leavevmode
\_nblines\_headertoskip =1 for .peaks coming from fit2d

\end{description}

col\_Ipixmax = 10 for .dat from LT peak search using method “Local Maxima”
(TODO : bug in Ipixmax for method “convolve”)

\end{fulllineitems}

\index{convert2corfile() (in module LaueTools.LaueGeometry)}

\begin{fulllineitems}
\phantomsection\label{\detokenize{Simulation_Module:LaueTools.LaueGeometry.convert2corfile}}\pysiglinewithargsret{\sphinxcode{\sphinxupquote{LaueTools.LaueGeometry.}}\sphinxbfcode{\sphinxupquote{convert2corfile}}}{\emph{filename}, \emph{calibparam}, \emph{dirname\_in=None}, \emph{dirname\_out=None}, \emph{pixelsize=0.08056640625}, \emph{CCDCalibdict=None}, \emph{add\_props=False}}{}
Convert .dat (peaks list from peaksearch procedure) to .cor (adding scattering angles 2theta chi)

From X,Y pixel positions in peak list file (x,y,I,…) and detector plane geometry comptues scattering angles 2theta chi
and creates a .cor file (ascii peaks list (2theta chi X Y int …))
\begin{quote}\begin{description}
\item[{Parameters}] \leavevmode\begin{itemize}
\item {} 
\sphinxstyleliteralstrong{\sphinxupquote{calibparam}} \textendash{} list of 5 CCD cakibration parameters (used if CCDCalibdict is None or  CCDCalibdict{[}‘CCDCalibPameters’{]} is missing)

\item {} 
\sphinxstyleliteralstrong{\sphinxupquote{pixelsize}} \textendash{} CCD pixelsize (in mm) (used if CCDCalibdict is None or CCDCalibdict{[}‘pixelsize’{]} is missing)

\item {} 
\sphinxstyleliteralstrong{\sphinxupquote{CCDCalibdict}} \textendash{} dictionary of CCD file and calibration parameters

\item {} 
\sphinxstyleliteralstrong{\sphinxupquote{add\_props}} \textendash{} add all peaks properties to .cor file instead of the 5 columns

\end{itemize}

\end{description}\end{quote}

\end{fulllineitems}

\index{convert2corfile\_fileseries() (in module LaueTools.LaueGeometry)}

\begin{fulllineitems}
\phantomsection\label{\detokenize{Simulation_Module:LaueTools.LaueGeometry.convert2corfile_fileseries}}\pysiglinewithargsret{\sphinxcode{\sphinxupquote{LaueTools.LaueGeometry.}}\sphinxbfcode{\sphinxupquote{convert2corfile\_fileseries}}}{\emph{fileindexrange}, \emph{filenameprefix}, \emph{calibparam}, \emph{suffix=''}, \emph{nbdigits=4}, \emph{dirname\_in=None}, \emph{dirname\_out=None}, \emph{pixelsize=0.08056640625}, \emph{fliprot='no'}}{}
convert a serie of peaks list ascii files to .cor files (adding scattering angles).

Filename is decomposed as following for incrementing file index in \#\#\#\#:
prefix\#\#\#\#suffix
example: myimage\_0025.myccd =\textgreater{} prefix=myimage\_ nbdigits=4 suffix=.myccd
\begin{quote}\begin{description}
\item[{Parameters}] \leavevmode\begin{itemize}
\item {} 
\sphinxstyleliteralstrong{\sphinxupquote{nbdigits}} \textendash{} nb of digits of file index in filename (with zero padding)
(example: for myimage\_0002.ccd nbdigits = 4

\item {} 
\sphinxstyleliteralstrong{\sphinxupquote{calibparam}} \textendash{} list of 5 CCD cakibration parameters

\end{itemize}

\end{description}\end{quote}

\end{fulllineitems}

\index{convert2corfile\_multiprocessing() (in module LaueTools.LaueGeometry)}

\begin{fulllineitems}
\phantomsection\label{\detokenize{Simulation_Module:LaueTools.LaueGeometry.convert2corfile_multiprocessing}}\pysiglinewithargsret{\sphinxcode{\sphinxupquote{LaueTools.LaueGeometry.}}\sphinxbfcode{\sphinxupquote{convert2corfile\_multiprocessing}}}{\emph{fileindexrange}, \emph{filenameprefix}, \emph{calibparam}, \emph{dirname\_in=None}, \emph{suffix=''}, \emph{nbdigits=4}, \emph{dirname\_out=None}, \emph{pixelsize=0.08056640625}, \emph{fliprot='no'}, \emph{nb\_of\_cpu=6}}{}
launch several processes in parallel to convert .dat file to .cor file

\end{fulllineitems}

\index{vec\_normalTosurface() (in module LaueTools.LaueGeometry)}

\begin{fulllineitems}
\phantomsection\label{\detokenize{Simulation_Module:LaueTools.LaueGeometry.vec_normalTosurface}}\pysiglinewithargsret{\sphinxcode{\sphinxupquote{LaueTools.LaueGeometry.}}\sphinxbfcode{\sphinxupquote{vec\_normalTosurface}}}{\emph{mat\_labframe}}{}
solve Mat * X = (0,0,1) for X
for pure rotation invMat = transpose(Mat)

TODO: add option sample angle and axis

\end{fulllineitems}

\index{vec\_onsurface\_alongys() (in module LaueTools.LaueGeometry)}

\begin{fulllineitems}
\phantomsection\label{\detokenize{Simulation_Module:LaueTools.LaueGeometry.vec_onsurface_alongys}}\pysiglinewithargsret{\sphinxcode{\sphinxupquote{LaueTools.LaueGeometry.}}\sphinxbfcode{\sphinxupquote{vec\_onsurface\_alongys}}}{\emph{mat\_labframe}}{}
solve Mat * X = (0,1,0) for X
for pure rotation invMat = transpose(Mat)

\end{fulllineitems}

\index{convert\_xycam\_from\_sourceshift() (in module LaueTools.LaueGeometry)}

\begin{fulllineitems}
\phantomsection\label{\detokenize{Simulation_Module:LaueTools.LaueGeometry.convert_xycam_from_sourceshift}}\pysiglinewithargsret{\sphinxcode{\sphinxupquote{LaueTools.LaueGeometry.}}\sphinxbfcode{\sphinxupquote{convert\_xycam\_from\_sourceshift}}}{\emph{OMs}, \emph{IIp}, \emph{calib}, \emph{verbose=0}}{}
From x,y on CCD camera (OMs) and source shift (IIprime)
compute modified x,y values for the SAME calibration (calib)(for further analysis)

return new value of x,y

\end{fulllineitems}

\index{lengthInSample() (in module LaueTools.LaueGeometry)}

\begin{fulllineitems}
\phantomsection\label{\detokenize{Simulation_Module:LaueTools.LaueGeometry.lengthInSample}}\pysiglinewithargsret{\sphinxcode{\sphinxupquote{LaueTools.LaueGeometry.}}\sphinxbfcode{\sphinxupquote{lengthInSample}}}{\emph{depth}, \emph{twtheta}, \emph{chi}, \emph{omega}, \emph{verbose=False}}{}
compute geometrical lengthes in sample from impact point (I) at the surface to a point (B)
where xray are scattered (or fluorescence is emitted) and finally escape from inside at point (C) lying at the sample surface
(intersection of line with unit vector u with sample surface plane tilted by omega)

\begin{sphinxadmonition}{warning}{Warning:}
twtheta and chi angles can be misleading.
Assumption is made that angles of unit vector from B to C (or to detector frame pixel) are
\(2 \theta\) and \(\chi\).
For large depth D, unit vector scattered beam direction is not given by \(2 \theta\) and \(\chi\) angles
as they are used for describing the scattering direction from point I and a given detector frame position
(you should then compute the two angles correction , actually \(\chi\) is unchanged, and the \(2 \theta\) change is approx
d/ distance .i.e. 3 10-4 for d=20 µm and CCD at 70 mm)
\end{sphinxadmonition}

\begin{sphinxadmonition}{note}{Note:}\begin{description}
\item[{incoming beam coming from the right positive x direction with}] \leavevmode\begin{itemize}
\item {} 
IB = (-D,0,0)

\item {} 
BC =(xc+D,yc,zc)

\item {} 
and length BC is proportional to the depth D

\end{itemize}

\end{description}
\end{sphinxadmonition}

\end{fulllineitems}



\subsection{Multiple Grains and Strain/orientation Distribution}
\label{\detokenize{Simulation_Module:module-LaueTools.multigrainsSimulator}}\label{\detokenize{Simulation_Module:multiple-grains-and-strain-orientation-distribution}}\index{LaueTools.multigrainsSimulator (module)}
Module to compute Laue Patterns from several crystals in various geometry

Main author is J. S. Micha:   micha {[}at{]} esrf {[}dot{]} fr

version July 2019
from LaueTools package for python2 hosted in

\sphinxurl{http://sourceforge.net/projects/lauetools/}

or for python3 and 2 in

\sphinxurl{https://gitlab.esrf.fr/micha/lauetools}
\index{Read\_GrainListparameter() (in module LaueTools.multigrainsSimulator)}

\begin{fulllineitems}
\phantomsection\label{\detokenize{Simulation_Module:LaueTools.multigrainsSimulator.Read_GrainListparameter}}\pysiglinewithargsret{\sphinxcode{\sphinxupquote{LaueTools.multigrainsSimulator.}}\sphinxbfcode{\sphinxupquote{Read\_GrainListparameter}}}{\emph{param}}{}
Read dictionary of  input key parameters for simulation

\end{fulllineitems}

\index{Construct\_GrainsParameters\_parametric() (in module LaueTools.multigrainsSimulator)}

\begin{fulllineitems}
\phantomsection\label{\detokenize{Simulation_Module:LaueTools.multigrainsSimulator.Construct_GrainsParameters_parametric}}\pysiglinewithargsret{\sphinxcode{\sphinxupquote{LaueTools.multigrainsSimulator.}}\sphinxbfcode{\sphinxupquote{Construct\_GrainsParameters\_parametric}}}{\emph{SelectGrains\_parametric}}{}
return list of simulation parameters for each grain set (mother and children grains)

\end{fulllineitems}

\index{dosimulation\_parametric() (in module LaueTools.multigrainsSimulator)}

\begin{fulllineitems}
\phantomsection\label{\detokenize{Simulation_Module:LaueTools.multigrainsSimulator.dosimulation_parametric}}\pysiglinewithargsret{\sphinxcode{\sphinxupquote{LaueTools.multigrainsSimulator.}}\sphinxbfcode{\sphinxupquote{dosimulation\_parametric}}}{\emph{\_list\_param, Transform\_params=None, SelectGrains=None, emax=25.0, emin=5.0, detectordistance=68.7, detectordiameter=165.0, posCEN=(1024.0, 1024.0), cameraAngles=(0.0, 0.0), gauge=None, kf\_direction='Z\textgreater{}0', pixelsize=0.08056640625, framedim=(2048, 2048), dictmaterials=\{'Al': {[}'Al', {[}4.05, 4.05, 4.05, 90, 90, 90{]}, 'fcc'{]}, 'Al2Cu': {[}'Al2Cu', {[}6.063, 6.063, 4.872, 90, 90, 90{]}, 'no'{]}, 'Al2O3': {[}'Al2O3', {[}4.785, 4.785, 12.991, 90, 90, 120{]}, 'Al2O3'{]}, 'Al2O3\_all': {[}'Al2O3\_all', {[}4.785, 4.785, 12.991, 90, 90, 120{]}, 'no'{]}, 'AlN': {[}'AlN', {[}3.11, 3.11, 4.98, 90.0, 90.0, 120.0{]}, 'wurtzite'{]}, 'Au': {[}'Au', {[}4.078, 4.078, 4.078, 90, 90, 90{]}, 'fcc'{]}, 'CCDL1949': {[}'CCDL1949', {[}9.89, 17.85, 5.31, 90, 107.5, 90{]}, 'h+k=2n'{]}, 'CdHgTe': {[}'CdHgTe', {[}6.46678, 6.46678, 6.46678, 90, 90, 90{]}, 'dia'{]}, 'CdHgTe\_fcc': {[}'CdHgTe\_fcc', {[}6.46678, 6.46678, 6.46678, 90, 90, 90{]}, 'fcc'{]}, 'CdTe': {[}'CdTe', {[}6.487, 6.487, 6.487, 90, 90, 90{]}, 'fcc'{]}, 'CdTeDiagB': {[}'CdTeDiagB', {[}4.5721, 7.9191, 11.1993, 90, 90, 90{]}, 'no'{]}, 'Crocidolite': {[}'Crocidolite', {[}9.811, 18.013, 5.326, 90, 103.68, 90{]}, 'no'{]}, 'Crocidolite\_2': {[}'Crocidolite\_2', {[}9.76, 17.93, 5.35, 90, 103.6, 90{]}, 'no'{]}, 'Crocidolite\_2\_72deg': {[}'Crocidolite\_2', {[}9.76, 17.93, 5.35, 90, 76.4, 90{]}, 'no'{]}, 'Crocidolite\_small': {[}'Crocidolite\_small', {[}3.2533333333333334, 5.976666666666667, 1.7833333333333332, 90, 103.6, 90{]}, 'no'{]}, 'Crocidolite\_whittaker\_1949': {[}'Crocidolite\_whittaker\_1949', {[}9.89, 17.85, 5.31, 90, 107.5, 90{]}, 'no'{]}, 'Cu': {[}'Cu', {[}3.6, 3.6, 3.6, 90, 90, 90{]}, 'fcc'{]}, 'Cu6Sn5\_monoclinic': {[}'Cu6Sn5\_monoclinic', {[}11.02, 7.28, 9.827, 90, 98.84, 90{]}, 'no'{]}, 'Cu6Sn5\_tetra': {[}'Cu6Sn5\_tetra', {[}3.608, 3.608, 5.037, 90, 90, 90{]}, 'no'{]}, 'DIA': {[}'DIA', {[}5.0, 5.0, 5.0, 90, 90, 90{]}, 'dia'{]}, 'DIAs': {[}'DIAs', {[}3.56683, 3.56683, 3.56683, 90, 90, 90{]}, 'dia'{]}, 'DarinaMolecule': {[}'DarinaMolecule', {[}9.4254, 13.5004, 13.8241, 61.83, 84.555, 75.231{]}, 'no'{]}, 'FCC': {[}'FCC', {[}5.0, 5.0, 5.0, 90, 90, 90{]}, 'fcc'{]}, 'Fe': {[}'Fe', {[}2.856, 2.856, 2.856, 90, 90, 90{]}, 'bcc'{]}, 'Fe2Ta': {[}'Fe2Ta', {[}4.83, 4.83, 0.788, 90, 90, 120{]}, 'no'{]}, 'FeAl': {[}'FeAl', {[}5.871, 5.871, 5.871, 90, 90, 90{]}, 'fcc'{]}, 'GaAs': {[}'GaAs', {[}5.65325, 5.65325, 5.65325, 90, 90, 90{]}, 'dia'{]}, 'GaN': {[}'GaN', {[}3.189, 3.189, 5.185, 90, 90, 120{]}, 'wurtzite'{]}, 'GaN\_all': {[}'GaN\_all', {[}3.189, 3.189, 5.185, 90, 90, 120{]}, 'no'{]}, 'Ge': {[}'Ge', {[}5.6575, 5.6575, 5.6575, 90, 90, 90{]}, 'dia'{]}, 'Ge\_compressedhydro': {[}'Ge\_compressedhydro', {[}5.64, 5.64, 5.64, 90, 90, 90.0{]}, 'dia'{]}, 'Ge\_s': {[}'Ge\_s', {[}5.6575, 5.6575, 5.6575, 90, 90, 89.5{]}, 'dia'{]}, 'Getest': {[}'Getest', {[}5.6575, 5.6575, 5.6574, 90, 90, 90{]}, 'dia'{]}, 'Hematite': {[}'Hematite', {[}5.03459, 5.03459, 13.7533, 90, 90, 120{]}, 'no'{]}, 'In': {[}'In', {[}3.2517, 3.2517, 4.9459, 90, 90, 90{]}, 'h+k+l=2n'{]}, 'InGaN': {[}'InGaN', {[}3.3609999999999998, 3.3609999999999998, 5.439, 90, 90, 120{]}, 'wurtzite'{]}, 'InN': {[}'InN', {[}3.533, 3.533, 5.693, 90, 90, 120{]}, 'wurtzite'{]}, 'Magnetite': {[}'Magnetite', {[}8.391, 8.391, 8.391, 90, 90, 90{]}, 'dia'{]}, 'Magnetite\_fcc': {[}'Magnetite\_fcc', {[}8.391, 8.391, 8.391, 90, 90, 90{]}, 'fcc'{]}, 'Magnetite\_sc': {[}'Magnetite\_sc', {[}8.391, 8.391, 8.391, 90, 90, 90{]}, 'no'{]}, 'Nd45': {[}'Nd45', {[}5.4884, 5.4884, 5.4884, 90, 90, 90{]}, 'fcc'{]}, 'Ni': {[}'Ni', {[}3.5238, 3.5238, 3.5238, 90, 90, 90{]}, 'fcc'{]}, 'NiO': {[}'NiO', {[}2.96, 2.96, 7.23, 90, 90, 120{]}, 'no'{]}, 'NiTi': {[}'NiTi', {[}3.5506, 3.5506, 3.5506, 90, 90, 90{]}, 'fcc'{]}, 'SC': {[}'SC', {[}1.0, 1.0, 1.0, 90, 90, 90{]}, 'no'{]}, 'SC5': {[}'SC5', {[}5.0, 5.0, 5.0, 90, 90, 90{]}, 'no'{]}, 'SC7': {[}'SC7', {[}7.0, 7.0, 7.0, 90, 90, 90{]}, 'no'{]}, 'Sb': {[}'Sb', {[}4.3, 4.3, 11.3, 90, 90, 120{]}, 'no'{]}, 'Si': {[}'Si', {[}5.4309, 5.4309, 5.4309, 90, 90, 90{]}, 'dia'{]}, 'Sn': {[}'Sn', {[}5.83, 5.83, 3.18, 90, 90, 90{]}, 'h+k+l=2n'{]}, 'Ti': {[}'Ti', {[}2.95, 2.95, 4.68, 90, 90, 120{]}, 'no'{]}, 'Ti2AlN': {[}'Ti2AlN', {[}2.989, 2.989, 13.624, 90, 90, 120{]}, 'Ti2AlN'{]}, 'Ti2AlN\_w': {[}'Ti2AlN\_w', {[}2.989, 2.989, 13.624, 90, 90, 120{]}, 'wurtzite'{]}, 'Ti\_beta': {[}'Ti\_beta', {[}3.2587, 3.2587, 3.2587, 90, 90, 90{]}, 'bcc'{]}, 'Ti\_omega': {[}'Ti\_omega', {[}4.6085, 4.6085, 2.8221, 90, 90, 120{]}, 'no'{]}, 'Ti\_s': {[}'Ti\_s', {[}3.0, 3.0, 4.7, 90.5, 89.5, 120.5{]}, 'no'{]}, 'UO2': {[}'UO2', {[}5.47, 5.47, 5.47, 90, 90, 90{]}, 'fcc'{]}, 'VO2M1': {[}'VO2M1', {[}5.75175, 4.52596, 5.38326, 90.0, 122.6148, 90.0{]}, 'VO2\_mono'{]}, 'VO2M2': {[}'VO2M2', {[}4.5546, 4.5546, 2.8514, 90.0, 90, 90.0{]}, 'no'{]}, 'VO2R': {[}'VO2R', {[}4.5546, 4.5546, 2.8514, 90.0, 90, 90.0{]}, 'rutile'{]}, 'W': {[}'W', {[}3.1652, 3.1652, 3.1652, 90, 90, 90{]}, 'bcc'{]}, 'YAG': {[}'YAG', {[}9.2, 9.2, 9.2, 90, 90, 90{]}, 'no'{]}, 'ZnO': {[}'ZnO', {[}3.252, 3.252, 5.213, 90, 90, 120{]}, 'wurtzite'{]}, 'ZrO2': {[}'ZrO2', {[}5.1505, 5.2116, 5.3173, 90, 99.23, 90{]}, 'VO2\_mono'{]}, 'ZrO2Y2O3': {[}'ZrO2Y2O3', {[}5.1378, 5.1378, 5.1378, 90, 90, 90{]}, 'fcc'{]}, 'alphaQuartz': {[}'alphaQuartz', {[}4.9, 4.9, 5.4, 90, 90, 120{]}, 'no'{]}, 'betaQuartznew': {[}'betaQuartznew', {[}4.9, 4.9, 6.685, 90, 90, 120{]}, 'no'{]}, 'bigpro': {[}'bigpro', {[}112.0, 112.0, 136.0, 90, 90, 90{]}, 'no'{]}, 'dummy': {[}'dummy', {[}4.0, 8.0, 2.0, 90, 90, 90{]}, 'no'{]}, 'ferrydrite': {[}'ferrydrite', {[}2.96, 2.96, 9.4, 90, 90, 120{]}, 'no'{]}, 'hexagonal': {[}'hexagonal', {[}1.0, 1.0, 3.0, 90, 90, 120.0{]}, 'no'{]}, 'inputB': {[}'inputB', {[}1.0, 1.0, 1.0, 90, 90, 90{]}, 'no'{]}, 'quartz\_alpha': {[}'quartz\_alpha', {[}4.913, 4.913, 5.404, 90, 90, 120{]}, 'no'{]}, 'smallpro': {[}'smallpro', {[}20.0, 4.8, 49.0, 90, 90, 90{]}, 'no'{]}, 'test\_reference': {[}'test\_reference', {[}3.2, 4.5, 5.2, 83, 92.0, 122{]}, 'wurtzite'{]}, 'test\_solution': {[}'test\_solution', {[}3.252, 4.48, 5.213, 83.2569, 92.125478, 122.364{]}, 'wurtzite'{]}, 'testindex': {[}'testindex', {[}2.0, 1.0, 4.0, 90, 90, 90{]}, 'no'{]}, 'testindex2': {[}'testindex2', {[}2.0, 1.0, 4.0, 75, 90, 120{]}, 'no'{]}\}}}{}
Simulation of orientation or deformation gradient.
From parent grain simulate a list of transformations (deduced by a parametric variation)

\_list\_param   : list of parameters for each grain  {[}grain parameters, grain name{]}

posCEN =(Xcen, Ycen)
cameraAngles =(Xbet, Xgam)
\begin{quote}\begin{description}
\item[{Returns}] \leavevmode
(list\_twicetheta,
list\_chi,
list\_energy,
list\_Miller,
list\_posX,
list\_posY,
ParentGrainName\_list,
list\_ParentGrain\_transforms,
calib,
total\_nb\_grains)

\end{description}\end{quote}

TODO:simulate for any camera position
TODO: simulate spatial distribution of laue pattern origin

\end{fulllineitems}



\section{Modules for Digital Image processing, Peak Search \& Fitting}
\label{\detokenize{LaueToolsModules:modules-for-digital-image-processing-peak-search-fitting}}

\subsection{PeakSearchGUI}
\label{\detokenize{PeakSearchGUI::doc}}\label{\detokenize{PeakSearchGUI:peaksearchgui}}\label{\detokenize{PeakSearchGUI:id1}}
The module PeakSearchGUI.py is made to provide graphical tools to perform peak search and fitting. It enables also tools to get mosaic from a set of images.


\subsubsection{Read Images and Binary files}
\label{\detokenize{PeakSearchGUI:read-images-and-binary-files}}
First select the detector you have used for the data collection in the menu Calibration

\noindent\sphinxincludegraphics{{clickcalib}.png}

Choose the correct camera to define the binary image file parameters

\noindent\sphinxincludegraphics{{ccdfileselect}.png}

View an image by selecting in the Menu  File/Open Image and Peak Search.

\noindent\sphinxincludegraphics{{peaksearchBoard}.png}

Then you obtain a board enabling to browse your set of images and to search for peaks

\noindent\sphinxincludegraphics[scale=0.6]{{peaksearchGUI}.png}

This board is composed by composed by 4 TOP tabs:
\begin{itemize}
\item {} 
parameters of display (View \& color)

\item {} 
Digital image processing (Image \& Filter)

\item {} 
Browse a set of images and select a ROI (Browse \& crop)

\item {} 
Mosaic or image-related monitor from a ROI over a set of images

\end{itemize}

And 5 tabs at the middle
\begin{itemize}
\item {} 
1 Selection of local maxima search Method

\item {} 
2 Fitting procedure parameters

\item {} 
3 Peaks List Manager

\end{itemize}


\subsubsection{View \& Color}
\label{\detokenize{PeakSearchGUI:view-color}}
This panel helps for viewing images with an appropriate color LUT, getting some infos on pixels intensity on the whole image (distribution) or along line (line profilers)
\begin{itemize}
\item {} 
Color Mapping (LUT) of the displayed image. \sphinxcode{\sphinxupquote{ShowHisto}} displays the histogram of intensity distribution (nb of pixel as a function of pixel intensity)
\begin{quote}

\noindent\sphinxincludegraphics{{lutparam}.png}
\end{quote}

\item {} 
intensities limits taken into account by the LUT
\begin{quote}

\noindent\sphinxincludegraphics{{imagedisplaylimits}.png}
\end{quote}

\item {} 
\sphinxcode{\sphinxupquote{Open LineProfiler}}: 1D plot of pixel intensities along a movable-by-user line. And \sphinxcode{\sphinxupquote{Eneable X Y Profiler}}: 1D plot of pixel intensities along X and Y from a clicked pixel position

\end{itemize}

\noindent\sphinxincludegraphics{{profilers}.png}


\subsubsection{Image \& Filter}
\label{\detokenize{PeakSearchGUI:image-filter}}
This panel supplies digital image processing tools to filter current image and particularly remove background.

\sphinxcode{\sphinxupquote{Blur Image}} computes the image filtered by a gaussian kernel (similar to a low pass filter). By checking \sphinxcode{\sphinxupquote{Substract blur as background}} the raw image minus the filtered image can be displayed and used for the local maxima (blob) search.

\sphinxcode{\sphinxupquote{Calculate image with A and B}} allows with an arithmetical formula with A (current image) and B (additional image to be open) the computation of a new image A’ on which will be performed the local maxima (blob) searchChecking. By default this image will not be used to refine the position and shape of local maxima but the former and initial A image. Check \sphinxcode{\sphinxupquote{use also for fit}} to apply the fit procedures on pixel intensities A’.

\sphinxcode{\sphinxupquote{Save results}} saves on hard disk the A’ image with the header contains and format of A.
\begin{quote}

\noindent\sphinxincludegraphics{{imagefilterbckg}.png}
\end{quote}


\subsubsection{Browse \& Crop}
\label{\detokenize{PeakSearchGUI:browse-crop}}
One can navigate on a set of images provided the images file name contains a numerical index with a constant prefix (e.g. myimage\_0032.ccd). Navigation with button with small step \sphinxcode{\sphinxupquote{index-1}} and \sphinxcode{\sphinxupquote{index+1}} corresponds to consecutive images collected with time or along a line on sample.  Navigation with button with larger step (Nb of images per line (step index)) \sphinxcode{\sphinxupquote{index-10}} and \sphinxcode{\sphinxupquote{index+10}} (for instance with larger step equals to 10) permits to look at the images collected along the direction perpendicular to the direction corresponding to the small step.

The \sphinxcode{\sphinxupquote{Go To index}} button allows to read directly an image with an other in the same dataset. \sphinxcode{\sphinxupquote{Auto index+1}} button will display the next image and wait for it if it is not already in the folder.
\begin{quote}

\noindent\sphinxincludegraphics{{imagebrowser}.png}
\end{quote}

To navigate and display faster the image when browsing on a particular region of interest (ROI) of the images, you can crop the data (\sphinxcode{\sphinxupquote{CropData}}) by specifying the half sizes (\sphinxcode{\sphinxupquote{boxsize}}) of the cropping box in the two directions.
\begin{quote}

\noindent\sphinxincludegraphics{{cropdata}.png}
\end{quote}


\subsubsection{Mosaic \& Monitor}
\label{\detokenize{PeakSearchGUI:mosaic-monitor}}
Several counters can be defined from pixel intensities in the same ROI (centered a clicked pixel with half box sizes given by user) over a set of images.
\begin{itemize}
\item {} 
Mosaic: Recompose a 2D raster scan from the selected ROI of each images as a function of image index

\item {} 
Mean Value: plot a 1D graph or a 2D raster scan from the mean pixel value in the selected ROI of each images as a function of image index

\item {} 
Max Value: plot a 1D graph or a 2D raster scan from the maximum pixel value in the selected ROI of each images as a function of image index

\item {} 
Peak-to-peak Value (or ‘peak to valley’): plot a 1D graph or a 2D raster scan from the largest pixel amplitude value in the selected ROI of each images as a function of image index

\item {} 
Peak position: plot two 1D graphes of X and Y peak position of the peak in the selected ROI of each images as a function of image index

\end{itemize}


\subsubsection{Plot Tools}
\label{\detokenize{PeakSearchGUI:plot-tools}}
The standard toolbar is provided by the graphical ‘matplotlib’ library with the following features:
\begin{itemize}
\item {} 
A ROI can be set to zoom in the data.
\begin{quote}

\noindent\sphinxincludegraphics{{zoominrectsmall}.png}
\end{quote}

\item {} 
This ROI can be moved easily with the pan button
\begin{quote}

\noindent\sphinxincludegraphics{{panbtnsmall}.png}
\end{quote}

\item {} 
When hovering the mouse on the image pixel position and corresponding intensity are displayed
\begin{quote}

\noindent\sphinxincludegraphics{{localinfosmall}.png}
\end{quote}

\item {} 
Previous selected ROIs are stored and can be recalled by arrows (\sphinxcode{\sphinxupquote{Home}} icon recalls the initial full image)

\end{itemize}


\paragraph{Local Maxima or Blob Search}
\label{\detokenize{PeakSearchGUI:local-maxima-or-blob-search}}
To guess the initial parameters fitting, 3 methods leads to a list of local maxima (or blobs)

\noindent\sphinxincludegraphics{{tablocalmaxima}.png}


\subsubsection{Method 1: pixels above a threshold on raw image}
\label{\detokenize{PeakSearchGUI:method-1-pixels-above-a-threshold-on-raw-image}}
The basic method consists in considering every pixel higher than \sphinxcode{\sphinxupquote{intensityThreshold}} as a local maxima pixel.
If \sphinxcode{\sphinxupquote{intensityThreshold}} is too high you will get only few pixels at the submit of laue spots.
If \sphinxcode{\sphinxupquote{intensityThreshold}} is too small you will too much pixels that you may stuck the software.
\sphinxcode{\sphinxupquote{MinimumDistance}} is the smallest distance separating two local maxima.
A good habit is too check the highest background level (e.g close to the image centre) and set \sphinxcode{\sphinxupquote{intensityThreshold}} to a larger value. But even in this case if (fluorescence) background varies a lot, you will miss peaks whose maximum intensities are below the threshold… This is why removing the background is mandatory.

\noindent\sphinxincludegraphics{{method1small}.png}


\subsubsection{Method 2: hottest pixel in a box (by array shift)}
\label{\detokenize{PeakSearchGUI:method-2-hottest-pixel-in-a-box-by-array-shift}}
Second Method finds the hottest pixel in a small box given by \sphinxcode{\sphinxupquote{PixelNearRadius}}. It shifts the whole data array in 8 directions and determines every pixel hotter than the others lying in these 8 directions.
The \sphinxstylestrong{thresholding} with the \sphinxcode{\sphinxupquote{IntensityThreshold}} level is performed on the intensity of these hot pixels \sphinxstylestrong{with respect to local background level} (set to the lowest pixel intensity in the box around the hot pixel).
One drawback of this method is that 2 hot pixels at the top of the peak but with strictly the same intensity are not detected (coincidence or more likely when peak is saturated).

\noindent\sphinxincludegraphics{{method2small}.png}


\subsubsection{Method 3: Peak enhancement by convolution on raw image}
\label{\detokenize{PeakSearchGUI:method-3-peak-enhancement-by-convolution-on-raw-image}}
This is the fastest method as soon as you have found the few parameters value (for batch). The \sphinxstyleemphasis{Raw image} data (unsigned 16 bits integers) are convolved by a 2D gaussian (mexican-hat-like) kernel. The resulting \sphinxstyleemphasis{convolved Image} (floats) have intense region (called blobs) where pixel intensity 2D profile on raw data is similar to the kernel intensity profile.
A first threshold with the level \sphinxcode{\sphinxupquote{ThresholdConvolve}} (float) allows to select enhanced blob above a background. An good estimate of \sphinxcode{\sphinxupquote{ThresholdConvolve}} value can be found by means of the pixel intensity histogram (\sphinxcode{\sphinxupquote{ShowHisto}}): it corresponds to the (float) intensity that separates the hottest pixel population that belong to the peak (large abscisssa value) from the weakest one that belong to background.
With this blobs list, a second thresholding at \sphinxcode{\sphinxupquote{Intensity Threshold (raw data)}} level is performed in the raw data pixel intensity with respect to local background level (like in method 2).
\sphinxcode{\sphinxupquote{PixelNearRadius}} value enables the user to reject too much closed spots.
\sphinxcode{\sphinxupquote{Max.Intensity}} value is used for the display of the convolved Image by clicking on \sphinxcode{\sphinxupquote{Show Conv. Image}}. Thresholding can be visualize by checking \sphinxcode{\sphinxupquote{Show thresholding}}.

\noindent\sphinxincludegraphics{{method3small}.png}


\paragraph{Peak Position Fit}
\label{\detokenize{PeakSearchGUI:peak-position-fit}}
The goal of the peak search is to have systematic values of Laue peaks and intensities. You can decide to fit or not the blob found. The Button \sphinxcode{\sphinxupquote{Search All Peaks}} launches  the local maxima chosen method and apply a fitting procedure (or not) to each found blob. A general peak list is built.
By clicking close to a Laue spot in the image and clicking on the \sphinxcode{\sphinxupquote{Fit Peak}} button, a gaussian fit is performed. This result can be added to the current general peak list with the button \sphinxcode{\sphinxupquote{Add Peak}}. Clicking close to a Laue spot that belongs to the peak list (blue circle marker on top of the image) and pressing \sphinxcode{\sphinxupquote{Remove Peak}} removes the laue spot from the list.

\noindent\sphinxincludegraphics{{finalbtns}.png}

Parameters to choose the fitting model or no fit and to reject or not the fitting result of each peak according to deviation from the initial guessed position (\sphinxcode{\sphinxupquote{FitPixelDeviation}}).

\noindent\sphinxincludegraphics{{selectfitornotsmall}.png}

\noindent\sphinxincludegraphics{{fitparam}.png}


\paragraph{Module functions}
\label{\detokenize{PeakSearchGUI:module-functions}}
The next documentation comes from the docstring in the header of function or class definition.


\subsubsection{PeakSearchGUI.py (PeakSearchBoard)}
\label{\detokenize{PeakSearchGUI:peaksearchgui-py-peaksearchboard}}\label{\detokenize{PeakSearchGUI:module-LaueTools.GUI.PeakSearchGUI}}\index{LaueTools.GUI.PeakSearchGUI (module)}\index{ViewColorPanel (class in LaueTools.GUI.PeakSearchGUI)}

\begin{fulllineitems}
\phantomsection\label{\detokenize{PeakSearchGUI:LaueTools.GUI.PeakSearchGUI.ViewColorPanel}}\pysiglinewithargsret{\sphinxbfcode{\sphinxupquote{class }}\sphinxcode{\sphinxupquote{LaueTools.GUI.PeakSearchGUI.}}\sphinxbfcode{\sphinxupquote{ViewColorPanel}}}{\emph{parent}}{}
class to play with color LUT and intensity scale
\index{OnReplot() (LaueTools.GUI.PeakSearchGUI.ViewColorPanel method)}

\begin{fulllineitems}
\phantomsection\label{\detokenize{PeakSearchGUI:LaueTools.GUI.PeakSearchGUI.ViewColorPanel.OnReplot}}\pysiglinewithargsret{\sphinxbfcode{\sphinxupquote{OnReplot}}}{\emph{\_}}{}
trigger main self.mainframe.OnReplot(1)

\end{fulllineitems}

\index{getclickposition() (LaueTools.GUI.PeakSearchGUI.ViewColorPanel method)}

\begin{fulllineitems}
\phantomsection\label{\detokenize{PeakSearchGUI:LaueTools.GUI.PeakSearchGUI.ViewColorPanel.getclickposition}}\pysiglinewithargsret{\sphinxbfcode{\sphinxupquote{getclickposition}}}{\emph{event}}{}
return closest pixel integer coordinates to clicked pt

\end{fulllineitems}

\index{OnShowLineXYProfiler() (LaueTools.GUI.PeakSearchGUI.ViewColorPanel method)}

\begin{fulllineitems}
\phantomsection\label{\detokenize{PeakSearchGUI:LaueTools.GUI.PeakSearchGUI.ViewColorPanel.OnShowLineXYProfiler}}\pysiglinewithargsret{\sphinxbfcode{\sphinxupquote{OnShowLineXYProfiler}}}{\emph{event}}{}
show line X and Y profilers

\end{fulllineitems}

\index{bestposition() (LaueTools.GUI.PeakSearchGUI.ViewColorPanel method)}

\begin{fulllineitems}
\phantomsection\label{\detokenize{PeakSearchGUI:LaueTools.GUI.PeakSearchGUI.ViewColorPanel.bestposition}}\pysiglinewithargsret{\sphinxbfcode{\sphinxupquote{bestposition}}}{\emph{LINEPROFILE\_WIDTH}, \emph{LINEPROFILE\_HEIGHT}}{}
return xp, yp (best position)

\end{fulllineitems}

\index{restrictxylimits\_to\_imagearray() (LaueTools.GUI.PeakSearchGUI.ViewColorPanel method)}

\begin{fulllineitems}
\phantomsection\label{\detokenize{PeakSearchGUI:LaueTools.GUI.PeakSearchGUI.ViewColorPanel.restrictxylimits_to_imagearray}}\pysiglinewithargsret{\sphinxbfcode{\sphinxupquote{restrictxylimits\_to\_imagearray}}}{\emph{xmin}, \emph{ymin}, \emph{xmax}, \emph{ymax}}{}
return compatible extremal value of x,y
xmin, ymin, xmax, ymax

\end{fulllineitems}

\index{getlineprofiledata() (LaueTools.GUI.PeakSearchGUI.ViewColorPanel method)}

\begin{fulllineitems}
\phantomsection\label{\detokenize{PeakSearchGUI:LaueTools.GUI.PeakSearchGUI.ViewColorPanel.getlineprofiledata}}\pysiglinewithargsret{\sphinxbfcode{\sphinxupquote{getlineprofiledata}}}{\emph{x0}, \emph{y0}, \emph{x2}, \emph{y2}}{}
get pixel intensities array between two points (x0,y0) and (x2,y2)

return dataX, dataY

\end{fulllineitems}

\index{OnShowLineProfiler() (LaueTools.GUI.PeakSearchGUI.ViewColorPanel method)}

\begin{fulllineitems}
\phantomsection\label{\detokenize{PeakSearchGUI:LaueTools.GUI.PeakSearchGUI.ViewColorPanel.OnShowLineProfiler}}\pysiglinewithargsret{\sphinxbfcode{\sphinxupquote{OnShowLineProfiler}}}{\emph{\_}}{}
show a movable and draggable line profiler and draw line and circle

\end{fulllineitems}

\index{updateLineXYProfile() (LaueTools.GUI.PeakSearchGUI.ViewColorPanel method)}

\begin{fulllineitems}
\phantomsection\label{\detokenize{PeakSearchGUI:LaueTools.GUI.PeakSearchGUI.ViewColorPanel.updateLineXYProfile}}\pysiglinewithargsret{\sphinxbfcode{\sphinxupquote{updateLineXYProfile}}}{\emph{event}}{}
recompute line section intensity profile
horizontal (x, zx)
vertical (y, zy)

\end{fulllineitems}

\index{onOpenDetFile() (LaueTools.GUI.PeakSearchGUI.ViewColorPanel method)}

\begin{fulllineitems}
\phantomsection\label{\detokenize{PeakSearchGUI:LaueTools.GUI.PeakSearchGUI.ViewColorPanel.onOpenDetFile}}\pysiglinewithargsret{\sphinxbfcode{\sphinxupquote{onOpenDetFile}}}{\emph{\_}}{}
open and read .det file with geometry detection calibration parameters

set self.DetFilename
set self.mainframe.DetFilename

\end{fulllineitems}

\index{showImage() (LaueTools.GUI.PeakSearchGUI.ViewColorPanel method)}

\begin{fulllineitems}
\phantomsection\label{\detokenize{PeakSearchGUI:LaueTools.GUI.PeakSearchGUI.ViewColorPanel.showImage}}\pysiglinewithargsret{\sphinxbfcode{\sphinxupquote{showImage}}}{}{}
branching from button of ViewColorPanel class: show blur/raw image

\end{fulllineitems}


\end{fulllineitems}

\index{FilterBackGroundPanel (class in LaueTools.GUI.PeakSearchGUI)}

\begin{fulllineitems}
\phantomsection\label{\detokenize{PeakSearchGUI:LaueTools.GUI.PeakSearchGUI.FilterBackGroundPanel}}\pysiglinewithargsret{\sphinxbfcode{\sphinxupquote{class }}\sphinxcode{\sphinxupquote{LaueTools.GUI.PeakSearchGUI.}}\sphinxbfcode{\sphinxupquote{FilterBackGroundPanel}}}{\emph{parent}}{}
class to handle image background tools
\index{Computefilteredimage() (LaueTools.GUI.PeakSearchGUI.FilterBackGroundPanel method)}

\begin{fulllineitems}
\phantomsection\label{\detokenize{PeakSearchGUI:LaueTools.GUI.PeakSearchGUI.FilterBackGroundPanel.Computefilteredimage}}\pysiglinewithargsret{\sphinxbfcode{\sphinxupquote{Computefilteredimage}}}{}{}
remove background if self.blurimage exists

\end{fulllineitems}

\index{onComputeBlurImage() (LaueTools.GUI.PeakSearchGUI.FilterBackGroundPanel method)}

\begin{fulllineitems}
\phantomsection\label{\detokenize{PeakSearchGUI:LaueTools.GUI.PeakSearchGUI.FilterBackGroundPanel.onComputeBlurImage}}\pysiglinewithargsret{\sphinxbfcode{\sphinxupquote{onComputeBlurImage}}}{\emph{\_}}{}
Compute background, blurred, filtered or low frequency spatial image
from current image

set self.blurimage

\end{fulllineitems}

\index{OnSwitchBlurRawImage() (LaueTools.GUI.PeakSearchGUI.FilterBackGroundPanel method)}

\begin{fulllineitems}
\phantomsection\label{\detokenize{PeakSearchGUI:LaueTools.GUI.PeakSearchGUI.FilterBackGroundPanel.OnSwitchBlurRawImage}}\pysiglinewithargsret{\sphinxbfcode{\sphinxupquote{OnSwitchBlurRawImage}}}{\emph{\_}}{}
set viewing of raw image (- background) or background (=filtered image)

\end{fulllineitems}

\index{onSaveBlurImage() (LaueTools.GUI.PeakSearchGUI.FilterBackGroundPanel method)}

\begin{fulllineitems}
\phantomsection\label{\detokenize{PeakSearchGUI:LaueTools.GUI.PeakSearchGUI.FilterBackGroundPanel.onSaveBlurImage}}\pysiglinewithargsret{\sphinxbfcode{\sphinxupquote{onSaveBlurImage}}}{\emph{\_}}{}
save on hard disk blurred or background image obtained from current image

\end{fulllineitems}

\index{onSaveFormulaResultImage() (LaueTools.GUI.PeakSearchGUI.FilterBackGroundPanel method)}

\begin{fulllineitems}
\phantomsection\label{\detokenize{PeakSearchGUI:LaueTools.GUI.PeakSearchGUI.FilterBackGroundPanel.onSaveFormulaResultImage}}\pysiglinewithargsret{\sphinxbfcode{\sphinxupquote{onSaveFormulaResultImage}}}{\emph{\_}}{}
save image on hard disk of data obtained by arithmetical formula

\end{fulllineitems}

\index{onGetBImagefilename() (LaueTools.GUI.PeakSearchGUI.FilterBackGroundPanel method)}

\begin{fulllineitems}
\phantomsection\label{\detokenize{PeakSearchGUI:LaueTools.GUI.PeakSearchGUI.FilterBackGroundPanel.onGetBImagefilename}}\pysiglinewithargsret{\sphinxbfcode{\sphinxupquote{onGetBImagefilename}}}{\emph{\_}}{}
open image as B image
set self.BImageFilename

\end{fulllineitems}

\index{onGetBlackListfilename() (LaueTools.GUI.PeakSearchGUI.FilterBackGroundPanel method)}

\begin{fulllineitems}
\phantomsection\label{\detokenize{PeakSearchGUI:LaueTools.GUI.PeakSearchGUI.FilterBackGroundPanel.onGetBlackListfilename}}\pysiglinewithargsret{\sphinxbfcode{\sphinxupquote{onGetBlackListfilename}}}{\emph{\_}}{}
open black peakslist file
set self.BlackListFilename
set self.mainframe.BlackListFilename

\end{fulllineitems}

\index{OnChangeUseFormula() (LaueTools.GUI.PeakSearchGUI.FilterBackGroundPanel method)}

\begin{fulllineitems}
\phantomsection\label{\detokenize{PeakSearchGUI:LaueTools.GUI.PeakSearchGUI.FilterBackGroundPanel.OnChangeUseFormula}}\pysiglinewithargsret{\sphinxbfcode{\sphinxupquote{OnChangeUseFormula}}}{\emph{evt}}{}
change arithmetical formula

\end{fulllineitems}


\end{fulllineitems}

\index{BrowseCropPanel (class in LaueTools.GUI.PeakSearchGUI)}

\begin{fulllineitems}
\phantomsection\label{\detokenize{PeakSearchGUI:LaueTools.GUI.PeakSearchGUI.BrowseCropPanel}}\pysiglinewithargsret{\sphinxbfcode{\sphinxupquote{class }}\sphinxcode{\sphinxupquote{LaueTools.GUI.PeakSearchGUI.}}\sphinxbfcode{\sphinxupquote{BrowseCropPanel}}}{\emph{parent}}{}
class to handle crop operation on images

\end{fulllineitems}

\index{MosaicAndMonitor (class in LaueTools.GUI.PeakSearchGUI)}

\begin{fulllineitems}
\phantomsection\label{\detokenize{PeakSearchGUI:LaueTools.GUI.PeakSearchGUI.MosaicAndMonitor}}\pysiglinewithargsret{\sphinxbfcode{\sphinxupquote{class }}\sphinxcode{\sphinxupquote{LaueTools.GUI.PeakSearchGUI.}}\sphinxbfcode{\sphinxupquote{MosaicAndMonitor}}}{\emph{parent}}{}~\index{OnMosaic() (LaueTools.GUI.PeakSearchGUI.MosaicAndMonitor method)}

\begin{fulllineitems}
\phantomsection\label{\detokenize{PeakSearchGUI:LaueTools.GUI.PeakSearchGUI.MosaicAndMonitor.OnMosaic}}\pysiglinewithargsret{\sphinxbfcode{\sphinxupquote{OnMosaic}}}{\emph{\_}}{}
launch main procedure of computing mosaic and displaying it

\end{fulllineitems}


\end{fulllineitems}

\index{ROISelection (class in LaueTools.GUI.PeakSearchGUI)}

\begin{fulllineitems}
\phantomsection\label{\detokenize{PeakSearchGUI:LaueTools.GUI.PeakSearchGUI.ROISelection}}\pysiglinewithargsret{\sphinxbfcode{\sphinxupquote{class }}\sphinxcode{\sphinxupquote{LaueTools.GUI.PeakSearchGUI.}}\sphinxbfcode{\sphinxupquote{ROISelection}}}{\emph{parent}}{}~\index{onCenterROI() (LaueTools.GUI.PeakSearchGUI.ROISelection method)}

\begin{fulllineitems}
\phantomsection\label{\detokenize{PeakSearchGUI:LaueTools.GUI.PeakSearchGUI.ROISelection.onCenterROI}}\pysiglinewithargsret{\sphinxbfcode{\sphinxupquote{onCenterROI}}}{\emph{evt}}{}
simply click and later press q

\end{fulllineitems}


\end{fulllineitems}

\index{PlotPeakListPanel (class in LaueTools.GUI.PeakSearchGUI)}

\begin{fulllineitems}
\phantomsection\label{\detokenize{PeakSearchGUI:LaueTools.GUI.PeakSearchGUI.PlotPeakListPanel}}\pysiglinewithargsret{\sphinxbfcode{\sphinxupquote{class }}\sphinxcode{\sphinxupquote{LaueTools.GUI.PeakSearchGUI.}}\sphinxbfcode{\sphinxupquote{PlotPeakListPanel}}}{\emph{parent}}{}
panel class to handle peaks list within GUI
\index{OnShowLineXYProfiler() (LaueTools.GUI.PeakSearchGUI.PlotPeakListPanel method)}

\begin{fulllineitems}
\phantomsection\label{\detokenize{PeakSearchGUI:LaueTools.GUI.PeakSearchGUI.PlotPeakListPanel.OnShowLineXYProfiler}}\pysiglinewithargsret{\sphinxbfcode{\sphinxupquote{OnShowLineXYProfiler}}}{\emph{event}}{}
show line X and Y profilers

\end{fulllineitems}

\index{getlineprofiledata() (LaueTools.GUI.PeakSearchGUI.PlotPeakListPanel method)}

\begin{fulllineitems}
\phantomsection\label{\detokenize{PeakSearchGUI:LaueTools.GUI.PeakSearchGUI.PlotPeakListPanel.getlineprofiledata}}\pysiglinewithargsret{\sphinxbfcode{\sphinxupquote{getlineprofiledata}}}{\emph{x0}, \emph{y0}, \emph{x2}, \emph{y2}}{}
get pixel intensities array between two points (x0,y0) and (x2,y2)

\end{fulllineitems}

\index{OnShowLineProfiler() (LaueTools.GUI.PeakSearchGUI.PlotPeakListPanel method)}

\begin{fulllineitems}
\phantomsection\label{\detokenize{PeakSearchGUI:LaueTools.GUI.PeakSearchGUI.PlotPeakListPanel.OnShowLineProfiler}}\pysiglinewithargsret{\sphinxbfcode{\sphinxupquote{OnShowLineProfiler}}}{\emph{\_}}{}
show a movable and draggable line profiler and draw line and circle

\end{fulllineitems}


\end{fulllineitems}

\index{findLocalMaxima\_Meth\_1 (class in LaueTools.GUI.PeakSearchGUI)}

\begin{fulllineitems}
\phantomsection\label{\detokenize{PeakSearchGUI:LaueTools.GUI.PeakSearchGUI.findLocalMaxima_Meth_1}}\pysiglinewithargsret{\sphinxbfcode{\sphinxupquote{class }}\sphinxcode{\sphinxupquote{LaueTools.GUI.PeakSearchGUI.}}\sphinxbfcode{\sphinxupquote{findLocalMaxima\_Meth\_1}}}{\emph{parent}}{}
class of method 1 for local maxima search (intensity threshold)

\end{fulllineitems}

\index{findLocalMaxima\_Meth\_2 (class in LaueTools.GUI.PeakSearchGUI)}

\begin{fulllineitems}
\phantomsection\label{\detokenize{PeakSearchGUI:LaueTools.GUI.PeakSearchGUI.findLocalMaxima_Meth_2}}\pysiglinewithargsret{\sphinxbfcode{\sphinxupquote{class }}\sphinxcode{\sphinxupquote{LaueTools.GUI.PeakSearchGUI.}}\sphinxbfcode{\sphinxupquote{findLocalMaxima\_Meth\_2}}}{\emph{parent}}{}
class of method parameters for 2nd method of local maxima(shifted arrays)

\end{fulllineitems}

\index{findLocalMaxima\_Meth\_3 (class in LaueTools.GUI.PeakSearchGUI)}

\begin{fulllineitems}
\phantomsection\label{\detokenize{PeakSearchGUI:LaueTools.GUI.PeakSearchGUI.findLocalMaxima_Meth_3}}\pysiglinewithargsret{\sphinxbfcode{\sphinxupquote{class }}\sphinxcode{\sphinxupquote{LaueTools.GUI.PeakSearchGUI.}}\sphinxbfcode{\sphinxupquote{findLocalMaxima\_Meth\_3}}}{\emph{parent}}{}
class of method 3 for local maxima search (convolution by a mexican hat kernel)
\index{OnSwitchImageDisplay() (LaueTools.GUI.PeakSearchGUI.findLocalMaxima\_Meth\_3 method)}

\begin{fulllineitems}
\phantomsection\label{\detokenize{PeakSearchGUI:LaueTools.GUI.PeakSearchGUI.findLocalMaxima_Meth_3.OnSwitchImageDisplay}}\pysiglinewithargsret{\sphinxbfcode{\sphinxupquote{OnSwitchImageDisplay}}}{\emph{evt}}{}
switch between raw and convolved image display (performed by MainPeakSearchFrame class)

\end{fulllineitems}


\end{fulllineitems}

\index{FitParametersPanel (class in LaueTools.GUI.PeakSearchGUI)}

\begin{fulllineitems}
\phantomsection\label{\detokenize{PeakSearchGUI:LaueTools.GUI.PeakSearchGUI.FitParametersPanel}}\pysiglinewithargsret{\sphinxbfcode{\sphinxupquote{class }}\sphinxcode{\sphinxupquote{LaueTools.GUI.PeakSearchGUI.}}\sphinxbfcode{\sphinxupquote{FitParametersPanel}}}{\emph{parent}}{}
\end{fulllineitems}

\index{PeakListOLV (class in LaueTools.GUI.PeakSearchGUI)}

\begin{fulllineitems}
\phantomsection\label{\detokenize{PeakSearchGUI:LaueTools.GUI.PeakSearchGUI.PeakListOLV}}\pysiglinewithargsret{\sphinxbfcode{\sphinxupquote{class }}\sphinxcode{\sphinxupquote{LaueTools.GUI.PeakSearchGUI.}}\sphinxbfcode{\sphinxupquote{PeakListOLV}}}{\emph{parent}}{}
panel embedding an ObjectListViewer from ObjectListView module

need of:
self.grangranparent.peaklistPixels
self.grangranparent.onRemovePeaktoPeaklist
self.grangranparent.OnReplot
self.grangranparent.framedim
and lot of other things with mainframe…
\index{OnRemove() (LaueTools.GUI.PeakSearchGUI.PeakListOLV method)}

\begin{fulllineitems}
\phantomsection\label{\detokenize{PeakSearchGUI:LaueTools.GUI.PeakSearchGUI.PeakListOLV.OnRemove}}\pysiglinewithargsret{\sphinxbfcode{\sphinxupquote{OnRemove}}}{\emph{\_}}{}
remove one element corresponding to the current highlighted row

\end{fulllineitems}


\end{fulllineitems}

\index{MainPeakSearchFrame (class in LaueTools.GUI.PeakSearchGUI)}

\begin{fulllineitems}
\phantomsection\label{\detokenize{PeakSearchGUI:LaueTools.GUI.PeakSearchGUI.MainPeakSearchFrame}}\pysiglinewithargsret{\sphinxbfcode{\sphinxupquote{class }}\sphinxcode{\sphinxupquote{LaueTools.GUI.PeakSearchGUI.}}\sphinxbfcode{\sphinxupquote{MainPeakSearchFrame}}}{\emph{parent}, \emph{\_id}, \emph{\_initialParameter}, \emph{title}, \emph{size=4}}{}
Class to show CCD frame pixel intensities
and provide tools for searching peaks
\index{create\_main\_panel() (LaueTools.GUI.PeakSearchGUI.MainPeakSearchFrame method)}

\begin{fulllineitems}
\phantomsection\label{\detokenize{PeakSearchGUI:LaueTools.GUI.PeakSearchGUI.MainPeakSearchFrame.create_main_panel}}\pysiglinewithargsret{\sphinxbfcode{\sphinxupquote{create\_main\_panel}}}{}{}
Creates the main panel with all the controls on it:
* mpl canvas
* mpl navigation toolbar
* Control panel for interaction

\end{fulllineitems}

\index{toplayout2() (LaueTools.GUI.PeakSearchGUI.MainPeakSearchFrame method)}

\begin{fulllineitems}
\phantomsection\label{\detokenize{PeakSearchGUI:LaueTools.GUI.PeakSearchGUI.MainPeakSearchFrame.toplayout2}}\pysiglinewithargsret{\sphinxbfcode{\sphinxupquote{toplayout2}}}{}{}
init top notebook tabs for image visualisation and processing

\end{fulllineitems}

\index{line\_select\_callback() (LaueTools.GUI.PeakSearchGUI.MainPeakSearchFrame method)}

\begin{fulllineitems}
\phantomsection\label{\detokenize{PeakSearchGUI:LaueTools.GUI.PeakSearchGUI.MainPeakSearchFrame.line_select_callback}}\pysiglinewithargsret{\sphinxbfcode{\sphinxupquote{line\_select\_callback}}}{\emph{eclick}, \emph{erelease}}{}
eclick and erelease are the press and release events

\end{fulllineitems}

\index{OnTabChange\_nb0() (LaueTools.GUI.PeakSearchGUI.MainPeakSearchFrame method)}

\begin{fulllineitems}
\phantomsection\label{\detokenize{PeakSearchGUI:LaueTools.GUI.PeakSearchGUI.MainPeakSearchFrame.OnTabChange_nb0}}\pysiglinewithargsret{\sphinxbfcode{\sphinxupquote{OnTabChange\_nb0}}}{\emph{event}}{}
handling changing tab of top notebook

\end{fulllineitems}

\index{askUserForDirname() (LaueTools.GUI.PeakSearchGUI.MainPeakSearchFrame method)}

\begin{fulllineitems}
\phantomsection\label{\detokenize{PeakSearchGUI:LaueTools.GUI.PeakSearchGUI.MainPeakSearchFrame.askUserForDirname}}\pysiglinewithargsret{\sphinxbfcode{\sphinxupquote{askUserForDirname}}}{}{}
provide a dialog to browse the folders and files

\end{fulllineitems}

\index{OnSetFileCCDParam() (LaueTools.GUI.PeakSearchGUI.MainPeakSearchFrame method)}

\begin{fulllineitems}
\phantomsection\label{\detokenize{PeakSearchGUI:LaueTools.GUI.PeakSearchGUI.MainPeakSearchFrame.OnSetFileCCDParam}}\pysiglinewithargsret{\sphinxbfcode{\sphinxupquote{OnSetFileCCDParam}}}{\emph{\_}}{}
Enter manually CCD file params
Launch Entry dialog

\end{fulllineitems}

\index{OnSetPlotSize() (LaueTools.GUI.PeakSearchGUI.MainPeakSearchFrame method)}

\begin{fulllineitems}
\phantomsection\label{\detokenize{PeakSearchGUI:LaueTools.GUI.PeakSearchGUI.MainPeakSearchFrame.OnSetPlotSize}}\pysiglinewithargsret{\sphinxbfcode{\sphinxupquote{OnSetPlotSize}}}{\emph{\_}}{}
set marker size

\end{fulllineitems}

\index{onClick() (LaueTools.GUI.PeakSearchGUI.MainPeakSearchFrame method)}

\begin{fulllineitems}
\phantomsection\label{\detokenize{PeakSearchGUI:LaueTools.GUI.PeakSearchGUI.MainPeakSearchFrame.onClick}}\pysiglinewithargsret{\sphinxbfcode{\sphinxupquote{onClick}}}{\emph{event}}{}
onclick

\end{fulllineitems}

\index{onKeyPressed() (LaueTools.GUI.PeakSearchGUI.MainPeakSearchFrame method)}

\begin{fulllineitems}
\phantomsection\label{\detokenize{PeakSearchGUI:LaueTools.GUI.PeakSearchGUI.MainPeakSearchFrame.onKeyPressed}}\pysiglinewithargsret{\sphinxbfcode{\sphinxupquote{onKeyPressed}}}{\emph{event}}{}
Handle key pressed

\end{fulllineitems}

\index{gettime() (LaueTools.GUI.PeakSearchGUI.MainPeakSearchFrame method)}

\begin{fulllineitems}
\phantomsection\label{\detokenize{PeakSearchGUI:LaueTools.GUI.PeakSearchGUI.MainPeakSearchFrame.gettime}}\pysiglinewithargsret{\sphinxbfcode{\sphinxupquote{gettime}}}{}{}
set self.currentime to current time

\end{fulllineitems}

\index{onToggle() (LaueTools.GUI.PeakSearchGUI.MainPeakSearchFrame method)}

\begin{fulllineitems}
\phantomsection\label{\detokenize{PeakSearchGUI:LaueTools.GUI.PeakSearchGUI.MainPeakSearchFrame.onToggle}}\pysiglinewithargsret{\sphinxbfcode{\sphinxupquote{onToggle}}}{\emph{event}}{}
handling on auto index button

\end{fulllineitems}

\index{onToggleCrop() (LaueTools.GUI.PeakSearchGUI.MainPeakSearchFrame method)}

\begin{fulllineitems}
\phantomsection\label{\detokenize{PeakSearchGUI:LaueTools.GUI.PeakSearchGUI.MainPeakSearchFrame.onToggleCrop}}\pysiglinewithargsret{\sphinxbfcode{\sphinxupquote{onToggleCrop}}}{\emph{event}}{}
activate/deactivate crop image mode

\end{fulllineitems}

\index{update() (LaueTools.GUI.PeakSearchGUI.MainPeakSearchFrame method)}

\begin{fulllineitems}
\phantomsection\label{\detokenize{PeakSearchGUI:LaueTools.GUI.PeakSearchGUI.MainPeakSearchFrame.update}}\pysiglinewithargsret{\sphinxbfcode{\sphinxupquote{update}}}{\emph{\_}}{}
update at each time step time

\end{fulllineitems}

\index{CurrentFileIsReady() (LaueTools.GUI.PeakSearchGUI.MainPeakSearchFrame method)}

\begin{fulllineitems}
\phantomsection\label{\detokenize{PeakSearchGUI:LaueTools.GUI.PeakSearchGUI.MainPeakSearchFrame.CurrentFileIsReady}}\pysiglinewithargsret{\sphinxbfcode{\sphinxupquote{CurrentFileIsReady}}}{}{}
return True if self.imagefilename is in folder and entire (correct size)

\end{fulllineitems}

\index{getIndex\_fromfilename() (LaueTools.GUI.PeakSearchGUI.MainPeakSearchFrame method)}

\begin{fulllineitems}
\phantomsection\label{\detokenize{PeakSearchGUI:LaueTools.GUI.PeakSearchGUI.MainPeakSearchFrame.getIndex_fromfilename}}\pysiglinewithargsret{\sphinxbfcode{\sphinxupquote{getIndex\_fromfilename}}}{}{}
get index of image from the image filename

\end{fulllineitems}

\index{setfilename() (LaueTools.GUI.PeakSearchGUI.MainPeakSearchFrame method)}

\begin{fulllineitems}
\phantomsection\label{\detokenize{PeakSearchGUI:LaueTools.GUI.PeakSearchGUI.MainPeakSearchFrame.setfilename}}\pysiglinewithargsret{\sphinxbfcode{\sphinxupquote{setfilename}}}{}{}
set filename from self.imagefilename, self.imageindex,
CCDLabel=self.CCDlabel

\end{fulllineitems}

\index{OnLargePlus() (LaueTools.GUI.PeakSearchGUI.MainPeakSearchFrame method)}

\begin{fulllineitems}
\phantomsection\label{\detokenize{PeakSearchGUI:LaueTools.GUI.PeakSearchGUI.MainPeakSearchFrame.OnLargePlus}}\pysiglinewithargsret{\sphinxbfcode{\sphinxupquote{OnLargePlus}}}{\emph{\_}}{}
increase self.imageindex by self.stepindex (vertical descending in sample raster scan)
and read new image and plot

\end{fulllineitems}

\index{OnLargeMinus() (LaueTools.GUI.PeakSearchGUI.MainPeakSearchFrame method)}

\begin{fulllineitems}
\phantomsection\label{\detokenize{PeakSearchGUI:LaueTools.GUI.PeakSearchGUI.MainPeakSearchFrame.OnLargeMinus}}\pysiglinewithargsret{\sphinxbfcode{\sphinxupquote{OnLargeMinus}}}{\emph{\_}}{}
decrease self.imageindex by self.stepindex (vertical ascendindg in sample raster scan)
and read new image and plot

\end{fulllineitems}

\index{OnPlus() (LaueTools.GUI.PeakSearchGUI.MainPeakSearchFrame method)}

\begin{fulllineitems}
\phantomsection\label{\detokenize{PeakSearchGUI:LaueTools.GUI.PeakSearchGUI.MainPeakSearchFrame.OnPlus}}\pysiglinewithargsret{\sphinxbfcode{\sphinxupquote{OnPlus}}}{\emph{\_}}{}
increase  self.imageindex by 1 (horizontal ascending to the right in sample raster scan)
and read new image and plot

\end{fulllineitems}

\index{OnMinus() (LaueTools.GUI.PeakSearchGUI.MainPeakSearchFrame method)}

\begin{fulllineitems}
\phantomsection\label{\detokenize{PeakSearchGUI:LaueTools.GUI.PeakSearchGUI.MainPeakSearchFrame.OnMinus}}\pysiglinewithargsret{\sphinxbfcode{\sphinxupquote{OnMinus}}}{\emph{\_}}{}
decrease  self.imageindex by 1 (horizontal descending to the left in sample raster scan)
and read new image and plot

\end{fulllineitems}

\index{OnGoto() (LaueTools.GUI.PeakSearchGUI.MainPeakSearchFrame method)}

\begin{fulllineitems}
\phantomsection\label{\detokenize{PeakSearchGUI:LaueTools.GUI.PeakSearchGUI.MainPeakSearchFrame.OnGoto}}\pysiglinewithargsret{\sphinxbfcode{\sphinxupquote{OnGoto}}}{\emph{\_}}{}
read image with selected self.imageindex and plot

\end{fulllineitems}

\index{onChangeIndex\_slider\_imagevert() (LaueTools.GUI.PeakSearchGUI.MainPeakSearchFrame method)}

\begin{fulllineitems}
\phantomsection\label{\detokenize{PeakSearchGUI:LaueTools.GUI.PeakSearchGUI.MainPeakSearchFrame.onChangeIndex_slider_imagevert}}\pysiglinewithargsret{\sphinxbfcode{\sphinxupquote{onChangeIndex\_slider\_imagevert}}}{\emph{\_}}{}
plot new image obtained by new index changed by vertical (slow axis) slider

\end{fulllineitems}

\index{read\_data() (LaueTools.GUI.PeakSearchGUI.MainPeakSearchFrame method)}

\begin{fulllineitems}
\phantomsection\label{\detokenize{PeakSearchGUI:LaueTools.GUI.PeakSearchGUI.MainPeakSearchFrame.read_data}}\pysiglinewithargsret{\sphinxbfcode{\sphinxupquote{read\_data}}}{\emph{secondaryImage=False}, \emph{secondaryImagefilename=None}}{}
read binary image file
\begin{description}
\item[{if secondaryImage update}] \leavevmode
self.dataimage\_ROI\_B

\item[{else update}] \leavevmode
self.dataimage\_ROI

\end{description}

\end{fulllineitems}

\index{OnCheckPlotValues() (LaueTools.GUI.PeakSearchGUI.MainPeakSearchFrame method)}

\begin{fulllineitems}
\phantomsection\label{\detokenize{PeakSearchGUI:LaueTools.GUI.PeakSearchGUI.MainPeakSearchFrame.OnCheckPlotValues}}\pysiglinewithargsret{\sphinxbfcode{\sphinxupquote{OnCheckPlotValues}}}{\emph{\_}}{}
enable or disable drawing of numerical pixel intensity value on plot

\end{fulllineitems}

\index{PlotValues() (LaueTools.GUI.PeakSearchGUI.MainPeakSearchFrame method)}

\begin{fulllineitems}
\phantomsection\label{\detokenize{PeakSearchGUI:LaueTools.GUI.PeakSearchGUI.MainPeakSearchFrame.PlotValues}}\pysiglinewithargsret{\sphinxbfcode{\sphinxupquote{PlotValues}}}{}{}
Draw numerical pixel intensity value on plot

\end{fulllineitems}

\index{init\_figure\_draw() (LaueTools.GUI.PeakSearchGUI.MainPeakSearchFrame method)}

\begin{fulllineitems}
\phantomsection\label{\detokenize{PeakSearchGUI:LaueTools.GUI.PeakSearchGUI.MainPeakSearchFrame.init_figure_draw}}\pysiglinewithargsret{\sphinxbfcode{\sphinxupquote{init\_figure\_draw}}}{}{}
init the figure

\end{fulllineitems}

\index{Show\_Image() (LaueTools.GUI.PeakSearchGUI.MainPeakSearchFrame method)}

\begin{fulllineitems}
\phantomsection\label{\detokenize{PeakSearchGUI:LaueTools.GUI.PeakSearchGUI.MainPeakSearchFrame.Show_Image}}\pysiglinewithargsret{\sphinxbfcode{\sphinxupquote{Show\_Image}}}{\emph{event}, \emph{datatype='Raw Image'}}{}
show self.dataimage\_ROI\_display

\end{fulllineitems}

\index{Show\_ConvolvedImage() (LaueTools.GUI.PeakSearchGUI.MainPeakSearchFrame method)}

\begin{fulllineitems}
\phantomsection\label{\detokenize{PeakSearchGUI:LaueTools.GUI.PeakSearchGUI.MainPeakSearchFrame.Show_ConvolvedImage}}\pysiglinewithargsret{\sphinxbfcode{\sphinxupquote{Show\_ConvolvedImage}}}{\emph{event}, \emph{datatype='Convolved Image'}}{}
set displayed data to be convolved data

\end{fulllineitems}

\index{updatePlotTitle() (LaueTools.GUI.PeakSearchGUI.MainPeakSearchFrame method)}

\begin{fulllineitems}
\phantomsection\label{\detokenize{PeakSearchGUI:LaueTools.GUI.PeakSearchGUI.MainPeakSearchFrame.updatePlotTitle}}\pysiglinewithargsret{\sphinxbfcode{\sphinxupquote{updatePlotTitle}}}{\emph{datatype=None}}{}
update plot title

\end{fulllineitems}

\index{normalizeplot() (LaueTools.GUI.PeakSearchGUI.MainPeakSearchFrame method)}

\begin{fulllineitems}
\phantomsection\label{\detokenize{PeakSearchGUI:LaueTools.GUI.PeakSearchGUI.MainPeakSearchFrame.normalizeplot}}\pysiglinewithargsret{\sphinxbfcode{\sphinxupquote{normalizeplot}}}{}{}
normalize current displayed array according to vmin vmax sliders

\end{fulllineitems}

\index{update\_draw() (LaueTools.GUI.PeakSearchGUI.MainPeakSearchFrame method)}

\begin{fulllineitems}
\phantomsection\label{\detokenize{PeakSearchGUI:LaueTools.GUI.PeakSearchGUI.MainPeakSearchFrame.update_draw}}\pysiglinewithargsret{\sphinxbfcode{\sphinxupquote{update\_draw}}}{\emph{\_}}{}
update 2D plot taken into account change of LUT table and vmin vamax values

\end{fulllineitems}

\index{getDisplayedImageSize() (LaueTools.GUI.PeakSearchGUI.MainPeakSearchFrame method)}

\begin{fulllineitems}
\phantomsection\label{\detokenize{PeakSearchGUI:LaueTools.GUI.PeakSearchGUI.MainPeakSearchFrame.getDisplayedImageSize}}\pysiglinewithargsret{\sphinxbfcode{\sphinxupquote{getDisplayedImageSize}}}{}{}
get xmin, xmax, ymin, ymax from current displayed image

\end{fulllineitems}

\index{set\_circleradius() (LaueTools.GUI.PeakSearchGUI.MainPeakSearchFrame method)}

\begin{fulllineitems}
\phantomsection\label{\detokenize{PeakSearchGUI:LaueTools.GUI.PeakSearchGUI.MainPeakSearchFrame.set_circleradius}}\pysiglinewithargsret{\sphinxbfcode{\sphinxupquote{set\_circleradius}}}{\emph{self.viewingLUTpanel.drg.artists}}{}
\end{fulllineitems}

\index{activateCrop() (LaueTools.GUI.PeakSearchGUI.MainPeakSearchFrame method)}

\begin{fulllineitems}
\phantomsection\label{\detokenize{PeakSearchGUI:LaueTools.GUI.PeakSearchGUI.MainPeakSearchFrame.activateCrop}}\pysiglinewithargsret{\sphinxbfcode{\sphinxupquote{activateCrop}}}{\emph{\_}}{}
set boxx and boxy from ctrls

\end{fulllineitems}

\index{readdata\_updateplot\_aftercrop\_uncrop() (LaueTools.GUI.PeakSearchGUI.MainPeakSearchFrame method)}

\begin{fulllineitems}
\phantomsection\label{\detokenize{PeakSearchGUI:LaueTools.GUI.PeakSearchGUI.MainPeakSearchFrame.readdata_updateplot_aftercrop_uncrop}}\pysiglinewithargsret{\sphinxbfcode{\sphinxupquote{readdata\_updateplot\_aftercrop\_uncrop}}}{}{}
read data and update data to be displayed and redraw

\end{fulllineitems}

\index{reinit\_aftercrop\_draw() (LaueTools.GUI.PeakSearchGUI.MainPeakSearchFrame method)}

\begin{fulllineitems}
\phantomsection\label{\detokenize{PeakSearchGUI:LaueTools.GUI.PeakSearchGUI.MainPeakSearchFrame.reinit_aftercrop_draw}}\pysiglinewithargsret{\sphinxbfcode{\sphinxupquote{reinit\_aftercrop\_draw}}}{}{}
reinit the figure

\end{fulllineitems}

\index{buildMosaic() (LaueTools.GUI.PeakSearchGUI.MainPeakSearchFrame method)}

\begin{fulllineitems}
\phantomsection\label{\detokenize{PeakSearchGUI:LaueTools.GUI.PeakSearchGUI.MainPeakSearchFrame.buildMosaic}}\pysiglinewithargsret{\sphinxbfcode{\sphinxupquote{buildMosaic}}}{\emph{parent=None}}{}
launch MOS.buildMosaic3() with GUI inputs as arguments

\end{fulllineitems}

\index{onMotion\_ToolTip() (LaueTools.GUI.PeakSearchGUI.MainPeakSearchFrame method)}

\begin{fulllineitems}
\phantomsection\label{\detokenize{PeakSearchGUI:LaueTools.GUI.PeakSearchGUI.MainPeakSearchFrame.onMotion_ToolTip}}\pysiglinewithargsret{\sphinxbfcode{\sphinxupquote{onMotion\_ToolTip}}}{\emph{event}}{}
tool tip to show data when mouse hovers on plot
Some pixels at the image border could not be detected

\end{fulllineitems}

\index{OnSpinCtrl\_ImaxDisplayed() (LaueTools.GUI.PeakSearchGUI.MainPeakSearchFrame method)}

\begin{fulllineitems}
\phantomsection\label{\detokenize{PeakSearchGUI:LaueTools.GUI.PeakSearchGUI.MainPeakSearchFrame.OnSpinCtrl_ImaxDisplayed}}\pysiglinewithargsret{\sphinxbfcode{\sphinxupquote{OnSpinCtrl\_ImaxDisplayed}}}{\emph{event}}{}
on change Imax by spin control

\end{fulllineitems}

\index{Get\_XYI\_from\_fit2dpeaksfile() (LaueTools.GUI.PeakSearchGUI.MainPeakSearchFrame method)}

\begin{fulllineitems}
\phantomsection\label{\detokenize{PeakSearchGUI:LaueTools.GUI.PeakSearchGUI.MainPeakSearchFrame.Get_XYI_from_fit2dpeaksfile}}\pysiglinewithargsret{\sphinxbfcode{\sphinxupquote{Get\_XYI\_from\_fit2dpeaksfile}}}{\emph{filename}}{}
useless ?

\end{fulllineitems}

\index{draw\_cursor() (LaueTools.GUI.PeakSearchGUI.MainPeakSearchFrame method)}

\begin{fulllineitems}
\phantomsection\label{\detokenize{PeakSearchGUI:LaueTools.GUI.PeakSearchGUI.MainPeakSearchFrame.draw_cursor}}\pysiglinewithargsret{\sphinxbfcode{\sphinxupquote{draw\_cursor}}}{\emph{event}}{}
event is a MplEvent.  Draw a cursor over the axes

\end{fulllineitems}

\index{getConvolvedData() (LaueTools.GUI.PeakSearchGUI.MainPeakSearchFrame method)}

\begin{fulllineitems}
\phantomsection\label{\detokenize{PeakSearchGUI:LaueTools.GUI.PeakSearchGUI.MainPeakSearchFrame.getConvolvedData}}\pysiglinewithargsret{\sphinxbfcode{\sphinxupquote{getConvolvedData}}}{}{}
convolve data according to check value of

\end{fulllineitems}

\index{SavePeakList\_PSPfile() (LaueTools.GUI.PeakSearchGUI.MainPeakSearchFrame method)}

\begin{fulllineitems}
\phantomsection\label{\detokenize{PeakSearchGUI:LaueTools.GUI.PeakSearchGUI.MainPeakSearchFrame.SavePeakList_PSPfile}}\pysiglinewithargsret{\sphinxbfcode{\sphinxupquote{SavePeakList\_PSPfile}}}{\emph{\_}}{}
save peak list and save .psp file

\end{fulllineitems}

\index{onSaveROIsList() (LaueTools.GUI.PeakSearchGUI.MainPeakSearchFrame method)}

\begin{fulllineitems}
\phantomsection\label{\detokenize{PeakSearchGUI:LaueTools.GUI.PeakSearchGUI.MainPeakSearchFrame.onSaveROIsList}}\pysiglinewithargsret{\sphinxbfcode{\sphinxupquote{onSaveROIsList}}}{\emph{\_}}{}
save rois list from current peaks list with boxsize of fitting procedure

\end{fulllineitems}

\index{onFitOnePeak() (LaueTools.GUI.PeakSearchGUI.MainPeakSearchFrame method)}

\begin{fulllineitems}
\phantomsection\label{\detokenize{PeakSearchGUI:LaueTools.GUI.PeakSearchGUI.MainPeakSearchFrame.onFitOnePeak}}\pysiglinewithargsret{\sphinxbfcode{\sphinxupquote{onFitOnePeak}}}{\emph{\_}}{}
fit one peak centered on where user has clicked

in displayed image coordinates: self.centerx, self.centery

\end{fulllineitems}

\index{OnFit() (LaueTools.GUI.PeakSearchGUI.MainPeakSearchFrame method)}

\begin{fulllineitems}
\phantomsection\label{\detokenize{PeakSearchGUI:LaueTools.GUI.PeakSearchGUI.MainPeakSearchFrame.OnFit}}\pysiglinewithargsret{\sphinxbfcode{\sphinxupquote{OnFit}}}{}{}
fit image array in a ROI with a 2D gaussian shape

\end{fulllineitems}

\index{onRemovePeaktoPeaklist() (LaueTools.GUI.PeakSearchGUI.MainPeakSearchFrame method)}

\begin{fulllineitems}
\phantomsection\label{\detokenize{PeakSearchGUI:LaueTools.GUI.PeakSearchGUI.MainPeakSearchFrame.onRemovePeaktoPeaklist}}\pysiglinewithargsret{\sphinxbfcode{\sphinxupquote{onRemovePeaktoPeaklist}}}{\emph{\_}, \emph{centerXY=None}}{}
remove picked peak from the current peaks list

\end{fulllineitems}

\index{onRemoveAllPeakstoPeaklist() (LaueTools.GUI.PeakSearchGUI.MainPeakSearchFrame method)}

\begin{fulllineitems}
\phantomsection\label{\detokenize{PeakSearchGUI:LaueTools.GUI.PeakSearchGUI.MainPeakSearchFrame.onRemoveAllPeakstoPeaklist}}\pysiglinewithargsret{\sphinxbfcode{\sphinxupquote{onRemoveAllPeakstoPeaklist}}}{\emph{\_}}{}
remove all spots of the peaks list and update the plot (remove circular markers)

\end{fulllineitems}

\index{getClosestPeak() (LaueTools.GUI.PeakSearchGUI.MainPeakSearchFrame method)}

\begin{fulllineitems}
\phantomsection\label{\detokenize{PeakSearchGUI:LaueTools.GUI.PeakSearchGUI.MainPeakSearchFrame.getClosestPeak}}\pysiglinewithargsret{\sphinxbfcode{\sphinxupquote{getClosestPeak}}}{\emph{centerXY=None}}{}
return peak in self.peaklistPixels
that is close to the clicked pixel position or the given value

\end{fulllineitems}

\index{deleteOnePeak() (LaueTools.GUI.PeakSearchGUI.MainPeakSearchFrame method)}

\begin{fulllineitems}
\phantomsection\label{\detokenize{PeakSearchGUI:LaueTools.GUI.PeakSearchGUI.MainPeakSearchFrame.deleteOnePeak}}\pysiglinewithargsret{\sphinxbfcode{\sphinxupquote{deleteOnePeak}}}{\emph{index\_close}, \emph{peakXY}}{}
delete one peak and update display and lists

\end{fulllineitems}

\index{deleteAllPeaks() (LaueTools.GUI.PeakSearchGUI.MainPeakSearchFrame method)}

\begin{fulllineitems}
\phantomsection\label{\detokenize{PeakSearchGUI:LaueTools.GUI.PeakSearchGUI.MainPeakSearchFrame.deleteAllPeaks}}\pysiglinewithargsret{\sphinxbfcode{\sphinxupquote{deleteAllPeaks}}}{}{}
delete all peaks and update display and lists

\end{fulllineitems}

\index{OnPeakSearch() (LaueTools.GUI.PeakSearchGUI.MainPeakSearchFrame method)}

\begin{fulllineitems}
\phantomsection\label{\detokenize{PeakSearchGUI:LaueTools.GUI.PeakSearchGUI.MainPeakSearchFrame.OnPeakSearch}}\pysiglinewithargsret{\sphinxbfcode{\sphinxupquote{OnPeakSearch}}}{\emph{\_}}{}
launch peak search by calling methods in readmccd.py

\end{fulllineitems}


\end{fulllineitems}



\subsection{Peak Search and Fit (readmccd.py)}
\label{\detokenize{PeakSearch::doc}}\label{\detokenize{PeakSearch:peak-search-and-fit-readmccd-py}}

\subsubsection{Module functions}
\label{\detokenize{PeakSearch:module-functions}}
The next documentation comes from the docstring in the header of function or class definition.


\paragraph{readmccd.py}
\label{\detokenize{PeakSearch:readmccd-py}}\label{\detokenize{PeakSearch:module-LaueTools.readmccd}}\index{LaueTools.readmccd (module)}
readmccd module is made for reading data contained in binary image file
fully or partially.
It can process a peak or blob search by various methods
and refine the peak by a gaussian or lorentzian 2D model

More tools can be found in LaueTools package at sourceforge.net and gitlab.esrf.fr
\index{setfilename() (in module LaueTools.readmccd)}

\begin{fulllineitems}
\phantomsection\label{\detokenize{PeakSearch:LaueTools.readmccd.setfilename}}\pysiglinewithargsret{\sphinxcode{\sphinxupquote{LaueTools.readmccd.}}\sphinxbfcode{\sphinxupquote{setfilename}}}{\emph{imagefilename}, \emph{imageindex}, \emph{nbdigits=4}, \emph{CCDLabel=None}}{}
reconstruct filename string from imagefilename and update filename index with imageindex
\begin{quote}\begin{description}
\item[{Parameters}] \leavevmode\begin{itemize}
\item {} 
\sphinxstyleliteralstrong{\sphinxupquote{imagefilename}} \textendash{} filename string (full path or not)

\item {} 
\sphinxstyleliteralstrong{\sphinxupquote{imageindex}} (\sphinxstyleliteralemphasis{\sphinxupquote{string}}) \textendash{} index in filename

\end{itemize}

\item[{Return filename}] \leavevmode
input filename with index replaced by input imageindex

\item[{Return type}] \leavevmode
string

\end{description}\end{quote}

\end{fulllineitems}

\index{getIndex\_fromfilename() (in module LaueTools.readmccd)}

\begin{fulllineitems}
\phantomsection\label{\detokenize{PeakSearch:LaueTools.readmccd.getIndex_fromfilename}}\pysiglinewithargsret{\sphinxcode{\sphinxupquote{LaueTools.readmccd.}}\sphinxbfcode{\sphinxupquote{getIndex\_fromfilename}}}{\emph{imagefilename}, \emph{nbdigits=4}, \emph{CCDLabel=None}, \emph{stackimageindex=-1}}{}
get integer index from imagefilename string
\begin{quote}\begin{description}
\item[{Parameters}] \leavevmode
\sphinxstyleliteralstrong{\sphinxupquote{imagefilename}} \textendash{} filename string (full path or not)

\item[{Returns}] \leavevmode
file index

\end{description}\end{quote}

\end{fulllineitems}

\index{readheader() (in module LaueTools.readmccd)}

\begin{fulllineitems}
\phantomsection\label{\detokenize{PeakSearch:LaueTools.readmccd.readheader}}\pysiglinewithargsret{\sphinxcode{\sphinxupquote{LaueTools.readmccd.}}\sphinxbfcode{\sphinxupquote{readheader}}}{\emph{filename}, \emph{offset=4096}, \emph{CCDLabel='MARCCD165'}}{}
return header in a raw format

default offset for marccd image

\end{fulllineitems}

\index{read\_header\_marccd() (in module LaueTools.readmccd)}

\begin{fulllineitems}
\phantomsection\label{\detokenize{PeakSearch:LaueTools.readmccd.read_header_marccd}}\pysiglinewithargsret{\sphinxcode{\sphinxupquote{LaueTools.readmccd.}}\sphinxbfcode{\sphinxupquote{read\_header\_marccd}}}{\emph{filename}}{}
return string of parameters found in header in marccd image file .mccd
\begin{itemize}
\item {} 
print allsentences  displays the header

\item {} 
use allsentences.split(‘n’) to get a list

\end{itemize}

\end{fulllineitems}

\index{read\_header\_marccd2() (in module LaueTools.readmccd)}

\begin{fulllineitems}
\phantomsection\label{\detokenize{PeakSearch:LaueTools.readmccd.read_header_marccd2}}\pysiglinewithargsret{\sphinxcode{\sphinxupquote{LaueTools.readmccd.}}\sphinxbfcode{\sphinxupquote{read\_header\_marccd2}}}{\emph{filename}}{}
return string of parameters comments and exposure time
found in header in marccd image file .mccd
\begin{itemize}
\item {} 
print allsentences  displays the header

\item {} 
use allsentences.split(‘n’) to get a list

\end{itemize}

\end{fulllineitems}

\index{read\_header\_scmos() (in module LaueTools.readmccd)}

\begin{fulllineitems}
\phantomsection\label{\detokenize{PeakSearch:LaueTools.readmccd.read_header_scmos}}\pysiglinewithargsret{\sphinxcode{\sphinxupquote{LaueTools.readmccd.}}\sphinxbfcode{\sphinxupquote{read\_header\_scmos}}}{\emph{filename}}{}
return string of parameters comments and exposure time
found in header in scmis image file .tif
\begin{itemize}
\item {} 
print allsentences  displays the header

\item {} 
use allsentences.split(‘n’) to get a list

\end{itemize}

\end{fulllineitems}

\index{readCCDimage() (in module LaueTools.readmccd)}

\begin{fulllineitems}
\phantomsection\label{\detokenize{PeakSearch:LaueTools.readmccd.readCCDimage}}\pysiglinewithargsret{\sphinxcode{\sphinxupquote{LaueTools.readmccd.}}\sphinxbfcode{\sphinxupquote{readCCDimage}}}{\emph{filename}, \emph{CCDLabel='MARCCD165'}, \emph{dirname=None}, \emph{stackimageindex=-1}, \emph{verbose=0}}{}
Read raw data binary image file.

Read raw data binary image file and return pixel intensity 2D array such as
to fit the data (2theta, chi) scattering angles representation convention.
\begin{quote}\begin{description}
\item[{Parameters}] \leavevmode\begin{itemize}
\item {} 
\sphinxstyleliteralstrong{\sphinxupquote{filename}} (\sphinxhref{https://docs.python.org/3/library/stdtypes.html\#str}{\sphinxstyleliteralemphasis{\sphinxupquote{str}}}) \textendash{} path to image file (fullpath if {}` dirname{}` =None)

\item {} 
\sphinxstyleliteralstrong{\sphinxupquote{CCDLabel}} (\sphinxhref{https://docs.python.org/3/library/stdtypes.html\#str}{\sphinxstyleliteralemphasis{\sphinxupquote{str}}}\sphinxstyleliteralemphasis{\sphinxupquote{, }}\sphinxstyleliteralemphasis{\sphinxupquote{optional}}) \textendash{} label, defaults to “MARCCD165”

\item {} 
\sphinxstyleliteralstrong{\sphinxupquote{dirname}} (\sphinxhref{https://docs.python.org/3/library/stdtypes.html\#str}{\sphinxstyleliteralemphasis{\sphinxupquote{str}}}\sphinxstyleliteralemphasis{\sphinxupquote{, }}\sphinxstyleliteralemphasis{\sphinxupquote{optional}}) \textendash{} folder path, defaults to None

\item {} 
\sphinxstyleliteralstrong{\sphinxupquote{stackimageindex}} (\sphinxhref{https://docs.python.org/3/library/functions.html\#int}{\sphinxstyleliteralemphasis{\sphinxupquote{int}}}\sphinxstyleliteralemphasis{\sphinxupquote{, }}\sphinxstyleliteralemphasis{\sphinxupquote{optional}}) \textendash{} index of images bunch, defaults to -1

\item {} 
\sphinxstyleliteralstrong{\sphinxupquote{verbose}} (\sphinxhref{https://docs.python.org/3/library/functions.html\#int}{\sphinxstyleliteralemphasis{\sphinxupquote{int}}}\sphinxstyleliteralemphasis{\sphinxupquote{, }}\sphinxstyleliteralemphasis{\sphinxupquote{optional}}) \textendash{} 0 or 1, defaults to 0

\end{itemize}

\item[{Raises}] \leavevmode
\sphinxhref{https://docs.python.org/3/library/exceptions.html\#ValueError}{\sphinxstyleliteralstrong{\sphinxupquote{ValueError}}} \textendash{} if data format and CCD parameters from label are not compatible

\item[{Returns}] \leavevmode
\begin{itemize}
\item {} 
dataimage, 2D array image data pixel intensity properly oriented

\item {} 
framedim, iterable of 2 integers shape of dataimage

\item {} 
fliprot : string, key for CCD frame transform to orient image

\end{itemize}


\item[{Return type}] \leavevmode
tuple of 3 elements

\end{description}\end{quote}

\end{fulllineitems}

\index{readoneimage() (in module LaueTools.readmccd)}

\begin{fulllineitems}
\phantomsection\label{\detokenize{PeakSearch:LaueTools.readmccd.readoneimage}}\pysiglinewithargsret{\sphinxcode{\sphinxupquote{LaueTools.readmccd.}}\sphinxbfcode{\sphinxupquote{readoneimage}}}{\emph{filename}, \emph{framedim=(2048}, \emph{2048)}, \emph{dirname=None}, \emph{offset=4096}, \emph{formatdata='uint16'}}{}
returns a 1d array of integers from a binary image file (full data)
\begin{quote}\begin{description}
\item[{Parameters}] \leavevmode\begin{itemize}
\item {} 
\sphinxstyleliteralstrong{\sphinxupquote{filename}} (\sphinxhref{https://docs.python.org/3/library/stdtypes.html\#str}{\sphinxstyleliteralemphasis{\sphinxupquote{str}}}) \textendash{} image file name (full path if dirname=0)

\item {} 
\sphinxstyleliteralstrong{\sphinxupquote{framedim}} (\sphinxstyleliteralemphasis{\sphinxupquote{tuple of 2 integers}}\sphinxstyleliteralemphasis{\sphinxupquote{, }}\sphinxstyleliteralemphasis{\sphinxupquote{optional}}) \textendash{} detector dimensions, defaults to (2048, 2048)

\item {} 
\sphinxstyleliteralstrong{\sphinxupquote{dirname}} (\sphinxhref{https://docs.python.org/3/library/stdtypes.html\#str}{\sphinxstyleliteralemphasis{\sphinxupquote{str}}}\sphinxstyleliteralemphasis{\sphinxupquote{, }}\sphinxstyleliteralemphasis{\sphinxupquote{optional}}) \textendash{} folder path, defaults to None

\item {} 
\sphinxstyleliteralstrong{\sphinxupquote{offset}} (\sphinxhref{https://docs.python.org/3/library/functions.html\#int}{\sphinxstyleliteralemphasis{\sphinxupquote{int}}}\sphinxstyleliteralemphasis{\sphinxupquote{, }}\sphinxstyleliteralemphasis{\sphinxupquote{optional}}) \textendash{} file header in byte (octet), defaults to 4096

\item {} 
\sphinxstyleliteralstrong{\sphinxupquote{formatdata}} (\sphinxhref{https://docs.python.org/3/library/stdtypes.html\#str}{\sphinxstyleliteralemphasis{\sphinxupquote{str}}}\sphinxstyleliteralemphasis{\sphinxupquote{, }}\sphinxstyleliteralemphasis{\sphinxupquote{optional}}) \textendash{} numpy format of raw binary image pixel value, defaults to “uint16”

\end{itemize}

\item[{Returns}] \leavevmode
dataimage : image data pixel intensity

\item[{Return type}] \leavevmode
1D array

\end{description}\end{quote}

\end{fulllineitems}

\index{readoneimage\_crop\_fast() (in module LaueTools.readmccd)}

\begin{fulllineitems}
\phantomsection\label{\detokenize{PeakSearch:LaueTools.readmccd.readoneimage_crop_fast}}\pysiglinewithargsret{\sphinxcode{\sphinxupquote{LaueTools.readmccd.}}\sphinxbfcode{\sphinxupquote{readoneimage\_crop\_fast}}}{\emph{filename}, \emph{dirname=None}, \emph{CCDLabel='MARCCD165'}, \emph{firstElemIndex=0}, \emph{lastElemIndex=2047}}{}
Returns a 2d array of integers from a binary image file. Data are taken only from a rectangle

with respect to firstElemIndex and lastElemIndex.
\begin{quote}\begin{description}
\item[{Parameters}] \leavevmode\begin{itemize}
\item {} 
\sphinxstyleliteralstrong{\sphinxupquote{filename}} \textendash{} string, path to image file (fullpath if {}` dirname{}`=None)

\item {} 
\sphinxstyleliteralstrong{\sphinxupquote{offset}} \textendash{} integer, nb of file header bytes

\item {} 
\sphinxstyleliteralstrong{\sphinxupquote{framedim}} \textendash{} iterable of 2 integers, shape of expected 2D data

\item {} 
\sphinxstyleliteralstrong{\sphinxupquote{formatdata}} \textendash{} string, key for numpy dtype to decode binary file

\end{itemize}

\item[{Returns}] \leavevmode
dataimage : 1D array image data pixel intensity

\end{description}\end{quote}

\end{fulllineitems}

\index{readrectangle\_in\_image() (in module LaueTools.readmccd)}

\begin{fulllineitems}
\phantomsection\label{\detokenize{PeakSearch:LaueTools.readmccd.readrectangle_in_image}}\pysiglinewithargsret{\sphinxcode{\sphinxupquote{LaueTools.readmccd.}}\sphinxbfcode{\sphinxupquote{readrectangle\_in\_image}}}{\emph{filename}, \emph{pixx}, \emph{pixy}, \emph{halfboxx}, \emph{halfboxy}, \emph{dirname=None}, \emph{CCDLabel='MARCCD165'}, \emph{verbose=True}}{}
returns a 2d array of integers from a binary image file. Data are taken only from a rectangle
centered on pixx, pixy
\begin{quote}\begin{description}
\item[{Returns}] \leavevmode
dataimage : 2D array, image data pixel intensity

\end{description}\end{quote}

\end{fulllineitems}

\index{readoneimage\_crop() (in module LaueTools.readmccd)}

\begin{fulllineitems}
\phantomsection\label{\detokenize{PeakSearch:LaueTools.readmccd.readoneimage_crop}}\pysiglinewithargsret{\sphinxcode{\sphinxupquote{LaueTools.readmccd.}}\sphinxbfcode{\sphinxupquote{readoneimage\_crop}}}{\emph{filename}, \emph{center}, \emph{halfboxsize}, \emph{CCDLabel='PRINCETON'}, \emph{dirname=None}}{}
return a cropped array of data read in an image file
\begin{quote}\begin{description}
\item[{Parameters}] \leavevmode\begin{itemize}
\item {} 
\sphinxstyleliteralstrong{\sphinxupquote{filename}} \textendash{} string, path to image file (fullpath if {}` dirname{}`=None)

\item {} 
\sphinxstyleliteralstrong{\sphinxupquote{center}} \textendash{} iterable of 2 integers, (x,y) pixel coordinates

\item {} 
\sphinxstyleliteralstrong{\sphinxupquote{halfboxsize}} \textendash{} integer or iterable of 2 integers, ROI half size in both directions

\end{itemize}

\item[{Returns}] \leavevmode
dataimage : 1D array, image data pixel intensity

\end{description}\end{quote}

\#TODO: useless?

\end{fulllineitems}

\index{readoneimage\_manycrops() (in module LaueTools.readmccd)}

\begin{fulllineitems}
\phantomsection\label{\detokenize{PeakSearch:LaueTools.readmccd.readoneimage_manycrops}}\pysiglinewithargsret{\sphinxcode{\sphinxupquote{LaueTools.readmccd.}}\sphinxbfcode{\sphinxupquote{readoneimage\_manycrops}}}{\emph{filename}, \emph{centers}, \emph{boxsize}, \emph{stackimageindex=-1}, \emph{CCDLabel='MARCCD165'}, \emph{addImax=False}, \emph{use\_data\_corrected=None}}{}
reads 1 image and extract many regions
centered on center\_pixel with xyboxsize dimensions in pixel unit

Parameters
filename : string,fullpath to image file
centers : list or array of {[}int,int{]} centers (x,y) pixel coordinates
use\_data\_corrected : enter data instead of reading data from file
\begin{quote}

must be a tuple of 3 elements:
fulldata, framedim, fliprot
where fulldata is a numpy.ndarray
as output by {\hyperref[\detokenize{PeakSearch:LaueTools.readmccd.readCCDimage}]{\sphinxcrossref{\sphinxcode{\sphinxupquote{readCCDimage()}}}}}
\end{quote}
\begin{description}
\item[{boxsize}] \leavevmode{[}iterable 2 elements or integer{]}
boxsizes {[}in x, in y{]} direction or integer to set a square ROI

\end{description}

Returns
Data : list of 2D array pixel intensity

Imax :
returns:
array of data: list of 2D array of intensity

\end{fulllineitems}

\index{readoneimage\_multiROIfit() (in module LaueTools.readmccd)}

\begin{fulllineitems}
\phantomsection\label{\detokenize{PeakSearch:LaueTools.readmccd.readoneimage_multiROIfit}}\pysiglinewithargsret{\sphinxcode{\sphinxupquote{LaueTools.readmccd.}}\sphinxbfcode{\sphinxupquote{readoneimage\_multiROIfit}}}{\emph{filename}, \emph{centers}, \emph{boxsize}, \emph{stackimageindex=-1}, \emph{CCDLabel='PRINCETON'}, \emph{baseline='auto'}, \emph{startangles=0.0}, \emph{start\_sigma1=1.0}, \emph{start\_sigma2=1.0}, \emph{position\_start='max'}, \emph{fitfunc='gaussian'}, \emph{showfitresults=1}, \emph{offsetposition=0}, \emph{verbose=0}, \emph{xtol=1e-08}, \emph{addImax=False}, \emph{use\_data\_corrected=None}}{}
Fit several peaks in one image

Parameters
filename : string, full path to image file
centers : list or array like with shape=(n,2)
\begin{quote}

list of centers of selected ROI
\end{quote}
\begin{description}
\item[{boxsize}] \leavevmode{[}(Truly HALF boxsize: fuill boxsize= 2 {\color{red}\bfseries{}*}halfboxsize +1){]}
iterable 2 elements or integer
boxsizes {[}in x, in y{]} direction or integer to set a square ROI

\end{description}

Optional parameters
baseline : string
\begin{quote}

‘auto’ (ie minimum intensity in ROI) or array of floats
\end{quote}
\begin{description}
\item[{startangles}] \leavevmode{[}float or iterable of 2 floats{]}
elliptic gaussian angle (major axis with respect to X direction),
one value or array of values

\item[{start\_sigma1, start\_sigma2: floats}] \leavevmode
gaussian standard deviation (major and minor axis) in pixel,

\item[{position\_start}] \leavevmode{[}string{]}
starting gaussian center:’max’ (position of maximum intensity in ROI),
‘centers’ (centre of each ROI)

\item[{offsetposition}] \leavevmode{[}integer{]}
0 for no offset
1  XMAS compatible, since XMAS consider first pixel as index 1 (in array, index starts with 0)
2  fit2d, since fit2d for peaksearch put pixel labelled n at the position n+0.5 (between n and n+1)

\item[{use\_data\_corrected}] \leavevmode{[}tuple of 3 elements{]}
Enter data instead of reading data from file:
fulldata, framedim, fliprot
where fulldata is a ndarray

\end{description}

returns
params\_sol : list of results
\begin{quote}
\begin{quote}

bkg,  amp  (gaussian height-bkg), X , Y ,
\end{quote}

major axis standard deviation ,minor axis standard deviation,
major axis tilt angle / Ox
\end{quote}

\# TODO: setting list of initial guesses can be improve with
scipy.ndimages of a concatenate array of multiple slices?

\end{fulllineitems}

\index{getindices2cropArray() (in module LaueTools.readmccd)}

\begin{fulllineitems}
\phantomsection\label{\detokenize{PeakSearch:LaueTools.readmccd.getindices2cropArray}}\pysiglinewithargsret{\sphinxcode{\sphinxupquote{LaueTools.readmccd.}}\sphinxbfcode{\sphinxupquote{getindices2cropArray}}}{\emph{center}, \emph{halfboxsizeROI}, \emph{arrayshape}, \emph{flipxycenter=False}}{}
return array indices limits to crop array data

Parameters
center : iterable of 2 elements
\begin{quote}

(x,y) pixel center of the ROI
\end{quote}
\begin{description}
\item[{halfboxsizeROI}] \leavevmode{[}integer or iterable of 2 elements{]}
half boxsize ROI in two dimensions

\item[{arrayshape}] \leavevmode{[}iterable of 2 integers{]}
maximal number of pixels in both directions

\end{description}

Options
flipxycenter : boolean
\begin{quote}

True: swap x and y of center with respect to others
parameters that remain fixed
\end{quote}

Return
imin, imax, jmin, jmax : 4 integers
\begin{quote}

4 indices allowing to slice a 2D np.ndarray
\end{quote}

\begin{sphinxadmonition}{note}{\label{PeakSearch:index-0}Todo:}
merge with check\_array\_indices()
\end{sphinxadmonition}

\end{fulllineitems}

\index{check\_array\_indices() (in module LaueTools.readmccd)}

\begin{fulllineitems}
\phantomsection\label{\detokenize{PeakSearch:LaueTools.readmccd.check_array_indices}}\pysiglinewithargsret{\sphinxcode{\sphinxupquote{LaueTools.readmccd.}}\sphinxbfcode{\sphinxupquote{check\_array\_indices}}}{\emph{imin}, \emph{imax}, \emph{jmin}, \emph{jmax}, \emph{framedim=None}}{}
Return 4 indices for array slice compatible with framedim

Parameters
imin, imax, jmin, jmax: 4 integers
\begin{quote}

mini. and maxi. indices in both directions
\end{quote}
\begin{description}
\item[{framedim}] \leavevmode{[}iterable of 2 integers{]}
shape of the array to be sliced by means of the 4 indices

\end{description}

Return
imin, imax, jmin, jmax: 4 integers
\begin{quote}

mini. and maxi. indices in both directions
\end{quote}

\begin{sphinxadmonition}{note}{\label{PeakSearch:index-1}Todo:}
merge with getindices2cropArray()
\end{sphinxadmonition}

\end{fulllineitems}

\index{to8bits() (in module LaueTools.readmccd)}

\begin{fulllineitems}
\phantomsection\label{\detokenize{PeakSearch:LaueTools.readmccd.to8bits}}\pysiglinewithargsret{\sphinxcode{\sphinxupquote{LaueTools.readmccd.}}\sphinxbfcode{\sphinxupquote{to8bits}}}{\emph{PILimage}, \emph{normalization\_value=None}}{}
convert PIL image (16 bits) in 8 bits PIL image
returns:
{[}0{]}  8 bits image
{[}1{]} corresponding pixels value array

TODO: since not used, may be deleted

\end{fulllineitems}

\index{writeimage() (in module LaueTools.readmccd)}

\begin{fulllineitems}
\phantomsection\label{\detokenize{PeakSearch:LaueTools.readmccd.writeimage}}\pysiglinewithargsret{\sphinxcode{\sphinxupquote{LaueTools.readmccd.}}\sphinxbfcode{\sphinxupquote{writeimage}}}{\emph{outputname}, \emph{\_header}, \emph{data}, \emph{dataformat=\textless{}class 'numpy.uint16'\textgreater{}}}{}
from data 1d array of integers
with header coming from a f.open(‘imagefile’); f.read(headersize);f.close()
WARNING: header contain dimensions for subsequent data. Check before the compatibility of
data with header infos(nb of byte per pixel and array dimensions

\end{fulllineitems}

\index{write\_rawbinary() (in module LaueTools.readmccd)}

\begin{fulllineitems}
\phantomsection\label{\detokenize{PeakSearch:LaueTools.readmccd.write_rawbinary}}\pysiglinewithargsret{\sphinxcode{\sphinxupquote{LaueTools.readmccd.}}\sphinxbfcode{\sphinxupquote{write\_rawbinary}}}{\emph{outputname}, \emph{data}, \emph{dataformat=\textless{}class 'numpy.uint16'\textgreater{}}}{}
write a binary file without header of a 2D array

\end{fulllineitems}

\index{SumImages() (in module LaueTools.readmccd)}

\begin{fulllineitems}
\phantomsection\label{\detokenize{PeakSearch:LaueTools.readmccd.SumImages}}\pysiglinewithargsret{\sphinxcode{\sphinxupquote{LaueTools.readmccd.}}\sphinxbfcode{\sphinxupquote{SumImages}}}{\emph{prefixname}, \emph{suffixname}, \emph{ind\_start}, \emph{ind\_end}, \emph{dirname=None}, \emph{plot=0}, \emph{output\_filename=None}, \emph{CCDLabel=None}, \emph{nbdigits=0}}{}
sum images and write image with 32 bits per pixel format (4 bytes)

\end{fulllineitems}

\index{Add\_Images() (in module LaueTools.readmccd)}

\begin{fulllineitems}
\phantomsection\label{\detokenize{PeakSearch:LaueTools.readmccd.Add_Images}}\pysiglinewithargsret{\sphinxcode{\sphinxupquote{LaueTools.readmccd.}}\sphinxbfcode{\sphinxupquote{Add\_Images}}}{\emph{prefixname}, \emph{ind\_start}, \emph{ind\_end}, \emph{plot=0}, \emph{writefilename=None}}{}
Add continuous sequence of images

\begin{sphinxadmonition}{note}{Note:}
Add\_Images2 exists
\end{sphinxadmonition}

Parameters
prefixname : string
\begin{quote}

prefix common part of name of files
\end{quote}
\begin{description}
\item[{ind\_start}] \leavevmode{[}int{]}
starting image index

\item[{ind\_end}] \leavevmode{[}int{]}
final image index

\end{description}

Optional Parameters
writefilename: string
\begin{quote}

new image filename where to write datastart (with last image file header read)
\end{quote}

Returns
datastart : array
\begin{quote}

accumulation of 2D data from each image
\end{quote}

\end{fulllineitems}

\index{diff\_pix() (in module LaueTools.readmccd)}

\begin{fulllineitems}
\phantomsection\label{\detokenize{PeakSearch:LaueTools.readmccd.diff_pix}}\pysiglinewithargsret{\sphinxcode{\sphinxupquote{LaueTools.readmccd.}}\sphinxbfcode{\sphinxupquote{diff\_pix}}}{\emph{pix}, \emph{array\_pix}, \emph{radius=1}}{}
returns
index in array\_pix which is the closest to pix if below the tolerance radius

array\_pix: array of 2d pixel points
pix: one 2elements pixel point

\end{fulllineitems}

\index{getExtrema() (in module LaueTools.readmccd)}

\begin{fulllineitems}
\phantomsection\label{\detokenize{PeakSearch:LaueTools.readmccd.getExtrema}}\pysiglinewithargsret{\sphinxcode{\sphinxupquote{LaueTools.readmccd.}}\sphinxbfcode{\sphinxupquote{getExtrema}}}{\emph{data2d}, \emph{center}, \emph{boxsize}, \emph{framedim}, \emph{ROIcoords=0}, \emph{flipxycenter=True}}{}
return  min max XYposmin, XYposmax values in ROI

Parameters
ROIcoords : 1 in local array indices coordinates
\begin{quote}

0 in X,Y pixel CCD coordinates
\end{quote}
\begin{description}
\item[{flipxycenter}] \leavevmode{[}boolean like{]}
swap input center coordinates

\item[{data2d}] \leavevmode{[}2D array{]}
data array as read by {\hyperref[\detokenize{PeakSearch:LaueTools.readmccd.readCCDimage}]{\sphinxcrossref{\sphinxcode{\sphinxupquote{readCCDimage()}}}}}

\end{description}

Return
min, max, XYposmin, XYposmax:
\begin{itemize}
\item {} 
min : minimum pixel intensity

\item {} 
max : maximum pixel intensity

\item {} 
XYposmin : list of absolute pixel coordinates of lowest pixel

\item {} 
XYposmax : list of absolute pixel coordinates of largest pixel

\end{itemize}

\end{fulllineitems}

\index{getIntegratedIntensities() (in module LaueTools.readmccd)}

\begin{fulllineitems}
\phantomsection\label{\detokenize{PeakSearch:LaueTools.readmccd.getIntegratedIntensities}}\pysiglinewithargsret{\sphinxcode{\sphinxupquote{LaueTools.readmccd.}}\sphinxbfcode{\sphinxupquote{getIntegratedIntensities}}}{\emph{fullpathimagefile}, \emph{list\_centers}, \emph{boxsize}, \emph{CCDLabel='MARCCD165'}, \emph{thresholdlevel=0.2}, \emph{flipxycenter=True}}{}
read binary image file and compute integrated intensities of peaks
whose center is given in list\_centers

return
array of
column 0: integrated intensity
column 1: absolute minimum intensity threshold
column 2: nb of pixels composing the peak

\end{fulllineitems}

\index{getIntegratedIntensity() (in module LaueTools.readmccd)}

\begin{fulllineitems}
\phantomsection\label{\detokenize{PeakSearch:LaueTools.readmccd.getIntegratedIntensity}}\pysiglinewithargsret{\sphinxcode{\sphinxupquote{LaueTools.readmccd.}}\sphinxbfcode{\sphinxupquote{getIntegratedIntensity}}}{\emph{data2d}, \emph{center}, \emph{boxsize}, \emph{framedim}, \emph{thresholdlevel=0.2}, \emph{flipxycenter=True}}{}
return  crude estimate of integrated intensity of peak above a given relative threshold

Parameters
ROIcoords : 1 in local array indices coordinates
\begin{quote}

0 in X,Y pixel CCD coordinates
\end{quote}
\begin{description}
\item[{flipxycenter}] \leavevmode{[}boolean like{]}
swap input center coordinates

\item[{data2d}] \leavevmode{[}2D array{]}
data array as read by {\hyperref[\detokenize{PeakSearch:LaueTools.readmccd.readCCDimage}]{\sphinxcrossref{\sphinxcode{\sphinxupquote{readCCDimage()}}}}}

\item[{Thresholdlevel}] \leavevmode{[}relative level above which pixel intensity must be taken into account{]}
I(p)- minimum\textgreater{} Thresholdlevel* (maximum-minimum)

\end{description}

Return
integrated intensity, minimum absolute intensity, nbpixels used for the summation

\end{fulllineitems}

\index{getMinMax() (in module LaueTools.readmccd)}

\begin{fulllineitems}
\phantomsection\label{\detokenize{PeakSearch:LaueTools.readmccd.getMinMax}}\pysiglinewithargsret{\sphinxcode{\sphinxupquote{LaueTools.readmccd.}}\sphinxbfcode{\sphinxupquote{getMinMax}}}{\emph{data2d}, \emph{center}, \emph{boxsize}, \emph{framedim}}{}
return min and max values in ROI

Parameters:
\begin{description}
\item[{data2d}] \leavevmode{[}2D array{]}
array as read by readCCDimage

\end{description}

\end{fulllineitems}

\index{LoGArr() (in module LaueTools.readmccd)}

\begin{fulllineitems}
\phantomsection\label{\detokenize{PeakSearch:LaueTools.readmccd.LoGArr}}\pysiglinewithargsret{\sphinxcode{\sphinxupquote{LaueTools.readmccd.}}\sphinxbfcode{\sphinxupquote{LoGArr}}}{\emph{shape=(256}, \emph{256)}, \emph{r0=None}, \emph{sigma=None}, \emph{peakVal=None}, \emph{orig=None}, \emph{wrap=0}, \emph{dtype=\textless{}class 'numpy.float32'\textgreater{}}}{}
returns n-dim Laplacian-of-Gaussian (aka. mexican hat)
if peakVal   is not None
\begin{quote}

result max is peakVal
\end{quote}

if r0 is not None: specify radius of zero-point (IGNORE sigma !!)

credits: “Sebastian Haase \textless{}\sphinxhref{mailto:haase@msg.ucsf.edu}{haase@msg.ucsf.edu}\textgreater{}”

\end{fulllineitems}

\index{ConvolvebyKernel() (in module LaueTools.readmccd)}

\begin{fulllineitems}
\phantomsection\label{\detokenize{PeakSearch:LaueTools.readmccd.ConvolvebyKernel}}\pysiglinewithargsret{\sphinxcode{\sphinxupquote{LaueTools.readmccd.}}\sphinxbfcode{\sphinxupquote{ConvolvebyKernel}}}{\emph{Data}, \emph{peakVal=4}, \emph{boxsize=5}, \emph{central\_radius=2}}{}
Convolve Data array witn mexican-hat kernel

inputs:
Data                            : 2D array containing pixel intensities
peakVal \textgreater{} central\_radius        : defines pixel distance from box center where weights are positive
\begin{quote}

(in the middle) and negative farther to converge back to zero
\end{quote}

boxsize                            : size of the box

ouput:
array  (same shape as Data)

\end{fulllineitems}

\index{LocalMaxima\_KernelConvolution() (in module LaueTools.readmccd)}

\begin{fulllineitems}
\phantomsection\label{\detokenize{PeakSearch:LaueTools.readmccd.LocalMaxima_KernelConvolution}}\pysiglinewithargsret{\sphinxcode{\sphinxupquote{LaueTools.readmccd.}}\sphinxbfcode{\sphinxupquote{LocalMaxima\_KernelConvolution}}}{\emph{Data}, \emph{framedim=(2048}, \emph{2048)}, \emph{peakValConvolve=4}, \emph{boxsizeConvolve=5}, \emph{central\_radiusConvolve=2}, \emph{thresholdConvolve=1000}, \emph{connectivity=1}, \emph{IntensityThreshold=500}, \emph{boxsize\_for\_probing\_minimal\_value\_background=30}, \emph{return\_nb\_raw\_blobs=0}, \emph{peakposition\_definition='max'}}{}
return local maxima (blobs) position and amplitude in Data by using
convolution with a mexican hat like kernel.
\begin{description}
\item[{Two Thresholds are used sequently:}] \leavevmode\begin{itemize}
\item {} 
thresholdConvolve : level under which intensity of kernel-convolved array is discarded

\item {} 
IntensityThreshold : level under which blob whose local intensity amplitude in raw array is discarded

\end{itemize}

\end{description}

Parameters

Data : 2D array containing pixel intensities

peakValConvolve, boxsizeConvolve, central\_radiusConvolve : convolution kernel parameters
\begin{description}
\item[{thresholdConvolve}] \leavevmode{[}minimum threshold (expressed in unit of convolved array intensity){]}
under which convoluted blob is rejected.It can be zero
(all blobs are accepted but time consuming)

\item[{connectivity}] \leavevmode{[}shape of connectivity pattern to consider pixels belonging to the{]}\begin{description}
\item[{same blob.}] \leavevmode\begin{itemize}
\item {} 
1: filled square  (1 pixel connected to 8 neighbours)

\item {} 
0: star (4 neighbours in vertical and horizontal direction)

\end{itemize}

\end{description}

\end{description}

IntensityThreshold : minimum local blob amplitude to accept
\begin{description}
\item[{boxsize\_for\_probing\_minimal\_value\_background}] \leavevmode{[}boxsize to evaluate the background{]}
and the blob amplitude

\item[{peakposition\_definition}] \leavevmode{[}string (‘max’ or ‘center’){]}
key to assign to the blob position its hottest pixel position
or its center (no weight)

\end{description}

Returns:
\textendash{}
\begin{description}
\item[{peakslist}] \leavevmode{[}array like (n,2){]}
list of peaks position (pixel)

\item[{Ipixmax}] \leavevmode{[}array like (n,1) of integer{]}
list of highest pixel intensity in the vicinity of each peak

\item[{npeaks}] \leavevmode{[}integer{]}
nb of peaks (if return\_nb\_raw\_blobs =1)

\end{description}

\end{fulllineitems}

\index{LocalMaxima\_from\_thresholdarray() (in module LaueTools.readmccd)}

\begin{fulllineitems}
\phantomsection\label{\detokenize{PeakSearch:LaueTools.readmccd.LocalMaxima_from_thresholdarray}}\pysiglinewithargsret{\sphinxcode{\sphinxupquote{LaueTools.readmccd.}}\sphinxbfcode{\sphinxupquote{LocalMaxima\_from\_thresholdarray}}}{\emph{Data}, \emph{IntensityThreshold=400}, \emph{rois=None}, \emph{framedim=None}, \emph{verbose=False}}{}
return center of mass of each blobs composes by pixels above IntensityThreshold

if Centers = list of (x,y, halfboxsizex, halfboxsizey)  perform only blob search in theses ROIs

!warning!: center of mass of blob where all intensities are set to 1

\end{fulllineitems}

\index{localmaxima() (in module LaueTools.readmccd)}

\begin{fulllineitems}
\phantomsection\label{\detokenize{PeakSearch:LaueTools.readmccd.localmaxima}}\pysiglinewithargsret{\sphinxcode{\sphinxupquote{LaueTools.readmccd.}}\sphinxbfcode{\sphinxupquote{localmaxima}}}{\emph{DataArray}, \emph{n}, \emph{diags=1}, \emph{verbose=0}}{}
from DataArray 2D  returns (array({[}i1,i2,…,ip{]}),array({[}j1,j2,…,jp{]}))
of indices where pixels value is higher in two direction up to n pixels

this tuple can be easily used after in the following manner:
DataArray{[}tupleresult{]} is an array of the intensity of the hottest pixels in array

in similar way with only four cardinal directions neighbouring (found in the web):
import numpy as N
def local\_minima(array2d):
\begin{quote}
\begin{description}
\item[{return ((array2d \textless{}= np.roll(array2d,  1, 0)) \&}] \leavevmode
(array2d \textless{}= np.roll(array2d, -1, 0)) \&
(array2d \textless{}= np.roll(array2d,  1, 1)) \&
(array2d \textless{}= np.roll(array2d, -1, 1)))

\end{description}
\end{quote}

WARNING: flat top peak are not detected !!

\end{fulllineitems}

\index{writepeaklist() (in module LaueTools.readmccd)}

\begin{fulllineitems}
\phantomsection\label{\detokenize{PeakSearch:LaueTools.readmccd.writepeaklist}}\pysiglinewithargsret{\sphinxcode{\sphinxupquote{LaueTools.readmccd.}}\sphinxbfcode{\sphinxupquote{writepeaklist}}}{\emph{tabpeaks}, \emph{output\_filename}, \emph{outputfolder=None}, \emph{comments=None}, \emph{initialfilename=None}}{}
write peaks properties and comments in file with extension .dat added

\end{fulllineitems}

\index{fitoneimage\_manypeaks() (in module LaueTools.readmccd)}

\begin{fulllineitems}
\phantomsection\label{\detokenize{PeakSearch:LaueTools.readmccd.fitoneimage_manypeaks}}\pysiglinewithargsret{\sphinxcode{\sphinxupquote{LaueTools.readmccd.}}\sphinxbfcode{\sphinxupquote{fitoneimage\_manypeaks}}}{\emph{filename}, \emph{peaklist}, \emph{boxsize}, \emph{stackimageindex=-1}, \emph{CCDLabel='PRINCETON'}, \emph{dirname=None}, \emph{position\_start='max'}, \emph{type\_of\_function='gaussian'}, \emph{guessed\_peaksize=(1.0}, \emph{1.0)}, \emph{xtol=0.001}, \emph{FitPixelDev=2.0}, \emph{Ipixmax=None}, \emph{MaxIntensity=100000000000}, \emph{MinIntensity=0}, \emph{PeakSizeRange=(0}, \emph{200)}, \emph{verbose=0}, \emph{position\_definition=1}, \emph{NumberMaxofFits=500}, \emph{ComputeIpixmax=False}, \emph{use\_data\_corrected=None}, \emph{reject\_negative\_baseline=True}, \emph{purgeDuplicates=True}}{}
fit multiple ROI data to get peaks position in a single image

Ipixmax  :  highest intensity above background in every ROI centered on element of peaklist
\begin{description}
\item[{use\_data\_corrected}] \leavevmode{[}enter data instead of reading data from file{]}
must be a tuple of 3 elements:
fulldata, framedim, fliprot
where fulldata is an ndarray

\end{description}

purgeDuplicates    : True   remove duplicates that are close within pixel distance of ‘boxsize’ and keep the most intense peak
\begin{description}
\item[{use\_data\_corrected}] \leavevmode{[}enter data instead of reading data from file{]}
must be a tuple of 3 elements:
fulldata, framedim, fliprot
where fulldata  ndarray

\end{description}

\begin{sphinxadmonition}{note}{Note:}
used in PeakSearchGUI
\end{sphinxadmonition}

\end{fulllineitems}

\index{PeakSearch() (in module LaueTools.readmccd)}

\begin{fulllineitems}
\phantomsection\label{\detokenize{PeakSearch:LaueTools.readmccd.PeakSearch}}\pysiglinewithargsret{\sphinxcode{\sphinxupquote{LaueTools.readmccd.}}\sphinxbfcode{\sphinxupquote{PeakSearch}}}{\emph{filename}, \emph{stackimageindex=-1}, \emph{CCDLabel='PRINCETON'}, \emph{center=None}, \emph{boxsizeROI=(200}, \emph{200)}, \emph{PixelNearRadius=5}, \emph{removeedge=2}, \emph{IntensityThreshold=400}, \emph{thresholdConvolve=200}, \emph{paramsHat=(4}, \emph{5}, \emph{2)}, \emph{boxsize=15}, \emph{verbose=0}, \emph{position\_definition=1}, \emph{local\_maxima\_search\_method=1}, \emph{peakposition\_definition='max'}, \emph{fit\_peaks\_gaussian=1}, \emph{xtol=1e-05}, \emph{return\_histo=1}, \emph{FitPixelDev=25}, \emph{write\_execution\_time=1}, \emph{Saturation\_value=65535}, \emph{Saturation\_value\_flatpeak=65535}, \emph{MinIntensity=0}, \emph{PeakSizeRange=(0}, \emph{200)}, \emph{Data\_for\_localMaxima=None}, \emph{Fit\_with\_Data\_for\_localMaxima=False}, \emph{Remove\_BlackListedPeaks\_fromfile=None}, \emph{maxPixelDistanceRejection=15.0}, \emph{NumberMaxofFits=5000}, \emph{reject\_negative\_baseline=True}, \emph{formulaexpression='A-1.1*B'}, \emph{listrois=None}}{}
Find local intensity maxima as starting position for fittinng and return peaklist.

Parameters
filename : string
\begin{quote}

full path to image data file
\end{quote}
\begin{description}
\item[{stackimageindex}] \leavevmode{[}integer{]}
index corresponding to the position of image data on a stacked images file
if -1  means single image data w/o stacking

\item[{CCDLabel}] \leavevmode{[}string{]}
label for CCD 2D detector used to read the image data file see dict\_LaueTools.py

\end{description}

center : position \#TODO: to be removed: position of the ROI center in CCD frame
\begin{description}
\item[{boxsizeROI}] \leavevmode{[}dimensions of the ROI to crop the data array{]}
only used if center != None

\item[{boxsize}] \leavevmode{[}half length of the selected ROI array centered on each peak:{]}
for fitting a peak
for estimating the background around a peak
for shifting array in second method of local maxima search (shifted arrays)

\item[{IntensityThreshold}] \leavevmode{[}integer{]}\begin{quote}

pixel intensity level above which potential peaks are kept for fitting position procedure
\end{quote}

for local maxima method 0 and 1, this level is relative to zero intensity
for local maxima method 2, this level is relative to lowest intensity in the ROI (local background)
start with high value
If too high, few peaks are found (only the most important)
If too low, too many local maxima are found leading to time consuming fitting procedure

\item[{thresholdConvolve}] \leavevmode{[}integer{]}
pixel intensity level in convolved image above which potential peaks are kept for fitting position procedure
This threshold step on convolved image is applied prior to the local threshold step with IntensityThreshold on initial image (with respect to the local background)

\end{description}

paramsHat :  mexican hat kernel parameters (see \sphinxcode{\sphinxupquote{LocalMaxima\_ndimage()}})
\begin{description}
\item[{PixelNearRadius: integer}] \leavevmode\begin{quote}

pixel distance between two regions considered as peaks
\end{quote}

start rather with a large value.
If too low, there are very much peaks duplicates and
this is very time consuming

\item[{local\_maxima\_search\_method}] \leavevmode{[}integer{]}\begin{quote}

Select method for find the local maxima, each of them will fitted
\end{quote}

: 0   extract all pixel above intensity threshold
: 1   find pixels are highest than their neighbours in horizontal, vertical
\begin{quote}

and diagonal direction (up to a given pixel distance)
\end{quote}
\begin{description}
\item[{: 2   find local hot pixels which after numerical convolution give high intensity}] \leavevmode
above threshold (thresholdConvolve)
then threshold (IntensityThreshold) on raw data

\end{description}

\item[{peakposition\_definition}] \leavevmode{[}‘max’ or ‘center’  for local\_maxima\_search\_method == 2{]}\begin{description}
\item[{to assign to the blob position its hottest pixel position}] \leavevmode
or its center (no weight)

\end{description}

\end{description}

Saturation\_value\_flatpeak        :  saturation value of detector for local maxima search method 1
\begin{description}
\item[{Remove\_BlackListedPeaks\_fromfile}] \leavevmode{[}None or full file path to a peaklist file containing peaks{]}
that will be deleted in peak list resulting from
the local maxima search procedure (prior to peak refinement)

\item[{maxPixelDistanceRejection}] \leavevmode{[}maximum distance between black listed peaks and current peaks{]}
(found by peak search) to be rejected

\end{description}

NumberMaxofFits            : highest acceptable number of local maxima peak to be refined with a 2D modelPeakSearch
\begin{description}
\item[{fit\_peaks\_gaussian}] \leavevmode{[}0  no position and shape refinement procedure performed from local maxima (or blob) results{]}
:    1  2D gaussian peak refinement
:    2  2D lorentzian peak refinement

\end{description}

xtol  : relative error on solution (x vector)  see args for leastsq in scipy.optimize
FitPixelDev            :  largest pixel distance between initial (from local maxima search)
\begin{quote}

and refined peak position
\end{quote}
\begin{description}
\item[{position\_definition: due to various conventional habits when reading array, add some offset to fitdata XMAS or fit2d peak search values}] \leavevmode
= 0    no offset (python numpy convention)
= 1   XMAS offset
= 2   fit2d offset

\item[{return\_histo}] \leavevmode{[}0   3 output elements{]}
: 1   4 elemts, last one is histogram of data
: 2   4 elemts, last one is the nb of raw blob found after convolution and threshold

\item[{Data\_for\_localMaxima}] \leavevmode{[}object to be used only for initial step of finding local maxima (blobs) search{]}\begin{quote}

(and not necessarly for peaks fitting procedure):
\end{quote}
\begin{itemize}
\item {} 
ndarray     = array data

\item {} 
‘auto\_background’  = calculate and remove background computed from image data itself (read in file ‘filename’)

\item {} 
path to image file (string)  = B image to be used in a mathematical operation with Ato current image

\end{itemize}

\item[{Fit\_with\_Data\_for\_localMaxima}] \leavevmode{[}use ‘Data\_for\_localMaxima’ object as image when refining peaks position and shape{]}
with initial peak position guess from local maxima search

\item[{formulaexpression}] \leavevmode{[}string containing A (raw data array image) and B (other data array image){]}
expressing mathematical operation,e.g:
‘A-3.2*B+10000’
for simple background substraction (with B as background data):
‘A-B’ or ‘A-alpha*B’ with alpha \textgreater{} 1.

\item[{reject\_negative\_baseline}] \leavevmode{[}True  reject refined peak result if intensity baseline (local background) is negative{]}
(2D model is maybe not suitable)

\end{description}

returns:

peak list sorted by decreasing (integrated intensity - fitted bkg)
peak\_X,peak\_Y,peak\_I,peak\_fwaxmaj,peak\_fwaxmin,peak\_inclination,Xdev,Ydev,peak\_bkg
\begin{description}
\item[{for fit\_peaks\_gaussian == 0 (no fitdata) and local\_maxima\_search\_method==2 (convolution)}] \leavevmode
if peakposition\_definition =’max’ then X,Y,I are from the hottest pixels
if peakposition\_definition =’center’ then X,Y are blob center and I the hottest blob pixel

\end{description}

nb of output elements depends on ‘return\_histo’ argument

\end{fulllineitems}

\index{peaksearch\_on\_Image() (in module LaueTools.readmccd)}

\begin{fulllineitems}
\phantomsection\label{\detokenize{PeakSearch:LaueTools.readmccd.peaksearch_on_Image}}\pysiglinewithargsret{\sphinxcode{\sphinxupquote{LaueTools.readmccd.}}\sphinxbfcode{\sphinxupquote{peaksearch\_on\_Image}}}{\emph{filename\_in}, \emph{pspfile}, \emph{background\_flag='no'}, \emph{blacklistpeaklist=None}, \emph{dictPeakSearch=\{\}}, \emph{CCDLabel='MARCCD165'}, \emph{outputfilename=None}, \emph{psdict\_Convolve=\{'Data\_for\_localMaxima': 'auto\_background'}, \emph{'FitPixelDev': 2.0}, \emph{'IntensityThreshold': 10}, \emph{'NumberMaxofFits': 5000}, \emph{'PixelNearRadius': 10}, \emph{'boxsize': 15}, \emph{'fit\_peaks\_gaussian': 1}, \emph{'local\_maxima\_search\_method': 2}, \emph{'position\_definition': 1}, \emph{'removeedge': 2}, \emph{'return\_histo': 0}, \emph{'thresholdConvolve': 500}, \emph{'verbose': 0}, \emph{'write\_execution\_time': 0}, \emph{'xtol': 0.001\}}}{}
Perform a peaksearch by using .psp file

\# still not very used and checked?
\# missing dictPeakSearch   as function argument for formulaexpression  or dict\_param??

\end{fulllineitems}

\index{read\_background\_flag() (in module LaueTools.readmccd)}

\begin{fulllineitems}
\phantomsection\label{\detokenize{PeakSearch:LaueTools.readmccd.read_background_flag}}\pysiglinewithargsret{\sphinxcode{\sphinxupquote{LaueTools.readmccd.}}\sphinxbfcode{\sphinxupquote{read\_background\_flag}}}{\emph{background\_flag}}{}
interpret the background flag (field used in FileSeries/Peak\_Search.py)

return two values to put in dict\_param of peaksearch\_series

\end{fulllineitems}

\index{plot\_image\_markers() (in module LaueTools.readmccd)}

\begin{fulllineitems}
\phantomsection\label{\detokenize{PeakSearch:LaueTools.readmccd.plot_image_markers}}\pysiglinewithargsret{\sphinxcode{\sphinxupquote{LaueTools.readmccd.}}\sphinxbfcode{\sphinxupquote{plot\_image\_markers}}}{\emph{image}, \emph{markerpos}, \emph{position\_definition=1}}{}
plot 2D array (image) with markers at first two columns of (markerpos)

\begin{sphinxadmonition}{note}{Note:}
used in LaueHDF5. Could be better implementation in some notebooks
\end{sphinxadmonition}

\end{fulllineitems}

\index{applyformula\_on\_images() (in module LaueTools.readmccd)}

\begin{fulllineitems}
\phantomsection\label{\detokenize{PeakSearch:LaueTools.readmccd.applyformula_on_images}}\pysiglinewithargsret{\sphinxcode{\sphinxupquote{LaueTools.readmccd.}}\sphinxbfcode{\sphinxupquote{applyformula\_on\_images}}}{\emph{A}, \emph{B}, \emph{formulaexpression='A-B'}, \emph{SaturationLevel=None}, \emph{clipintensities=True}}{}
calculate image data array from math expression

A, B            : ndarray  of the same shape

SaturationLevel  :  saturation level of intensity

clipintensities    :   clip resulting intensities to zero and saturation value

\end{fulllineitems}

\index{peaksearch\_fileseries() (in module LaueTools.readmccd)}

\begin{fulllineitems}
\phantomsection\label{\detokenize{PeakSearch:LaueTools.readmccd.peaksearch_fileseries}}\pysiglinewithargsret{\sphinxcode{\sphinxupquote{LaueTools.readmccd.}}\sphinxbfcode{\sphinxupquote{peaksearch\_fileseries}}}{\emph{fileindexrange}, \emph{filenameprefix}, \emph{suffix=''}, \emph{nbdigits=4}, \emph{dirname\_in='/home/micha/LaueProjects/AxelUO2'}, \emph{outputname=None}, \emph{dirname\_out=None}, \emph{CCDLABEL='MARCCD165'}, \emph{KF\_DIRECTION='Z\textgreater{}0'}, \emph{dictPeakSearch=None}}{}
peaksearch function to be called for multi or single processing

\end{fulllineitems}

\index{peaksearch\_multiprocessing() (in module LaueTools.readmccd)}

\begin{fulllineitems}
\phantomsection\label{\detokenize{PeakSearch:LaueTools.readmccd.peaksearch_multiprocessing}}\pysiglinewithargsret{\sphinxcode{\sphinxupquote{LaueTools.readmccd.}}\sphinxbfcode{\sphinxupquote{peaksearch\_multiprocessing}}}{\emph{fileindexrange}, \emph{filenameprefix}, \emph{suffix=''}, \emph{nbdigits=4}, \emph{dirname\_in='/home/micha/LaueProjects/AxelUO2'}, \emph{outputname=None}, \emph{dirname\_out=None}, \emph{CCDLABEL='MARCCD165'}, \emph{KF\_DIRECTION='Z\textgreater{}0'}, \emph{dictPeakSearch=None}, \emph{nb\_of\_cpu=2}}{}
launch several processes in parallel

\end{fulllineitems}

\index{gauss\_kern() (in module LaueTools.readmccd)}

\begin{fulllineitems}
\phantomsection\label{\detokenize{PeakSearch:LaueTools.readmccd.gauss_kern}}\pysiglinewithargsret{\sphinxcode{\sphinxupquote{LaueTools.readmccd.}}\sphinxbfcode{\sphinxupquote{gauss\_kern}}}{\emph{size}, \emph{sizey=None}}{}
Returns a normalized 2D gauss kernel array for convolutions

\end{fulllineitems}

\index{blur\_image() (in module LaueTools.readmccd)}

\begin{fulllineitems}
\phantomsection\label{\detokenize{PeakSearch:LaueTools.readmccd.blur_image}}\pysiglinewithargsret{\sphinxcode{\sphinxupquote{LaueTools.readmccd.}}\sphinxbfcode{\sphinxupquote{blur\_image}}}{\emph{im}, \emph{n}, \emph{ny=None}}{}~\begin{description}
\item[{blurs the image by convolving with a gaussian kernel of typical}] \leavevmode
size n. The optional keyword argument ny allows for a different
size in the y direction.

\end{description}

\end{fulllineitems}

\index{blurCCD() (in module LaueTools.readmccd)}

\begin{fulllineitems}
\phantomsection\label{\detokenize{PeakSearch:LaueTools.readmccd.blurCCD}}\pysiglinewithargsret{\sphinxcode{\sphinxupquote{LaueTools.readmccd.}}\sphinxbfcode{\sphinxupquote{blurCCD}}}{\emph{im}, \emph{n}}{}
apply a blur filter to image ndarray

\end{fulllineitems}

\index{circularMask() (in module LaueTools.readmccd)}

\begin{fulllineitems}
\phantomsection\label{\detokenize{PeakSearch:LaueTools.readmccd.circularMask}}\pysiglinewithargsret{\sphinxcode{\sphinxupquote{LaueTools.readmccd.}}\sphinxbfcode{\sphinxupquote{circularMask}}}{\emph{center}, \emph{radius}, \emph{arrayshape}}{}
return a boolean ndarray of elem in array inside a mask

\end{fulllineitems}

\index{compute\_autobackground\_image() (in module LaueTools.readmccd)}

\begin{fulllineitems}
\phantomsection\label{\detokenize{PeakSearch:LaueTools.readmccd.compute_autobackground_image}}\pysiglinewithargsret{\sphinxcode{\sphinxupquote{LaueTools.readmccd.}}\sphinxbfcode{\sphinxupquote{compute\_autobackground\_image}}}{\emph{dataimage}, \emph{boxsizefilter=10}}{}
return 2D array of filtered data array
:param dataimage: array of image data
:type dataimage: 2D array

\end{fulllineitems}

\index{computefilteredimage() (in module LaueTools.readmccd)}

\begin{fulllineitems}
\phantomsection\label{\detokenize{PeakSearch:LaueTools.readmccd.computefilteredimage}}\pysiglinewithargsret{\sphinxcode{\sphinxupquote{LaueTools.readmccd.}}\sphinxbfcode{\sphinxupquote{computefilteredimage}}}{\emph{dataimage}, \emph{bkg\_image}, \emph{CCDlabel}, \emph{kernelsize=5}, \emph{formulaexpression='A-B'}, \emph{usemask=True}}{}
return 2D array of initial image data without background given by bkg\_image data
\begin{description}
\item[{usemask}] \leavevmode{[}True  then substract bkg image on masked raw data{]}
False  apply formula on all pixels (no mask)

\end{description}
\begin{quote}\begin{description}
\item[{Parameters}] \leavevmode\begin{itemize}
\item {} 
\sphinxstyleliteralstrong{\sphinxupquote{dataimage}} (\sphinxstyleliteralemphasis{\sphinxupquote{2D array}}) \textendash{} array of image data

\item {} 
\sphinxstyleliteralstrong{\sphinxupquote{bkg\_image}} (\sphinxstyleliteralemphasis{\sphinxupquote{2D array}}) \textendash{} array of filtered image data (background)

\item {} 
\sphinxstyleliteralstrong{\sphinxupquote{CCDlabel}} (\sphinxstyleliteralemphasis{\sphinxupquote{string}}) \textendash{} key for CCD dictionary

\end{itemize}

\end{description}\end{quote}

\end{fulllineitems}

\index{filterimage() (in module LaueTools.readmccd)}

\begin{fulllineitems}
\phantomsection\label{\detokenize{PeakSearch:LaueTools.readmccd.filterimage}}\pysiglinewithargsret{\sphinxcode{\sphinxupquote{LaueTools.readmccd.}}\sphinxbfcode{\sphinxupquote{filterimage}}}{\emph{image\_array}, \emph{framedim}, \emph{blurredimage=None}, \emph{kernelsize=5}, \emph{mask\_parameters=None}, \emph{clipvalues=None}, \emph{imageformat=\textless{}class 'numpy.uint16'\textgreater{}}}{}
compute a difference of images inside a region defined by a mask

blurredimage:    ndarray image to substract to image\_array
kernelsize:    pixel size of gaussian kernel if blurredimage is None

mask\_parameters: circular mask parameter: center=(x,y), radius, value outside mask

\end{fulllineitems}

\index{blurCCD\_with\_binning() (in module LaueTools.readmccd)}

\begin{fulllineitems}
\phantomsection\label{\detokenize{PeakSearch:LaueTools.readmccd.blurCCD_with_binning}}\pysiglinewithargsret{\sphinxcode{\sphinxupquote{LaueTools.readmccd.}}\sphinxbfcode{\sphinxupquote{blurCCD\_with\_binning}}}{\emph{im}, \emph{n}, \emph{binsize=(2}, \emph{2)}}{}
blur the array by rebinning before and after aplying the filter

\end{fulllineitems}

\index{remove\_minimum\_background() (in module LaueTools.readmccd)}

\begin{fulllineitems}
\phantomsection\label{\detokenize{PeakSearch:LaueTools.readmccd.remove_minimum_background}}\pysiglinewithargsret{\sphinxcode{\sphinxupquote{LaueTools.readmccd.}}\sphinxbfcode{\sphinxupquote{remove\_minimum\_background}}}{\emph{im}, \emph{boxsize=10}}{}
remove to image array the array resulting from minimum\_filter

\end{fulllineitems}

\index{purgePeaksListFile() (in module LaueTools.readmccd)}

\begin{fulllineitems}
\phantomsection\label{\detokenize{PeakSearch:LaueTools.readmccd.purgePeaksListFile}}\pysiglinewithargsret{\sphinxcode{\sphinxupquote{LaueTools.readmccd.}}\sphinxbfcode{\sphinxupquote{purgePeaksListFile}}}{\emph{filename1}, \emph{blacklisted\_XY}, \emph{dist\_tolerance=0.5}, \emph{dirname=None}}{}
remove in peaklist .dat file peaks that are in blacklist

blacklisted\_XY:         {[}X1,Y1{]},{[}X2,Y2{]}

\end{fulllineitems}

\index{write\_PurgedPeakListFile() (in module LaueTools.readmccd)}

\begin{fulllineitems}
\phantomsection\label{\detokenize{PeakSearch:LaueTools.readmccd.write_PurgedPeakListFile}}\pysiglinewithargsret{\sphinxcode{\sphinxupquote{LaueTools.readmccd.}}\sphinxbfcode{\sphinxupquote{write\_PurgedPeakListFile}}}{\emph{filename1}, \emph{blacklisted\_XY}, \emph{outputfilename}, \emph{dist\_tolerance=0.5}, \emph{dirname=None}}{}
write a new .dat file where peaks in blacklist are omitted

\end{fulllineitems}

\index{removePeaks\_inPeakList() (in module LaueTools.readmccd)}

\begin{fulllineitems}
\phantomsection\label{\detokenize{PeakSearch:LaueTools.readmccd.removePeaks_inPeakList}}\pysiglinewithargsret{\sphinxcode{\sphinxupquote{LaueTools.readmccd.}}\sphinxbfcode{\sphinxupquote{removePeaks\_inPeakList}}}{\emph{PeakListfilename}, \emph{BlackListed\_PeakListfilename}, \emph{outputfilename}, \emph{dist\_tolerance=0.5}, \emph{dirname=None}}{}
read peaks PeakListfilename and remove those in BlackListed\_PeakListfilename
and write a new peak list file

\begin{sphinxadmonition}{note}{Note:}
Not used ??
\end{sphinxadmonition}

\end{fulllineitems}

\index{merge\_2Peaklist() (in module LaueTools.readmccd)}

\begin{fulllineitems}
\phantomsection\label{\detokenize{PeakSearch:LaueTools.readmccd.merge_2Peaklist}}\pysiglinewithargsret{\sphinxcode{\sphinxupquote{LaueTools.readmccd.}}\sphinxbfcode{\sphinxupquote{merge\_2Peaklist}}}{\emph{filename1}, \emph{filename2}, \emph{dist\_tolerance=5}, \emph{dirname1=None}, \emph{dirname2=None}, \emph{verbose=0}}{}
return merge spots data from two peaklists and removed duplicates within dist\_tolerance (pixel)

\end{fulllineitems}

\index{writefile\_mergedPeaklist() (in module LaueTools.readmccd)}

\begin{fulllineitems}
\phantomsection\label{\detokenize{PeakSearch:LaueTools.readmccd.writefile_mergedPeaklist}}\pysiglinewithargsret{\sphinxcode{\sphinxupquote{LaueTools.readmccd.}}\sphinxbfcode{\sphinxupquote{writefile\_mergedPeaklist}}}{\emph{filename1}, \emph{filename2}, \emph{outputfilename}, \emph{dist\_tolerance=5}, \emph{dirname1=None}, \emph{dirname2=None}, \emph{verbose=0}}{}
write peaklist file from the merge of spots data from two peaklists
(and removed duplicates within dist\_tolerance (pixel))

\end{fulllineitems}



\section{Modules for Laue Pattern Indexation}
\label{\detokenize{LaueToolsModules:modules-for-laue-pattern-indexation}}

\section{Modules for Crystal unit cell refinement}
\label{\detokenize{LaueToolsModules:modules-for-crystal-unit-cell-refinement}}

\section{Modules for batch processing}
\label{\detokenize{LaueToolsModules:modules-for-batch-processing}}

\renewcommand{\indexname}{Python Module Index}
\begin{sphinxtheindex}
\def\bigletter#1{{\Large\sffamily#1}\nopagebreak\vspace{1mm}}
\bigletter{l}
\item {\sphinxstyleindexentry{LaueTools.CrystalParameters}}\sphinxstyleindexpageref{Simulation_Module:\detokenize{module-LaueTools.CrystalParameters}}
\item {\sphinxstyleindexentry{LaueTools.GUI.PeakSearchGUI}}\sphinxstyleindexpageref{PeakSearchGUI:\detokenize{module-LaueTools.GUI.PeakSearchGUI}}
\item {\sphinxstyleindexentry{LaueTools.lauecore}}\sphinxstyleindexpageref{Simulation_Module:\detokenize{module-LaueTools.lauecore}}
\item {\sphinxstyleindexentry{LaueTools.LaueGeometry}}\sphinxstyleindexpageref{Simulation_Module:\detokenize{module-LaueTools.LaueGeometry}}
\item {\sphinxstyleindexentry{LaueTools.multigrainsSimulator}}\sphinxstyleindexpageref{Simulation_Module:\detokenize{module-LaueTools.multigrainsSimulator}}
\item {\sphinxstyleindexentry{LaueTools.readmccd}}\sphinxstyleindexpageref{PeakSearch:\detokenize{module-LaueTools.readmccd}}
\end{sphinxtheindex}

\renewcommand{\indexname}{Index}
\printindex
\end{document}